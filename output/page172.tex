\section*{Wesen und Begriff der Theologie.}\hfill 161\par\parauf dem Wissensgebiet möglich, das es nicht mit wirklich existierenden, sondern nur mit gedachten Dingen zu tun hat, wie dies bei der \"reinen\" Mathematik (im Unterschiede von der \"angewandten\" Mathematik) der Fall ist. Die spekulative Systembildung ist schon in ihrer Anwendung auf naturwissenschaftliche und geschichtliche Tatsachen nicht nur unwissenschaftlich, sondern geradezu unsinnig, weil ihr der Wahn zugrunde liegt, dass sich Tatsachen nach dem menschlichen Denken richten. Mit Recht ist der philosophische Idealismus eine Krankheit des menschlichen Geistes genannt worden, in der der Mensch sich einbildet, dass seine Gedanken (Ideen) Regel und Maß der Dinge seien. \emph{E d m. Hoppe} bemerkte in der Zeitschrift \"Der Alte Glaube\": \"Die Natur\" und wir segen analog hinzu: auch die Geschichte \"ist nicht so liebenswürdig, sich an das Schema des Lehrbuchs zu binden.\"\textsuperscript{556} Wollte indes es in der Theologie jede menschliche oder spekulative Systembildung schlechthin ausgeschlossen, weil die christliche Lehre in der Schrift als eine göttliche Größe gegeben ist, an der menschliches Denken nichts ändern kann noch soll. Hier ist jeder Zusatz und jeder Abzug ausdrücklich untersagt.\textsuperscript{557} Die Tätigkeit des Theologen besteht daher weder darin, die christliche Lehre aus einem obersten Grundsatz oder aus einer Tatsache, z.B. aus der Tatsache der Wiedergeburt, durch Denken zu entwickeln, noch auch darin, aus dem sogenannten \"Ganzen der Schrift\", was ein logisches Monstrum ist, zu konstruieren, sondern lediglich darin, die christliche Lehre in allen ihren Teilen direkt den Schriftausagen zu entnehmen, die von den betreffenden Lehren handeln (sedes doctrinae). Wenn wir so das, was die Schrift selbst über die einzelnen Lehren aussagt, an einen Ort zusammenstellen, haben wir das geordnete Wissen der christlichen Lehre, das in diesem Leben für uns Menschen erreichbar und nötig ist. Auch über den Zusammenhang der in der Schrift vorliegenden Lehren können wir selbstverständlich nur so viel lehren, als darüber in der Schrift gesagt ist. Bei dieser Methode bleiben Lücken für das menschliche Begreifen in diesem Leben. Als Beispiele hierfür wurden schon früher angeführt die Schriftlehren vom Seligmacherwerden Sola Dei gratia und vom Verlorengehen Sola hominum culpa. Beide Lehren sind klar in der Schrift geoffenbart. Wer sie aber systematisieren, das heißt, zu einer Einheit im Sinne der menschlichen Vernunft ver-\par\vspace{1em}\hrule\begin{footnotesize}\par\vspace{0.5em}\noindent\textsuperscript{556} Zitiert in \emph{L. u. A.} 1907, S. 316.\par\noindent\textsuperscript{557} Iov. 23, 6; Matth. 5, 17--19; Joh. 10, 35; 8, 31. 32; Gal. 1, 6--9.\par\noindent W. Rieber, Dogmatik. I.\hfill 11\end{footnotesize}