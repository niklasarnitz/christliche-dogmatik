\markright{Wesen und Begriff der Theologie.}\thispagestyle{plain}\setcounter{page}{166}Stoff nur zusammenordnen, indem sie zugleich ihn empirisch erkennt. Zweitens kann solch zusammengeordneter empirischer Stoff ein in sich geschlossenes Ganzes, ein System, nur dann ergeben, wenn dieser Stoff selbst in sich ein Ganzes ist. \dots{} Woraus wir denn auch zugleich sehen, dass solch System sich vor allem seine empirischen Quellen suchen und sich fortwährend an dieselben halten muss, sich nur Hand in Hand mit der empirischen Forschung erbauen und vor Absichtlich dieser selbst nicht fertig werden kann.\setcounter{footnote}{561}\footnote{Mit andern Worten und in Erinnerung an Edm. Hoppe: Die Natur, als etwas tatsächlich Gegebenes, ist nicht so liebenswürdig, uns nach dem Naturforscher zu richten, sondern der Naturforscher muss sich, solange er vernünftig bleibt, nach der Natur richten. Ebensowenig sind die geschichtlichen Tatsachen so liebenswürdig, sich nach dem Geschichtsschreiber zu richten, sondern der Geschichtsschreiber muss sich, wenn er nicht in einen Romanschreiber ausarten will, nach den bezeugten geschichtlichen Tatsachen richten. Wo der Naturforscher und der Geschichtsschreiber anfangen, zur Hypothese zu greifen oder „Geseze zu treiben“, da sollen sie das sagen, um so dem Publikum kenntlich zu machen, wo ihrer -- des Naturforschers und des Geschichtsschreibers -- Wissenschaft aufgehört habe.} \dots{} Von ihrer Grundverschiedenheit aber mit jener eigenen Äußerungen zweifellos vor, welche den spekulativen Philosophen aller Zeiten vorgeschwebt hat; sie wollten, ausgehend von irgendeinem Einfachsten, aus diesem Einfachsten unter Zurückweisung aller Empirie durch Selbstentfaltung jenes Einfachsten ein System von Erkenntnissen hervorgehen lassen der Hoffnung, die unverbrüchliche Notwendigkeit dieses Entfaltungsprozesses werde solchem System und seinen einzelnen Sätzen eine solche Richtigkeit und Gewissheit geben, dass es dann hinterher nicht allein mit allem, was die Empirie uns erkennen lässt, sich decken, sondern auch für das empirische Erkennen erst den rechten Schlüssel hergeben werde. \dots{} Es bedarf nicht erst des Nachweises, dass diese grundsätzlich von der Empirie absehende spekulative Art von Systembildung ganz etwas anderes ist als jene bloß auf Stoffentsprechende Zusammenordnung empirisch gewonnener Erkenntnisse ausgehende erste Art. Sehen wir nun beide auf ihr Verhältnis zu der Heilslehre an, so ist mir außer Zweifel, dass die spekulative Methode weder ganz noch halb anwendlich auf dieselbe ist, da Gott sein Heil geschichtlich in Wort und Werk offenbart hat, also auch will, dass es auf empirische Weise von uns erkannt werde. Dagegen ist ebenso gewiss und selbstverständlich, dass ich die erste empirische Art systematischer Behandlung für anwendlich auf die Heilslehre halte, ja, dass sie mir auf die Heilslehre mehr als auf irgend etwas anderes in der