\section*{Vorwort.}\n\nMit dem Erscheinen dieses Bandes liegt meine \emph{Christliche Dogmatik} nun vollständig gedruckt vor. Es ist öffentlich gefragt worden, warum der zweite und dritte Band zuerst erschienen sind. Der Grund ist der, dass der Wunsch geäußert wurde, es möchte im großen Jubiläumsjahr 1917 zuerst der Band gedruckt werden, in dem die Lehren von der Gnade Gottes in Christo, von Christi Person und Werk und von der Rechtfertigung zur Darstellung kommen. An den zweiten Band schloss sich naturgemäß der dritte Band, in dem die Folgen der christlichen Rechtfertigungslehre beschrieben werden.\n\nIn dem vorliegenden Bande nehmen die ersten zwei Kapitel, \emph{Wesen und Begriff der Theologie} und \emph{Die Heilige Schrift}, mehr als die Hälfte des Raumes ein. Dies erklärt sich aus der Tatsache, dass in der modernen protestantischen Theologie unchristliche Vorstellungen vom Wesen und Begriff der Theologie sich eingebürgert haben. Dies ist aber nur die notwendige Folge des Abfalls von der christlichen Wahrheit, dass die Heilige Schrift Gottes eigenes unfehlbares Wort ist. Wie wir in der römischen Kirche einen völligen prinzipiellen Zusammenhang der christlichen Theologie vor Augen haben, weil dort die subjektive Anschauung des Papstes die alles bestimmende Macht ist, so haben wir nun dieselbe Sachlage in der modern-protestantischen Theologie, weil diese die objektive göttliche Autorität der heiligen Schrift preisgegeben und sich in das \emph{christliche Erlebnis}, das ist, in die subjektive Anschauung \emph{des theologisierenden Subjekts}, geflüchtet hat. Dies erklärt, wie gesagt, die ausführliche Behandlung der