150\hfill Wesen und Begriff der Theologie.\par\parἐν τῷ λόγω τοῦ Χριστοῦ). Luther weiß schlagend nach, dass die ersten alten „Hauptkonzilia“ mit ihrem ὁμοούσιος, θεοτόκος usw. nichts Neues gemacht, sondern nur die Lehren bekannt haben, die die Christenheit von dem Anfang an auf Grund der Schrift geglaubt hat.\footnote{512}{Weiß Luther in seiner Schrift „Von den Konzilien und Kirchen“ von den einzelnen Konzilien nach und sagt dann in Zusammenfassung St. V. XVI 2248 f.: „Also haben wir die vier Hauptkonzilia und die Ursachen, warum sie gehalten sind. Das erste, zu Nizäa, hat die Gottheit Christi wider Areium verteidigt, das andere, zu Konstantinopel, die Gottheit des Heiligen Geistes wider Macedonium verteidigt, das dritte, zu Ephesus, in Christo eine Person wider Nestorium verteidigt, das vierte, in Chalcedon, zwei Naturen in Christo wider Eutychen verteidigt, aber damit keinen neuen Artikel des Glaubens gestellt. Denn solche vier Artikel und viel und reichlicher und gewaltiger, auch allen in St. Johannis Evangelio gestellet, wenngleich die anderen Evangelisten und St. Paulus, St. Petrus hievon nichts hatten geschrieben, die doch dieses alles auch gewaltiglich lehren und zeugen samt allen Propheten. Haben nun diese vier Hauptkonzilia (welche von den Bischöfen zu Rom den vier Evangelien nach ihrem Defekt gleichschalten sind, gerade als fänden solche Stücke nicht viel reichlicher neben allen Artikeln in den Evangelien — so sein versehen die Geschichte, was Evangelia oder Konzilia sind) nichts Neues wollen noch können in Glaubensartikeln machen oder setzen, wie sie selbst bekennen, wieviel weniger kann man solche Macht neben den andern Konzilien, die man geringer Macht hält, wo diese vier sollen die Hauptkonzilia sein und heißen... Sehet sie aber etwas Neues im Glauben oder guten Werken, so sei gewiss, dass der heilige Geist nicht da sei, sondern der unheilige Geist mit seinem Engeln.“} Auch durch die Reformation der Kirche ist die christliche Lehre nicht im geringsten fortgebildet, sondern lediglich die alte Lehre der Schrift aus dem papistischen Wust der Menschenlehren wieder hervorgezogen, gelehrt und bekannt worden. Dessen war Luther sich wohl bewusst. Er sagt:\footnote{513}{St. VI, 2, XVII, 1324.} „Wir erdichten nichts Neues, sondern halten und bleiben bei dem alten Gotteswort, wie es die alte Kirche gehabt; darum sind wir mit derselben die rechte alte Kirche, die einerlei Gotteswort lehret und glaubet. Darum lästern die Papisten abermal Christum selbst, die Apostel und ganze Christenheit, wenn sie uns Neue und Ketzer schelten. Denn sie finden nichts bei uns denn allein das Alte der alten Kirche.“ Es liegt ein Nichtverstehen der Reformation vor, wenn moderne Theologen, um sich mit ihrer Lehrfortbildung unter Luthers Protektion zu stellen, von einem „neuen Verständnis des Christentums in der Reformation“ reden. Dass sie nichts Neues lehre, ist auch der Ruhm der Augustana.\footnote{514}{Art. XXI. M. 47, 1–5; 48, 1–6.} Ebenso will die Konkordienformel nichts Neues wegen, wie aus ihrer prinzipiellen Erklärung hervorgeht: „dass die einzige