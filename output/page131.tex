Wissenschaft auf dem Gebiete der Theologie eine höhere Stufe der Erkenntnis im Vergleich mit der Erkenntnis des Glaubens, so ist auch in diesem Sinne die Theologie nicht eine Wissenschaft. Der Grund hierfür liegt in der Tatsache, dass auch die gelehrtesten Theologen mit ihrer Erkenntnis von geistlichen Dingen nicht über die ausdrückliche Offenbarung der heiligen Schrift hinauskommen. Auch bei ihnen, wie bei allen Christen, bleibt in diesem Leben das Organ der Erkenntnis der geistlichen Dinge (\emph{das medium cognoscendi}) lediglich der Glaube. Noch anders ausgedrückt: auch der Theologe erkennt von geistlichen Dingen nur so viel, als er auf Grund des geoffenbarten Wortes Gottes glaubt.\footnote{1 Kor. 13, 12; Joh. 8, 31. 32; Röm. 1, 6.} „Anfang, Mitte und Ende der Theologie ist, Gottes Wort glauben.“ Dass die Erkenntnis des Theologen in der Regel extensiv größer ist, kommt daher, dass derselbe infolge anhaltenden Studiums mehr Einzelheiten oder Nebenumstände aus der Offenbarung der Schrift erkennt. Nicht steht es etwa so, dass der Theologe das wüsste, was die anderen Christen nur glauben. Auch bei dem Theologen ist Wissen Glaube und Glaube Wissen. Die philologischen, philosophischen, historischen usw. Kenntnisse des Theologen, die er vor anderen Christen voraushat, gehören zum äußeren theologischen Apparat, nicht zum inneren Wesen des Erkennens, und dienen, recht verwendet, nur der Erkenntnis des Glaubens. Sie dienen nämlich dem genauen Verständnis des Schriftwortes, also der genauen Auffassung der göttlichen Offenbarung, nicht befähigen oder berechtigen sie den Theologen, eigene, das heißt, aus sich selbst geschöpfte, Gedanken über geistliche Dinge zu haben. Wenn eine Anzahl neuerer Theologen die Theologie mit Vorliebe als die Wissenschaft vom Christentum in dem Sinne definieren, als ob es die Aufgabe der Theologie sei, den Glauben zum Wissen zu erheben, so liegt eine große Selbsttäuschung und ein Abfall vom Erkenntnisprinzip der Theologie vor.\par Versteht man unter Wissenschaft ein gewisses Wissen im Gegensatz zu bloßen Ansichten, Meinungen, Hypothesen usw., dann ist die Theologie die Wissenschaft \emph{par exochén}, das ist, die vollendetste Wissenschaft, die es auf Erden geben kann. Der Grund dafür ist dieser: Während wir auf allen Wissensgebieten, die in das Reich der Natur gehören, nur eine Zusammenstellung von menschlichen Beobachtungen und menschlichen Schlussfolgerungen haben, die der Natur der Sache nach (\emph{errare humanum est}) mehr oder minder unsichere Resultate ergeben (Philosophie, Astronomie, Medizin usw.), so