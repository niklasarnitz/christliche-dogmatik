Halten wir so den schriftgemäßen Begriff von offenen Fragen und theologischen Problemen fest, so liegt auf der Hand, dass es töricht wäre, auf ihre Behandlung viel Zeit und Kraft zu verwenden. Nachdem Reusch die theologischen Probleme als Dinge beschrieben hat, denen das Schriftzeugnis fehlt, fügt er hinzu: Inu-tilis est eorum cognitio, et vanae sunt de eisdem disputationes.\footnote{389) Annotationes in Baieri Comp. 1757, p. 52.}\par Dannhauer erinnert daran, dass von den Scholastikern viele Fragen behandelt werden, die in der Schrift entweder gar nicht oder doch nicht klar beantwortet sind; aber in bezug auf das Resultat und den praktischen Nutzen fügt er hinzu: „Einer meint den Bock, der andere hält das Sieb unter.“\footnote{390) Hodosophia, Phaen. XI, p. 667, ed. 1713: Unus hircum mulget, alter supponit cribrum.}\par Wir haben wahrlich genug zu tun, wenn wir das lernen und lehren und dabei bleiben, was in der Schrift geoffenbart vorliegt. Luther rechnet das Behandeln unnütziger Fragen, die uns nicht geboten sind, zu den „Hindernissen des Evangeliums“, weil dadurch die gebotenen großen Hauptsachen in den Hintergrund gedrängt werden und erziehungsmäßig die große Menge nur zu leicht für Menschengedanken zu haben ist, die die Neugierde befriedigen. So ist's, sagt Luther, den Juden mit der Erforschung ihrer Geschlechtsregister ergangen, und so hat man auch im Papsttum um unnötige Kalb- und Gewäsche gezankt, weil jeder recht haben wollte.\footnote{391) Zu I Tim. 1, 3. 4. Z. V. [IX, 863 f.]}\par Luthers warnende Worte lauten so: „Das sind zwei Hindernisse des Evangeliums: eines, so man anders lehrt, also dass man das Gesetz und Werke hinein auf die Gewissen treibt; das andere, so der Teufel, wo er sieht, dass er den Glauben nicht stracks umstoßen kann, mit List fährt und von hinten hereinschleicht und unnütze Fragen aufwirft, damit man sich bekümmere und dieweil das Hauptstück dahinbleibe, als da sind von toten Heiligen und abgeschiedenen Seelen, wo sie bleiben, ob sie schlafen, und dergleichen.“\footnote{392) Hierbei ist über den Zustand der Seelen zwischen Tod und Auferstehung aus der Schrift gewiss wissen, III, 574 ff. dargelegt.}\par Da geht immer eine Frage nach der andern auf, das ihrer kein Ende ist. Da bekümmert sich der leidige Vorwitz um unnötige und unnütze Dinge, das weder geboten ist noch zur Sache dient. So kommt der Teufel hinter die Leute, sperrt ihnen das Maul auf, dass sie danach gaffen und jenes verlieren. Und tritt denn ein Narr auf, der auch gelesen will sein, wirkt etwas Neues und Seltsames auf,