\pagenumbering{arabic}
\setcounter{page}{85}
\markright{Wesen und Begriff der Theologie.}
unrichtig sei.\footnote{Frank, a. a. O., S. 124.} Auch neuerdings lasen wir wieder: „Unter dem Titel ‚Wort Gottes‘ die Teilung in Gesetz und Evangelium und das Verhältnis beider zu behandeln, wie die alte Dogmatik tat, dazu liegt keine Veranlassung vor.“ Frank urteilt mit Recht: „Es ist für die Lage der Dinge in der Gegenwart charakteristisch, dass auch dieses Stück der evangelischen Lehre als ungeeignet befunden und der rechten evangelischen Erkenntnis zuwiderlaufend bezeichnet wird.“\footnote{Horst Stephan, Glaubenslehre, 1921, S. 183.} Das ist allerdings für die Lage der Dinge in der Gegenwart „charakteristisch“. Aber zu gleicher Zeit ist es auch sehr selbstverständlich, weil eine notwendige Folge der allgemein gewordenen Leugnung der \textit{satisfactio Christi vicaria}. Denn die Sachlage ist doch die: Hat Gott nicht dadurch, dass er Christum an Stelle der Menschen unter die Pflicht und Strafe des den Menschen gegebenen Gesetzes stellte, die ganze Menschheit vollkommen mit sich versöhnt, so ist die notwendige Folge, dass der Mensch noch irgendwie durch eigenes Tun und durch eigene Güte Gott versöhnen oder die Versöhnung, die durch Christum geschehen ist, ergänzen muss. Und die moderne Theologie lehrt dies auch sehr reichlich. Auch der positiven Richtung zugesellte Theologen lassen die durch Christum geschehene Versöhnung durch die menschliche Heiligung ergänzt werden. Einer von ihnen meint: „Wir sind darauf hingewiesen, die Umgestaltung der Menschheit in den Begriff des Versöhnungswertes aufzunehmen.“\footnote{U. a. O., S. 104.} Damit ist dann der Unterschied von Gesetz und Evangelium aufgehoben. „Die Dinge werden“, wie Frank es ausdrückt, „in einen Brei zusammengerührt.“\footnote{Kiren, RE.3 XX, 574. Ebenso Ev. Dogmatik3, S. 118. Vgl. den Abschnitt „Nähere Beschreibung moderner Versöhnungstheorien“ II, 429 ff.}

Nach der Schrift kommt außerordentlich viel darauf an, dass wir Gesetz und Evangelium nicht „in einen Brei zusammenrühren“, sondern scharf unterscheiden. Bekanntlich behandelt die Schrift das wichtige Thema, wie der sündige Mensch die Vergebung seiner Sünden und die Seligkeit erlangt. Zugleich sagt sie sehr bestimmt, dass dies nur auf eine Weise geschehen könne, nämlich so, dass dabei das Gesetz völlig ausgeschieden wird und allein das Evangelium in Geltung tritt. Gottes Methode der Vergebung der Sünden und der Verleihung der Seligkeit lautet dahin: „ohne Gesetz“,