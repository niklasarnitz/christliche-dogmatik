christlichen Glauben durch die Verkürzung des Antichrists drohen, liegt auf der Hand und ist später noch näher darzulegen. Hier sei nur noch daran erinnert, dass auch die alten Lutherischen Lehrer die Lehre vom Antichrist nicht zu einem „\textit{Fundamentalartikel}“ gemacht haben, wie man ihnen wohl zugeschrieben hat. Vielmehr haben sie ausdrücklich erklärt, dass es vor und selbst nach der Offenbarung des Antichrists durch die Reformation viele Christen gab und gibt, die im Baptismus nicht den Antichrist erkannt haben.

Wenn wir hier den Begriff „\textit{Fundamentallehren}“ im Unterschied von nichtfundamentalen Lehren bestimmen wollen, so handelt es sich um die Frage, welche Lehren die Schrift dem christlichen Glauben zum Fundament gibt. Welche Lehren sind dies?

Bekanntlich wird sonderlich zu unserer Zeit im Interesse der Lehrfreiheit behauptet, dass der Begriff „\textit{Fundamentallehren}“ sich nicht klar bestimmen lasse, wie auch die Erfahrung genugsam beweise. So meint z. B. der Erlanger Theologe Hofmann, dass „über den Unterschied von Fundamentalem und Nichtfundamentalem bis auf diesen Tag fruchtloser Streit gewesen“ sei.\footnote{313) Der Schriftbeweis I 2, 9, 10.} Dagegen ist zu sagen, dass wir über die \textit{articuli fundamentales} nur so lange im Ungewissen bleiben können, als wir nicht den Schriftgemäßen Begriff vom Objekt des seligmachenden Glaubens festhalten. Der seligmachende Glaube, den die Schrift lehrt, ist der Glaube an die Vergebung der Sünden um Christi \textit{satisfactio vicaria} willen; noch anders ausgedrückt: der Glaube an die göttliche Rechtfertigung ohne des Gesetzes Werke, durch den Glauben. Nur wer diese Vergebung der Sünden oder Rechtfertigung durch Wirkung des Heiligen Geistes glaubt, ist an Christum gläubig im Sinne der Schrift\footnote{314) Gal. 2, 16: „Weil wir wissen (\textgreek{εἰδότες}), dass der Mensch durch des Gesetzes Werke nicht gerecht wird, sondern durch den Glauben an JEsum Christum, so glauben wir auch an Christum JEsum.“} und ein Glied der christlichen Kirche.\footnote{315) Apostf. 5, 14: \textgreek{προσετίθετο κύριος [zur Gemeinde] σῳζομένους ἐπὶ τὸ αὐτό}.} Wer diese Lehre nicht glaubt, befindet sich nach der sehr bestimmten Aussage der Schrift außerhalb der Zahl der Gläubigen oder außerhalb der christlichen Kirche.\footnote{316) Gal. 3, 6—10.} Hierher gehören Luthers Worte: \textit{Hic locus [justificationis] caput et angularis lapis est, qui solus ecclesiam gignit, nutrit, aedificat, servat, defendit; ac sine eo ecclesia Dei non potest una hora subsistere.}\footnote{317) Opp. v. a. VII, 512; St. L. XIV, 168.} Ferner: \textit{Quotquot sunt in mundo, qui eam}