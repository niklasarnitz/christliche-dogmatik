{\leavevmode\rlap{\normalsize 160}\hfil\normalsize Wesen und Begriff der Theologie.\hfil\mbox{}}

fordern, keine christliche Kirche, die die Gemeinde der Gläubigen ist,
keine Seligkeit, die durch den Glauben an Christum erlangt wird.\footnote{Die ausführliche Darlegung unter dem Abschnitt „Objektive und sub-\njektive Versöhnung“ II, 411 ff.}
\textbf{Luther:} \emph{In philosophia modicus error in principio in fine est
maximus. Sic in theologia modicus error totam doctrinam ever-
tit. ... Est enim doctrina instar mathematiei puncti; non potest
igitur dividi, hoc est, neque ademptionem neque additionem ferre
potest. ... Debet igitur doctrina esse unus quidam perpetuus et
rotundus aureus circulus, in quo nulla sit fissura. Ea accedente
vel minima, circulus non est amplius integer.}\footnote{Ad Gal. \emph{Präf.} II, 334 sq.; \emph{St. L.} IX, 614 f.} So entschieden
lehrt Luther das festgeschlossene Ganze der christlichen Lehre und in
diesem Sinne ein \emph{„System“} der christlichen Lehre. Der festge-
schlossene innere Zusammenhang der christlichen Lehre vom Zentrum
der Rechtfertigungslehre aus liegt auch darin zutage, dass ohne den
Artikel von der Rechtfertigung tatsächlich kein anderer Artikel der
christlichen Lehre geglaubt wird. Es steht nicht so, dass man den
Artikel z. B. von der Dreieinigkeit oder den von der Person Christi
glauben und den von der Rechtfertigung nicht glauben könnte. Frei-
lich, die fides humana an jene Artikel kann da sein, aber nicht die
fides divina, die der heilige Geist wirkt. Denn der heilige Geist
hält ja erst mit dem Rechtfertigungsglauben Einzug in ein Menschen-
herz.\footnote{Gal. 3, 1—3.} Erst wenn ich durch Wirkung des Heiligen Geistes glaube,
dass Gott mir um Christi satisfactio vicaria willen meine Sünden
vergeben hat, glaube ich auch durch Wirkung des Heiligen Geistes,
dass es einen Gott gibt, dass Gott dreieinig ist, dass Christus Gott
und Mensch ist, dass es eine Auferstehung der Toten und ein ewiges
Leben gibt usw. So sehr ist vom Artikel der Rechtfertigung aus die
christliche Lehre unus quidam perpetuus et rotundus aureus circulus.
Dass trotzdem bei uns der Ausdruck \emph{„System“} zur Charakterisierung
der christlichen Lehre nicht gerade beliebt ist, hat seinen Grund darin,
dass die modernen Theologen die Theologie zumeist in dem Sinne
ein System nennen, in welchem sie nicht ein System ist.

Versteht man nämlich unter \emph{„System“} ein solches zusammen-
hängendes Ganzes, das unter Ablehnung von dem tatsächlich Gege-
benen aus einem obersten Grundsatz durch Denken abgeleitet
oder entwickelt wird (spekulatives \emph{System}), so ist die christliche Lehre
kein System. Die Systembildung durch menschliches Denken ist nur