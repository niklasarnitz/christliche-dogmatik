Welt anwendlich erscheint, weil die Worte und Werke Gottes zum Heil gewisslich in sich selbst ein harmonisches Ganzes sind. Wie denn auch alle christlichen Dogmatiker aller Zeiten von dieser Methode Gebrauch gemacht haben. Nur bestehe ich eben darum mit allen diesen Dogmatikern auch darauf, dass dann auch die Gesetze dieser Methode innegehalten werden sollen, dass die Bildung der christlichen systematischen Theologie nur vis-à-vis der ihr eignen Empirie erfolgen dürfe. . . . Vergleich ich nun hiermit v. Hofmanns Systemforderungen, so lag in seinen eigenen Äußerungen zweifellos vor, dass ihm die empirische Art der systematischen Behandlung nicht wissenschaftlich genug dünkt, sondern dass er den Weg spekulativer Systembildung empfiehlt. Denn er sucht ein Einfaches als Ausgangspunkt; das soll sich selbst entfalten, aus dem soll in unverbrüchlicher Notwendigkeit hergeleitet werden. Dabei soll von Schrift und Kirchenlehre abgesehen werden; aber was herauskommt, wird sich mit dem hinterher zu Vergleichenden decken. Von alledem, von diesen Kategorien von Notwendigkeit, Selbstentfaltung, Herleitung, weiß die empirische Art systematischer Behandlung nichts; aber die spekulative weiß nicht allein davon, sondern sie hat in ihnen ihr Wesen. . . . Es ist nicht wahr, was Hofmann von sich selbst zeugt: seine Systematik ist nicht wesentlich dieselbe wie die Wissenschaftlichkeit des Augustinus; denn es handelt sich nicht um die Strenge oder die Nichtstrenge in den wissenschaftlichen Anforderungen, sondern es handelt sich um ein Herleiten und Sich-selbst-Entfaltenlassen mit Notwendigkeit und unter Absehen von der Schrift, wovon Augustinus usw. nicht ein Wort gewusst haben. Ebensowenig lässt seine Systematik sich als eine bloße Berührung der Methode früherer Dogmatiker fassen, sondern jene ist eine von dieser verschiedene Art: die früheren Dogmatiker haben die empirische Art der systematischen Behandlung, während v. Hofmann die spekulative will.\n\n5. Der Wunsch, Luther zum Protektor zu haben, hat die neueren systembildenden Theologen veranlasst, dem Reformator der Kirche nachzusagen, dass er das Ganze der christlichen Lehre aus dem Artikel von der Rechtfertigung \"genetisch entwickelt habe\". So schon Luthardt. Die \"genetische Entwicklung\" im Sinne Luthardts würde Luther allerdings unter die Systembauer einreihen. Luthardt sagt: \footnote{563} „Wenn die Dogmatik die systematische Darstellung des christlichen Glaubens sein soll, so muss sie das Ganze der christlichen Lehre