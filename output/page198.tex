187\section*{Wesen und Begriff der Theologie.}\par genossen keinen Eindruck machen, dass sie \"für die Kirche nicht ohne Gefahr\" seien. Ganz neuerdings berichtet auch Horst Stephan das modern-theologische Lager zunächst mit der Ver-sicherung: \"Heute ist die Inspirationslehre von der wissenschaftlichen Theologie aufgegeben\", setzt aber doch hinzu: \"Sie wirkt nur in der Laienorthodorie \dots noch kräftig nach.\"\footnote{Glaubenslehre, 1921, S. 52.} Es regt sich also die Furcht vor einer Reaktion aus den Laienkreisen. In dem neuesten Heft der \"Neuen Kirchlichen Zeitschrift\"\footnote{Neue Kirchl. Zeitschrift, 1923, S. 110.} rechnet ein Artikelschreiber mit der Möglichkeit, dass in der Gegenwart \"in ähnlicher Weise der Rückzug auf einen unevangelischen Autoritäts-standpunkt vollzogen wird [gemeint ist der Rückzug auf die Schrift als Gottes Wort], wie man es an der Repristinationstheologie der ersten Hälfte des 19. Jahrhunderts wahrnehmen kann\". Zu-gleich leben wir aus diesen Äußerungen der Furcht, dass der modernen Theologie die christliche Erkenntnis in dem Umfange abhanden ge-kommen ist, dass sie die Rückkehr zur Schrift \"als Gottes Wort für ein Unglück hält, das mit Macht zu bekämpfen sei.\"\footnote{Auch hierfür liefert der oben erwähnte Artikel der Neuen Kirchl. Zeit-schrift einen Beweis. Es wird als ein Trachten nach verwerflicher Repristinations-Theologie angesehen, dass --- nach einem Bericht Edwards --- Mitte der dreißiger Jahre des vorigen Jahrhunderts auf der Universität Erlangen die lutherischen Studenten der Theologie sich schon im ersten Studienjahr die symbolischen Bücher anschaffen.}\par Wir dürfen uns daher nicht wundern, dass sonderlich auch die amerikanischen Lutheraner, die am Schriftprinzip fest-halten und infolgedessen auch in der christlichen Lehre einig sind, für einen wenig wünschenswerten Teil der christlichen Kirche ge-halten und als Vertreter der \"Repristinationstheologie\" in den Hintergrund gewiesen werden. Diese Kritik gilt insonderheit der Missourisynode und ihren Schriften. D. Walther, der allerdings mit Recht die Führerrolle unter den Vätern der Missourisynode zugeschrieben wird, erscheint in Jöcklers \"Handbuch der theologischen Wissenschaften\" als Kuriosität neben Kohlbrügge, Gaußen und Kumper, weil er \"im altorthodoxen Sinne\" die Inspiration der Heiligen Schrift lehre.\footnote{Handbuch 2 III, 149.} Walther wird der Klasse der \"Repristi-nationstheologen\" des 19. Jahrhunderts zugezählt. Andere haben ihn \"Zitantentheologe\" genannt und damit --- wiewohl nicht immer --- andeuten wollen, dass er nicht als Schrifttheologe zu klassi-