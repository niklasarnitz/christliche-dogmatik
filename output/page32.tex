\par\noindent und ,Fleisch' einander entgegen. Er nennt hier, wie ich schon oben erinnert habe, ,Fleisch' nicht die Wollust, viehische Leidenschaften oder sinnliche Begierden; denn hier handelt er nicht von der Wollust oder anderen Blüten des Fleisches, sondern von der Vergebung der Sünden, von der Rechtfertigung des Gewissens, von der Erlangung der Gerechtigkeit, die vor Gott gilt, von der Befreiung vom Gesetze, von der Sünde und vom Tode. \dots{} Darum ist [hier] ,Fleisch' nichts anderes als die Gerechtigkeit, die Weisheit des Fleisches und die Gedanken der Vernunft, welche sich bemüht, durch das Gesetz gerecht zu werden.\par So steht auf Grund der Schrift fest, dass jede Religion, die irgendwie durch menschliche Leistung dem Menschen Gottes Gnade zuwenden will, nicht Weisheit von Gott, sondern Weisheit von unten her ist, kurz, ,,man-made''.\par Die Religion des Evangeliums hingegen, die den Inhalt hat, dass der Sohn Gottes Mensch geworden ist, durch sein Werk die Menschen mit Gott versöhnt hat, so dass die Menschen nun ohne eigene Werke (\textgreek{χωρὶς ἔργων νόμου}), durch den Glauben an Christi Werk, selig werden --- diese Religion hat ihren Ursprung nicht im Menschenherzen, \textgreek{ἐν καρδίαις ἀνθρώπων οὐχ ἀρεῖν}, sondern war von Ewigkeit in Gottes Herzen verborgen und ist in der Zeit nur durch Gottes Offenbarung im Wort bekannt geworden. Die christliche Gnadenreligion ist im strikten Sinne des Wortes Gottes Weisheit, \textgreek{σοφία θεοῦ}. Auch die Elite der Menschheit, die Obersten dieser Welt, sind nicht auf die Idee gekommen, dass Gott ohne des Menschen eigene Werke, um des gekreuzigten Christus willen, den Menschen die Sünde vergibt.\footnotemark[69] Kurz, die christliche Gnadenreligion ist in keinem Sinne eine menschliche Erfindung, sondern ,,God-made'' und hat ihre einzige Erkenntnisquelle (\emph{principium cognoscendi}) in Gottes Wortoffenbarung, die nun die Kirche bis an den jüngsten Tag in dem geschriebenen Wort der Apostel und Propheten besitzt.\footnotemark[70] Es steht daher so, dass alle auch innerhalb der äußeren Christenheit vorgetragenen Lehren, durch welche die Erlangung der Gnade Gottes von Menschenwerken und ,,sittlichen Leistungen'' in verschiedenem Umfange und unter verschiedenen Namen (Pelagianismus, Semipelagianismus, Synergismus) abhängig gemacht wird, menschlichen Ursprungs sind. Sie gehören nicht dem Gebiet der göttlichen, in Gottes Wort geoffenbarten Religion an, sondern dem Gebiet der von Menschen erdachten Religionen.\par\vspace{0.5\baselineskip}\small\noindent 69) 1 Kor. 2, 6 ff.\par\noindent 70) Röm. 16, 25. 26; Eph. 2, 20; 1 Joh. 1, 4.