\section*{Wesen und Begriff der Theologie.}

Die Lehrbücher, in denen die Gottesgelehrtheit aller Christen nach den Hauptstücken zusammengestellt ist (Religion objektiv genommen), werden gewöhnlich Katechismen, Religionslehre, Handbücher der christlichen Lehre usw. genannt.\footnote{161}

Die Lehrbücher, in denen die besondere Gottesgelehrtheit der Lehrer der Kirche zur Darstellung kommt, gehen in kirchlichem Sprachgebrauch unter den Namen: Lehrbücher der Theologie (Theologie objektiv genommen), Dogmatik, systematische Theologie, wissenschaftliche Theologie, die christliche Lehre in wissenschaftlicher Darstellung; in älterer Zeit: loci communes, systema theologiae Christianae usw. In englischer Sprache und auch in unserm Lande sind wohl die am meisten gebrauchten Ausdrücke: Doctrinal Theology, Systematic Theology, Dogmatic Theology oder auch kurz Christian Dogmatics. In welchem Sinne -- und in welchem Sinne nicht -- die Ausdrücke „\emph{Wissenschaft}“ und „\emph{System}“ auf die Theologie anwendbar sind, wird der Wichtigkeit wegen unter den Abschnitten „Theologie und Wissenschaft“ und „Theologie und System“ näher dargelegt werden.

\subsection*{8. Die christliche Theologie.}

Die Etymologie und somit die Wortbedeutung von „Theologie“ ist nicht, wie bei dem Wort „Religion“, zweifelhaft. \emph{Theologia} ist offenbar \foreignlanguage{greek}{λόγος περὶ τοῦ θεοῦ} und bezeichnet, subjektiv genommen, die Kenntnis von Gott oder die Gottesgelehrtheit; objektiv genommen, die Lehre von Gott.\footnote{162} Ähnliche Wortbildungen sind Psychologie, Physiologie, Biologie, Astrologie usw. \emph{Thomas von Aquino} sagt bekanntlich von der Theologie: \emph{Theologia a Deo docetur, Deum docet et ad Deum ducit}.\footnote{163} Das ist sachlich richtig. Mit Recht erinnert aber \emph{Baier}: Nomen \foreignlanguage{greek}{θεοῦ} in compositione cum nomine \foreignlanguage{greek}{λόγος} obiectum denotat.\footnote{164}

\footnote{161} Vgl. Zeitschrift, Bd.\,3, VII, 585\,ff. Luther bezeichnet den Inhalt des Katechismus in der „kurzen Vorrede“ zum Großen Katechismus als „einen Unterricht für die Kinder und Einfältigen“, als „eine Kinderlehre, so ein jeglicher Christ zur Not wissen soll, also dass, wer solches nicht weiß, nicht könnte unter die Christen gezählt und zu keinem Sakrament zugelassen werden“, über Katechismen im Allgemeinen und Luthers Katechismus im besonderen mit Angabe auch der neueren Literatur cf. B.\,Bente in Concordia Triglotta, Histor.\,Introduction, S.\,62--93.
\footnote{162} Huthardt, Kompendium, S.\,3. Walther, L.\,u.\,M.\,14, 5. Augustinus, De Civ.\,Dei VIII, 1: Verbo Graeco \foreignlanguage{greek}{θεολογίας} significari intelligimus de divinitate rationem sive sermonem.
\footnote{163} Bei Quenstedt, Systema I, 1.
\footnote{164} Comp.\,ed.\,Walther, I, 2.