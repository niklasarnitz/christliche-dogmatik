Die Theologie ist nur der, der durch den Geist vermittelte des Wortes wiedergeboren ist. In seiner Ausgabe von Baier führt er dazu (S. 69) Luthers Wort an: „Einen Doktor der heiligen Schrift wird dir niemand machen denn alleine der heilige Geist vom Himmel, wie Christus saget Joh. 6, 45.“ Aus dem Wort der Theologie, und was er treibt, ist wieder das Wort. — Die Schrift war Walther Gottes Wort und nichts anderes. An der altkirchlichen Inspirationslehre ließ er nicht rütteln. Das rühmt an ihm Rohnert, dass er in den letzten Jahrzehnten wohl am entschiedensten für die altdogmatische Verbalinspiration eingetreten ist (Dogmatik, S. 105). An der Inspiration der Schrift hielt Walther fest, weil er wohl sah, dass, wenn man auch nur im geringsten hier nachgebe, man damit aufgebe, dass die Schrift allein Quelle und Norm der Theologie sei. — „Offene Fragen“, wie die Iowa-Synode, erkennt Walther nicht an. Es bedarf nicht erst der symbolischen Bearbeitung, um eine Lehre zur Kirchenlehre zu machen. Die Bekenntnisse machen keine neuen Kirchenlehren, sondern stellen sie nur dar. Die Schrift ist das Entscheidende. Daher ist die Bibellehre auch Kirchenlehre, wenn sie auch in den Symbolen noch nicht behandelt ist (L. u. W. 14, 133 ff.). Dabei hat aber Walther die Bekenntnisse hochgeschätzt. Überall zieht er dieselben in seinen Schriften an und dazu die Aussagen der treutherischen Lehrer. Aber seine Theologie ist doch nicht im üblen Sinne eine Revisionsnationaltheologie, wie die neueren Theologen drüben sie verschreien; denn bei ihm sind es nicht die alten Dogmatiker oder die Symbole, die den Ausschlag geben, sondern die Schrift. — Als Schriftstheologe hat er keine besonderen Lehren, die er vorzugsweise getrieben hat, aber die Zeitläufe brachten es mit sich, dass er etliche Lehren besonders treiben musste und sie energisch durchgearbeitet hat: die Lehre von Kirche und Amt gegenüber der Buffalojynode, die Lehre von der Wahl und Berufung gegenüber der Ohio- und der Iowa-Synode, die Lehre von der Rechtfertigung und Versöhnung gegenüber der Erlangertheologie und dem Sektwesen des Landes.

„So könnte über Walther. Der eigentliche Kampf um die Lehre von Kirche und Amt, der fast gleichzeitig auch in Deutschland geführt wurde, war schon beendet, als Hönecke nach den Vereinigten Staaten kam. Aber auch in diesen Lehren gibt er dem Irrtum gegenüber, der sich links und rechts erhoben hat, der christlichen Wahrheit Zeugnis, wie aus seiner Dogmatik hervorgeht. Er lehrt: \footnote{631} Die

\footnotetext{631}{Die}
\footnotetext{632}{LV, 175 ff.}
\footnotetext{633}{So die Leipziger Allgemeine Ev.-Luth. Kirchenzeitung 1893, Nr. 2.}