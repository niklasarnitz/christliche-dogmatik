\pagenumbering{arabic}\setcounter{page}{31}\section*{Wesen und Begriff der Theologie.}\par chen der Alleinwirksamkeit Gottes zuscheidt.\textsuperscript{105}\footnote{105) Eph. 1, 19; Phil. 2, 29; 1 Kor. 2, 14; 1, 23. Die ausführliche Darlegung II, 546 ff., 564 ff.; III, 107 ff.} Zugleich wird durch diesen arminianischen Glauben, der ein teilweises Menschenwerk ist, die christliche Lehre von der Rechtfertigung \emph{z\={w}ris n\={o}mou, o\={u}k \={e}x \={e}rgon}, mitten ins Herz getroffen. Wie Luther gegen Erasmus, der ebenfalls die Zulassung der facultas se applicandi ad gratiam zum Zustandekommen der Bekehrung forderte, bemerkt: „Du bist mir an die Kehle gefahren.“\textsuperscript{106}\footnote{106) Sl. S. XVIII, 1967. Opp. v. a. VII, 367.}\par Dasselbe gilt natürlich auch von den synergistischen Lutheranern, die ebenfalls jene arminianische Mitwirkung zur Erlangung der Gnade Gottes unter verschiedenen Ausdrücken (rechtes Verhalten, Selbstbiegung, Selbstentscheidung, geringere Schuld im Vergleich mit andern usw.) lehren und dadurch die Entstehung des christlichen Glaubens verhindern, weil der christliche Glaube zu allen Zeiten die Art an sich hat, dass er nur in zerschlagenen Herzen entsteht\textsuperscript{107}\footnote{107) Luther, Opp. v. a. VII, 154: Quamdiu homo persuasus fuerit, sese vel tantulum posse pro salute sua, manet in fiducia sui, nec se de penitüs desperat, ideo non humiliatur coram Deo, sed locum, tempus, opus aliquod sibi praesumit vel sperat vel optat saltem, quo tandem perveniat ad salutem.} und auf die sola gratia baut.\textsuperscript{108}\footnote{108) Apologie, 97, 56: „Sooft die Schrift vom Glauben redet, meinet sie den Glauben der auf lauter Gnade baut.“} Dass es unter denen, die den Synergismus in Worten und Schriften lehren, dennoch Kinder Gottes gibt, kommt daher, dass sie inkonsequent werden, nämlich vor Gott und in ihrem Gebetskammerlein ihre eigene Lehre nicht glauben. Hrant meint mit Recht, dass Melanchthon seine synergistische Theorie nicht geglaubt habe.\textsuperscript{109}\footnote{109) Theologie der Konkordienformel I, 135. Über den Verzicht seitens theosreißiger Synergisten, mit ihrem Synergismus vor Gott hinzutreten, vgl. auch Luther, De Servo Arbitrio, Opp. v. a. VII, 166; G. G. XVIII, 1729 f. Ebenso Read, Irenic Theology, p. 163.} Welche Parteiung und Trennung aber der Synergismus in der christlichen Kirche von Melanchthon an bis auf unsere Zeit angerichtet hat, ist genugsam bekannt.\par In neuerer Zeit ist eine besonders fruchtbare Quelle der Uneinigkeit und Parteiung innerhalb der äußeren Christenheit dadurch hervorgebrochen, dass die meisten als tonangebend geltenden öffentlichen Lehrer den christlichen Begriff von der Heiligen Schrift