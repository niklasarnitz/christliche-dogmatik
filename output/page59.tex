studierenden Jugend unaufhörlich darlegen, dass nicht ein Wissen, sondern eine Selbsttäuschung und ein Nichtwissen vorliegt, wenn jemand meint, mit seiner Gotteserkenntnis und Gotteslehre über den Glauben an das geschriebene Wort Christi hinauszukommen. Dies wurde schon ausführlich unter dem Abschnitt „Die Zahl der Religionen in der Welt“ dargelegt.\footnote{S. 18 f.}Weil es sich, wie bereits bemerkt wurde, bei den Ausdrücken „Theologie“ und „Theologe“ nicht um einen in der Schrift gegebenen Sprachgebrauch, sondern nur um einen kirchlichen Ausdruck handelt, so ziehen wir es vor, mit älteren lutherischen Theologen unter Theologie die Gottesgelehrtheit zu verstehen, welche zur Verwaltung des öffentlichen Predigtamts erforderlich ist. Wir verstehen also unter Theologie, subjektiv oder konkret genommen, die vom heiligen Geist in einem Christen gewirkte Tüchtigkeit (\textit{ἱκανότης}, \textit{habitus}), die die Funktionen des öffentlichen Predigtamts zu vollziehen, also Gottes Wort aus der Schrift öffentlich und sonderlich rein zu lehren, die auferstehenden Irrlehren zu widerlegen und also sündige Menschen zum Glauben an Christum und zur Seligkeit zu führen.\footnote{Quenstedt: Theologia habitualiter et concretive considerata est habitus intellectualis theologicus practicus per Verbum et Spiritu Sancto homini de vera religione collatus, ut eius opera homo peccator per fidem in Christum ad Deum et salutem aeternam perducatur. (Systema I, 16.) Ebenso Gerhard (L. De Natura Theologiae, \S 31).} Die einzelnen Teile dieser Definition sind später noch näher zu beschreiben. Unter Theologie, objektiv oder als Lehre genommen, verstehen wir dann die mündlich oder schriftlich dargestellte christliche Lehre (\textit{doctrina}) in dem Umfang und in der Gestalt, wie sie ein Verwalter des öffentlichen Predigtamts innehaben sollte.\footnote{Quenstedt, I, 16: Theologia, systematice et abstractive spectata, est doctrina ex Verbo Dei exstructa, qua homines in fide vera et vita pia erudiantur ad vitam aeternam, vel est doctrina e revelatione divina hausta, monstrans, quomodo homines de Dei per Christum cultu ad vitam aeternam informandi.} Beide Begriffe sind der Sache nach in der Schrift enthalten. Der subjektive Begriff findet sich 2. Kor. 3, 5. 6: \textit{Ἡ ἱκανότης ἡμῶν ἐκ τοῦ θεοῦ, ὃς καὶ ἱκάνωσεν ἡμᾶς διακόνους καινῆς διαθήκης}. Der objektive Begriff liegt vor z. B. 2. Tim. 1, 13: \textit{Ὑποτύπωσιν ἔχε ὑγιαινόντων λόγων ὧν παρ’ ἐμοῦ ἤκουσας}.\footnote{Es ist ein feines Wesen geistlicher Spott, wenn im Interesse der „Lehrfreiheit“ gefragt worden ist, ob vielleicht der Apostel Paulus eine „Normal-} Auch dieser Begriff ist auf Grund der Schrift näher darzulegen.