bereits versöhnt ist. „Weil wir wissen“ (\emph{εἰδότες}) — sagt Paulus im Namen aller Christen — „dass der Mensch durch des Gesetzes Werke nicht gerecht wird, sondern durch den Glauben an Jesum Christum, so glauben wir auch an Christum Jesum, auf dass wir gerecht werden durch den Glauben an Christum und nicht durch des Gesetzes Werke; denn durch des Gesetzes Werke wird kein Fleisch gerecht.“\footnote{17}{Gal. 2, 16. Apologie (M. 188, 19): Fide consequimur remissionem peccatorum propter Christum, non propter nostra opera praecedentia aut sequentia.} „Die Christen mögen den Verlust des ewigen Heils und der Herrlichkeit des Herrn verlieren, die ihr durch das Gesetz gerecht werden wollt, und seid von der Gnade gefallen.“\footnote{18}{Gal. 5, 4: \emph{Χριστοῦ ἐξεπέσατε.} Luther übersetzt dabei auch, wenn er im Großen Katechismus (M. 458, 56) sagt: „Es haben sich alle selbst herausgeworfen und gesondert [von der christlichen Kirche]; die nicht durchs Evangelium und Vergebung der Sünde, sondern durch ihre Werke Seligkeit suchen und verdienen wollen.“}\n\n\subsection*{3. Die Zahl der Religionen in der Welt.}\n\nFragen wir nach der Zahl der wesentlich verschiedenen Religionen, so geht aus der vorstehenden Darlegung bereits hervor, dass es nicht tausend,\footnote{19}{So z. B. Weiler, Großes Konversationslexikon 6, XVI, 784.} auch nicht vier,\footnote{20}{Die heidnische, jüdische, mohammedanische und christliche Religion. Diese Auffassung findet sich auch hin und wieder in lutherischen Katechismen, indem nicht sowohl der wesentliche Inhalt der genannten Religionen als ihr historisches Auftreten in der Welt ins Auge gefasst wird. Doch ist dies auch historisch insofern nicht richtig, als die christliche Religion mit der Verzeihung von dem Weibesamen aus dem Geschlecht der Menschheit ins Dasein trat.} sondern nur zwei wesentlich verschiedene Religionen in der Welt gibt: die Religion des Gesetzes, das ist, die Bemühung um die Versöhnung mit Gott auf dem Wege der eigenen menschlichen Werke, und die Religion des Evangeliums, das ist die Versöhnung mit Gott auf dem Wege der göttlichen Gnade.