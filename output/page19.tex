\section*{Wesen und Begriff der Theologie.}\hfill 8

„Weil wir wissen“ (\textit{εἰδότες}) — sagt Paulus im Namen aller Christen —, „dass der Mensch durch des Gesetzes Werke nicht gerecht wird, sondern durch den Glauben an Christum und nicht durch des Gesetzes Werke; denn durch des Gesetzes Werke wird kein Fleisch gerecht.“\footnote{Gal. 2, 16. Apologie (M. 188, 19): Fide consequimur remissionem peccatorum propter Christum, non propter nostra opera praecedentia aut sequentia.} Hier haben Christen also schon die Erlösung, die Heiden noch nicht. Wenn nun die Heiden noch nicht wissen, dass Gott in Christo die ganze Welt mit sich versöhnt hat, so fallen sie damit auf den heidnischen Religionsbegriff zurück und suchen sich ihren Religionsbegriff außerhalb der Christenheit zu erhalten. „Ihr habt Christum verloren, die ihr durch das Gesetz gerecht werden wollt, und seid von der Gnade gefallen.“\footnote{Gal. 5, 4: \textit{Chrestou ara oude Christou} (ihr seid – in Bezug auf eure Verhältnisse zu Christo – abgetan, los von Christo, habt eine Gemeinschaft mit ihm); vgl. \textit{Energou} und \textit{nomophylax}), \textit{hoper en nomo ekathydaze} (die ihr in der Meinung steht, dass ihr durch das Gesetz gerecht werdet), \textit{to charitos epekotate}. Richtig Weder p. 31: „Die Rechtfertigung durchs Gesetz und die Rechtfertigung um Christi willen sind nämlich \textit{opposita} (Werke — Glaube), so dass die eine die andere ausschließt.“ Luther übertreibt daher auch nicht, wenn er im Großen Katechismus (M. 458, 56) sagt: „Es haben sich alle selbst herausgeworfen und gesondert [von der christlichen Kirche]; die nicht durchs Evangelium und Vergebung der Sünde, sondern durch ihre Werke Seligkeit suchen und verdienen wollen.“}

\subsection*{3. Die Zahl der Religionen in der Welt.}

Fragen wir nach der Zahl der wesentlich verschiedenen Religionen, so geht aus der vorstehenden Darlegung bereits hervor, dass es nicht tausend,\footnote{so z. B. Wener, Großes Konversationslexikon 6, XVI, 784.} auch nicht vier,\footnote{Die heidnische, jüdische, mohammedanische und christliche Religion. Diese Auffassung findet sich auch hin und wieder in lutherischen Katechismen, indem nicht sowohl der wesentliche Inhalt der genannten Religionen als ihr historisches Auftreten in der Welt ins Auge gefasst wird. Doch ist dies auch historisch insofern nicht richtig, als die christliche Religion mit der Verziehung von dem Wesentlichen aus dem Erlöser der Menschheit ins Dasein trat.} sondern nur zwei wesentlich verschiedene Religionen in der Welt gibt: die Religion des Gesetzes, das ist, die Bemühung um die Versöhnung mit Gott auf dem Wege der eigenen menschlichen Werke, und die Religion des Evangeliums,