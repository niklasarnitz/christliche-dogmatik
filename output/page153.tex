\begin{center}Wesen und Begriff der Theologie.\end{center}\null\hfill 142 gefährliches Ding ist. Gott kann die Kritik seines Wortes nicht leiden. Christus hat den Menschen Gottes Wort gegeben, damit sie es glauben. Alle, die es kritisieren, anstatt es zu glauben, geraten in Bezug auf die Wahrheitserkenntnis unter das Gericht, das Christus in den Worten Matth. 11, 25 beschreibt: „Du hast solches den Weisen und Klugen verborgen und hast es den Unmündigen offenbaret.“ Gottes Wort hat primo loco eine erleuchtende, secundo loco eine verblendende Wirkung. Wer es durch Wirkung des Heiligen Geistes im Wort nicht als Menschenwort, sondern, wie es denn wahrhaftig ist, als Gottes Wort aufnimmt, wie die Thessaloniker \textsuperscript{489}, den erleuchtet es; wer ihm die Kritik seines Ich entgegensetzt, den verblendet es. Und diese richterlich blind machende Wirkung bindet auch die natürliche Urteilskraft. Wir weisen geistlich auf die Versuche hin, die wir so ziemlich bei allen Erlebnistheologen finden, die Theologie unter die Protektion Luthers zu stellen. Schon Luther soll, wenn auch noch nicht ganz konsequent, die Theologen vom Schriftwort weggewiesen und das fromme Ich zum Richter über das Schriftwort eingesetzt haben. Äußerungen, in denen Luther sehr richtig betont, dass zum äußeren Schriftwort der Glaube hinzukommen müsse, werden so gedeutet, als ob Luther den Glauben vom äußeren Schriftwort loslösen wolle. Dieses Missverständnis der Äußerungen Luthers reicht über das gewöhnliche Maß der Urteilsschwäche hinaus, das allen Menschen seit dem Sündenfall eigen ist.\textsuperscript{490} Sodann sehen wir uns der Tatsache gegenüber, dass die Vertreter der Erlebnistheologie einander des „Subjektivismus“ beschuldigen. Diese Anschuldigungen sind sinnlos, weil die Prüfung derselben sofort zeigt, dass der Ankläger nicht minder als der Angeklagte im Hospital des Subjektivismus krank daniederliegt. A. Herrmann beschuldigt Frank des Subjektivismus, wie Ahmels berichtet und ausdrücklich darlegt.\textsuperscript{491} Herrmann aber will seinerseits die christliche Gewissheit dadurch begründen, dass das menschliche Subjekt das „innere Leben Jesu“ in sich nachahmt. Als ob das nicht Subjektivismus im eminenten Sinne des Wortes wäre! Franks Subjektivismus wird auch von Ahmels beanstandet. Ahmels weist darauf hin, dass Frank die „ethische Betrachtungsweise“ (die moralische Umwandlung des christlichen Subjekts\vspace{2em}\noindent \textsuperscript{489)} 1 Thess. 2, 13.\noindent \textsuperscript{490)} Dieser Punkt wird bei der Lehre von der Heiligen Schrift unter dem Abschnitt „Luther und die Heilige Schrift“ näher dargelegt.\noindent \textsuperscript{491)} Die Frkf. Wahrheitsgewissheit, S. 124--167.