\subsection*{13. Der Zweck der Theologie, den sie an den Menschen erreichen will.}

Der Theologe hat sich mit großem Fleiß davor zu hüten, dass ihm nicht falsche Zwecke seiner Tätigkeit untergeschoben werden. Zweck der Theologie, auf die Menschen gesehen, ist erstens nicht Kultur und bürgerliche Gerechtigkeit, wiewohl das Christentum am sichersten und schnellsten Menschen kultiviert und zu guten Staatsbürgern macht. Zweck der Theologie ist zweitens auch nicht die „Befriedigung des intellektuellen Bedürfnisses der Menschen“ und die „Bereicherung des menschlichen Wissens im allgemeinen, wiewohl die Theologie aus der Heiligen Schrift auf viele Fragen Antwort gibt, deren Beantwortung die menschliche Forschung vergeblich erstrebt.“\footnote{415} Der Zweck, den die Theologie am Menschen nach dem Sündenfall erreichen soll und will, ist die Errettung von der ewigen Verdammnis, der sämtliche Individuen des Menschengeschlechts verfallen sind, oder, was dasselbe ist, die Führung des Menschen zur ewigen Seligkeit (\textit{\textgreek{σωτηρια}}, \textit{salus aeterna}). Dieser Zweck der christlichen Theologie ist 1 Tim. 4, 16 ausgesprochen: \textit{\textgreek{τουτο ποιων}} (nämlich, wenn du des christlichen Lehramtes wartest) \textit{\textgreek{και σωσεεις και σεαυτον σωσεεις και τους ακουοντας σου}}. Nach Matth. 13, 52 ist ein jeglicher Schriftgelehrter „zum Himmelreich gelehrt“ (\textit{\textgreek{γραμματευς τη βασιλεια των ουρανων}}). Durch diesen Zweck ist das kirchliche Lehramt das wichtigste Amt auf Erden, das \textit{\textgreek{καλον εργον}} im eminenten Sinne.\footnote{416} Wenn Luthardt an den alten lutherischen Theologen die „unmittelbare Beziehung der Theologie zur Seligkeit“ zwar respektiert, aber zugleich als „wissenschaftlich nicht richtig“ tadelt,\footnote{417} so hat dieser Tadel darin seinen Grund, dass Luthardt eine Theologie vertritt, die ihren Daseinszweck aus den Augen verloren hat. Es ist dies die Theologie, die den Glauben schon in diesem Leben in ein Wissen verwandeln will und zur Erreichung dieses Zweckes die christliche Lehre nicht aus der Heiligen Schrift, sondern aus dem Ich des dogmatisierenden Subjekts beziehen will. Es ist zwar anerkennenswert, dass Luthardt diese Theologie nicht „unmittelbar“ auf die Seligkeit bezogen haben wollte. Es wäre ja erschrecklich, wenn die Seligkeit auch nur irgendwie von einer Theologie abhinge, die prinzipiell statt aus dem

\vfill

\footnotesize
\noindent

        
415415

      

So gibt z. B. die Theologie zuversichtlichen Aufschluss über die metaphysischen Probleme des Seins und Werdens (Kol. 1, 16. 17; 1 Mos. 1, 12. 13), über deren Lösung die Philosophie bekanntlich in alle Windrichtungen auseinandergeht.\

        
416416

      

1 Tim. 3, 1.\

        
417417

      

Kompendium II, S. 4.
