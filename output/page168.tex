hängt auf die Tatsache hin, dass es zwei Klassen von Hörern, resp. Lesern seines Wortes gibt: solche, die sein Wort als Gottes Wort erkennen und willig annehmen, und solche, die sein Wort nicht als Gottes Wort erkennen, sondern für eine lästige „äußere Autorität“ halten und dagegen rebellieren. Den Grund für diese ungleiche Stellung seinem Worte gegenüber gibt Christus mit den Worten an: „Wer aus Gott ist (\textgreek{ὃς ἐκ τοῦ θεοῦ}), der höret Gottes Wort“,\footnote{539} und im Bilde ausgedrückt: „Die Schafe folgen ihm [Christo] nach, denn sie kennen seine Stimme. Einem Fremden aber folgen sie nicht nach, sondern fliehen vor ihm, denn sie kennen der Fremden Stimme nicht.“\footnote{540} Von der andern Klasse sagt Christus, nämlich von den Juden, die sich gegen sein Wort als gegen eine unerträgliche „äußere Autorität“ ablehnend verhielten: „Ihr höret nicht, denn ihr seid nicht von Gott“ (\textgreek{ἐκ τοῦ θεοῦ οὐκ ἐστέ}), und: „Ihr kennet meine Sprache nicht, denn ihr könnt ja mein Wort nicht hören.“\footnote{541} Kürzer ausgedrückt, sagt Christus: Gottes Kinder erkennen in Christi Wort Gottes Wort; die noch nicht Gottes Kinder sind, sind noch nicht in dem Zustande, in Christi Wort Gottes Wort erkennen und annehmen zu können. Deshalb ruft Christus den über sein Wort murrenden Juden gleichsam beschwichtigend zu: „Murret nicht untereinander! Es kann niemand zu mir kommen, es sei denn, dass ihn ziehe der Vater, der mich gesandt hat.“\footnote{542} Auf Grund dieser Belehrung Christi halten wir nicht dafür, dass wir einem Menschen, der noch außerhalb der christlichen Kirche steht, zuerst mit Vernunftgründen die göttliche Autorität der Schrift beweisen sollten, um ihn danach zur Erkenntnis seiner Sünden und zum Glauben an Christum, den Sündenheiliger, zu führen, sondern wir predigen ihm, auch ohne die Schrift nur zu erwähnen, Buße und Vergebung der Sünden. Ist ein Mensch auf diese Weise --- und es gibt keine andere Weise --- ein Christ, ein Schäflein der Herde Christi, ein Kind Gottes geworden, dann erkennt er in dem Schriftwort Gottes Wort, gerade wie die Kinder Gottes, unter den Juden Christi mündlich verkündigtes Wort als Gottes Wort erkannten und von Herzen annahmen. Ist, mit Ludwig Hosacker zu reden, das Kamel durch das Nadelöhr gegangen, ohne Bild ausgedrückt: ist ein Mensch auf dem Wege der vom Heiligen Geist gewirkten contritio und fides ein Christ geworden, dann lässt er auch solche Stellen der Schrift unkritisiert, die er noch nicht versteht, weil ihm die Wahrheit