\null\vspace*{-1em}\noindent\begin{minipage}[t]{0.5\textwidth}Wesen und Begriff der Theologie.\end{minipage}%\begin{minipage}[t]{0.5\textwidth}\raggedleft 194\end{minipage}\par\vspace{1em}sind, so doch wir mit Lust und Freude Tag und Nacht daraus lernen, sind doch weder unsere Bibel noch unser Bekenntnis, vielmehr gewahren wir selbst in ihnen schon hie und da eine Trübung jenes Stromes, der im 16. Jahrhundert so kristallhell hervorprudelte." Auch in neueren theologischen Schriften finden wir die Notiz, dass Baiers \emph{Compendium Theologiae Positivae} von Walther in St. Louis wieder neu herausgegeben sei. Dadurch ist wohl bei manchen die Meinung entstanden, als ob die alten Dogmatiker und speziell Baier von Walther und überhaupt innerhalb der Missourisynode als die eigentlichen „\emph{Normaltheologen}“ angesehen worden seien. So lesen wir z. B. bei Ritsch-Stephan\footnote{611) Lehrbuch der ev. Dogmatik, 3. Aufl., bearbeitet von Horst Stephan 1912, S. 29.} die Notiz: „Baier verfasste das \emph{Compendium Theologiae Positivae} (1686), das die milde Orthodoxie des Musäus zusammenfasst; es verbreitete sich überaus rasch und weit und vermittelt noch heute den neuorthodoxen Lutheranern [!], besonders Amerikas, die altprotestantische Dogmatik; neu herausgegeben Breuth, Berlin 1864, und von Walther, St. Louis 1879 ff.“ Wir haben schon oben als bemerkt, dass die Walthersche Ausgabe nicht einen großen Abdruck von Baiers Compendium darbietet, sondern durch Einfügung von reichlichen und oft sehr ausführlichen Zitaten zu einem ganz neuen Buch erweitert worden ist. In den eingefügten Zitaten kommen nicht nur Luther und die Repräsentanten der altprotestantischen Dogmatik, sondern auch die Hauptvertreter der Dogmatik des 19. Jahrhunderts zu Worte. Der Zweck der Waltherschen Ausgabe ist, wie schon bemerkt wurde, der, den Studenten der Theologie ein möglichst reiches Quellenmaterial darzubieten, wodurch sie imstand gesetzt werden, sich über den Stand der Theologie in der Vergangenheit und in der Gegenwart zu orientieren. Walther schreibt darüber:\begin{quote}„Wir [amerikanischen Lutheraner] suchen uns selbst von dem, was gegenwärtig gegen die christliche Wahrheit geschrieben wird, eine genaue Kenntnis zu verschaffen, und verschweigen die Angriffe der Gegenwart mit ihrem speziellen Apparat selbst unter studierenden Jugend nicht, überzeugt, dass derjenige, welcher die Wahrheit gründlich und lebendig erlernt hat, darin das sichere Präservativ gegen Infektion auch mit dem scheinbarsten Irrtum besitzt.“\footnote{612) S. 175, Fußnote 583.}\end{quote}Ein Teil der Zitate, die Walther seiner Ausgabe von Baier eingefügt hat, sind auch als\footnote{613) E. u. W. 1875, S. 68.}