lehnen, verflucht.\footnote{401} Dennoch ist es kein kirchliches Dogma, sondern in der christlichen Kirche ausdrücklich verboten, weil Christus der Alleinreformierte und die einzige Lehrautorität in seiner Kirche ist. Auf reformiertem Kirchengebiet begegnen wir dem Dogma von der unmittelbaren göttlichen Gnadenoffenbarung und Gnadenwirkung.\footnote{402} \textit{Dux vel vehiculum Spiritui non est necessarium.} \textit{Efficacious grace acts immediately.} Auch dies Dogma strebte und strebt so energisch nach Anerkennung, dass die gegenteilige lutherische Lehre als die Majestät Gottes schädigend und ein bloßes Verstandeschristentum (Intellektualismus) fördernd verworfen wird.\footnote{403} Dennoch haben wir es an diesem Punkte nicht mit einem kirchlichen Dogma, sondern mit einer menschlichen Einbildung zu tun, für die es auch nicht einen Schein des Schriftbeweises gibt.\footnote{404} Um vielmehr direkt dem (\textgreek{διὰ λόγου}, \textgreek{τοῦ ἐναργοῦς}, \textgreek{ἐξ ἀκοῆς}, \textgreek{διὰ λόγων πάρεργος οὖς}, \textgreek{ἐξ ἀκοῆς τοῦ ἔτους ἔδαφος νίδροσιτησιτ})\footnote{405} widerspricht. Bei den arminianischen Reformierten und den synergistischen Lutheranern treffen wir auf das Dogma von der menschlichen Mitwirkung zur Bekehrung und Seligkeit.\footnote{406} Auch dieses Dogma dringt mit nicht geringer Energie auf Anerkennung. Seine Protektoren behaupten, dass ohne „Einschränkung“ der \textit{sola gratia} eine Zwangsbekehrung, eine \textit{gratia particularis} und anderes Unglück die notwendige Folge sei.\footnote{407} Dennoch ist der Synergismus nicht ein kirchliches Dogma, weil die Schrift durchweg die \textit{sola gratia} lehrt.\footnote{408} Die neueren Theologen, und zwar auch solche, die den Dogmen abhold sind, treten bei großer Nichtübereinstimmung in der Lehre mit großer Übereinstimmung für die These ein, dass die christliche Lehre nicht aus der Heiligen Schrift, sondern aus dem Zuwendigen, dem „Erlebnis“ usw. des theologisierenden Subjekts zu schöpfen und zu normieren sei. Auch dieses sonderbare Dogma tritt keineswegs bescheiden, sondern mit der Behauptung auf, dass Intel-\footnotesize \begin{enumerate} \item[\textsuperscript{401}] Das Dekret des Vatikanischen Konzils, abgedruckt bei Günther, a.\,a.\,O., S.\,379. \item[\textsuperscript{402}] Vgl. Ruf. 22, 25; Matth. 23, 8. 10. \item[\textsuperscript{403}] Vgl. den Abschnitt „Die Gnadenmittel und die Entschlafenen“, III, 150~ff.; ferner den Abschnitt „Zusammenfassende Beurteilung der reformierten Gnadenmittellehre“, III, 168~ff. \item[\textsuperscript{404}] Die Vertreter der reformierten Gnadenmittellehre auf dem Kriegspfad, III, 192~ff. 150~ff. \item[\textsuperscript{405}] Die Auflösung des reformierten Schriftbeweises III, 175~ff. \item[\textsuperscript{406}] Joh. 17, 20; 1~Kor. 4, 15; Röm. 10, 17; Tit. 3, 5; Eph. 5, 26. \item[\textsuperscript{407}] Die ausführliche Darlegung II, 564--598. \item[\textsuperscript{408}] Vgl. den Abschnitt „Die beweisende Ursache der Bekehrung“ II, S.\,546~ff. \end{enumerate}