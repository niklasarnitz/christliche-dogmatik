
\begin{center}
Wesen und Begriff der Theologie. \hfill 45
\end{center}

\setcounter{footnote}{164} % Set footnote counter to 164 so the next footnote is 165.

Was den Sprachgebrauch (\emph{usus loquendi}) betrifft, so kommt das Wort Theologie in der Heiligen Schrift nicht vor. Die alten Theologen sagen: Theologia est vox non Ἑλληνικος, sed ἀρχαικος, quamvis non ἀντιγραφος. Zur Überschrift der Offenbarung St. Johannis: Ἀποκάλυψις Ἰωάννου τοῦ Θεολόγου bemerkt Calov nach Gerhard: Theologi nomen non Iohannes sibi sumpsit, sed qui inscriptionem libri fecit.\footnote{Biblia Illustr. 3. St.} In Bezug auf den Sprachgebrauch ist weiter zu bemerken, dass das Wort Theologie und Theologe nicht nur in der christlichen Kirche, sondern auch bei den Heiden gefunden wird. Und das kann uns nicht befremden. Weil die Heiden wissen, dass es einen Gott gibt,\footnote{Röm. 1 und 2.} so haben sie sich auch um eine Kenntnis und Lehre von Gott bemüht und die Leute, welche hierin nach ihrer Meinung etwas Besonderes geleistet haben, „Theologen“ und das Resultat ihrer Bemühung „Theologie“ genannt. Beispiele hierfür sind reichlich vorhanden.\footnote{Aristoteles sagt (Metaph. I, 3) von Thales und denen, die vor Thales über den Ursprung der Dinge speculierten, dass sie Theologisten (\emph{Θεολόγους}) waren. Nach Josephus (c. Apionem I, 2) schrieb Pherecydes von Syros schon im 6. Jahrhundert ein Werk unter dem Titel \emph{Θεολογία}, worin er περὶ τῶν οὐρανίων καὶ θείων philosophierte. Cicero sagt (De Nat. Deorum III, 21): \emph{Principio Ioves tres numerant, ii, qui theologi nominantur}. Augustinus erwähnt (De Civ. Dei VI, 5) nach Varro, einem Zeitgenossen Ciceros, drei Arten von heidnischer Theologie: \emph{mythicum genus, quo maxime utuntur poetae, physicum, quo philosophi utuntur, civile, quo populi et sacerdotes, hosse et administrare debent}. Interessant ist Augustinus, der die heidnischen Theologen so darstellt und den folgenden Kapiteln. Über den Gebrauch des Wortes „Theologie“ bei den Heiden vgl. Buddeus, Inst., 1741, p. 48 sqq. Augusti Sahm, Lehrb. d. christl. Gl. I, 104 f. F. W. Walther, Lehre u. Wehre 1868, S. 5 f.} Calov sagt daher: Si primam vocis theologiae impositionem attendas, videtur ea gentilibus tribuenda, a quibus postmodum in ecclesiae usum dimanavit.\footnote{Isagoge 2 I, 8.}

Was den Sprachgebrauch innerhalb der christlichen Kirche betrifft, so ist das Wort „Theologie“ und auch das Konkretum „Theologe“ nicht immer in demselben Sinne gebraucht worden. Dabei sollte im voraus an folgendes erinnert werden: Weil es sich bei den Worten nicht um einen in der Schrift gegebenen Ausdruck, sondern um einen kirchlichen Sprachgebrauch handelt, so sollte um die Worte auch kein Streit geführt werden, solange mit den Worten Begriffe verbunden werden, die der Schrift nicht widersprechen, son-
