Die alten Dogmen sehr wohl beibehalten werden könnten, nur müssten sie „fortgebildet“ oder „liberalisiert“ werden.\textsuperscript{394}

Auf das undogmatische Christentum gehen wir hier nicht weiter ein. Es verzichtet ja von vornherein auf alle kirchlichen Dogmen. Wenden wir uns dem anderen Teil zu, der Dogmen für notwendig hält, so stehen wir vor der Tatsache, dass uns verschiedene Definitionen vom kirchlichen Dogma dargeboten werden.\textsuperscript{395} Um herauszustellen, welche Dogmen mit Recht das Prädikat „kirchlich“ verdienen, geben wir von einer ungenügenden Definition aus. Es ist der Vorschlag gemacht worden, solche Lehren kirchliche Dogmen zu nennen, die „kirchliche Anerkennung suchen oder beanspruchen“. Diese Definition wird viel gebraucht, ist aber ungenügend, weil die Erfahrung lehrt, dass gerade unkirchliche Lehren mit größter Entschiedenheit Anerkennung beanspruchen. Einige Beispiele beweisen dies. Ein Dogma Roms lautet dahin, dass die Rechtfertigung vor Gott nicht nur durch den Glauben an das Evangelium sich vollziehe, sondern auch durch das Halten der Gebote Gottes und der Kirche bedingt sei.\textsuperscript{396} Rom bringt auch so energisch auf Anerkennung dieses Dogmas, dass es über alle, die zur Erlangung der Rechtfertigung vor Gott allein auf Gottes Barmherzigkeit in Christo ohne des Gesetzes Werke vertrauen, das Anatnema ausspricht.\textsuperscript{397} Dennoch ist dies römische Dogma nicht kirchlich, sondern so unkirchlich, dass es alle, die es glauben, von der Kirche ausschließt. „\emph{Ihr habt Christum verloren (\textgreek{κατηργηθήτε ἀπὸ Χριστοῦ}), die ihr durch das Gesetz gerecht werden wollt, und seid von der Gnade gefallen.“} „\emph{Die mit des Ge-setzes Werken umgehen (\textgreek{ὅσοι ἐξ ἔργων νόμου εἰσίν}), die sind unter dem Fluch.“}\textsuperscript{398} Rom hat auch das Dogma von der Oberherrschaft und Infallibilität des Papstes.\textsuperscript{399} Auch für dieses Dogma beansprucht Rom Anerkennung in dem Maße, dass es alle, die es ab-

\begingroup\footnotesize
\itemize
    \item[\textsuperscript{394}] Winschefer Donald, The Expansion of Religion, 1896, S. 125. Die ausführliche Darlegung s. u. W. 1920, S. 270 ff.; „Die moderne Diesseitigkeitstheologie.“ Auch s. u. W. 1921, S. 2 ff.; „Das Christentum als Jenseitsreligion.“ Hier finden sich auch die Literaturangaben. Auf denselben Gegenstand gehen ein die Lehrerverhandlungen des Michigan-Distrikts, Bericht 1919, S. 44 f.
    \item[\textsuperscript{395}] Zur Literatur über diesen Punkt vgl. N. Seeberg, „Brauchen wir ein neues Dogma?“ 1892; derselbe, Grundwahrheiten der christlichen Religion\textsuperscript{5}, 1910, S. 61 ff.; Theodor Kolon, Moderne Theologie des alten Glaubens\textsuperscript{2}, 1906; Loofs unter „Dogmengeschichte“ N.E.A IV. 753 ff.; Ritschl-Stephan, „Dogmatik“\textsuperscript{3}, 2 ff. 47 ff.; Horst Stephan, „Glaubenslehre“, 1921, S. 19 ff.
    \item[\textsuperscript{396}] In der Schrift kommt das Wort sowohl von kantischen als kirchlichen Verordnungen vor, wie aus Lut. 2, 1 und Apost. 16, 4 hervorgeht.
    \item[\textsuperscript{397}] Trsid. Sess. VI, can. 10. 11. 12. 20.
    \item[\textsuperscript{398}] Gal. 5, 4; 3, 10; 4, 21—31.
    \item[\textsuperscript{399}] Vgl. Die Belege III, Note 1639. Ganz ausführlich bei Günther, Symb.3, S. 378 f.
\endgroup