Geheim vom Christo, von der Welt her verschwiegen gewesen ist, nun aber offenbart und kundgemacht durch die Propheten Schriften aus Befehl des ewigen Gottes, der Gehorsam des Glaubens aufzurichten unter allen Heiden.\footnote{592) Röm. 16, 25. 26.} Wie Johannes sowohl sein Evangelium als auch seine erste Epistel anfangs von dem ewigen, in der Zeit erschienenen Sohn Gottes und von hier aus bald beim Zentrum anlangt, nämlich bei der Vergebung der Sünden durch den Glauben an das Blut und den Tod des Sohnes Gottes. Wir könnten auch anfangen bei der Ewigkeit, die für uns Menschen auf diese Zeitlichkeit folgt, nämlich bei der ewigen Seligkeit derer, die aus großer Trübsal gekommen sind und ihre Kleider gewaschen und ihre Kleider helle gemacht haben im Blut des Lammes.\footnote{593) Offenb. 7, 14–17; Matth. 25, 34.} Auch hier schließt sich nicht schwer an, was die Schrift vom Menschen und seinem Sündenelend lehrt sowie von der Erlösung, die durch Christum geschehen ist, und von der Aneignung der geschehenen Erlösung. Wir könnten auch in der Mitte anfangen, etwa bei der Predigt des Engels auf dem Felde bei Bethlehem: \textquotedblleft Fürchtet euch nicht! Siehe, ich verkündige euch große Freude, die allem Volk widerfahren wird; denn euch ist heute der Heiland geboren, welcher ist Christus, der Herr.\footnote{594) Luc. 2, 10. 11.} Von hier aus rückwärts und vorwärts gehend, könnten wir anschließen, was die Schrift von der tötenden Sündenschuld des Menschen und von der seligmachenden Gnade Gottes lehrt. Kurz, man kann bei der Darstellung der christlichen Lehre an verschiedenen Punkten anfangen, ohne dass es ein Unheil gäbe, vorausgesetzt, dass man mit der Schrift anfängt, fortfährt und aufhört. Nur ein Anfangspunkt ist innerhalb der christlichen Kirche unheilvoll und verboten.\nAnfangspunkt der Theologie darf nicht sein \textquotedblleft das religiöse Erlebnis\textquotedblright, \textquotedblleft das fromme Selbstbewusstsein des theologisierenden Subjekts\textquotedblright, des Subjekts, das die heilige Schrift als Gottes unfehlbares Wort verwirft und daher auch, was es lehrt, nicht aus der heiligen Schrift, sondern aus dem eigenen Subjekt nehmen und normieren will, im Widerspruch mit der Meinung Christi, bei seinem Wort zu bleiben, und im Widerspruch mit dem Urteil seines Apostels, der jedem Lehrer, der nicht bei den gefunden Worten Christi bleibt, die \textit{licentia docendi} entzieht, weil solcher an Einbildung und Unwissenheit leidet. Diesen in der Schrift verbotenen Anfangspunkt will auch Ihmels als den einzig mög-