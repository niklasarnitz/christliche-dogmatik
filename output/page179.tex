\noindent 168 \hfill Wesen und Begriff der Theologie.\par aus einer fundamentalen Einheit genetisch entwickeln, also nicht etwa bloß aus einem obersten Grundsatz ableiten, sondern den Tatbestand des Christentums selbst, wie er prinzipiell zusammengefasst ist, auseinanderlegen. Als solches genetische Prinzip bezeichnet Luther den Artikel von der Glaubensgerechtigkeit: \"In ihm hält uns David vor die Summa der ganzen christlichen Lehre und die helle liebe Sonne, welche die christliche Gemeinde erleuchtet. Wenn dieser Artikel mit gewissem und festem Glauben gefasst und behalten wird, so kommen und folgen die andern allmählich nach, als von der Dreieinigkeit ufm.\" W. B. Erl. Ausg., Exeg. opp. Lat. XX, p. 193: Stante enim hac doctrina stat ecclesia etc. Comm. in ep. ad Gal. II, 23, ed. Irmischer. Locus igitur iustificationis, ut saepe moneo, diligenter discendus est. In eo enim comprehenduntur omnes alii fidei nostrae articuli, eoque salvo salvi sunt et reliqui. Aber die folgende Dogmatik führte diesen Gedanken nicht durch und bezeichnete nur die Schrift (im Gegensatz zum römischen Fundamentalartikel von der Infallibilität des Papstes; vgl. Gerhard, Conf. Cathol. I, 2, 1, p. 71) als unicum principium cognoscendi, aus welchem die Glaubenslehren nicht nur bewiesen, sondern entwickelt wurden.\" Der hier von Luthardt behauptete Gegensatz zwischen Luther und \"der folgenden Dogmatik\" ist eine Fiktion. Was Luther von der zentralen Stellung der Rechtfertigungslehre sagt, dass nämlich dieser Artikel die Summa der ganzen christlichen Lehre enthält, über alle andern Lehren das rechte Licht wirft und dem Eindringen von Irrtümern in allen andern Lehren wehrt ufm., das findet sich auch bei den lutherischen Dogmatikern, indem sie z. B. alle andern Lehren in ihrem Verhältnis zu dem Artikel von der Rechtfertigung als articuli antecedentes und consequentes beschreiben.\textsuperscript{564)} Und was Luthardt den Dogmatikern im Unterschied von Luther zuschreibt, dass ihnen die Schrift unicum principium cognoscendi war, woraus die Glaubenslehren nicht nur zu normieren, sondern auch direkt zu entnehmen seien, das findet sich auch durchweg und noch gewaltiger bei Luther. Luther kennt schlechterdings keine Konstruktionsmethode in der Theologie, auch keine Konstruktion der Glaubenslehren aus dem Zentralartikel von der Rechtfertigung unter Absehung von dem Schriftwort. Das sagt Luther nicht nur an den zahlreichen Stellen, in denen er jeden \"schriftlosen\" Gedanken verwirft und jedem\par\vspace{\baselineskip}\noindent \textsuperscript{564)} Düenstedt, Systema I, 352 sqq., die Thesen 6. 7. 8.