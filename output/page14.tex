„freiere“ Stellung zur Schrift eingenommen habe als die späteren lutherischen Theologen. Über diese Behauptung, wo sie bona fide aufgestellt wird, beruht auf Unkenntnis der geschichtlichen Tatsachen, wie bei der Lehre von der heiligen Schrift darzulegen ist.\n\nBekanntlich behaupten die neueren Theologen, die an der Stelle der Schrift als Quelle und Norm der Theologie ihr eigenes frommes Bewusstsein sehen, dass gerade ihr frommes Selbstbewusstsein und ihr durch die neuere Wissenschaft scharf entwickelter „Wirklichkeitssinn“ sie davon abhalte, Schrift und Gottes Wort zu identifizieren. Es wird uns erlaubt sein, „Erlebnis“ gegen „Erlebnis“ und „Wirklichkeitssinn“ gegen „Wirklichkeitssinn“ zu setzen. Wir unsererseits erleben es mit Millionen Christen und dürfen es durch Gottes Gnade noch immerfort erleben, dass die heilige Schrift wirklich Gottes Wort ist. Und dies Erlebnis vermittelt uns für das geschriebene Wort der Apostel geradeso, wie es sich im Herzen der korinthischen Christen in Bezug auf die mündliche Verkündigung des Apostels Paulus vermittelte.\footnotemark{6}) Weil die heilige Schrift Gottes Wort ist, so wartet sie nicht darauf, dass sie vom Papst oder von irgendeinem andern theologisierenden Individuum anerkannt und bestätigt werde, sondern sie verschafft sich selbst Anerkennung durch Hervorbringung des Glaubens infolge der Wirksamkeit des heiligen Geistes, die mit dem Wort verbunden ist, gerade wie die Werke Gottes im Reiche der Natur sich als göttlich selbst bezeugen, ohne auf eine Bestätigung seitens der Vertreter der Naturwissenschaften warten zu müssen. Hingegen bewegen sich die schriftpflichtigen neueren Theologen auf dem Gebiet der Selbsttäuschung und gehen an dem der Erkenntnis der Wahrheit vorbei, weil sie den Glauben von seinem Entstehungs- und Erkenntnisgrund abrücken und ihn unmittelbar aus dem eigenen Innern emporsteigen lassen wollen. Dass dabei eine Selbsttäuschung vorliegt, ist deshalb gewiss, weil Christus sehr klar und bestimmt die Erkenntnis der Wahrheit an das Bleiben an seinem Wort bindet: „So ihr bleiben werdet an meiner Rede, \textit{ἐν τῷ λόγῳ τῷ ἐμῷ} \dots \textit{γνώσεσθε τὴν ἀλήθειαν}, Joh. 8, 31. 32. Wollen wir also als theologische Lehrer nicht Irrtum, sondern Wahrheit erkennen und lehren, so müssen wir an Christi Wort bleiben, das wir bis an den jüngsten Tag in dem Wort seiner Apostel haben, wie uns Christus gleichfalls sehr klar unterrichtet, wenn er Joh. 17, 20 sagt, dass alle durch das Apostel Wort (\textit{διὰ λόγον αὐτῶν}) an ihn\footnotetext[6]{1 Kor. 2, 1–5.}