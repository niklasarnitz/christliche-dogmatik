\section*{Wesen und Begriff der Theologie.}\hfill 64 Er sagt: „Ich weiß nicht, ob jemand auf den Schriftgrund dieser Einteilung [in $\delta ó \xi o \upsilon o \varsigma$ und $ἔ \xi o \upsilon o \varsigma$], dieses Begriffes der Theologie, aufmerksam gemacht hat. Allein, wo könnte er wohl zu suchen sein als in den Worten des Herrn Matth. 11, 27: Niemand kennet den Sohn denn nur der Vater, und niemand kennet den Vater denn nur der Sohn, und wem es der Sohn will offenbaren.“\footnote{225} Unsere alten Theologen haben den Schriftgrund reichlich ausgezeigt. Schertzer: Theologia $\delta ó \xi o \upsilon o \varsigma$ ist ipsius Dei de se ipso cognitio, Matth. 11, 27; 1 Cor. 2, 10sq.\footnote{226} Die alten Theologen führen die folgenden Gedanken aus: 1. Nur Gott erkennt Gott; für uns Menschen wohnt Gott in einem uns unzugänglichen Licht, 1 Kor. 2, 10. 11; Joh. 1, 18a; Matth. 11, 27 --- 1 Tim. 6, 16. 2. Gott ist aus dem uns Menschen unzugänglichen Licht durch Selbstoffenbarung herausgetreten. Die göttliche Selbstoffenbarung liegt in zweifacher Weise vor: in Gottes Naturreich und in seinem Wort. Die im Reich der Natur vorliegende Selbstoffenbarung (Röm. 1, 19ff. 32; 2, 14. 15; Apost. 14, 17; 17, 26. 27) ist die Erkenntnisquelle der natürlichen Theologie; die in Gottes Wort vorliegende Selbstoffenbarung (Joh. 1, 18b; 8, 31. 32; Eph. 2, 20) ist die einzige Erkenntnisquelle der christlichen Theologie. Daher ist in der Christenheit nur die Theologie existenzberechtigt, die ἔκδοχος, das ist, lediglich Wiedergabe der in der heiligen Schrift für uns vorliegenden göttlichen Lehre ist. Man kann hierüber Gerhard nachlesen.\footnote{227} Theologia ἔκδοχος ist sachlich nichts anderes, als wenn Luther von den christlichen Lehrern nach der Apostel\footnote{225) Zeitschpr. f. luth. Theol. u. k. 1848, 1, 7. Zitiert in Vater-Walther 1, 5.}\footnote{226) Systema, Proleg. de Theologia, p. 2. Bei Cuenstedt, Systema 1, 5 sq. unter Thesis IV.}\footnote{227) Gerhard sagt, L. de Natura Theologiae, § 15 sq., über die Einteilung der Theologie in theologiam $\delta ó \xi o \upsilon o \varsigma$ vel ἔκδοχος: “Ἀρχέτυπος seu πρωτότυπος est in Deo Creatore, quae Deum seipsum novit in seipso et extra se universa per seipsum acti, scientia indicabilis et immutabilis ...” ἔκδοχος theologia est a priori quasi expressa et efformata per gratiosum communicationem. Das Mittel der Mitteilung der Erkenntnis, die ursprünglich nur in Gott ist, ist das äußere Wort, quo [Deus] in tempore homines alloquitur. Daraus ergibt sich für die christliche Theologie: principium theologiae supernaturalis adaequatum et proprium esse divinam revelationem, quae cum hodie nonnisi in sacris literis, hoc est, in propheticis Veteris et apostolicis Novi Testamenti libris descripta exstat, inde scriptum Dei Verbum sive, quod idem ist, Scripturam Sacram dicimus esse unicum et proprium theologiae principium. Ebenso und noch ausführlicher Cuenstedt, Systema 1, 5 sqq.}