Stelle, wo er etwas unhöflich wird. Er schreibt: \footnote{460) U. a. O., S. 115.} „Wer [Philippi ist gemeint] mir die „objektive“ Versöhnungsthat [Christi] und das Wort Gottes entgegenhält statt meines „subjektiven“ Standpunktes, mit dem vermag ich mich nicht auseinanderzusetzen, weil er die Fragestellung nicht verstanden hat.“ Damit ist allerdings, wie Bachmann sagt, „bewusst und grundsätzlich die volle Selbstgewissheit des Christentums und seiner Theologie“ geleugnet. Nur möchten wir Bachmanns „bewusst und grundsätzlich“ in etwas limitieren. Frank vertrat ohne Zweifel „die volle Selbstgewissheit des Christentums und seiner Theologie“, wenn er auf dem Katheder saß oder in seiner Studierstube Bücher schrieb. Wir haben freilich Frank nicht persönlich gekannt. Aber auf Grund von Mitteilungen anderer und auf Grund dessen, was Frank sonst geschrieben hat, glauben wir doch annehmen zu dürfen, dass sich sein Verkehr mit Gott nicht auf der Basis seiner „Selbstgewissheit“ vollzogen hat, sondern sich genau auf der Basis vermittelte, an die Philippi ihn erinnerte, nämlich auf der Basis der objektiven Versöhnungsthat Christi und des objektiven Wortes Gottes. Wie Frank in seiner „Theologie der Konkordienformel“ \footnote{461) Bd. I. S. 135.} annimmt, das Melanchthon seinen Synergismus nie selbst geglaubt hat, so nehmen wir an, dass auch Frank „die Selbstgewissheit des Christentums und seiner Theologie“ selbst nie geglaubt hat. Wir bemerken dies hier auch in dem Zweck, um den Gedanken abzuwehren, als ob wir allen Vertretern der Selbstgewissheitstheorie das persönliche Christentum absprechen. So gewiss es auf Grund der Heiligen Schrift ist, dass diese Theorie bei konsequenter Durchführung in der Praxis das persönliche Christentum völlig unmöglich macht, so lehrt uns andererseits die Erfahrung, dass es auch auf dem Gebiet des theologischen Betriebs eine „glückliche Inkonsequenz“ oder eine „doppelte Buchführung“ gibt. Hier haben wir bereits früher erinnert, und daran werden wir auch später noch wiederholt erinnern. Vor einigen Jahren berichteten deutschländische Blätter, dass ein Theologe, der theoretisch ebenfalls die Theologie der Selbstgewissheit betrieb, auf dem Kranken- und Sterbebette die Äußerung tat, er finde nun seine ganze Theologie in Joh. 3, 16 zusammengefasst, womit er tatsächlich aus seinem Ich heraus und „über sich“ fuhr, wie Luther es ausdrückt, und sich damit auf ein außer ihm gelegenes Fundament stellte. Doch dies nur nebenbei.