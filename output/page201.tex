190

\subsection*{Wesen und Begriff der Theologie.}

derselbe schreibt: \textquotedblleft Wie es ein Wahnsinn wäre, zu sagen, man könne aus den Regeln des Schuhmacherhandwerks über die christliche Lehre urteilen, so irren die, welche der Philosophie ein Urteil über dieselbe zuschreiben.\textquotedblright (Scholia in epist. ad Col., S. 68.) Mag die Wissenschaft noch so zuversichtlich die Resultate ihrer Forschungen für absolut gewisse Wahrheiten ausgeben, so halten wir doch nicht sie, wohl aber die Schrift für infallibel. Widersprechen die Ergebnisse wissenschaftlicher Forschung der klaren Schrift, so ist uns von vornherein gewiss, dass sie nichts sind als gewisser Irrtum, selbst wenn wir nicht imstande sind, ihn als solchen anders als mit Berufung auf die Schrift nachzuweisen. Sooft wir zwischen Wissenschaft und Schrift zu wählen haben, sprechen wir dabei mit Christo, unserm Herrn: \textquotedblleft Die Schrift kann doch nicht gebrochen werden\textquotedblright, Joh. 10, 35, und mit dem heiligen Apostel: \textquotedblleft Wir nehmen gefangen alle Vernunft unter den Gehorsam Christi\textquotedblright, 2 Kor. 10, 5. Wir warten nicht darauf, dass die Wissenschaft uns unsern Grund erst erobere. Wir haben ihn schon, und er steht uns vor aller wissenschaftlichen Untersuchung oder Prüfung so fest als unser Gott, der ihn gelegt hat. Was auch immer die Wissenschaft zutage fördern mag, das gibt uns weder den Glauben noch nimmt sie ihn uns. Wir stehen auf einem Felsen, den wir mit wissen, dass denselben auch die Pforten der Hölle nicht, geschweige menschliche Wissenschaft, überwältigen kann, und lachen dabei aller Feinde und ihrer wissenschaftlichen Sturmböcke und Mauerbrecher, mit denen sie den aus den tobenden Gewässern der Welt emporragenden himmelhohen Felsen mit wahnsinniger Wut berennen. Denn also spricht der Herr: \textquotedblleft Wer auf diesen Stein fällt, der wird zerschellen; auf welchen er aber fällt, den wird er zermalmen\textquotedblright, Matth. 21, 44.

Walther deckt auch den Missbrauch auf, den die moderne Theologie mit dem sogenannten wiedergebornen Ich oder der erleuchteten Vernunft treibt. Er sagt: \textquotedblleft Durch die Erleuchtung erhält ja die Vernunft nicht ein eigenes Licht neben der Schrift, vielmehr besteht ihre Erleuchtung eben darin, dass durch Wirkung des Heiligen Geistes das Wort der Propheten und Apostel ihr einziges Licht in Sachen des Glaubens geworden ist\textquotedblright. Sofern sie aus ihren Prinzipien wider die Artikel des Glaubens disputieren will, insofern ist sie nicht wiedergeboren, weil die wiedergeborne Vernunft aus den Prinzipien des Wortes Gottes disputiert. Sehr eindringlich warnt Walther auch die christliche Kirche vor der Meinung, dass ihre Lehre dem Inhalte nach durch die Mittel der neueren Wissenschaft fort.