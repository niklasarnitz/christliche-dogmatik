122
\noindent Wesen und Begriff der Theologie.

sehen werde. So sagt Calvin: „Zu beachten ist, dass es sich nicht darum handelt, ob die Theologie in einem allgemeineren Sinne des Wortes eine Wissenschaft genannt werden könne, auch nicht darum, ob die Theologie wegen ihrer Vollkommenheit nicht vielmehr eine Wissenschaft als eine Meinung oder eine unvollkommene Fertigkeit (habitus) genannt werden sollte. . . . Beides geben wir leicht zu, weil sie auch vom heiligen Geist in einem weiteren Sinne Wissenschaft genannt wird. 1 Kor. 12, 8, und richtig von Thomas [von Aquino] wie auch von Augustinus hie und da gelehrt wird, dass sie nicht bloß eine Meinung, sondern auch eine Wissenschaft sei.“\footnote{439}

Zu unserer Zeit nennen wir die Theologie nicht gern eine Wissenschaft, weil das Wort durch neuere Theologen in Misskredit gebracht worden ist. Die Diskreditierung ist durch jene Theologen geschehen, die der Theologie die Aufgabe zuweisen, den Glauben zum Wissen zu erheben oder die christliche Lehre als wahr vor der menschlichen Vernunft zu erweisen. Dies ist unmöglich, weil der natürliche Mensch die Wahrheit des Evangeliums nicht erkennen kann, \textgreek{ψυχικὸς ἄνθρωπος οὐ δέχεται τὰ τοῦ πνεύματος τοῦ θεοῦ . . . οὐ δύναται γνῶναι}\footnote{440}. Deshalb lautet auch Christi Befehl nicht dahin, der Welt das Evangelium zu beweisen, sondern es der Welt zu verkündigen, \textgreek{κηρύξατε τὸ εὐαγγέλιον πάσῃ τῇ κτίσει}. Der Apostel Paulus besaß eine wissenschaftliche Bildung. Aber gerade dieser Apostel betont sehr stark, dass er sich der wissenschaftlichen Demonstration auch vor einem wissenschaftlich gebildeten Publikum, z. B. vor den Korinthern, enthalten habe, um ihrem Glauben nicht falsche Stützen unterzuschieben.\footnote{441} Alte Theologen drücken dies trefflich und kurz so aus: Theologia non est habitus demonstrativus, sed exhortativus. Sie wollen damit sagen: Die theologische Tüchtigkeit besteht darin, die christliche Lehre der Welt vorzulegen oder kundzutun, nicht darin, sie der Welt mit Vernunftgründen als wahr zu erweisen. Für den Wahrheitsbeweis horagt der heilige Geist, der mit dem verkündigten Wort verbunden ist, indem er durch die Verkündigung des Gesetzes Gottes die sicheren Herzen zerschlägt und durch die Verkündigung des Evan-

\footnotesize
439) Systema I, 42: Observandum autem, non disquiri, utrum theologia scientia dici possit laxiori vocis significatione, nec etiam, an theologia potius scientia sit sui perfectionem quam opinio aut imperfectus habitus dici debeat. . . . Utrumque illud facile admittimus, cum et a Spiritu Sancto vocetur scientia laxa significatione 1 Cor. 12, 8 etc. et recte a Thoma ut et ab Augustino passim doceatur, eam non solum opinionem, sed et scientiam esse.
440) 1 Kor. 2, 14; 1, 23.
441) 1 Kor. 2, 1—5.
\normalsize