\pagenumbering{arabic}\setcounter{page}{63}\section*{Wesen und Begriff der Theologie.}stehen''; denn ,,das kann man nicht halten''.\footnote{St. S. XVI, 2212.} Wohl aber will Luther mit dem ,,Nachsagen'' nachdrücklich einschärfen, dass der christliche Lehrer ,,soll außer der Schrift nichts lehren in göttlichen Sachen''. Seine Lehre soll inhaltlich eine bloße Wiedergabe der Lehre der Propheten und Apostel, also \textit{doctrina divina}, sein, ohne Beimischung von eigenen menschlichen Anschauungen. Denn das ist, setzt Luther hinzu, die Eigenart aller treuen Lehrer, die nach der Apostel Zeit das Lehramt in der Kirche verwalten, dass sie ,,nichts Eigenes noch Neues sehen, wie die Propheten tun, sondern lehren, das sie von den Propheten haben''. Die Lehrer an den theologischen Anstalten der Missourisynode übertreiben nicht, wenn sie die Studenten erinnern, ihre ausgearbeiteten Predigten noch einmal daraufhin genau zu prüfen, ob nicht etwa ein ,,schriftloser'' Gedanke [Luthers Ausdruck] sich eingeschlichen habe, und ihn schonungslos zu streichen, weil er als \textit{Eigeprodukt} keine Berechtigung in Gottes Kirche habe, die auf den Grund der Apostel und Propheten erbaut ist.Was die alten lutherischen Theologen betrifft, so wurde schon zu Anfang dieses Abschnitts darauf hingewiesen, dass ihnen die christliche Theologie, als Lehre gefasst, nichts anderes sei als eine geordnete Zusammenstellung der in der heiligen Schrift vorliegenden göttlichen Lehre. Daher ihre Definitionen des Lehr-Objekts der Theologie: \textit{doctrina e revelatione divina hausta}, \textit{doctrina ex Verbo Dei exstructa} usw. Daher auch ihre Mahnung, dass in den kirchlichen Lehrleib (\textit{doctrinae corpus}) nur das gehöre, was sich als in der Schrift vorliegend nachweisen lasse. Um diese Beschaffenheit der christlichen Lehre nachdrücklich hervorzuheben, nennen die alten Theologen die Theologie, als Lehre gefasst, auch \textit{theologia \textgreek{Exilistos}}, abbildliche Theologie, das ist, eine Wiedergabe oder einen Abdruck der \textit{theologia \textgreek{archetypos}}, der urbildlichen Theologie, das ist, der Erkenntnis von Gott und göttlichen Dingen, die sich ursprünglich nur in Gott findet, die Gott aber aus Gnaden den Menschen durch sein Wort vermittelt oder mitgeteilt hat. Diese Terminologie ist nicht überflüssig und veraltet, wie man gemeint hat,\footnote{Bretschneider, Systematische Entwicklung aller in der Dogmatik vorkommenden Begriffe, S. 68.} sondern durchaus schriftgemäß und für die Theologen aller Zeiten sehr instruktiv. Von neueren Theologen hat dies Knudelbach anerkannt. Er sagt: ,,Ich weiß nicht, ob jemand