34\hfill Wesen und Begriff der Theologie.\par der lutherischen Lehre ist natürlich auf dem Wege der Induktion zu führen. Wie Luther in seinem Glaubensbekenntnis vom Jahre 1529 sagt:\footnote{St. L. XX, 1004 ff. Erl. 30, 303 ff.} „Ob jemand nach meinem Tode würde sagen: Wo der Luther jetzt lebte, würde er diesen oder diesen Artikel anders lehren und halten, denn er hat ihn nicht genugsam bedacht usw.: davider sage ich jetzt als dann und dann als jetzt, dass ich von Gottes Gnaden alle diese Artikel aufs fleißigste bedacht, durch die Schrift und wieder durchdrungen oftmals gezogen und so gewiss dieselbigen wollte verfechten, als ich jetzt habe das Sakrament des Altars verfochten.“ -- Hier sei nur noch darauf hingewiesen, dass die lutherische Kirche das Examen in Bezug auf die Schriftmäßigkeit ihrer Lehre auch an dem Punkte besteht, an welchem die große Mehrzahl der Theologen seit Augustin bis auf die neueste Zeit aus rationalistischen Erwägungen das Schriftprinzip verlassen hat. Wir meinen den Punkt, den man die \textit{crux theologorum} genannt hat und den wir als die schwerste Belastungsprobe für das Festhalten am Schriftprinzip bezeichnen möchten. Man meint nämlich in Bezug auf die Gnade Gottes, dass die \textit{universalis gratia} und die \textit{sola gratia} nicht beide zumal feilgehalten werden könnten. Die Calvinisten behaupten, wie wir sahen, dass zur Rettung der \textit{sola gratia} die \textit{universalis gratia} preisgegeben sei; die Synergisten fordern, dass zur Rettung der \textit{universalis gratia} die \textit{sola gratia} geopfert werde. Beides zugleich festzuhalten, sei unmöglich. Die lutherische Kirche ist sich der Schwierigkeit, die hier für das menschliche Begreifen vorliegt, klar bewusst. Dennoch hält sie beides, sowohl die \textit{universalis gratia} als auch die \textit{sola gratia}, ohne Einschränkung fest, weil beide Lehren klar in der Schrift bezeugt sind. Sie erwartet die Lösung der Schwierigkeit, die hier für das menschliche Begreifen vorliegt, im ewigen Leben.\footnote{P. C. 709, 28, 29; 557, 17--19 (Bekenntnis zur \textit{gratia universalis}). -- F. C. 923, 9--11; 716, 57--64 (Bekenntnis zur \textit{sola gratia} und Verzicht auf die Lösung der hier für das menschliche Begreifen vorliegenden Schwierigkeit in diesem Leben). Ebenso Luther, \textit{De Servo Arbitrio}, Opp. v. a. VII, 365, \S. 2. XVIII, 1965 f.}\par Zur Besprechung der Parteien innerhalb der äußeren Christenheit gehört auch ein Hinweis auf die Motive für die Abweichung von der Schriftlehre und die meistens sich anschließende Parteibildung. Die heilige Schrift kennt für diese abnorme Erscheinung innerhalb der christlichen Kirche keine edlen, sondern nur fleischliche Motive.