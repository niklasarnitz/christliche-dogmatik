Wesen und Begriff der Theologie. 9\nDas ist, dass durch das Evangelium vom heiligen Geist gewirkte Glaube, dass wir durch die Versöhnung, die durch Christum geschehen ist, ohne eigene Werke ($\chi\omega\rho\grave{\iota}\varsigma \, ἔ\rho\gamma\omega\nu \, \nu\acute{o}\mu\omicron\upsilon$), einen gnädigen Gott haben.\n\nDie Zweizahl der Religionen, auf ihre wesentliche Beschaffenheit gesehen, ist durch die ganze Schrift klar gelehrt, wie im folgenden näher darzulegen ist. Die Zweizahl geht auch schon daraus hervor, dass die christliche Religion die Aufgabe hat, alle anderen Religionen zu verdrängen. Der an die christliche Kirche gerichtete Missionsbefehl ist durchaus universaler Natur: $\mu\alpha\theta\eta\tau\epsilon\acute{u}\sigma\alpha\tau\epsilon \, \pi\acute{\alpha}\nu\tau\alpha \, \tau\acute{\alpha} \, ἔ\theta\nu\eta$. \textsuperscript{21)}\footnote{21) Matth. 28, 19. So auch reichlich schon in den Weissagungen des Alten Testaments. \emph{Ps.} 2, 8: „ich will dir die Heiden zum Erbe geben und der Welt Ende zum Eigentum.“ \emph{Gen.} 49, 10; \emph{Ps.} 72, 8 usw. \emph{Jes.} 49, 6 (\emph{Christus} das „Licht der Heiden“ und „Gottes Heil“ bis an der Welt Ende).}\nund spricht daher nicht bloss einigen, sondern allen anderen Religionen die Existenzberechtigung ab mit der hinzugefügten Begründung, dass alle Religionen mit Ausnahme der christlichen praktisch wertlos sind, dass sie nämlich die Menschen in der Finsternis und in Satans Gewalt belassen. Es heisst in der Zweckbestimmung des Christentums als Weltreligion: „aufzutun ihre Augen, dass sie [Juden und Heiden] sich bekehren von der Finsternis zum Licht und von der Gewalt des Satans zu Gott, zu empfangen Vergebung der Sünden und das Erbe mit denen, die geheiligt werden, durch den Glauben an mich“, nämlich Christum ($\pi\acute{\iota}\sigma\tau\epsilon\iota \, \tau\tilde{\eta} \, \epsilon\acute{\iota}\varsigma \, ἐ\mu\acute{\epsilon}$, scil. $\epsilon\acute{\iota}\varsigma \, \chi\rho\iota\sigma\tau\acute{o}\nu$). \textsuperscript{22)}\footnote{22) \emph{Apostelgesch.} 26, 18. Des Luthers gemässigte Darlegung zu \emph{Jes.} 9, 2 f., dass alle intellektuellen und moralischen Bestrebungen der Heiden und der ungläubigen Juden die Menschen in trostloser Finsternis belassen, \emph{Kl.} 9, VI, 106 f. \emph{Herzog Kehler Schaffer} zu \emph{Apostelgesch.} 26, 18 in langes Kommentar: „The purpose of his [des \emph{Apostels Paulus}] mission is stated in such a manner that it can be understood only as referring to Gentiles.“ Dagegen richtig \emph{Meyer}: Els $\omicron\dot{u}\kappa$ ist nicht bloss auf $\tau\tilde{\omega}\nu \, ἔ\theta\nu\omega\nu$ zu beziehen, sondern auf $\tau\omicron\tilde{\iota}\varsigma \, \lambda\alpha\omicron\tilde{\iota}\varsigma$ mal $\tau\tilde{\omega}\nu \, ἔ\theta\nu\omega\nu$ zusammen, „was durch das pragmatische Verhältnis A. 19, 20 gefordert wird“. A. 20 berichtet \emph{Paulus} selbst, wie er den Befehl \emph{Christi} verstanden habe, nämlich so, dass er allseits Juden und Heiden Busse und Bekehrung zu Gott predigte, Barthäus bestreitet auch, dass nicht nur die Bekehrung von der Finsternis zum Licht, sondern auch die Bekehrung von der Gewalt des Satans zu Gott gleicherweise auf die Heiden und die ungläubigen Juden. Auch hier hat \emph{Meyer} zunächst die richtige Beziehung auf beide Klassen. Wenn er aber in Bezug auf die Bedeutung „von der Gewalt des Satans zu Gott“ limitierend hinzufügt, sie „berücksichtigt mit vorherrschender Beziehung die Heiden, welche ἀφ’ $o\tilde{u} \, \tau\tilde{\omega} \, \kappa\acute{o}\sigma\mu\omega$, \emph{Eph.} 2, 12, sind, unter der Gewalt des Satans, des ἄρχοντος}