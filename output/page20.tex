Wesen und Begriff der Theologie. \hfill 9

dass, der durch das Evangelium vom heiligen Geist gewirkte Glaube, dass wir durch die Versöhnung, die durch Christum geschehen ist, ohne eigene Werke (\textlatin{\$\chi\omega\rho\grave{i}\varsigma\ \stackrel{\acute{e}}{\rho}\gamma\omega\nu\ \nu\acute{o}\mu o\nu$}), einen gnädigen Gott haben.

Die Zweizahl der Religionen, auf ihre wesentliche Beschaffenheit gesehen, ist durch die ganze Schrift klar gelehrt, wie im folgenden näher darzulegen ist. Die Zweizahl geht auch schon daraus hervor, dass die christliche Religion die Aufgabe hat, alle andern Religionen zu verdrängen. Der an die christliche Kirche gerichtete Missionsbefehl ist durchaus universaler Natur: \textlatin{\$\Mu\alpha\theta\eta\tau\epsilon\acute{\upsilon}\sigma\alpha\tau\epsilon\ \pi\acute{\alpha}\nu\tau\alpha\ \tau\grave{\alpha}\ \stackrel{\prime}{\epsilon}\theta\nu\eta$}. \footnote{Matth. 28, 19. So auch reichlich schon in den Weissagungen des Alten Testaments. Ps. 2, 8: „Ich will dir die Heiden zum Erbe geben und der Welt Ende zum Eigentum.“ Gen. 49, 10; Ps. 72, 8 usw. Jes. 49, 6 (Christus das „Licht der Heiden“ und „Gottes Heil“ bis an der Welt Ende).} und spricht daher nicht bloß einigen, sondern allen andern Religionen die Existenzberechtigung ab mit der hinzugefügten Begründung, dass alle Religionen mit Ausnahme der christlichen praktisch wertlos sind, dass sie nämlich die Menschen in der Finsternis und in Satans Gewalt belassen. Es heißt in der Zweckbestimmung des Christentums als Weltreligion: „auftun ihre Augen, dass sie [Juden und Heiden] sich bekehren von der Finsternis zum Licht und von der Gewalt des Satans zu Gott, zu empfangen Vergebung der Sünden und das Erbe mit denen, die geheiligt werden, durch den Glauben an mich“, nämlich Christum (\textlatin{\$\pi\acute{\iota}\sigma\tau\epsilon\iota\ \tau\stackrel{\acute{\eta}}{\ \epsilon}\pi\stackrel{\prime}{\iota}\ \stackrel{\acute{\epsilon}}{\mu}\epsilon$}, scil. \textlatin{\$\epsilon\pi\stackrel{\prime}{\iota}\ \Chi\rho\iota\sigma\tau\acute{o}\nu$}). \footnote{Apost. 26, 18. Der Luthers geniale Darlegung zu Joh. 9, 2 ff., dass alle intellektuellen und moralischen Betrübungen der Heiden und der ungläubigen Juden die Menschen in trostloser Finsternis belassen, Ps. 1, 91, 10 ff. Vgl. Lechler, Schäffer im Apost. 26, 18 in langem Kommentar: „The purpose of his [des Apostels Paulus] mission is stated in such a manner that it can be understood only as referring to Gentiles.“ Dagegen richtig Meyer: \textlatin{\$\Omicron\stackrel{\acute{\iota}}{\zeta}\ \epsilon\sigma\tau\stackrel{\prime}{\iota}\ \omicron\stackrel{\prime}{\iota}\ \alpha\pi\stackrel{\prime}{\omicron}\ \tau\stackrel{\acute{\eta}}{\zeta}\ \tau\alpha\sigma\tau\rho\omicron\phi\stackrel{\acute{\eta}}{\zeta}\ \tau\stackrel{\acute{\eta}}{\zeta}\tau\epsilon\ \tau\stackrel{\prime}{\eta}\zeta\ \pi\nu\epsilon\acute{\upsilon}\mu\alpha\tau\acute{\iota}\chi\tilde{\eta}\zeta\ \kappa\alpha\iota\ \tau\stackrel{\acute{\eta}}{\zeta}\ \eta\theta\iota\chi\tilde{\eta}\zeta$}, ist nicht bloß auf \textlatin{\$\tau\acute{o}\nu\ \acute{\iota}\omicron\upsilon\delta\alpha\dot{\iota}\omicron\nu$ } und den \textlatin{\$\tau\grave{o}\nu\ \stackrel{\prime}{\epsilon}\lambda\lambda\eta\nu\alpha$ } zusammen, „was durch das pragmatische Verhältnis A. 19, 20 gefordert wird“. V. 20 berichtet Paulus selbst, wie er den Befehl Christi verstanden habe, nämlich so, dass er alsbald Juden und Heiden Buße und Bekehrung zu Gott predigte, wörtlich: bekehrt euch auch nicht nur die Bekehrung von der Finsternis zum Licht, sondern auch die Bekehrung von der Gewalt des Satans zu Gott gleicherweise auf die Heiden und die ungläubigen Juden. Auch hier hat Meyer zunächst die richtige Beziehung auf beide Klassen. Wenn er aber in Bezug auf die Beteuerung „von der Gewalt des Satans zu Gott“ limitierend hinzufügt, sie „berücksichtigt mit vorherrschender Beziehung die Heiden, welche \textlatin{\$\acute{\alpha}\beta\rho\omicron\iota\ \epsilon\nu\ \tau\stackrel{\tilde{\omega}}{\ \kappa}\acute{o}\sigma\mu\omega$}, Eph. 2, 12, sind, unter der Gewalt des Satans, des \textlatin{\$\acute{\alpha}\rho\chi\omega\nu\ \tau\omicron\tilde{v}\}.