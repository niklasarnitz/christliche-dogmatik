\section*{Wesen und Begriff der Theologie.} und als eine große Gabe Gottes anzusehen hat, ist die Schulung in allem weltlichen Wissen, das die Kirche als äußeren Apparat zur Ausrichtung ihres Berufes in der Welt, zum Lehren und Predigen des Evangeliums, nötig hat. Dahin gehört, was der allgemeinen Schulung des Geistes dient; überhaupt das, was man gewöhnlich unter humanistischer Bildung zusammenfasst. Insonderheit dringt Walther mit Luther auf das Studium der Geschichte, der klassischen Sprachen und für Theologen „in voller Ausrüstung“ auf das Studium der Originalsprachen der heiligen Schrift. Er sagt: „Der Geist Carlstadts, der Wiedertäufer und anderer die Wissenschaft als etwas Unnützes, ja Gefährliches und Fleischliches verachtender und dafür der Eingebungen des ‚Geistes‘ sich rühmender Schwärmer hat unter uns keine Stätte. Wir sind nicht des Sinnes, dass die Kirche in die Wüste fliehen, um ihrer Selbsterhaltung willen sich auf den Holzscheitischem legen, sich von der ungläubigen Welt abschließen, die Feinde außer ihr gewähren lassen, die antireligiösen Gebildeten, welchen das Evangelium nur in einer gewissen Form nahegebracht werden kann, preisgeben und dahinfahren lassen und sich nur an das ungebildete Volk wenden sollen. Nein; wir erkennen es als unsere heilige Pflicht, allen alles zu werden, auf dass wir allenthalben ja etliche selig machen. Wir stimmen von Herzen mit Melanchthon überein, wenn derselbe einst schrieb: ‚Eine Ilias von Übeln ist eine ungelehrte Theologie.‘ (\textit{Corpus Ref. XI, 278.}) “ Walther weist darauf hin, mit welchem Ernst Luther auf eine Bildung in allen „freien Künsten“ als allen Ständen nützlich gedrungen hat, und fügt hinzu: „Wie könnten wir uns auch Lutheraner, ja auch nur Christen nennen, wenn wir Wissenschaftsverächter wären?“ Freilich, besonders ausführlich spricht sich Walther den Zeitumständen entsprechend darüber aus, in welchem Sinne der Wissenschaft kein Heimatsrecht in der Kirche zu gestatten sei. Er schreibt: „Wir wollen nichts von einer Wissenschaft wissen, welche der Schrift gegenüber die Hausherrin und Meisterin spielen, anstatt nur auf Auffindung der in der Schrift enthaltenen Wahrheit behilflich zu sein, über dieselbe zu Gericht sitzen und aus sich korrigieren will, die anstatt in ihrer Sphäre zu bleiben, die zufällig auf ihrem Gebiete geltenden Gesetze zu allgemeinen erheben und dieselben auch dem Schriftgebiete aufnötigen will. Solche $\text{μεταφράσεις}$ als $\text{ἄλλο γένος}$ hatten wir für ebenso abgöttisch als unwissenschaftlich. Wir stimmen vollkommen mit Melanchthon überein, wenn