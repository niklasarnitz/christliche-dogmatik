\textsl{208}\hfill\textsl{Wesen und Begriff der Theologie.}\par hohe Alter hinein und vom reichlichen Hauskreuz nicht gelähmt, ist er bis an sein Ende einer der treuesten, eifrigsten und kräftigsten Verfechter strengen Luthertums geblieben. Dieser sein redlicher polemischer Eifer wird ihm vielfach vorgeworfen. Nach dem Urteil der Neueren ist er ein Mann, der wie ein Wachthund beständig vor dem Hause seines Herrn steht und bissig alle anbellt, die auch nur am Zaun etwas abbrechen wollen. So ein Mann gefällt solchen Theologen natürlich nicht, die schier alles, selbst die Kernlehren, preiszugeben willens sind. Zu Calovs Zeit, gerade wie heute, fing man an, unter korrekt klingender Phrase so leise und still am Zaun der reinen Lehre abzubrechen; da konnte er als treuer Wächter nicht schweigen und muss es sich daher gefallen lassen, dass man ihn heute viel schmäht. Vielfach, sogar mit ehrenrührigen Worten (Rn. I II, 507), wirft man ihm seine echte Ehe im hohen Alter vor, während man doch auf der andern Seite das wirklich unhöfliche, jahrelange Verhältnis Schleiermachers, dem die Neueren alle huldigen, gar nicht oder wenigstens äußerst schonend (Rn. XII, 743) berührt. „Sine ira et studio ist das gewiss nicht gehandelt.''\par Was Höneckes Einfluss auf die Wisconsinsynode betrifft, so wird er von Prof. J. Schaller (\dag 1920) als der Mann beschrieben, durch den die Wisconsinsynode zu einer klaren Lehrstellung gekommen ist. Schaller sagt: „Es handelte sich damals [als die Wisconsinsynode noch zum General Council gehörte] darum, der Wisconsinsynode eine unmissverständliche Lehrstellung zu verschaffen und ihr Verhältnis zu andern amerikanischen Kirchenkörpern wie auch zu der deutschländischen Kirche klarzustellen. Sie gehörte damals zum Generalkonzil, das zwar in seinem Bekenntnis zur lutherischen Lehre viel entschiedener stand als die Generalsynode, aber doch wegen unionistischer Praxis den entschiedenen Lutheranern, die der Wisconsinsynode angehörten, missfiel. Auf der andern Seite stand die Missourisynode mit ihrem unzweideutigen Bekenntnis zu den symbolischen Büchern der lutherischen Kirche und ihrem entschiedenen Zeugnis gegen diejenigen, die mit dem lutherischen Bekenntnis in der Praxis nicht Ernst machten. An den Verhandlungen, die über die Bekenntnisfrage geführt wurden, nahm der junge Pastor Hönecke regen und bald auch entscheidenden Anteil. Denn er hatte sich auf das Studium der alten lutherischen Dogmatiker geworfen und schnell nicht nur eine gründliche Bekanntschaft mit ihrer Lehrstellung erlangt, sondern auch die Herzensüberzeugung gewonnen, dass jede unionistische Verbrüderung nicht nur Verleugnung des lutherischen