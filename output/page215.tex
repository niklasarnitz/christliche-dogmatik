\centering Wesen und Begriff der Theologie. \hfill 204\par\parlasst uns wachen und beten, dass wir das Erbe dieser Zukunft nicht verlieren! Unter dem Druck der \textquotedblleft unwissenschaftlichen Wissenschaft\textquotedblright, wie Walther es auszudrücken pflegte, ist der spätere Delitzsch, wie bereits bemerkt wurde, von seinem eigenen Wahrheitszeugnis abgewichen. Aber das benimmt seinem früheren Zeugnis ebensowenig die Wahrheit, als des späteren Melanchthons Abirren vom rechten Wege die ursprünglich von ihm bekannte Wahrheit hinfällig macht. Nach und nach trat eine Entfremdung zwischen unserer amerikanisch-lutherischen Kirche und der Kirche Deutschlands ein. Wir blieben bei der Schrift als Gottes Wort und als der einzigen Quelle und Norm der Theologie und sahen in Luther, dem Reformator der Kirche, das rechte Vorbild, wie in der christlichen Kirche zu lehren sei. Die deutschländische Theologie gab je länger, je mehr die Schrift als Gottes Wort auf und wandelte in den Bahnen Schleiermachers, des \textquotedblleft Reformators des 19. Jahrhunderts\textquotedblright, der die Kirche und ihre Theologie nicht auf den Felsen des Wortes Gottes zurückführte, wie dies der Reformator des 16. Jahrhunderts tat, sondern Kirche und Theologie in den Sumpf des Subjektivismus hineinzerrte, indem er die Parole ausgab, die christliche Lehre anstatt aus der Schrift aus dem angeblich frommen Ich des theologisierenden Subjekts, dem \textquotedblleft Erlebnis\textquotedblright{} usw. zu beziehen. In diesem Sumpf des Subjektivismus bewegt sich gegenwärtig fast die ganze Theologie Deutschlands, soweit die öffentlichen Lehrer in Betracht kommen.\par\parEin Laie (ein Jurist) weist in der \textquotedblleft N. Kirchl. Zeitschr.\textquotedblright{} (1923, S. 116) auf die unglückliche politische Lage hin, in Deutschland sich gegenwärtig befindet und die wohl im allgemeinen den amerikanischen Lutheranern \textquotedblleft streng konfessioneller Richtung\textquotedblright{} am tiefsten zu Herzen geht. Möchte durch Gottes Gnade, wie in andern Ländern, so auch in Deutschland die Theologie aus dem Sumpf der Zeittheologie herausgeführt werden! Der Laie schreibt u.\,a.: \begin{quote}Das deutsche Volk ist so tief gesunken in der Zeit nach dem Dreißigjährigen Kriege. Aber bedenken wir, dass in jener Zeit sich das Glaubensleben in der evangelischen Kirche am reichsten entfaltet hat. Damals entstanden die schönsten unserer Kirchenlieder, die uns allen von Kindheit an ein unsterblicher Schatz des Gemütes sind. Hoffen wir, dass gerade in den Stürmen und Anfechtungen der neuen Zeit sich unsere Kirche als eine große Kraft des Lebens erweisen wird.\end{quote}Das würde geschehen, und die Kirche würde auch jetzt die politische Not wegglanden und wegbeten wie nach dem Dreißigjährigen Kriege, wenn die kirchlichen Verhältnisse so lägen wie damals. Damals hielten die Lehrer