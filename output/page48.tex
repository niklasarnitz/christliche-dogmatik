\section*{Wesen und Begriff der Theologie.}\n\hfill 37\n\nDie christliche Religion ist aus einem doppelten Grunde „absolut“ oder schlechthin vollkommen und unüberbietbar.\n\nErstlich deshalb, weil sie zur Versöhnung mit Gott nicht, wie alle nichtchristlichen Religionen, des Menschen eigene Werke oder eigene Tugend fordert, sondern im Gegenteil Glaube an die vollkommene und unüberbietbare Versöhnung ist, die dadurch zustande kam, dass Gott in Christo war und die Welt mit sich selbst versöhnte.\footnote{133) 2 Kor. 5, 18. 19.} Noch anders ausgedrückt: Die christliche Religion ist deshalb schlechthin vollkommen, weil sie nicht eine moralische Anweisung ist, wie die Menschen sich selbst Vergebung der Sünden erwerben können, sondern im Gegenteil Glaube an die Vergebung der Sünden, die durch Christi stellvertretende Gesetzeserfüllung und stellvertretendes Strafleiden erworben wurde\footnote{134) Gal. 4, 5; 3, 13.} und die nun in der Verheißung des Evangeliums dem Glauben zur Aneignung dargeboten wird.\footnote{135) Apost. 26, 18; Röm. 10, 17; 1 Kor. 2, 4. 5.} In dieser Tatsache, dass doch Christi satisfactio vicaria die Versöhnung oder Vergebung der Sünden vorhanden ist und im Evangelium verkündigt wird, ist es begründet, dass ein Mensch in demselben Augenblick, in welchem er durch Wirkung des Evangeliums\footnote{136) Röm. 3, 28; λογιζόμεθα οὖν νίσει δικαιοῦσθαι ἀνθρωπον χωρὶς ἔργων νόμου. Röm. 3, 1: διαλύωμεν οὖν τὲ νόμον? κτλ.} zum Glauben an das Evangelium kommt, durch diesen Glauben — ohne des Geistes Werk, ἔργων νόμον — vor Gott gerecht\footnote{137) Kol. 2, 10: „Ihr seid vollkommen in ihm; nämlich in Christo, das ist ἀνὰ πρᾶγμα ἔπραξεν. Diese Aussage bezieht sich nicht auf „besonders geforderte“ Christen, sondern auf alle, die Christum durch den Glauben angenommen haben. V. 5—7; also auf alle Christen. Und diese Vollkommenheit sollen sich die Christen nicht streitig machen lassen durch die Prätensionen der Philosophie. V. 8. Die Philosophie nämlich, die auf dem Gebiet der Religion sich geltend macht, kann doch} oder, was dasselbe ist, vor Gott vollkommen (πεπληρωμένος, τέλειος) wird.\footnote{138) Vom 10, 1.} Denn Gott hält es