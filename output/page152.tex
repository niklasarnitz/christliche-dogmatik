\thispagestyle{plain}\setcounter{page}{141}\markboth{Wesen und Begriff der Theologie.}{Wesen und Begriff der Theologie.}\par Apostel und Propheten beiseitegeschiebt, der gründet sich nicht auf Christum, den Erlösten, sondern setzt sich daneben. Luther drückt dies, wie wir oben hörten, so aus: „Wenn ich ohne Wort bin, nicht daran denke noch damit umgebe, so ist kein Christus daheim.“\par Zur Erklärung des geistlichen und geistigen Zusammenbruchs der Theologie der Selbstgewissheit, des Erlebnisses usw. gehört der Hinweis auf die Tatsache, dass wir schon eingangs dieses Abschnitts berührten. Es ist dies die Tatsache, dass die Erlebnistheologie nicht eine gläubige, sondern eine kritische Stellung gegen Gottes Wort und damit gegen Gott selbst einnimmt. Dieser kritischen Stellung verdankt sie ihr Entstehen und Bestehen. Weil sie die Schrift als Gottes unfehlbares Wort verwirft, so hat sie sich auf das „religiöse Erlebnis“ des theologisierenden Subjekts zurückgezogen und examiniert und kritisiert von hier aus die heilige Schrift. Während Christus sagt: „Ich habe ihnen gegeben dein Wort“ und diesem Wort das Zeugnis ausstellt: „Dein Wort ist die Wahrheit“.\footnote{Joh. 17, 14. 17.} sagt die Erlebnistheologie vom äußersten linken bis zum äußersten rechten Flügel teils ganz offen heraus, teils in etwas verdeckter Weise, aber unisono: Dein Wort ist nicht die Wahrheit, sondern mit Irrtum durchsetzt. Und während Christus seiner Kirche geboten hat, dass sie sich zum Zweck der Wahrheitserkenntnis auf die Basis seines Wortes stelle, um auf diese Weise von jedem menschlichen Ich, insonderheit auch vom Ich der Theologen, erlöst zu sein, erteilt die Erlebnistheologie der Kirche die Weisung, sich von dem der Kirche gegebenen Wort Christi loszumachen und sich in die „Sturmsfreie Burg“ des Selbstbewusstseins des christlichen Subjekts zurückzuziehen. Sie verschärft die Kritik noch durch die immer wiederholte Beteuerung: wer diesen Paktwechsel nicht mitmache, sondern die christliche Lehre noch aus der heiligen Schrift schöpfen und normieren wolle, wie die erste Kirche, die Kirche der Reformation und „sonderlich die alten Dogmatiker“ getan haben, der richte Unglück in der Kirche an; der vermittele nicht „lebendiges Christentum“ und „lebendigen Glauben“, sondern „Intellektualismus“, tote Orthodoxie. Das ist in genauer Darstellung die ausgesprochen kritische Stellung, die alle Erlebnistheologen gegen Gottes Wort und damit gegen Gott selbst einnehmen. Die Erlebnistheologie ist die Theologie der Erhebung über Gottes Wort. Es ist der Nietzsche'sche „Übermensch“ auf dem theologischen Gebiet. Nun wissen wir aber aus der heiligen Schrift, dass diese kritische Stellung gegen Gottes Wort ein überaus gefähr-