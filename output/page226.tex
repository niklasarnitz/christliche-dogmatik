\setcounter{footnote}{636} % Set footnote counter to continue from page 214 \begin{center}Wesen und Begriff der Theologie.\end{center}\null\hfill 215 bei dem andern nicht, \dots\ in dem verschiedenen Verhalten der Menschen gegen die angebotene Gnade suchen.“\footnote{Monatshefte 1872, S. 80, 87. 103.} „Also erklärt sich das verschiedene Wirken der bekehrenden und seligmachenden Gnade wohl aus dem verschiedenen Verhalten der Menschen ihr gegenüber.“\footnote{Seibstätter 1911, S. 526.} „Diese Lehre, wonach das „verschiedene Verhalten“ des Menschen als Erklärungs grund für die Tatsache verwendet wird, warum von zwei Menschen, die das Evangelium hören, der eine glaubt, während der andere nicht glaubt, und wonach die Bekehrung und Seligkeit in des Menschen eigener Hand steht, ist die Lehre des späteren Melanchthon, die in der Konkordienformel so nachdrücklich verworfen wird. Es ist dies auch die Lehre moderner Lutheraner wie Dieckhoff und Luthardt, welche meinen, die sola gratia streichen zu müssen, um die Kirche vor Calvinismus zu bewahren.“\footnote{II, Not. 1296; II, Not. 1317.} „Und weil die Missourisynode wie die allgemeine Gnade, so auch die sola gratia festhält, so ist die Sage von dem missourischen „Calvinismus“ in der Welt verbreitet worden, und darum redete auch die „allgemeine Ev.-Luth. Kirchenzeitung“ von der „bitteren missourisch-calvinistischen Pille“, die angeblich die Synoden von Wisconsin, Minnesota usw. verschluckt hätten. Mit welcher klaren Erkenntnis der Lehre der Heiligen Schrift, Luthers und des lutherischen Bekenntnisses diese, wir müssen sagen, so leichtsinnig angeklagten Synoden alle einschlägigen Lehrpunkte behandelt haben, ergibt sich aus ihrem Bericht über die Synodalsammlung im Jahre 1882\footnote{Handlung der 32. Versammlung der Synode von Wisconsin in Gemeinschaft mit der Synode von Minnesota in La Crosse, Wis., vom 8. bis zum 14. Juni 1882.} um zu verhüten, dass irgend jemand wider seine Überzeugung zu dem Resultat der Verhandlungen ja sage, wurde nach den eingehenden Verhandlungen „beschlossen, nachmittags wieder zu gemeinschaftlicher Sitzung zusammenzutreten und ein jedes Glied der beiden Synoden zu veranlassen, seine Übereinstimmung mit der vorgetragenen Lehre zu erklären oder, wo es sich zu derselben nicht bekennen könne, dies ebenfalls kundzugeben“\footnote{Bericht, S. 34 f.} Die Stellung der beiden Synoden zur Lehre der Schrift und des lutherischen Bekenntnisses geht aus den folgenden ausschlaggebenden Sätzen hervor:\footnote{A. a. O., S. 23. 39. 56. 41.} „Der Mensch kann zu seiner Bekehrung weder durch sein Tun noch durch sein Verhalten das Geringste beitragen, und dass einige bekehrt werden, während