\setcounter{page}{86}
\section*{Wesen und Begriff der Theologie.}

„ohne Gesetzes Werke“, „durch den Glauben an Jesum Chri-
stum“, „durch das Evangelium“ ($\chi^{\alpha ́}\rho\iota\zeta$ $\rho^{\prime}\omega\mu\nu$, $\chi^{\alpha ́}\rho\iota\zeta$ $\varepsilon^{\prime}\rho\gamma\omega\nu$ $\rho^{\prime}\omega\mu\nu$
— $\delta^{\iota}\alpha ́$ $\pi^{\iotá}\sigma\tau\varepsilon\omega\zeta$, $\pi^{\iotá}\sigma\tau\iota\zeta$ $\delta^{\iota} \alpha ́$ $\rho^{\prime}\omega\nu$ $\varepsilon^{\prime}\nu\varepsilon ́\rho\gamma\varepsilon\iota\alpha\nu$)\% TODO: Formatierung prüfen
\footnote{284) Röm. 3, 21. 28; Gal. 2, 16; Röm. 3, 22; 1 Kor. 15, 2; Eph. 2, 8.} Alle, die zur Er-
langung der Gnade und Seligkeit das Gesetz nicht ausscheiden wollen,
bleiben unter dem Fluch des Gesetzes, weil das Gesetz über jeden Men-
schen den Fluch ausspricht, der nicht geblieben ist in alle dem, das ge-
schrieben steht im Buche des Gesetzes, dass er’s tue.\footnote{285) Gal. 3, 10.} Daher sagt
Luther mit Recht, dass je der Christ die Kunst der Unterscheidung
von Gesetz und Evangelium verstehen müsse. „Wo es an diesem Stück
mangelt, da kann man einen Christen von einem Heiden oder Juden
nicht erkennen.“\footnote{286) St. E. IX, 798.} Ein Mensch ist ein Christ und bleibt ein
Christ nur dadurch, dass er in seinem Gewissen wider die Anklagen
des Gesetzes sich mit dem Evangelium tröstet, das ihm die Vergebung
seiner Sünden „ohne Gesetz“ zusagt. Auch sind Heiligung und
Werke nur bei den Menschen möglich, die nicht unter dem Gesetz,
sondern unter der Gnade sind.\footnote{287) Röm. 6, 14.} Dies alles wird ausführlich dar-
gelegt im Anschluss an die Lehre von den Gnadenmitteln unter dem
Abschnitt „Gesetz und Evangelium“.\footnote{288) III, 259 ff.} Hier, wo es sich um die
„Einteilung“ der Schriftlehren und um die Charakterisierung der
rechten „Theologie“ und eines rechten „Theologen“ handelt, möge
der folgende kurze Umriss Platz finden.

Dass in der Schrift vorliegende Wort Gottes teilt sich dem In-
halte nach in Gesetz und Evangelium.\footnote{289) In der ganzen Schrift ist Gottes Gesetz gelehrt (Matth. 22, 37—40);
ebenso ist in der ganzen Schrift Gottes Evangelium gelehrt (Röm. 1, 1. 2; 3, 21;
Apostl. 10, 43). Daher die Apologie ($\mathfrak{A}_{r}$. 87, 5): Universa Scriptura in hos duos
locos praecipuos distribui debet: in Legem et promissiones. Alias enim legem
tradit, alias tradit promissionem de Christo.} Beide muss der Theologe
unverkürzt und unverändert lehren. Er darf weder etwas vom Gesetz
preisgeben noch eine Änderung am Evangelium vornehmen. In
Bezug auf das Gesetz heißt es: „Bis dass Himmel und Erde vergehe,
wird nicht vergehen der kleinste Buchstabe noch ein Tüttel vom Gesetz
(\textit{lôna êv j} $\mu\iota\alpha$ $\chi\varepsilon\varphi\alpha\lambda\alpha$), mit der hinzugefügten Warnung: „Wer
eins von diesen kleinsten Geboten auflöset und lehret die Leute also,
der wird der kleinste heißen im Himmelreich.“\footnote{290) Matth. 5, 17—19.} Und was das
Evangelium betrifft, so spricht der Apostel Paulus den Fluch aus über
jeden, der ein anderes Evangelium lehrt, als er selbst gelehrt hat.\footnote{291) Gal. 1, 7—9.}