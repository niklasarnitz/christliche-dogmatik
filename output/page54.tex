dass allen Christen Gottesgelehrtheit zukommt,\footnote{Joh. 6, 45: \emph{διδακτοὶ} (von allen Christen ist die Mehr) \emph{διδακτοὶ} (von Gott).} als auch, dass den Lehrern der Christen eine besondere Gottesgelehrtheit eigen sein soll.\footnote{Die rhetorische Frage 1 Kor. 12, 29: \emph{Μὴ πάντες διδάσκαλοι};, hat den Sinn: Nicht alle Christen sind Lehrer. Auch 1 Tim. 3, 2: „Es soll aber ein Bischof sein \dots Lehrhaftig“ (\emph{διδακτικός}), bezeichnet lehrhaftig eben eine besondere Grad der Lehrhaftigkeit, wie auch V. 5 noch ausdrücklich hervorgehoben wird, weil der Bischof nicht bloß sich selbst und sein eigenes Haus, sondern auch die Gemeinde Gottes versorgen soll. Deshalb gibt Paulus auch Timotheus, 2 Tim. 2, 2, den Auftrag: „was du von mir gehört hast, das befiehl treuen Menschen, die da tüchtig sind, auch andere zu lehren“, \emph{οἵτινες ἱκανοὶ ἔσονται καὶ ἑτέρους διδάξαι}. Aus den hergeholten drei Eigenschaften, die an den zu bestellenden Lehrern sich finden sollen (1 Tim. 3, 1 ff. Tit. 1, 6 ff.), geht hervor, dass die Lehrer nicht etwa durch das Los, sondern nach besonderen Eigenschaften, zu welchen auch die besondere Lehrthüchtigkeit gehört, gewählt werden sollen.} Bei dieser Unterscheidung ist jedoch festzuhalten, dass beide Arten von Gottesgelehrtheit, also nicht nur die aller Christen, sondern auch die der Lehrer, nur die heilige Schrift zur Erkenntnisquelle haben. Neuere Theologen verhandeln, ohne eine Einigung erzielt zu haben, über das Verhältnis zwischen „religiösem“ und „theologischem Erkennen“. Die einen wollen es „möglichst ineinandergerückt“, die andern möglichst geschieden haben. In der Gegenwart wird viel über den Unterschied zwischen „religiösem“ und „theologischem Erkennen“ gehandelt.\footnote{Vgl. z. B. Richard Grützmacher, Studien zur dogm. Theol., 3. Heft, S. 120 ff.} Vom christlichen Standpunkt aus ist festzuhalten: „religiöses Erkennen“ und „theologisches Erkennen“ unterscheiden sich nicht prinzipiell, auf die Erkenntnisquelle und das Erkenntnismedium gesehen, sondern fallen prinzipiell zusammen, weil auch des theologischen Erkennens Anfang, Mitte und Ende nichts anderes ist, als Gottes Wort, wie es in der Schrift geoffenbart vorliegt, glauben. Der Grund ist der, dass auch die Theologen oder die Lehrer der Kirche in ihrer Erkenntnis der christlichen Lehre nicht um eine Linie über Gottes Offenbarung in seinem Wort hinauskommen, wie die Schrift so oft und mannigfaltig bezeugt.\footnote{Nach Joh. 8, 31. 32 wird die Erkenntnis der Wahrheit (\emph{γνώσεσθε τὴν ἀλήθειαν}) nur durch das Bleiben an Christi Wort im Wort der Apostel haben (Joh. 17, 20), vermittelt und das Bleiben an Christi Wort nähert sich immer nur in der Weise, dass Christi Wort geglaubt wird. Insofern ein Lehrer nicht bei Christi Worten bleibt, kommt ihm nicht Erkenntnis, sondern Unwissenheit zu (1 Tim. 6, 3. 4).}