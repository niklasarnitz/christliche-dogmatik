\begin{center}
38

Wesen und Begriff der Theologie.
\end{center}
also, dass dem in sich Gottlosen (\textit{\textalpha\sigma\epsilon\beta\eta\varsigma}) sein Glaube zur Gerechtigkeit gerechnet wird,\footnote{139) Röm. 4, 5.}
oder was wiederum dasselbe ist, dass Christi vollkommene Gerechtigkeit des Sünders eigene Ungerechtigkeit bedeckt.\footnote{140) 1 Joh. 2, 1. 2: Ob jemand sündiget, \textit{\textpi\alpha\rho\alpha\kappa\lambda\eta\tau\omicron\nu \epsilon\chi\omicron\mu\epsilon\nu \pi\rho\omicron\varsigma \tau\omicron\nu \pi\alpha\tau\epsilon\rho\alpha, \textiota\eta\sigma\omicron\upsilon\nu \Chi\rho\iota\sigma\tau\omicron\nu \delta\iota\kappa\alpha\iota\omicron\nu.}}
Zur Wahrung des absoluten Charakters der christlichen Religion gehört daher, dass wir voll und ganz an der satisfactio Christi vicaria festhalten. Wollten wir mit Rom zur Erlangung der Versöhnung mit Gott auch „die eingeflossene Gnade“, das Halten

als lauterer Betrug, kein Zwang, eingeschätzt werden, weil die Menschenliebe ist (\textit{\textchi\alpha\rho\alpha \tau\eta\nu \pi\alpha\rho\alpha\delta\omicron\upsilon\sigma\alpha\nu \tau\omicron\nu \alpha\nu\theta\rho\omega\pi\omicron\nu}), dürftiges Wissen der Welt (\textit{\textchi\alpha\rho\alpha \tau\alpha \sigma\tau\omicron\iota\chi\epsilon\iota\alpha \tau\omicron\upsilon \kappa\omicron\sigma\mu\omicron\upsilon}), nämlich Gesetz (so richtig von Neueren auch Cremer im Wörterbuch sub \textit{\sigma\tau\omicron\iota\chi\epsilon\iota\omicron}). Die Christen hingegen haben ihr „Sehrmnatrium“ (Meyer) an Christo (\textit{\textchi\alpha\rho\alpha \Chi\rho\iota\sigma\tau\omicron\nu}), und das sie dadurch vollkommen sind, begründet der Apostel mit einem Hinweis auf die hohe Person Christi und auf den ihnen durch Christum gewordenen Geist. Seiner Person nach ist Christus nicht ein bloßer Mensch, sondern der, in welchem die ganze Fülle der Gottheit leiblich wohnt, der auch über die Engelwelt erhaben ist, so dass die Christen von dieser Seite (von der Engelwelt her) keine höhere Vollkommenheit zu erwarten haben, wie die Irrelehrer behaupteten, V. 16. Was den Besitz der Christen betrifft, der ihnen in Christo geworden ist, so sind sie durch „die Beschneidung Christi“, die Taufe, vom Sündenrode zum geistlichen Leben gekommen, weil Christus ihnen alle ihre Sünden vergab, welche Sündenvergebung in Christo satisfactio vicaria ihren Grund hat, nämlich darin, dass Christus die durch das Gesetz entstandene Schuldpflicht (\textit{\chi\epsilon\iota\rho\omicron\gamma\rho\alpha\phi\omicron\nu}) durch seinen Tod am Kreuz ausgelöscht hat, V. 14, „der. Das objektive Sühnopfer durch den Tod Christi war vorangegangen und wird V. 14 beschrieben.“ Auch in 1 Kor. 2, 6: „Weisheit aber reden wir unter den Vollkommenen“, \textit{\sigma\omicron\phi\iota\alpha\nu \delta\epsilon \lambda\alpha\lambda\omicron\upsilon\mu\epsilon\nu \epsilon\nu \tau\omicron\iota\varsigma \tau\epsilon\lambda\epsilon\iota\omicron\iota\varsigma}, sind die „Vollkommenen“, \textit{\tau\epsilon\lambda\epsilon\iota\omicron\iota}, nicht die „getreisten“, in die höhere Sphäre gründlicher und umfassender Einsicht“, sonderlich in die „künftigen Verhältnisse des Messiasreichs“ eingedrungenen Christen (Meyer usf.), sondern nach dem vorhergehenden und nachfolgenden Kontext alle die durch Wirkung des Heiligen Geistes das der Welt, und ihren Lehren, verbundene Gemeinschaft von Christo, dem Genuss von Glauben als alle (=Christen Luther, Osthausen usf.). Vgl. die Auslassung und Rezension der verschiedenen Auslegungen bei Wolf. Curae, z. St., mit dem Resultat: Mihi quidem cum beato Luthero nostro et plerisque aliis interpretibus prior placet sententia, ut scilicet per \textit{\tau\epsilon\lambda\epsilon\iota\omicron\upsilon\varsigma} intelligantur credentes, quos cap. 1, 24 appellaverat \textit{\pi\iota\sigma\tau\omicron\upsilon\varsigma}, i.e., tales, qui factae vocationi obtemperassent. Es ist in dem ganzen Zusammenhang nicht von „zukünftigen Verhältnissen des Messiasreichs“ die Rede (eine Auffassung, bei der chiliastische Ideen hineinspielten), auch nicht speziell von der „künftigen Seligkeit der Christen, die V. 10 gelegentlich angewendet wird, sondern von dem, was die Christen gegenwärtig durch den Glauben an das Evangelium von dem gekreuzigten Christus haben.