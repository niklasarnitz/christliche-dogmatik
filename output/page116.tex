Fragen in diesem Sinne anzuerkennen sind, geht aus allen Schriftaussagen hervor, in denen das Hinzutun zu Gottes Wort verboten ist.\footnote{384} Jeder wahre Theologe muss nicht nur das Reden, sondern auch das Schweigen lernen. Er soll reden, wo und soweit Gottes Wort redet, aber auch schweigen, wo Gottes Wort schweigt, das ist, keinen Aufschluss gibt. Lernt er nicht diese Kunst des Schweigens, sondern erlaubt er sich zu reden, wo Gottes Wort schweigt, so gilt ihm das Wort: „So spricht der HERR Zebaoth: Gehorchet nicht den Worten der Propheten, so euch weissagen! Sie betrügen euch, denn sie predigen ihres Herzens Gesicht und nicht aus des HERRN Munde.“\footnote{385} Offene Fragen in diesem Sinne werden auch „theologische Probleme“ genannt, nämlich Probleme in dem Sinne, dass sie in der Kirche hier auf Erden nicht gelöst werden können, weil hier die göttliche Lösung durch die Heilige Schrift fehlt. Das ist der Sinn des alten Diktums, dass ein Theologe auf viele Fragen mit gutem Gewissen antworten könne: „Ich weiß nicht“, \emph{nescio}.\par Zu den theologischen Problemen in diesem Sinne gehört z. B. die Frage, wie die Sünde entstehen konnte, da doch alle Kreaturen, also auch alle Engel, ursprünglich gut geschaffen waren.\footnote{386} Hierher kann auch die Frage gerechnet werden, ob die Seele des Kindes jedesmal unmittelbar von Gott geschaffen (\emph{Kreatianismus}) oder von den Eltern auf das Kind übergeleitet, also mittelbar von Gott erschaffen werde (\emph{Traduzianismus}).\footnote{387} Zu den in diesem Leben unlös-