\phantomsection\label{page:114}\quad 114 \hfill Wesen und Begriff der Theologie.\parwieder aufzunehmen. --- Die praktische Theologie ist die vom heiligen Geist gewirkte Tüchtigkeit, die aus der heiligen Schrift erkannte reine christliche Lehre in allen Funktionen des öffentlichen Predigtamts praktisch anzuwenden, also in der öffentlichen Predigt und in der Privatseelsorge, in der katechetischen Unterweisung von Jung und Alt, in der Regierung der Gemeinde usw. Es liegt auf der Hand, wie der theologische Charakter auf dem praktischen Gebiet ganz direkt gefährdet wird, sobald hier außerbiblischer Raum gewinnt.\parHieraus ergibt sich, dass die theologischen Disziplinen sich nicht voneinander trennen lassen. Wie der Dogmatiker zugleich Exeget, Historiker und praktischer Theologe sein muss, so müssen auch der Exeget, Historiker und der praktische Theologe zugleich gute Dogmatiker in dem Sinne sein, dass sie die Schriftlehre in allen ihren Teilen genau kennen. Dem Verlangen nach einem „undogmatischen“ Christentum ist das Diktum entgegengestellt worden: „Nur die Dogmatik ist erbaulich.“ Das ist ganz richtig, wenn unter Dogmatik die doctrina divina verstanden wird, die in der Schrift geoffenbart vorliegt und die allein in der Kirche Christi gelehrt werden soll. Zu der christlichen Kirche kommt alles auf die Lehre an, wie aus Christi Generalinstruktion Matth. 28 hervorgeht: „Lehret sie halten alles, was ich euch befohlen habe!“ Das sollen sowohl die theologischen Lehrer als auch die praktischen Prediger nie vergessen. Alle Theologen, welche die direkte Mitteilung „übernatürlicher Wahrheiten“ aus der Schrift, also der Lehre der Schrift, als „Intellektualismus“ störend abweisen, offenbaren damit, dass sie vergessen haben, was ihres Amtes ist. Und was die praktischen Prediger betrifft, so sollten auch sie insonderheit nicht vergessen, dass sie vor allen Dingen Lehre, die in der heiligen Schrift vorliegende göttliche Lehre, zu predigen haben. Ihre Predigten müssen, wie wir es gewöhnlich ausdrücken, „Lehrpredigten“ sein. Über Lehrpredigten, was sie seien, wie sie wirken und warum sie vielfach unterlassen werden, möchten wir Walther zu Worte kommen lassen. Er schreibt:\footnote{414) „Mag eine Predigt noch so reich an Ermahnungen, Bestrafungen und Tröstungen sein, ist sie dabei ohne Lehre, so ist sie doch eine leere, magere Predigt, deren Ermahnungen, Bestrafungen und Tröstungen wie in der Luft schweben. Es ist nicht auszusagen, von wie vielen Predigern und wieviel in dieser Beziehung gesündigt wird. Kaum hat der Prediger seinen Text und Lehrgegenstand berührt, so fängt er auch schon an zu ermahnen oder zu strafen oder zu trösten. Seine}\footnote{414) Pastorale, S. 81 f.}