\usepackage{upgreek}

\subsection*{18. Theologie und Lehrfreiheit.}

Wie die Freiheit aller Christen darin besteht, dass sie vom eigenen Willen los und Gebundene der Knechtschaft Gottes geworden sind (\textgreek{δουλαύρεες τῷ θεῷ}),\footnote{521} so besteht insonderheit die Lehrfreiheit der Lehrer der Kirche in ihrem völligen Gebundensein durch Gottes Wort. Diese Erklärung der Lehrfreiheit gibt Christus ausdrücklich in den Worten: „So ihr bleiben werdet an meiner Rede, so \ldots{} werdet ihr die Wahrheit erkennen, und die Wahrheit wird euch freimachen“ (\textgreek{ἡ ἀλήθεια ἐλευθερώσει ὑμᾶς}).\footnote{522} Sofern der Theologe das für christliche Lehre hält und ausübt, was dem eigenen Ich oder dem Ich anderer Menschen entstammt, ist er der Menschenknechtschaft verfallen. Die Lehrfreiheit, die in unserer Zeit sonderlich für die Theologen gefordert worden ist und darin bestehen soll, dass der Theologe nicht durch das Wort der Heiligen Schrift gebunden sei (Buchstabenknechtschaft, unmündiger Lehrzwang, gesetzlicher Geist usw.), ist die Freiheit des Schlechten, noch anders ausgedrückt: die Freiheit des „übermenschlichen“, der über Gottes Wort und Willen erhaben zu sein meint.

Um zu erkennen, welche Abnormität in der allgemein geforderten Lehrfreiheit uns entgegentritt, achten wir noch auf die folgenden einzelnen Punkte:
\begin{enumerate}
    \item Die christliche Kirche hat bis an den Jüngsten Tag nur einen Lehrer. Das ist Christus.\footnote{524} Als \textgreek{ὅπου ἐσμὲν ὁ διδάσκαλος, πάντες οἱ ἡμεῖς ἀδελφοὶ ἐστε} = „wahryynhς \textgreek{ἡμῶν ἐσυρ ἔλε ὁ Χουοτός} 523), wo er, Christus, seinen Jüngern befohlen hat, sie sollen alle bis ans Ende der Tage lehren.“\footnote{527} Obwohl Christus im Stande der Erhöhung nach seiner sichtbaren Gegenwart der Kirche entzogen ist, so ist und bleibt er doch auch im Stande der Erhöhung der einzige Lehrer seiner Kirche durch sein Wort, das er durch seine Apostel der Kirche gegeben hat und auf welches Wort er seine Kirche bis an den Jüngsten Tag verweist.\footnote{528} So haben es auch die Apostel verstanden. Sie binden die Christen an ihre Lehre.\footnote{529} Aber das tun sie, weil sie wissen, dass sie nur Christi Wort reden.\footnote{530} So haben es auch nach der Apostel Zeit alle rechten Lehrer der Kirche verstanden. \textit{Luther}:\footnote{531} „Will jemand predigen, so
\end{enumerate}

\vspace{1em} 

\RaggedRight 
\footnotesize
\noindent 
\footnotetext[521]{521) Röm. 6, 22.}
\footnotetext[524]{524) Matth. 23, 8, 10.}
\footnotetext[525]{525) Röm. 3, 31. 32.}
\footnotetext[526]{526) Matth. 28, 19. 20.}
\footnotetext[527]{527) vgl. Matth. 23, 8. 10.}
\footnotetext[528]{528) Matth. 28, 19. 20.}
\footnotetext[529]{529) 2. Thess. 2, 15; Gal. 1, 6—9.}
\footnotetext[530]{530) 1. Kor. 14, 37; 2. Kor. 13, 3; 1. Tim. 6, 3.}
\footnotetext[531]{531) Vgl. S. XII, 1413.}