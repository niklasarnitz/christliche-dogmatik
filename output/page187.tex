Methode bedeutet Verschiedenheiten in der Stellung einzelner Lehren. Baier z. B. behandelt die letzten Dinge: Tod, Auferstehung, Jüngstes Gericht und Weltuntergang, vor der Lehre von der Sünde und der Erlösung durch Christum (nämlich im unmittelbaren Anschluss an den finis theologiae ex parte hominis, die Seligkeit). Quenstedt hingegen stellt die genannten Lehren ganz an das Ende seiner Dogmatik. Was das Urteil über die synthetische und analytische und jede andere Methode betrifft, so sollte nicht vergessen werden, dass die äußere Anordnung der einzelnen Lehren von untergeordneter Bedeutung ist, solange der Inhalt der Lehren nicht gefälscht wird. In der Theologie kommt, wie auch Kudelbach richtig bemerkt hat,ootnote{585} alles darauf an, dass der Begriff der göttlichen Offenbarung durchweg festgehalten wird, das heißt, dass alle Lehren lediglich aus der Schrift genommen und nicht irgendwie in Verfolgung der Methode des Prokrustes nach der Ansicht oder dem „Glauben“ des theologisierenden Subjekts zurechtgeschnitten werden. In der Theologie ist jede Methode zu verwerfen, die unter Abhebung von dem Schriftwort irgend etwas erfinden will, mag die Methode sich synthetisch oder analytisch, wissenschaftlich oder praktisch oder sonstwie nennen. In der Theologie gibt es nichts zu erfinden, weder was den Inhalt noch was den Zusammenhang der Lehren betrifft. Daher ist der Theologie immer nur eine solche Methode angemessen, die sich prinzipiell darauf beschränkt, das durch die Offenbarung der Schrift bereits Vorhandene zusammenzuordnen. Dies geschieht bei der sogenannten Lokalmethode, nach welcher an einen Ort zu- sammengestellt wird (daher die einzelnen Lehren passend loci genannt

\subsection*{Wesen und Begriff der Theologie.}


















































































































































































































































































































































































































































































































































































































































































































































































































































































































































































































































































































































































































































































































































































































































































































































































































































Wesen und Begriff der Theologie. 176