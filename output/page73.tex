\textbf{Wesen und Begriff der Theologie. \quad 62}

Theologen die strengste Gedankenzucht. Sie sollen bei sich schonungs-\los alle Gedanken über Gott und göttliche Dinge ausscheiden, die nicht in klaren Schriftworten ausgedrückt vorliegen. Dafür führt er sein eigenes Beispiel an. Es könne, sagt er, nicht ausbleiben, dass uns bei der Behandlung der hohen Dinge, die Gott und gött-\liche Dinge betreffen, eigene Gedanken kommen. Auch bei ihm seien mancherlei eigene Gedanken aufgetaucht. Aber Gott habe ihm die Gnade verliehen, sich die Gedanken wieder ausfallen zu lassen, die ohne Wort bei ihm sich einstellten. Luther kommt auf diesen Punkt, wenn Zwingli und Genossen ihm Mangel an „Geist“ vorwarfen und sein Hang an den Worten der Schrift als Kopfstarrsinn, Buchstabendienst, geistlose Theologie usw. bezeich-\neten. Dagegen bemerkt Luther, dass ihm wohl mehr eigene Ge-\danken gekommen seien als allen Schwärmern zusammengenommen. Er habe aber solche Gedanken fahren lassen weil sie sein Heimatrecht in der Kirche haben. Er schreibt:\footnote{221) St. z. XX, 792. Frl. 30, 46.} „O wie manche seine Einfälle hab’ ich in der Schrift gehabt, die ich hab’ müssen lassen fahren, welche, so sie ein Schwärmer hätte gehabt, wären ihm freilich alle Druckereien zu wenig gewesen.“ Luther nennt, wie wir schon hörten, alle Lehrer, die wirklich unter der Direktion des Heiligen Geistes stehen, „Katechumenen und Schüler der Propheten“, „als die wir nachsagen und predigen, was wir von den Propheten und Aposteln gehört und gelernt haben.“\footnote{222) St. z. III, 189. Frl. 37, 12.} Man hat sich an diesem „Nachsagen“ als einer Beschreibung der Lehrthätigkeit eines christlichen Lehrers gestoßen. Luther ver-\steht das „Nachsagen“ nicht dahin, dass ob der christliche Lehrer „nicht sollte brauchen mehr oder andere Worte, als in der Schrift

\par\noindent\hangindent=1.5em\small könnte als die Leser meiner Dogmatik wenden müssen.“ S. 118 f. gegen P. Munkel (Bescheidenheit: „Ich kann mir nicht denken, dass P. Munkel, der stas Doktor der Theologie schreibt, so wenig von Theologie versteht, dass er nicht wissen sollte, dass es Sophisteritäten sind, welche ausgestroßen werden müssen, Rothäulig und solche Untersuchungen nicht fürs Volk. Wer bringt sie denn aber ins Volk? Solche Blätter, wie sie P. Munkel schreibt. ... Er also, dieser Zwischenträger zwischen Wissenschaft und Volk, der keinem von beiden Kreisen recht angehört, er verwirrt das Volk, nicht ich. Wenn P. Munkel die Höhen nicht vertragen kann, wo Lamonen und Felsblöße fallen, so bleibe er doch in der Luneburger Heide bei den Heideschaukeln, pflege Bienen und ziehe Sporgel.“ So redet Kohnis von seinem Standpunkt aus, dass die heilige Schrift nicht „das inspizierte Gebrauch der reinen Lehre ist“ (S. 127) und daher erst durch die Wissenschaft der Theologen seinfedest werden müsse, was christliche Lehre sei.
