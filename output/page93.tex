82 Wesen und Begriff der Theologie.

noch soll. Richard Grützmacher sagt ganz richtig, obwohl er selbst nicht völligen Ernst damit macht, „dass uns die geschichtliche Heilsoffenbarung nur in der sie wiedergebenden Heiligen Schrift enthalten ist.“\footnotemark{270} Wenn nun neuere Theologen die „geschichtliche Art“ des Christentums dahin deuten und dazu verwenden, um die heilige Schrift als einzige Quelle und Norm der christlichen Lehre aus den Augen zu tun und die christliche Lehre der eigenen Inwendigkeit zu entnehmen, so stehen wir vor einem ganz ungeschichtlichen Angriff auf die wirkliche Geschichte des Christentums. Die Berufung auf die „Geschichte“ schließt daher eine Selbsttäuschung in sich. In Wirklichkeit geht die Tendenz dahin, dass sich das „dogmatisierende Subjekt“ in der Kirche Christi auf den Lehrstuhl setzt an Stelle Christi, der in seinem Wort die einzige Lehrautorität ($\text{εἷς ὁ διδάσκαλος}$) in der Kirche ist bis an das Ende der Zeit.\footnotemark{271}

Um die eben besprochenen Punkte zusammenzufassen: Wenn wir theologischen Lehrer die heilige Schrift nicht Gottes eigenes Wort sein lassen und deshalb auch nicht als einzige Quelle und Norm der Theologie verwenden, so lehren wir auch nicht Gottes Lehre (doctrinam divinam), sondern die eigene Anschauung (proprias opiniones). Es ist sachlich indifferent, ob wir die subjektive Quelle und Norm christliches Erlebnis oder Glaube und Glaubensbewusstsein oder wiedergeborenes Ich oder geschichtliche Auffassung des Christentums oder sonstwie nennen. Alle Wege, die an der Schrift als der einzigen Quelle und Norm der Theologie vorbeiführen, führen in die Ichtheologie, und wir Christen sind berechtigt, uns das schon zitierte Wort Luthers zuzurufen: „Sie reden solch Ding nur darum, dass sie uns aus der Schrift führen und sich selbst zu meistern über uns erheben, dass wir ihre Traumpredigten glauben sollen“, oder wie Luther den Ausdruck noch etwas drastischer gestaltet, „dass sie uns mögen aus der Schrift führen, den Glauben verdammen, sich selbst über die Eier setzen und unser Abgott werden“.\footnotemark{272}

Daher steht es in bezug auf die moderne Theologie so, dass sie erst nach prinzipieller Umkehr den Anspruch erheben kann, als christliche Theologie anerkannt zu werden. Die prinzipielle Umkehr besteht aber darin, dass sie die Schrift und Gottes Wort wieder „identifizieren“ lernt, das heißt, nach dem Vorgang Christi und

\vspace*{\fill}
\footnotetext{270}Studien zur systematischen Theologie, 3. Heft, S. 40.
\footnotetext{271}Matth. 23, 8; 28, 10. 20; Joh. 8, 31. 32; 17, 20.
\footnotetext{272}St. V, 336.