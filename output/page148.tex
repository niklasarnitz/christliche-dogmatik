Wesen und Begriff der Theologie.

Selbstbewusstseins", das sich von Christi Wort geschieden und damit gegen die Wahrheitserkenntnis eine Blockade errichtet hat.\footnote{481)}

3. Doch, wenn die Selbstgewissheit auch nicht zur Gewissheit, sondern zum Gegenteil dient, ist sie denn nicht wenigstens wissenschaftlich? Als Franks „System der christlichen Gewissheit“ in erster Auflage erschienen war, erregte sie großes Aufsehen. Aber es regte sich auch die Kritik. Namentlich wurde Frank von jemand, den er selbst in der zweiten Auflage seiner Schrift einen „hervorragenden“ Theologen nennt, der Gedanke nahegelegt, dass sein (Franks) großes Buch nur „geringen Nutzen“ haben dürfte, und zwar nach des Autors eigener Theorie. Wenn die „christliche Gewissheit“ wirklich, wie Frank behauptet, von jedem außer ihr gelegenen Dinge vollkommen unabhängig sei, so könne auch das Fränkische Buch, weil es doch auch zu den außerhalb des christlichen Subjekts gelegenen Dingen gehöre, weder Frank selbst noch irgendeinen anderen Menschen in der Welt irgend etwas zur Hervorbringung oder Erhaltung der christlichen Gewissheit nützen. Das

\footnote{481) Dass auch bei Frank aus der von ihm befolgten Methode die Urteile des christlichen Glaubens „verkrümmt und verkrüppelt hervorgehen“, ist in „Lehre und Wehre“ 1896, S. 65 ff. 97 ff. 129 ff. 161 ff. 201 ff. 262 ff., ausführlich dargelegt. Der Artikel unter der Überschrift „Franks Theologie“ ist von D. Stöckhardt geschrieben. Stöckhardt stellt Frank nicht in jeder Hinsicht auf gleiche Linie mit dem liberalen Flügel der Schleiermacherschen Theologie, er erkennt an, „dass Frank doch gewisse Elemente der christlichen Wahrheit stehen lässt, aber mit Recht hinzu: „Das liegt nicht im System (Franks), das ist Inkonsequenz“. Das ist ein Weh von Christentum, der sich dem fremdartigen theologischen Prinzip gegenüber noch behauptet hat. Und diese Inkonsequenz ist das Beste in seinem System und nicht sein Verdienst.“ Stöckhardt weiß nach, dass Frank die unfehlbare göttliche Autorität der heiligen Schrift seinem Ich zum Opfer bringt. Während Christus und seine Apostel ohne Bedenken Schrift und Gottes Wort identifizieren, sagt Frank: „Ich möchte die Verantwortung nicht auf mich nehmen, einen Christen zu lehren, dass der Glaube an die Christuswahrheit insbesondere den Glauben an die absolute Vertrauenswahrheit der heiligen Schrift.“ (L. u. W., a. a. O., S. 297.) Nach der satisfactio vicaria Christi geht Frank fallen. Während die Schrift ausdrücklich lehrt, dass Christus die Strafe erlitten hat, die wir Menschen hätten erleiden sollen (Jes. 53, 5; 2 Kor. 5, 21; Gal. 3, 13), erklärt Frank: „Müsste man behufs der Stellvertretung Christi fordern, dass er gelitten habe, was die verdammte Menschheit hätte leiden müssen, so wäre die satisfactio vicaria hinfällig, da Christus eben dies nicht erlitten hat.“ (L. u. W., a. a. O., S. 138.) In der Lehre von der Rechtfertigung redet Frank einerseits ganz orthodox von der iustitia extra nos posita, andererseits kommt auch Frank im Grunde zu der Rechtfertigung nicht bloß als modicum lyricos, sondern auch an Verhaltens des Menschen, als an der freien Selbstbestimmung, in Betracht. (L. u. W., a. a. O., S. 169.)}