143 Wesen und Begriff der Theologie.\par\par\setcounter{footnote}{491}im Vergleich mit dem früheren unchristlichen Subjekt) als „Grundlage des ganzen Christenstandes“ verwenden will. Ahmels urteilt sehr richtig, dass die ethische Beschaffenheit des Christen „mannigfachen Schwankungen“ unterworfen sei und deshalb nicht Fundament des Christenstandes sein könne. Frank selbst habe diese „mannigfachen Schwankungen“ zugegeben. Die Grundlage des Christenstandes, sagt Ahmels, könne nur die Rechtfertigung bilden. Aber auch Ahmels gelingt es nicht, über den Subjektivismus hinauszukommen, weil er, wie Frank, in den Rechtfertigungsglauben menschliches Tun mengt. Wenn die Worte, mit denen er seine „Zentralfragen“ schließt, ernst zu nehmen sind, so stellt er das Kommen zum gnädigen Gott und die Wahrheitserkenntnis auf das menschliche Wollen. Er sagt dort: \footnote{Zentralfragen 2, S. 166.} „Zulegt heißt diese Wahrheit [die christliche Wahrheit] Gott, und Gott kann sich nur dem erschließen, der Ihn will.“ Das ist wahrlich eine klar ausgesprochene subjektive Begründung sowohl der Heils- als auch der Wahrheitsgewissheit. So dachte sich, wie die Konkordienformel erinnert, auch Chrysostomus die Sache: Trahit Deus, sed volentem trahit; tantum velis, et Deus praeoccurrit. „Gott zieht, er zieht aber den, der da will; du musst nur wollen, so wird dir Gott zuvorkommen.“ Mit Recht warnt die Konkordienformel vor diesen Reden, weil sie „zur Bestätigung des natürlichen freien Willens . . . wider die Lehre von der Gnade Gottes eingeführt“ und „der Form gesunder Lehre nicht ähnlich“ seien. \footnote{M. 608, 86.} Im Reiche Gottes liegt es nicht an jemandes Wollen oder Laufen, sondern an Gottes Erbarmen. \footnote{Röm. 9, 16. 30--33.} Wie stehen wir armen Menschen uns doch im Wege, und wie töricht handeln wir, wenn wir nach einer subjektiven Begründung der Gewissheit suchen! Der Grund unserer Gewissheit liegt außer uns, in Gottes Wort, im Wort der Apostel und Propheten, worauf die christliche Kirche erbaut ist. Jeder Versuch, die Gewissheit, sei es die Heilsgewissheit, sei es die Wahrheitsgewissheit, durch etwas, was in uns selbst gelegen ist, zu begründen, macht die Gewissheit ungewiss, einerlei wie wir das in unser Ich verlegte Fundament benennen, ob Wiedergeburt oder sittliche Umwandlung oder Selbstbestimmung und Selbstzueignung oder menschliches Verhalten und geringere Schuld oder vorgängiges Wollen usw. Das hat Luther wohl erfahren. Deshalb ruft er Erasmus zu, als dieser ihm zumutete, das Kommen zum gnädigen Gott durch die facultas se applicandi ad gratiam bedingt sein zu