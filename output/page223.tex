\centerline{212}\centerline{Wesen und Begriff der Theologie.}\par Kirche im eigentlichen Sinne des Worts sind nur die gläubigen Christen, und nur sie sind die ursprünglichen Besitzer aller geistlichen Güter und Rechte. Sie sind es daher auch, die das öffentliche Predigtamt den dazu tüchtigen Personen durch den Beruf „übergeben“. In Bezug auf die Frage, ob das öffentliche Predigtamt göttliche oder menschliche Ordnung sei, lehrt er entschieden die göttliche Ordnung. \footnote{632) IV, 175 ff.} Er wendet sich einerseits gegen Grabau, Löhe, Kliefoth, Münchmeyer und andere, die, romantisierend, aus dem öffentlichen Amt ein Gnadenmittel neben Wort und Sakrament machen, andererseits gegen Hase, Köstlin, Hölfling, Luthardt und andere, welche die göttliche Ordnung des öffentlichen Predigtamts in dem Sinne, als ob es ein göttliches Gebot habe, leugnen und behaupten, dass das Amt in concreto ohne ausdrücklichen göttlichen Befehl mit innerer Notwendigkeit aus der christlichen Gemeinde hervorgehe.\par Was die Lehren von der Bekehrung und Gnadenwahl betrifft, so ist nicht nur in amerikanischen, sondern auch in europäischen Zeitschriften und Schriften gegen Hönecke, resp. gegen die Missourisynode und andere Synoden innerhalb der Synodalkonferenz die Anfrage erhoben worden, dass sie „keinerzeit bereitwillig die bittere missourisch-calvinistische Pille verschluckten“. \footnote{633) So die Leipziger Allgemeine Ev.-Luth. Kirchenzeitung 1893, Nr. 2.} Ohne Bild ausgedrückt, lautet die Anklage dahin, dass Hönecke und die Synoden, deren einflussreicherer Theologe er war, ohne eigene, ja wider ihre eigene Überzeugung in dem Streit um die lutherische Lehre von der Bekehrung und Gnadenwahl sich auf die Seite der Missourisynode gestellt hätten. Mit der angeblich verschluckten „bitteren missourisch-calvinistischen Pille“ hat es diese Verwandtnis: Anfangs der siebziger Jahre des vorigen Jahrhunderts wurde aus der Iowasynode heraus und etwa acht Jahre später auch aus der Ohiosynode heraus gegen die Missourisynode die Anklage erhoben, dass sie in einen „grundstürzenden Irrtum“, nämlich in „Calvinismus“, gefallen sei. Was war denn vorgefallen? Innerhalb der Missourisynode war das geschehen, was zu allen Zeiten geschehen ist, wenn in der christlichen Kirche über die christliche Lehre ernstlich gehandelt und nachgedacht wurde. Bei den Lehrverhandlungen auf Synoden und Konferenzen war man gelegentlich auch auf die sogenannte \emph{crux theologorum} gekommen, nämlich auf die Tatsache, die die Konkordienformel so