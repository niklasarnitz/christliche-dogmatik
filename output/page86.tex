Menschen noch auch durch einen Engel vom Himmel verändern lassen will.\textsuperscript{254} Wenn dies Evangelium Gottes in der Kirche verkündigt und gelehrt wird, dann ist da, wie Luther sagt, „die tröstliche Verheißung der Gnade durchs Evangelium“.\textsuperscript{255} Durch das Wort des Evangeliums entsteht der christliche Glaube, am Wort des Evangeliums hat er sein Objekt, und im Wort des Evangeliums ergreift er die von Christo erworbene Vergebung der Sünden. \emph{Luther}: Haec est fides apprehensiva Christi, pro peccatis nostris morientis et pro iustitia nostra resurgentis. Hanc fidem Paulus praedicat, quam Spiritus Sanctus ad vocem evangelii in cordibus audientium donat et servat.\textsuperscript{256} Kurz, das christliche Erlebnis von Sünde und Gnade kommt zustande, nicht durch unmittelbare Wirkung Gottes, auch nicht durch Gottes Wirkung im Reich der Natur und in der Geschichte, sondern lediglich durch die Offenbarung Gottes in seinem Wort. Sofern wir uns, wir seien „Laien“ oder „Theologen“, von der heiligen Schrift als Gottes eigenem, an unsere Adresse gerichtetem Wort losmachen, machen wir uns auch vom christlichen „Erlebnis“ los. Man beruft sich auf Ereignisse im Reich der Natur und in der Geschichte, durch welche Gott gewalttätig in unser Leben eingreift. Nun, solche Ereignisse können hinzukommen, und sie kommen hinzu, um äußerlich des Menschen Aufmerksamkeit auf die Verkündigung des Wortes Christi zu richten. Aber das Erlebnis von Sünde und Gnade, wodurch ein Mensch ein Christ wird und ein Christ bleibt, wird nur durch das Lehren des göttlichen Wortes gewirkt, einerlei ob die Schrift ausdrücklich zitiert wird oder nicht. Ohne die Verkündigung des Wortes Christi bedeckt Finsternis das Erdreich und Dunkel die Völker, obwohl die Völker von „Geschichte“ umgeben sind und auch reichlich Gottes Hand in Erdbeben, Kriegen, Hungersnot usw. „erleben“.\textsuperscript{257} Daher hatte und hat die Kirche Christi Missionspflicht unter allen Völkern bis an die Enden der Erde und bis ans Ende der Tage, einerlei was und wieviel sich dort auf dem Gebiet der Geschichte und im Reich der Natur ereignet. Denn wie sollen sie glauben, von dem sie nichts gehört haben? Der Glaube kommt aus der Predigt, das Predigen aber durch das Wort Gottes.\textsuperscript{258}\par\vspace{1em}\par\begin{footnotesize}\noindent 254) Gal. 1, 7--9.\\255) Opp. v. a. IV, 486.\\256) Vgl. weitere Darlegungen über diesen Punkt II, 128 ff., auch II, Note 255; II, 536.\\257) Röm. 10, 14. 17.\\258) Röm. 10, 14. 17.\end{footnotesize}