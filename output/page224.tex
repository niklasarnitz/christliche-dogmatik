Wesen und Begriff der Theologie.\hspace*{\fill}213
\par
formuliert: „Einer wird verstockt, verblendet, in verkehrten Sinn gegeben; ein anderer, so wohl in gleicher Schuld, wird wiederum bekehrt.“ Diese Frage wurde in der apostolischen Kirche behandelt,\footnote{634) Röm. 11, 33--36. Überhaupt gehört der ganze Abschnitt Kap. 9--11 hierher.} dann dem Pelagianismus und Semipelagianismus gegenüber im vierten, fünften und sechsten Jahrhundert,\footnote{635) Schriftgemäße Stellung zur crux theologorum in den Canones von Arausio (Orange) 529. Bei Mault VIII, 712 ff. Ich habe die 25 Sätze abdrucken lassen in „Die Grunddifferenz in der Lehre von der Bekehrung und Gnadenwahl“, 1903, S. 34 ff. vgl. auch „Zur Einigung“ 2, 1913, S. 3 ff.} namentlich auch zur Zeit der Reformation, z. B. in der scharf zugespitzten Frage, warum Saul verworfen, David angenommen ward, Petrus umkehrt, Judas verloren geht.\footnote{636) Der Gegensatz zwischen Luther und dem späteren Melanchthon II, 583, zwischen Melanchthon und Chemniß II, 585, Note 1367.} Was innerhalb der Missourisynode gelehrt wurde (und noch gelehrt wird), lässt sich in drei Sätze zusammenfassen: 1. Wir kennen aus der Schrift ganz genau den Grund der Bekehrung: es ist Gottes Gnadenwirkung allein. 2. Wir kennen auch aus der Schrift ganz genau den Grund der Nichtbekehrung: es ist allein des Menschen Widerstreben gegen die Wirksamkeit des Heiligen Geistes (sola hominis culpa). 3. Weil aber die Gnade Gottes sowohl allgemein als ernstlich ist und alle Menschen in dem gleichen gänzlichen Verderben liegen, so bleibt es in diesem Leben für unser menschliches Begreifen ein Geheimnis, warum die einen bekehrt werden und die andern nicht. Die Lösung dieses Geheimnisses erwarten wir im ewigen Leben. Jeder Lösungsversuch in diesem Leben bringt uns in Widerspruch mit der Schrift. Das Geheimnis wäre leicht gelöst, wenn wir mit Erasmus und dem späteren Melanchthon irgend etwas im Menschen (aliquid in homine) als Grund oder Erklärungsgrund für die Bekehrung eines Menschen annehmen dürften (facultas se applicandi ad gratiam, verschiedenes Verhalten, Unterlassung des mutwilligen Widerstrebens usw.), oder wenn in der christlichen Kirche zu lehren erlaubt wäre, dass die Bekehrung nicht allein von Gottes Gnade, sondern auch vom Verhalten des Menschen abhänge. Über diese Lösung tritt in Widerspruch zu all den Schriftaussagen, die so klar die Entstehung des Glaubens der Gnaden- und Allmachtswirkung Gottes zuschreiben und dem Menschen nicht nur jede Neigung zum Evangelium absprechen, sondern ihm auch Feindschaft gegen das Evangelium zuschreiben. Das Geheimnis