81 Wesen und Begriff der Theologie.\par „nicht auf die Worte, sondern auf den Inhalt der Schrift kommt es an“ sind hinter dem sicheren Studiertisch entstanden. Vom Gesetz Gottes getroffene Gewissen kommen nur dadurch zur Ruhe, dass sie sich auf das nicht wankende Fundament stellen können, auf das die ganze christliche Kirche erbaut ist, nämlich auf das Wort der Apostel und Propheten, Eph. 2, 20, das Christi eigenes Wort ist, Joh. 17, 20. Lessings Seufzer: „Wer erlöst uns von dem unerträglichen Joche des Buchstabens!“ (nämlich des Schriftwortes) ist an der modernen Theologie vermittelt der Trennung des Inhalts der Schrift von dem Wort der Schrift in Erfüllung gegangen. Aber wir erinnern uns dabei der Tatsache, dass es Lessing nicht um die Wahrheiten, sondern um den Zweifel zu tun war, nach seinem viel zitierten Anspruch: „Wenn Gott in seiner rechten Hand alle Wahrheit hielte und in seiner linken Hand das stete Suchen nach Wahrheit, obgleich mit dem Zusatz, immer und ewig zu irren, so würde er (Lessing) die linke Hand wählen.“\footnote[267]{Bei Eduard König, Der Glaubenssatz, S. 63. Auch in Concise Dictionary of Religious Knowledge, by Jackson, sub “Lessing”.} Bei Lessing stand es so: „Die Begriffe Sünde und Erlösung sind für Lessing als ihn persönlich angehend nicht vorhanden, und darum ist auch für ihn eine übernatürliche Offenbarung [das Schriftwort] wertlos.“\footnote[268]{Bertheau in RG.2 VIII, 611.}\par Eine Selbsttäuschung liegt endlich auch vor in der Berufung auf die „geschichtliche Art“ des Christentums. Das Christentum hat freilich eminent geschichtliche Art an sich. Die Schrift bezeugt uns, dass der ewige Sohn Gottes Mensch geworden und damit in die „Geschichte“ eingegangen ist. Der Ewige ist zeitlich geworden. Die Schrift bezeugt uns auch, dass dieses wunderbare göttliche Geheimnis auf Befehl des ewigen Gottes durch die Propheten Schriften in menschlicher Sprache geoffenbart vorliegt\footnote[269]{Röm. 16, 25. 26. Daselbst geschah natürlich auch vor der schriftlichen Fixierung durch das mündlich verkündigte Wort Gottes. Gott hat sein geschichtliches Tun zur Erfassung der Menschheit von allem Anfang an durch ein geschichtliches Wort begleitet, damit die Menschen nicht auf eigene Anschauungen über die Erscheinung des Sohnes Gottes im Fleisch angewiesen wären.} und damit in „geschichtliche“ Erscheinung getreten ist. Aber zur geschichtlichen Erscheinung des Christentums in der Welt gehört nun auch die Tatsache, dass die „Heilsoffenbarung“ mit dem Wort Christi, das wir im Werk seiner Apostel haben, so völlig abgeschlossen ist, dass die nachfolgende „Geschichte“ daran nicht das geringste ändern kann.