\setcounter{footnote}{410}
\noindent 112 \hfill Wesen und Begriff der Theologie.

\bigskip

fragte, ob sie Christum homousion hielten, wie der Schrift Meinung in allen Worten ist.“\footnote{St. I. XVI. 2212. (Text I 25, 292.}

Was von Synoden und Konzilien und von allen großen und kleinen kirchlichen Versammlungen gilt, gilt natürlich auch von dem einzelnen Theologen und seiner Theologie und speziell von dem „Dogmatiker“ und seiner „Dogmatik“. Die Theologen und speziell auch die Dogmatiker sind nur insofern kirchlich, als sie prinzipiell nur die Schrift als Quelle und Norm der Theologie anerkennen und das praktische Resultat, die Lehre, nicht ein mixtum compositum von Schriftlehre und Menschengedanken, sondern theologia \textgreek{θεόπνευστος}, nur Wiedergabe in der der Schrift geoffenbart vorliegenden Lehre, ist. Alle bloß menschlichen Lehren, auch wenn sie die schriftgemäße Lehre darbieten, gebraucht der Theologe und speziell auch der Dogmatiker nicht als Quelle und Norm der Lehre, sondern nur als \textit{testes veritatis}, „als Zeugen, welcher Gestalt nach der Apostel Zeit und an welchen Orten folge Lehre der Propheten und Apostel erhalten worden ist“.\footnote{Konkordienformel. M. 568, 1.} Die wirklich lutherischen Dogmatiker beziehen dies auch auf die Symbole der lutherischen Kirche; denn sie bekennen sich „erstlich zu den prophetischen und apostolischen Schriften Altes und Neues Testaments als zu dem reinen, lauten Brunnen Israels, welche allein die einige, wahrhaftige Richtschnur ist, nach der alle Lehrer und Lehre zu richten und zu urteilen sind“.\footnote{Konkordienformel. M. 568, 3.} Nebenbei ist hier für die „alte Dogmatik“ ein gutes Wort einzulegen. Ziemlich allgemein wird behauptet, dass die alten lutherischen Dogmatiker ihre Lehre nicht aus der Schrift dargestellt, sondern die Schrift nur als „eine Sammlung von Beweisstellen“ für die fertig herübergenommene Kirchenlehre benützt hätten. Diese Behauptung ist geschichtlich unrichtig und beruht, wo sie bona fide auftritt, auf Unkenntnis der Sachlage. In der alten Dogmatik, wie sie z. B. durch Quenstedt repräsentiert wird, ist die christliche Lehre nicht bloß durch die Schrift bewiesen, sondern auch aus der Schrift dargestellt. Davon kann sich jeder überzeugen, der sich die Mühe gibt, in Quenstedts Systema Theologicum bei den einzelnen Lehren die \textgreek{θέσις} und \textgreek{βέβαιωσις} nachzulesen. Was für ein unvollziehbarer Gedanke uns in „dem Ganzen der Schrift“ zugemutet wird, womit die moderne Theologie pro domo kämpft und die Schrift als Quelle und Norm der Theologie beseitigt, ist später noch darzulegen.