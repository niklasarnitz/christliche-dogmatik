\hfill 121\n\nWesen und Begriff der Theologie.\n\nhat der christliche Theologe über alle Gegenstände, mit denen er es als Theologie zu tun hat, in der heiligen Schrift Gottes Wort, also Gottes eigene Beobachtung, Anschauung und Lehre, wobei natürlich (errare in Deum non cadit) jeder Irrtum und jede Unsicherheit schlechthin ausgeschlossen ist. Joh. 17, 17: „Dein Wort ist die Wahrheit“ und Joh. 10, 35: „Die Schrift kann nicht gebrochen werden.“\n\nDagegen wurde und wird eingewendet: Zugegeben, dass das, was die Schrift lehrt, Wahrheit oder objektiv gewiss ist, so bleibt noch immer die \emph{subjektive} Gewissheit in Frage, nämlich, ob auch der Mensch die in der Schrift geoffenbarte Lehre recht erkenne oder auffasse. Es wird etwa gesagt: „Die reine Objektivität derer, die sich lediglich an die heilige Schrift halten wollen, nämlich nicht nur als Norm, sondern auch als Quelle, ist nur Schein.“ Und dies wird damit begründet, dass der Schriftinhalt notwendig durch die subjektive Auffassung des Theologen hindurch müsse.\footnote{435) Bei Ritschl-Stephan, S. 11.} Auch ist zu sagen: Auch der Glaube, durch den der christliche Theologe, wie jeder andere Christ, Gottes eigene Lehre in der Schrift aufsucht oder erkennt, ist nicht eine von Menschen selbst erzeugte Meinung oder Anschauung (fides humana), sondern eine vom heiligen Geist durch das göttliche Wort selbst gewirkte Erkenntnis oder Überzeugung (fides divina), also ganz \emph{sicheres} Wissen oder \emph{sichere} Erkenntnis. Weil der christliche Glaube sein Entstehen und Bestehen nicht \textgreek{ἐν σοφίᾳ ἀνθρώπων}, sondern \textgreek{ἐν δυνάμει Θεοῦ} hat,\footnote{436) 1 Kor. 2, 5.} so wird er im Gegensatz zum Wissen der Welt auch als ein gewisses Wissen in der Schrift beschrieben: „Wir haben nicht empfangen den Geist der Welt, sondern den Geist aus Gott, \textgreek{ἵνα εἰδῶμεν τὰ ὑπὸ τοῦ θεοῦ χαριδῶνα ἡμῖν}.“\footnote{437) 1 Kor. 2, 12.} Doch dieser Gegenstand wird unter dem Abschnitt „Theologie und Gewissheit“ weiter behandelt.\n\nEs sei hier nur noch darauf hingewiesen, wie Luther die Gewissheit beschreibt, welche in der Theologie herrscht. Er sagt: „Der heilige Geist ist kein Skeptiker und hat nicht Zweifel oder Meinungen in unsere Herzen geschrieben, sondern Behauptungen, die gewisser und fester sind als das Leben selbst und alle Erfahrung.“\footnote{438) Opp. v. a. VII, 123 sq.: Spiritus Sanctus non est scepticus nec dubia aut opiniones in cordibus nostris scripsit, sed assertiones ipsa vita et omni experientia certiores et firmiores. St. G. XVIII, 1680.} Auch spätere lutherische Dogmatiker, die scientia als Geltungsbegriff für die Theologie ablehnen, geben zu, dass man die Theologie sehr wohl eine Wissenschaft nennen könne, wenn auf die in ihr herrschende Gewissheit im Gegensatz zu einer Meinung ge-