24 \hspace{\fill} Wesen und Begriff der Theologie.\par\noindent Zertrennung und Ärgernis anrichten neben der Lehre, die ihr gelernt habt, und weichet von denselbigen!$^{82}$) Sonderlich in der korinthischen Gemeinde gab es Leute, die sich als „Propheten“ und „geistlich“ vorfanden und trotzdem, ja gerade deshalb entschieden die göttliche Autorität des Apostelwortes in Frage stellten. Paulus sieht sich zu dem scharfen Wort veranlasst: „So sich jemand lässt dünken, er sei ein Prophet oder geistlich, der erkenne [\textgreek{ἐπιγινωσκέτω}], erkenne an], was ich euch schreibe, denn es sind des Herrn Gebote.$^{83}$) Und dass damit in der apostolischen Kirche sich mächtig der Versuch regte, an die Stelle der christlichen Gnadenlehre Werkslehre zu setzen, geht sowohl aus Paulus' Verwunderung über den Abfall der Galater hervor: „Mich wundert, dass ihr euch so bald abwenden lasset von dem, der euch berufen hat in die Gnade Christi, auf ein anderes Evangelium“, als auch aus Paulus' überaus heftiger Polemik gegen die Werkslehrer: „So auch wir oder ein Engel vom Himmel euch würde Evangelium predigen anders, denn das wir euch gepredigt haben, der sei verflucht!“ und: „Wollte Gott, dass sie auch ausgerottet würden, die euch verführen!“$^{84}$) Diese Bestrebungen, der Apostel Wort und damit auch die christliche Gnadenlehre des Christentums beiseitezusetzen, erklären uns das Entstehen und das Bestehen der Parteien in der christlichen Kirche bis auf diesen Tag.\par Um dies nachzuweisen, müssen wir die hauptsächlichsten bestehenden Kirchenparteien, also die römische Partei, die reformierten Gemeinschaften, die Parteien innerhalb der lutherischen Kirche, auch die modern-theologische Richtung, auf ihre Art prüfen. Dazu ist nötig, dass wir hier in der Einleitung kurze Auszüge aus längeren Darlegungen bringen, die sich durch die ganze Dogmatik hindurchziehen und dort der ausführlichsten Darlegung der einzelnen Lehren in These und Antithese dienen. Hier behandeln wir das, was in neueren Dogmatiken und Monographien etwa unter die Titel gebracht wird: „Romanismus und Protestantismus“, „Der lutherische und der reformierte Protestantismus“, „Fortbildung des Luthertums“, „Moderne Theologie des alten Glaubens“, „Irenische Theologie“ usw.\par Was die zahlreiche römische Partei betrifft, so erkennt sie in thesi zwar die göttliche Autorität der Schrift an, fordert aber, dass nicht die Schrift selbst zu Worte komme, sondern im Sinne der römischen Kirche, der sancta mater ecclesia, das ist, in letzter Instanz\par\vspace{1em}\noindent$^{82}$) Röm. 16, 17.\par$^{83}$) 1 Kor. 14, 37.\par$^{84}$) Gal. 1, 6—9; 5, 12.