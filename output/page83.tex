\noindent Wesen und Begriff der Theologie.\hfill 72\par\nZeugnis ab, dass die Lehre des Heiligen Geistes „Wahrheit“, das ist, durchaus wahr sei, wenn er innerlich durch die von ihm geoffenbarte und in der Schrift begriffene Lehre in den Menschenherzen wirkt, sie berührt, zieht und bewegt, so dass sie die Lehre als von Gott stammende oder als wahrhaft \emph{göttliche} Lehre glauben.“ Es ist also in der alten Dogmatik „die \emph{psychologische Vermittlung}“ völlig in Ordnung. Und – so müssen wir hinzufügen – eine andere „\emph{psychologische Vermittlung}“ der übernatürlichen Wahrheiten des Christentums gibt es überhaupt nicht, weil das Evangelium von Christo, dem Gekreuzigten, nicht nur in keines Menschen Herz gekommen ist,\footnote{244) 1 Kor. 2, 9; Röm. 16, 25.} sondern auch jedem natürlichen Menschen als ein Ärgernis und eine Torheit erscheint.\footnote{245) 1 Kor. 1, 23; 2, 14.} Die gesamte apologetische Tätigkeit, die uns zur Verfügung steht, ist nicht imstande, das menschliche Herz zu ändern und so dem Evangelium von Christo innerlich Anerkennung zu verschaffen. Auch diesen Punkt stellt der Apostel Paulus klar ins Licht. Er berichtet 1 Kor. 2 ausdrücklich, dass er sich des „\emph{psychologischen Anknüpfungspunktes}“ durch vernünftige Menschen Weisheit ($\varepsilon\nu \pi \varepsilon \iota \theta \text{o}\tilde{\iota}\text{s} \sigma \omicron \phi \acute{\iota}\alpha \text{s} \lambda \text{o}\acute{\gamma}\text{o}\iota\text{s}$) enthalten habe, damit der Glaube der Korinther nicht auf Menschenweisheit zu stehen komme, sondern in Beweis des Geistes und der Kraft seinen Bestand habe. Es liegt daher eine große Verirrung in der Behauptung, dass durch direktes Lehren der göttlichen Wahrheiten aus der Schrift eine bloße Verstandeserkenntnis, „\emph{Intellektualismus}“, erzeugt werde oder eine „\emph{mechanische Eingießung übernatürlicher Wahrheiten}“ stattfinde. Und, nebenbei bemerkt: welche innerliche Überhebung der Zeittheologie tritt uns hier entgegen! Wir dürfen wohl auf allgemeine Zustimmung rechnen, wenn wir sagen: kann das geschriebene Wort der Apostel und Propheten Christi sich in den Menschenherzen nicht Anerkennung verschaffen oder sich „\emph{psychologisch}“ vermitteln, so wird es noch viel weniger das Wort tun, das doch nur dem „\emph{Erlebnis}“ moderner Theologen entstammt. Bekanntlich ist auf diesen Punkt auch in den schmalkaldischen Artikeln hingewiesen, wenn es dort von den Enthusiasten in der kräftigen Sprache Luthers heißt: „Das ist alles der alte Teufel und alle Schlange, der Adam und Eva auch zu Enthusiasten machte, vom äußerlichen Wort Gottes auf Geisterei und Eigendünkel (\emph{proprias opiniones}) führte und tat's doch auch durch äußere Worte. Gleichwie auch unsere Enthusiasten das