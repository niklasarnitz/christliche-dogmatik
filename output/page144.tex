\noindent Wesen und Begriff der Theologie. \hfill 133\vspace{1em}Hat ihren Grund in Luthers Stehen auf der Schrift als Gottes eigenem Wort, also auf einem außer seinem Ich und außer der ganzen Welt gelegenen Standpunkt. Von diesem Wort, als außer ihm gelegen, sagt Luther:\begin{quote}„Es ist größer denn hunderttausend Welten, ja größer denn Himmel und Erde. Dasselbe Wort soll mein treuer Ruf und starker Bann sein, daran ich mich halten will, auf das ich es ertragen und ausstehen könne. Wo wir uns an den Baum nicht halten, so ist unsere Natur viel zu schwach, dass sie den grimmigen Haß und Neid der Welt ertragen und die listigen Anschläge und feurigen Pfeile des Teufels ausstehen könne.“\footnotemark[470]\end{quote}Und um noch einmal an Luthers Bemerkungen zu 2 Sam. 23 zu erinnern: Allein Gottes Wort macht gewiss, \textit{certum est active}; der Mensch aber wird durch Gottes Wort gewiss gemacht, er ist \textit{certus passive}; aber so gewiss, dass das Herz „trostlich und hochmütiglich alles verachtet und spottet, was zweifeln, zagen, böse und zornig sein will; denn es weiß, dass ihm Gottes Wort nicht lügen kann“.\footnotemark[471]Wie steht es hingegen mit der Gewissheit bei der Selbstgewissheitstheologie? Dass die Theologie der Selbstgewissheit tatsächlich eine Theologie der Selbstungewissheit ist, geht aus einer ganzen Reihe von Tatsachen hervor. Auf Ungewissheit weist deutlich hin der herrschende Indifferentismus in Bezug auf die christliche Lehre. Die Übereinstimmung in der Lehre wird geradezu als eine Abnormalität angesehen. „Reine Lehre“ wird mit einem „sogenannt“ eingeführt und zu einem Objekt des Spottes gemacht. Nun steht es doch so: Wer der Wahrheit gewiss ist, der ist in Bezug auf die Lehre nicht indifferent, sondern hält entschieden auf reine Lehre, wie denn auch diese Qualität der christlichen Lehre, dass sie „rein“, das ist, nicht durch menschliche Zusatzprodukte befleckt sei, durchweg in der Schrift Alten und Neuen Testaments gefordert ist.\footnotemark[472] Auf Ungewissheit weist ferner hin der herrschende Unionismus, der wenig Bedenken trägt, auch mit Geistern, die ganz offenbar nicht von Gott sind, kirchliche Gemeinschaft zu halten. Wer der göttlichen Wahrheit gewiss ist, der denkt und handelt nach der Ordnung Christi: „Sehet euch vor den falschen Propheten!“ und nach der Mahnung Pauli: „Weichet von denselbigen!“ und nach der Weisung, die „der Apostel der Liebe“ in Bezug auf\vspace{2em}\footnotetext[470]{St. V. XIII, 2621.}\footnotetext[471]{St. V. III, 1887.}\footnotetext[472]{Vgl. die ausführlichere Darlegung unter dem Abschnitt „Die nähere Beschreibung der Theologie, als Tatsache gehört.“}