\subsection*{Wesen und Begriff der Theologie. \hfill 60}\setcounter{footnote}{210}reden.\footnote{1 Petr. 4, 11. Luther, i. St.: \"Dass, dass er gewiss sei, dass er Gottes Wort und nicht sein eigen Wort rede.\" (XII, 443.)} Wer anders lehrt (heterodidaskalei) und nicht bei den gesunden Worten Christi, die wir in dem Wort seiner Apostel haben,\footnote{Joh. 8, 31. 32, vgl. mit Joh. 17, 20.} bleibt, der ist für das Lehramt in der christlichen Kirche disqualifiziert, weil er, anstatt die göttliche Wahrheit zu lehren, in eigener menschlicher Meinung aufgeblasen ist, tetuphotai, nichts weiß, meden epistamenos, sondern krank daniederliegt an Disputationen und Wortstreitigkeiten (noson peri zeteses kai logomachias).\footnote{1 Tim. 6, 4.} Daher werden, wie im Alten, so auch im Neuen Testament die Christen angewiesen,\footnote{2 Joh. 8---11; Röm. 16, 17.} allen Lehrern, die nicht Christi Lehre (ten didachen tou Christou), also Gottes Lehre, bringen, die christliche Gemeinschaft zu versagen, weil solche Lehrer durch eigene Lehre, die sie sich erlauben, Spaltungen und Ärgernisse in der Kirche anrichten (ereidiasias kai ta skandalon poiountes), die Christen um die Güter ihres Christentandes bringen (meden apolesete ha ergazametha), nicht das köstliche Werk eines christlichen Lehrers verrichten, sondern bösen Werken (tois ergois tois ponerois), obliegen. So entschieden ist in der Schrift gefordert, dass die Lehre, die in der christlichen Kirche verkündigt wird, Gottes eigene Lehre, \emph{doctrina divina}, sei.Wie gewalttätig in Übereinstimmung mit der Heiligen Schrift Luther aus \emph{doctrina divina} in der christlichen Kirche dringt, sehen wir schon aus seiner Bemerkung zu Jer. 23, 16, in der er die Theologen so anredet: \"Die Theologen, wo wollt ihr hier vorüber? Meint ihr, dass ein gering Ding sei, wenn die hohe Majestät verbeut, was nicht aus Gottes Munde gehet und etwas anders denn Gottes Wort ist?\"\footnote{Zitiert unter dem vorhergehenden Abschnitt, S. 52.} Aber Luther lag die Beschaffenheit der Lehre, die in der christlichen Kirche gelehrt werden soll, so am Herzen, dass er die necessitas \emph{doctrinae divinae} in ecclesia tractandae et audiendae unter mehrfachen Gesichtspunkten einschärft. So bei der Unterscheidung von Kirche und Staat. Im \"Weltlichen und Hausregiment\", sagt Luther, haben menschliche Ansichten und Menschenwort ihre Berechtigung, weil dies Gebiet dem \"natürlichen Licht\", das ist, der menschlichen Vernunft, unterstellt ist. Hingegen sagt er in Bezug auf das Lehren in der Kirche: \"Will jemand predigen, so schweige er seiner Worte.\" \"Allhier in der Kirche soll er nichts