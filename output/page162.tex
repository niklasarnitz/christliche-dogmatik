\leavevmode\llap{\textbf{151}}\hfill Wesen und Begriff der Theologie.\par\bigskipRegel und Richtschnur, nach welcher zugleich alle Lehren und Lehrer gerichtet und gewürdigt werden sollen, sind allein die prophetischen und apostolischen Schriften Alten und Neuen Testaments. ... In deren Schriften aber der alten und neuen Lehrer, wie sie Namen haben, sollen der heiligen Schrift nicht gleichgehalten, alle zumal miteinander denselben unterworfen und anders oder weiter nicht angenommen werden denn als Zeugen, welcher Gestalt nach der Apostelzeit und an welchen Orten solche Lehre der Apostel und Propheten erhalten worden“. \footnote{515) Epitome. M. 517, 1. 2.} Doch in der Konkordienformel auch im Artikel von der Person Christi und speziell von der communicatio idiomatum seine neue Lehre vorgetragen sei, wird dogmenhistorisch im Catalogus Testimoniorum nachgewiesen.\footnote{516) M. 731--760.}\parDass es keine Fortbildung der christlichen Lehre gibt, tritt schließlich auch darin zutage, dass alle Fortbildungsversuche, bei Licht besehen, sich als Umbildung und Zerstörung der christlichen Lehre erweisen. Und das ist nicht nur das Urteil Luthers und derer, die von der Ichtheologie als Repristinationstheologen registriert werden, sondern so haben auch solche landeskirchliche Theologen des 19. Jahrhunderts geurteilt, denen das Wesen an der christlichen Lehre am Herzen lag. So schrieb D. Winkel im Jahre 1862 im Vorwort zu seinem „Neuen Zeitblatt“: \footnote{517) Zitiert in S. W. 1875, S. 71 ff.} „Schwerlich ist noch eine Lehre übriggeblieben, welche nicht Umbildung, Zusätze und Ausmerzungen in erheblichem Maße erfahren hat. Man hebe von der Dreieinigkeit an, gehe weiter zu den Lehren von der Person und dem Werke Christi, vom Glauben und der Gerechtigkeit, von den Sakramenten und der Kirche bis zu den letzten Dingen, man wird kaum noch etwas in seiner alten Gestalt und in seinem vormaligen Werte finden. Nicht selten ist es dermaßen verändert, dass nur der alte Rahmen noch an das alte Bild erinnert, und bisweilen ist sogar der Rahmen als gar zu knapp und altfränkisch zerschlagen. Eine kleine Probe mag das anschaulich machen. Wenn Christus nach der Kirchenlehre auch in seiner Niedrigkeit wahrhaftiger Gott ist, so hat man ihn jetzt der göttlichen Eigenschaften entleert, ohne welche die Gottheit gar nicht gedacht werden kann, oder man lässt sich seine Gottheit allmählich bis zur Auferstehung in ihn hineinarbeiten. Der Tod Christi hat es sich gefallen lassen müssen, dass er nicht mehr zur Sühne an unserer Statt und zur Versöhnung mit Gott geschehen ist. Die Gerechtigkeit des Glaubens durch die Gerechtfertigung Gottes soll