Kirchengemeinschaft stehen.\footnote{621} Aus der Wisconsin-Synode haben wir ein großes dogmatisches Werk von D. Adolf Hönecke (\textdagger 1908).\footnote{622} Diese Dogmatik ist ein Beweis für die Tatsache, dass ein Dogmatiker selbständig arbeiten und dabei doch in der Lehre mit andern vollständig übereinstimmen kann. Auch Hönecke gelangt zu dem Resultat, dass die Lehre, welche die Lutherische Kirche in ihren Symbolen bekennt, die Lehre der Heiligen Schrift ist, und diese Lehre wird von ihm sowohl aus der Schrift dargestellt, als auch in scharfsinniger Polemik gegen die alten, die neueren und die neuesten Irrlehrer siegreich behauptet. Besonders ausführlich sind die Prolegomena gearbeitet, in denen der Verfasser sich auch eingehend mit der modernen Erlebnistheologie auseinandersetzt. Hönecke weist nach, dass diese Erlebnistheologie, die prinzipiell die christliche Lehre nicht aus der Schrift, sondern aus dem Innern des theologisierenden Subjekts schöpfen und normieren will, sowohl schriftwidrig ist, als auch an Selbstwidersprüchen leidet.\footnote{623} Wir heben einige Hauptpunkte aus Höneckes Dogmatik hervor, die den Verfasser als Dogmatiker charakterisieren. Hönecke lehrt: Schrift und Gottes Wort sind schlechthin zu identifizieren. „Wir verwerfen alle Ansichten, nach welchen nicht alles in der Bibel Gottes Wort sein soll, oder, was dasselbe sagt, nach welchen nicht alles in der Schrift von Gott eingegeben\ldots“\par\vspace*{\fill}\footnotesize\noindent\textsuperscript{621)} Das sind in Amerika die Vereinigte Synode von Wisconsin, Minnesota, Michigan u. a. St., die Slowakische Synode von Amerika und die Norwegische Synode, in Deutschland die Ev.-Luth. Freikirche von Sachsen u. a. St., in Australien die Ev.-Luth. Synode in Australien.\par\noindent\textsuperscript{622)} Der vollständige Titel lautet: Ev.-Luth. Dogmatik von D. theol. Adolf Hönecke, weiland Direktor und Professor am Seminar der Allgemeinen Ev.-Luth. Synode von Wisconsin, Minnesota, Michigan u. a. St. zu Milwaukee, Wis. Zum Druck bearbeitet von seinen Söhnen Walter und Otto Hönecke. 1909 ff. North-western Publishing House, Milwaukee, Wis. Vier Bände zu je etwa 460 Seiten Großoktav.\par\noindent\textsuperscript{623)} Auch Fränks „System der christlichen Gewissheit“ bespricht Hönecke ausführlich mit der Begründung: „Dies nähere Eingehen hat seinen Grund nicht etwa in der Bedeutung dieses Systems; denn tatsächlich bringt es nichts Neues, weder nach Prinzip (das Operieren aus dem Bewusstsein ist, wie wir bisher gesehen, etwas, das sonst zuvor da war nach noch Methode und Tendenz (Schrifttatsache, abgesehen von der Schrift, ist doch auch etwas Altes)), allein das Werk macht Aufsehen und kann Ansätzen, welch letzteres rein dogmatisch ist. Dazu gibt Fränk die Sach als positiver Theologe von echt lutherischem Standpunkt. Man begegnet in deutschen theologischen Schriften viel dem Ausdruck des Vertrauens zu ihm, und zwar als lutherischem Theologen. So fordert es das Interesse der lutherischen Kirche, dass wir näher auf dies Werk eingehen.“ (I, 120--150.)