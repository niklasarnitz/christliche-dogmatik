festgehalten werden. Adolf Harnack, der auf dem Wege der abgeblich „geschichtlichen“ Kritik Wesentlichem und Unwesentlichem in der christlichen Religion unterscheiden will, sagt von seinem Standpunkt aus ganz richtig: „Ich meine, nach einigen hundert Jahren wird man auch in den Gedankengebilden, die wir zurückgelassen haben, viel Widerspruchsvolles entdecken und wird sich wundern, dass wir uns dabei beruhigt haben. Man wird an dem, was wir für den Kern der Dinge hielten, noch manche harte und spröde Schale finden; man wird es nicht begreifen, dass wir so kurzsichtig sein konnten und das Wesentliche nicht rein zu erfassen und auszuscheiden vermochten.“ootnote{145) Wesen des Christentums \S 3, S. 35.} Nicht alle neueren Theologen stehen der göttlichen Autorität der Heiligen Schrift in dem Grade negativ gegenüber wie Harnack. Aber auch die Theologen, welche zwar als positiv klassifiziert werden, aber ebenfalls die Inspiration und damit die unfehlbare göttliche Autorität der Schrift preisgegeben haben, täuschen sich selbst, wenn sie meinen, dass sie bei ihrer kritischen Stellung zur Schrift das Christentum als absolute Religion festhalten können. Indem sie die christliche Religion anstatt aus ihrer göttlichen Quelle und der Schrift, aus dem „christlichen Ich“ oder dem „christlichen Glaubensbewusstsein“ oder dem „religiösen Erlebnis“ usw. schöpfen wollen, verlegen sie die christliche Religion auf das Gebiet der subjektiven menschlichen Meinung, und an die Stelle der „absoluten“ christlichen Religion tritt zugestandenermaßen, wie wir bereits hörten, „eine schier endlose Fülle von Verschiedenheiten“ der religiösen Auffassung. Daher haben die theologischen Lehrer unserer Zeit die Pflicht, die Studierenden mit großem Ernst vor allen neueren Theologen zu warnen, die die Heilige Schrift nicht als Gottes unfehlbares Wort gelten lassen. Kurz, wer das Christentum als die absolute Religion festhalten will, muss sowohl Christi stellvertretende Genugtuung als auch die heilige Schrift als Gottes Wort festhalten.

Es mag hier auch auf die Tatsache hingewiesen werden, dass nach der Lehre der Schrift die christliche Religion von allem Anfang an als „absolute“ aufgetreten ist. Der Einwand, dass die christliche Religion in die Reihe der geschichtlichen Erscheinungen gehöre, alles Geschichtliche aber immer nur relativen, nicht absoluten Charakter tragen könne, ist nicht stichhaltig. Der Einwand schließt ein petitio principii in sich. Er setzt als selbst-