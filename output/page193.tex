\begin{flushleft}\textbf{Wesen und Begriff der Theologie.}\hfill 182\end{flushleft}lichen Schützen, wenn er sagt: „Selbst wenn auch die Schrift ein einheitliches Lehrganzes geben wollte [wollte? die Schrift will „einheitlich“ Gottes Wort sein\footnote{„δοῦλος Κυρίου“, 2 Tim. 3, 16; „τὸ λόγμα τοῦ Θεοῦ“, Röm. 3, 2; „οὐ δύναται λυθῆναι“, Joh. 10, 35; Altes und Neues Testament „einheitlich“ des Heiligen Geistes Wort, 1 Petr. 1, 10—12.}] und es also für die Dogmatik möglich wäre, lediglich dieses Lehrganze nachzubilden, so könnte man sich doch um deswillen in der Dogmatik nicht dabei beruhigen, weil die Dogmatik grundsätzlich ja die Erkenntnis des Glaubens darstellen will.“ Gewiss, die Dogmatik will grundsätzlich die Erkenntnis des Glaubens darstellen, aber die Erkenntnis des Glaubens, der an Christi Wort, am Wort der Apostel und Propheten, am Schriftwort bleibt, der immer nur vis-a-vis des Schriftworts aussagt. Der Glaube ohne Schriftwort ist ein Glaube „in die Luft“, nicht ein Glaube, welcher spricht: „Rede, Herr, denn dein Knecht höret!“ sondern Unglaube, der nicht glaubt dem Zeugnis, das Gott gezeugt hat von seinem Sohne, und daher Gott in seinem Wort zum Lügner macht.\footnote{1 Joh. 5, 9. 10.}Vielleicht ist hiermit das Nötige über die äußere Gruppierung der christlichen Lehren innerhalb eines Corpus doctrinae gesagt. Im Anschluss hieran tun wir noch einen Rückblick auf die Art der modernen Theologie, sofern sie von ihrem Standpunkt aus als Anklägerin gegen alle Vertreter der Schrifttheologie auftritt.Die christliche Kirche unserer Zeit darf sich nicht verhehlen, dass sie an den modernen Theologen, die anstatt allein aus der Schrift aus dem eigenen Bewusstsein lehren wollen, eine feindliche Macht in der eigenen Mitte hat, die darauf aus ist, sie, die christliche Kirche, vom Schriftwort und damit von dem Fundament ihres Glaubens abzudrängen. Diese Feindschaft gegen das Schriftprinzip wird auch klar aus der Kritik erkannt, die von der modernen Theologie an den Personen und Schriften der alten Theologen, die das Schriftprinzip festgehalten haben, geübt wird. Diese Kritik wird auch ausgedehnt auf die Personen und Schriften solcher Theologen unserer Zeit, die ebenfalls an dem sola Scriptura festhalten. Sie werden, wie wir bereits in einem andern Zusammenhang sahen, als „Novristinationstheologen“ registriert. Unter diese Kritik fällt auch die Kirchengemeinschaft, von welcher aus diese Dogmatik geschrieben ist, die Missourisynode. Andere Synoden hierzulande und in andern Ländern, die mit der Missourisynode in Kirchengemeinschaft stehen,