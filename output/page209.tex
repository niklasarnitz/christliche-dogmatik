\subsection*{Wesen und Begriff der Theologie.\hfill 198}
Theologie aus Deutschland stammt, und zwar aus Deutschlands bester Zeit im vorigen Jahrhundert. Nach den Freiheitskämpfen gegen die französische Weltherrschaft ging durch Deutschland eine bedeutende religiöse Erweckung. Sie ging vornehmlich von Laien-kreisen aus, erstreckte sich aber auch auf einen Teil der studierenden Jugend. Die Mehrzahl der Väter der Missourisynode gehörte auf der Universität Leipzig zu dem Häuflein gläubiger Studenten, das also beschrieben wird: „Sie versammelten sich an gewissen Lagen jeder Woche zu gemeinsamem Gebet, zu gemeinsamer Lesung der heiligen Schrift, zum Zweck der Erbauung und zu gegenseitigem Aus-tausch über das eine, das not ist.“\footnote{615) Hochstetter, Geschichte der Missourisynode, S. 65.} Auch Franz Delitzsch ($\dagger$ 1890) gehörte diesem Studentenkreise an. A. Köhler sagt in der herzogschen Realenzyklopädie, dass Delitzsch nach seiner Bekehrung vom Nationalismus zum Christentum das Studium der Theologie gemeinsam betrieb „mit seinen gleichgesinnten Freunden, welche später größten-teils die Begründer der streng konfessionellen Richtung in der lutherischen Kirche Nordamerikas wurden.“\footnote{616) RE$^3$ IV, 566.} In diesem Kreise war, wie weiter berichtet wird,\footnote{617) Hochstetter, a. a. O., S. 66.} anfänglich von dem Lehrunterschied zwischen den verschiedenen Kirchen keine Rede. Aber mit dem Wachstum in der Erkenntnis entstand nach einiger Zeit auch die Frage: Welches Glaubens seid ihr? Seid ihr lutherisch oder reformiert oder uniert? Die Folge hiervon war eine Richtung, aber die meisten von ihnen erkannten bald, dass kein anderer als der lutherische Glaube es sei, den der heilige Geist in den fleißigen und heilsbegierigen Schriftforschern als den wahren, in Not und An-fechtung allem Selbstlebenden verriegelt hatte, noch ehe sie wussten, welcher Kirche Glaube es sei. Nun hat Delitzsch das, was er nach seinem Übergang „aus der Schule Spinozas und Fichtes in die Schule Christi“\footnote{618) So beschreibt Delitzsch selbst seine Bekehrung zum Christentum im Vor-wort zu einer Schrift „Vom Hause Gottes oder Kirche“, Dresden 1849.} als christliche Wahrheit erkannt hatte, später zum großen Teil wieder vergessen. Der spätere Delitzsch ist ein Beispiel für die unvermeidliche Degeneration der Theologie, wenn sie von ihrer einzigen Quelle und Norm abrückt und unter dem blendenden Schein der „Wissenschaft“ sich der unwissenschaftlichen \emph{μετάβασις εἰς ἄλλο γένος} schuldig macht. Aber auch der spätere Delitzsch hat die Zeit seines Lebens, die er gemeinsam mit den Begründern „der streng konfessionellen Richtung in der lutherischen Kirche Nord-