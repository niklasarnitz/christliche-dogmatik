Erstlich zu, dass dem frommen Ich oder dem religiösen Erlebnis auch nicht ganz zu trauen sei, sondern vielmehr die Möglichkeit der Selbsttäuschung vorliege. Ferner wird zugestanden, dass bei der Methode sich eine schier unendliche Fülle von Verschiedenheiten ereignet habe und das Füllhorn der Verschiedenheiten sicherlich noch nicht leer sei. Aber noch mehr. Es wird weiterhin zugestanden, teils, dass die heilige Schrift als eine 	extquotedblleft authentische Urkunde	extquotedblright der geschichtlichen Gottesoffenbarung zu werten sei, teils, dass die Zeit, in der die heiligen Schriften geschrieben wurden, der Gottesoffenbarung immerhin 	extquotedblleft näher	extquotedblright gestanden habe als wir, die Spätgeborenen, im 19. und 20. Jahrhundert. Bei diesen Zugeständnissen legt sich entschieden der Gedanke nahe, dass es zum Zweck der Gewinnung einer 	extquotedblleft Einheit	extquotedblright doch wohl vernünftiger und sicherer wäre, bei der Schrift zu bleiben, als sich in das fromme Selbstbewusstsein des theologisierenden Subjekts zurückzuziehen, das sich möglicherweise selbst täuscht und das tatsächlich eine schier endlose Fülle von Verschiedenheiten aus sich herausgesagt hat.


	extbf{3.} Wollen wir in die Selbsttäuschung und die Täuschung anderer, die mit dieser menschlichen Systembildung in der Theologie verbunden ist, nicht hineingezogen werden, so müssen wir durchaus festhalten, dass wir uns auf dem Gebiet der Theologie völlig auf dem Gebiet der gegebenen und feststehenden Tatsachen befinden, an denen kein menschliches Denken und also auch keine menschliche Systembildung auch nur das Geringste ändern kann. Mit Recht ist auf die Analogie hingewiesen worden, die in dieser Beziehung zwischen der Naturwissenschaft und der Theologie besteht. Dafür treten Christus und seine Apostel mit ihrer Autorität ein, und zwar sowohl hinsichtlich des Alten Testaments als auch hinsichtlich des Neuen Testaments.ootnote{559) Hiob. 10, 35; 2 Tim. 3, 16; 1 Petr. 1, 10. 11; 2 Petr. 1, 21; Joh. 17, 14. 17; 8, 31. 32.} Die Naturwissenschaft findet die tatsächlich auf dem Gebiet der Natur gegebenen Dinge das 	extquotedblleft Wahrnehmungsgebiet	extquotedblright oder die 	extquotedblleft Wahrnehmungswelt	extquotedblright, wie es passend ausgedrückt worden ist. Alles menschliche Wissen von natürlichen Dingen reicht immer nur so weit, als die Beobachtung und Erfahrung der vorliegenden Tatsachen reicht. Wenn der Naturforscher Pflanzen oder Tiere systematisch zusammenordnen will, so tut er das nicht, solange er wirklich wissenschaftlich verfährt, nach seinem von der Außenwelt unabhängigen Ich, das ist, nicht nach seinen eigenen Gedanken, wie die Pflanzen und Tiere beschaffen sein sollten, sondern er fasst die Zusammenordnung gänzlich von der gegebenen Beschaffenheit der außer ihm gelegenen Objekte ab. Mit anderen Worten: Der wissenschaftlich arbeitende Naturforscher macht nicht oder konstruiert nicht sein System, sondern nimmt Notiz davon und er erkennt es an, wo es und