\setcounter{page}{5} \markright{Wesen und Begriff der Theologie.} Wesen und Begriff der Theologie. \par sicherer Studiertisch oder Katheder aus gegen die satisfactio vicaria redet, das persönliche Christentum ab. Auch Luther weist auf eine mögliche „glückliche Inkonsequenz“ hin, wenn er von den Theologen, die Erasmus gegen ihn ins Feld führte, sagt, dass sie anders inter disputandum geredet haben, als ihr Herz vor Gott stand.\footnote{9) Opp. v. a. VII, 166. Ep. Z. XVII, 1730.} Aber konsequenterweise besteht ein Zusammenhang zwischen der Leugnung der stellvertretenden Genugtuung Christi und der Ablehnung der Schrift als des Wortes Christi, wie auch konsequenterweise ein Kausalnexus besteht zwischen der Erkenntnis Christi als des Sündeheißlandes und der Erkenntnis der Schrift als des Wortes Christi. Dies ist bei der Lehre von der Schrift weiter auszuführen. \par Eine weitere böse Folge der Ichtheologie ist die Lehrer-irrung, die überall dort eingetreten ist, wo es der modernen Theologie gelang, die Kirche von ihrem Lehrfundament, dem Wort der Apostel und Propheten (Eph. 2, 20), abzurütteln und auf die Ich-basis zu stellen. Zwar ist die Meinung geäußert worden, dass auch nach Preisgabe der göttlichen Autorität der Schrift eine Einheit in der Lehre möglich sei. Die „rein subjektiven“, „Einfälle“ würden sich als solche verraten und von dem „kirchlichen Gemeingeist“ abgestoßen werden. Aber in derselben Schrift wird referiert, dass die weitgehende Übereinstimmung in den neuen dogmatischen Grundsätzen verbunden sei „mit einer schier unendlichen Fülle von Verschiedenheiten in der Anwendung dieser Grundsätze, wie sie bald mehr durch die religiöse Individualität des Dogmatikers, bald mehr durch den Grad seiner wissenschaftlichen Konsequenz verursacht wird“.\footnote{10) Ritsch: Stephan, Dogmatik, S. 13 und IX.} Diesem offenkundigen und zugestandenen Auseinanderwerfen in der Lehre, das sich im modern-theologischen Lager findet, kann nur auf eine Weise gewehrt werden: Die Theologen müssen die Ichbasis verlassen und sich wieder auf das Fundament stellen, auf dem die ganze christliche Kirche erbaut ist, nämlich auf das Wort der Apostel und Propheten Christi, auf Christi Wort, auf die heilige Schrift. Darum gestaltet sich unsre Lehren, die wir uns christliche Theologen nennen, so, wie es Luther beschreibt: „Das mögen wir tun, soferne wir auch heilig sind und den heiligen Geist haben, dass wir Katechumenen und Schüler der Propheten uns rühmen, als die wir nachsagen und predigen, was wir von den Propheten und Aposteln gehört und gelernt, und auch gewiss sind, dass es die Propheten gelehrt haben. Das heißen