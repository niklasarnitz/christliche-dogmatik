115\qquad Wesen und Begriff der Theologie.\n\nPredigt besteht fast aus nichts als Fragen und Exklamationen, Selig-\npreisungen und Weherufen, Aufforderungen zur Prüfung und Be-\narbeitungen des Gewissens und Geistes, so dass der Zuhörer, immer\nim Gemüt und Gewissen angefasst, zu gar keiner ruhigen Überlegung\nkommen kann. Weit entfernt aber, dass solches Predigen besonders\nzu Herzen gehen und wahres Leben wirken sollte, so ist es vielmehr\ndazu angetan, die Leute tot zu predigen, den etwa vorhandenen\nHunger nach dem Brot des Lebens zu ertöten und methodisch Über-\ndruß und Ekel an Gottes Wort zu wirken. Es muss notwendig jedem\nZuhörer widerlich werden, wenn er immer und immer, ohne dass zu-\nvor der Grund durch Lehre gelegt ist, sich ermahnt oder gestraft oder\nauch salzlös getröstet sieht. Es ist freilich leichter, dies aus dem Steig-\nreif so zu tun, dass die Predigt doch den Anschein hat, lebendig und\nkräftig zu sein, als eine Lehre deutlich und gründlich darzulegen. Und\ndass jenes leichter ist, mag wohl bei manchen die Hauptursache sein,\ndass sie so wenig Lehre predigen, dass sie meist selbst schon solche\nThemata wählen, die die Kenntnis der Sache bei den Zuhörern schon\nvoraussetzen und daher schon nur praktische Anwendung des Gegen-\nstandes versprechen. Bei vielen liegt aber der Grund hiervon ohne\nZweifel auch darin, dass sie, weil sie selbst keine gründliche\nKenntnis der geoffenbarten Lehren haben, dieselben natürlich auch\nandern nicht gründlich darlegen können. Noch andere aber mögen\nendlich wohl auch darum so wenig Lehre in ihren Predigten treiben,\nweil sie in dem Wahne stehen, ausführliche Lehrdarstellungen seien\nzu trocken, ließen die Zuhörer kalt, dienten nicht zur Erweckung,\nBekehrung und einem wahren lebendigen und tätigen Herzenschristen-\ntum. Es ist dies aber ein großer Irrtum. Gerade die in der Schrift\nuns Menschen zur Seligkeit geoffenbarten ewigen Gedanken des\nHerzens Gottes, gerade diese von der Welt her verschwiegen ge-\nwesen, aber durch der Propheten und Apostel Schriften uns kund-\ngemachten Wahrheiten, Ratslüsse und Glaubensgeheimnisse sind der\nhimmlische Same, der in die Herzen der Zuhörer gesenkt werden\nmuss, soll in denselben eine Frucht einer wahren Buße, eines ungefärb-\nten Glaubens und einer aufrichtigen, tätigen Liebe hervorwachsen.\nWahres Wachstum einer Gemeinde in christlichem Wesen ist ohne an-\ngründlicher Lehre reine Predigten nicht möglich. Wer es daran\nfehlen lässt, ist in seinem Amte nicht treu, mag er immerhin durch\nsein stetes eifriges Ermahnen, ernstes Strafen oder sonderlich evan-\ngelisch sein wollendes Trösten den Anschein haben, als ob er sich in\ntreuer Sorge für die ihm anvertrauten Seelen verzehrte.