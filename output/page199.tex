\section*{Wesen und Begriff der Theologie.}\hfill 188zieren sei. Walther ist sowohl in den Vereinigten Staaten als auch in Deutschland und andern Ländern sowohl schwer getadelt als auch, wenn auch mit Einschränkung, gelobt worden.\footnote{609) \textsc{R.G.S.} IX, 85; XVIII, 687 ff.} Sein Name erscheint bis auf diese Zeit in dogmatischen Lehrbüchern, und zwar auch zu dem Zweck, um zugleich die theologische Art und die kirchliche Stellung der Missourisynode zu charakterisieren. Sollen wir unser eigenes Urteil über Walther zusammenfassen, so möchten wir ihn den Apologeten der Schrifttheologie Luthers und der Schrifttheologie der altprotestantischen Dogmatiker nennen, soweit die letzteren sich als genuine Vertreter der Schrifttheologie Luthers erwiesen haben. Damit wird Walther zugleich der Apologet der Theologen der Gegenwart, die unter die Benennung „Repristinationstheologen“ gebracht werden, weil sie, was die Schriftmethode betrifft, in den Bahnen der altlutherischen Theologen wandeln. Die Anschauungen, welche von der modernen Theologie gegen Luther und die altprotestantischen Dogmatiker bis auf Hollaz erhoben werden, lauten, wie wir sahen, auf Mangel an wissenschaftlichem Sinn in der Erfassung der theologischen Probleme, auf mechanischen Schriftgebrauch durch direkte Berufung auf das „Es steht geschrieben“, auf Verbreitung von Verstandeschristentum („Intellektualismus“), auf prüfungslose Herübernahme traditioneller Lehren usw. Alle diese Anschauungen werden von Walther im einzelnen aufgenommen und als sachlich falsche Anklagen erwiesen. Und das geschieht nicht nur theoretisch, sondern auch praktisch. Walther bekennt frei und offen: Wir amerikanischen Lutheraner wandeln in den von euch bekämpften Bahnen Luthers und der alten Theologen. Aber kommt zu uns und prüft das bei uns vor Augen liegende Resultat. Geht durch unsere Gemeinden, besucht unsere Pastoralkonferenzen und unsere Synodalversammlungen und fragt euch dann, ob das, was ihr gehört und gesehen habt, wirklich den Charakter eines bloßen Verstandeschristentums und einer bloßen Repristinationstheologie trage. Der großen Schar der Ankläger gegenüber sollte dem Verteidiger in Bezug auf die Hauptanklagepunkte das Wort gegeben werden. Walther hat diese Punkte ausführlich im Vorwort zu „Lehre und Wehre“ vom Jahre 1875 behandelt. Was die Stellung der christlichen Kirche zur Wissenschaft betrifft, so legt Walther dar, welche Wissenschaft sie eifrig zu pflegen hat und welcher Wissenschaft sie nach Gottes Willen die Türe verschließen muss. Die Wissenschaft, welche die Kirche zu pflegen