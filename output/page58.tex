47 Wesen und Begriff der Theologie.\par 4. Die Kenntnis und Lehre von einzelnen Teilen der christlichen Religion, nämlich von der Gottheit Christi und im Zusammenhang damit von der Trinität.\footnotemark[170] Dieser Sprachgebrauch ist bis auf unsere Zeit ziemlich allgemein festgehalten worden. Auch wir nennen die Lehre von der Gottheit Christi und der Trinität „Theologie im engeren Sinne", im Unterschied von Kosmologie, Anthropologie, Christologie, Ekklesiologie usw.\par In den angeführten Bedeutungen werden „Theologie" und „Theologe" in einem Sinne gebraucht, der der Sache nach in der Schrift enthalten ist. Hingegen liegt ein schriftwidriger Sprachgebrauch vor, wenn Theologie eine Kenntnis und Lehre von Gott und göttlichen Dingen bezeichnen soll, die angeblich über den Glauben an das Schriftwort hinausgeht oder den Glauben zum Wissen erheben will. Dies ist das \foreignlanguage{greek}{μεσον ψευδος} der neueren Theologie in ihren verschiedenen Schattierungen. Wir müssen der \par iisque assentieren. Nach 1 Petr. 3, 15; Kol. 3, 16 usw. hätte Gerhard hinzufügen können, was er auch an anderen Stellen sagt (l. \emph{De Minist. Eccl.}, § 87): atque eos \emph{[articulos fidei]} docent et profitentur. Natürlich gibt es auch unter den sogenannten Laien wieder Unterschiede in der Gotteserkenntnis und der Lehretätigkeit. In unserer Synode ist der Ausdruck „Laientheologen" im Gebrauch. Wir verstehen darunter solche Laien, die unserer Ministerialen Kenntnis der christlichen Lehre und ihren Eifer für kirchliche Angelegenheiten über das Durchschnitts-maß hinauszeigen. Solche Laientheologen gab es in der apostolischen Kirche, wie wir aus der Urkirche Röm. 16 annehmen dürfen. In der großen Majorität der Fälle sind die „Laientheologen" der Kirche zum Segen geworden. Vgl. L. u. W. 1860, S. 352, über die unberechtigte Furcht vor „Laien auf den Synoden".\par \setcounter{footnote}{170}\footnote{171} So erhebt Gregor von Nazianz (f. um 390) den Beinamen \foreignlanguage{greek}{θεολογος} nach Veröffentlichung seiner Rede zur Verteidigung der Gottheit Christi. Gewiss ist auch, dass der Kirchenvater den Titel Theologe sogar dem Evangelisten Johannes beilegt, weil dieser mit besonderem Nachdruck die ewige, wesentliche Gottheit Christi lehrt. So spricht Athanasius (im vierten Jahrhundert) von dem Evangelium Johannis: \foreignlanguage{greek}{ἢ φημι καὶ ὁ θεολόγος, φημί· ἐν ἀρχῆ ἦν ὁ λόγος}. Die Theologie als Lehre von der göttlichen Natur Christi unterscheidet dann die Kirchenväter von der Ökonomie (\foreignlanguage{greek}{οἰκονομια}, \emph{dispensatio}) als der Lehre von Christo in seiner Menschwerdung. So sagt Gregor von Nazianz: \foreignlanguage{greek}{Ἄλλος ἐστὶ λόγος τῆς θεολογίας ἢ τῆς οἰκονομίας, ἄλλος τῆς οἰκονομίας}. Mit diesem speziellen Sinn von Theologie hängt es zusammen, dass das Verbum \emph{Theologein}, theologeieren, getroffen im Sinne von „als Gott bekennen" gebraucht wird. Walther zitiert (Q. L. W. 1868, Z. 7) die Worte von \emph{Matth}. 13, 52 f. (\emph{De S. Trin}. ed. ä. \emph{Opp}. ed. Bonifius II, 190 sq.): \foreignlanguage{greek}{Πῶς γὰρ δύνῃ θεολογεῖν τὸ πνεῦμα, ὁ μὴ φίλων εἰπεῖν τὴν αὐτὴν οὐσίαν καὶ δόξαν καὶ βούλην καὶ δύναμιν πατρὸς καὶ υἱοῦ}; Zur Bezeichnung der Lehre vom Geheimnis der Trinität gebraucht Basilius „Theologie" in den Worten: Quomodo non erit necessarium silere, ne theologiae dignitas verborum penuria et tenuitate periclitari videatur. (\emph{Sermo de fide et trinitate}. Opp. I, 371. L. u. W. 1868, S. 8.)\par \footnote{172} So bemerkt Gerhard (l. \emph{De Natura Theologiae}, § 1, 3 f.): „Theologia ist nicht primär eine spekulativ-wissbare Lehre, sondern ein praktisches Wissen, das der Seligkeit dient und mit allen christlichen Lehrartikeln verbunden ist."