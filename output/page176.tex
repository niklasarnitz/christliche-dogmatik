wurde, an die Tatsache, dass die fragmentarische Erkenntnis der geistlichen Dinge der für die Zeit dieses Lebens normale Zustand ist. --- Und noch auf einen Punkt sollte in diesem Zusammenhang hingewiesen werden. Nach die modernen Theologen, die die heilige Schrift als Quelle und Norm der Theologie aufgegeben und sich auf ihr Ich zurückgezogen haben, sollten selbst von ihrem Standpunkt aus die Systembildung oder die Konstruktion eines einheitlichen Ganzen als unmöglich erkennen. Wer ein System im Sinne eines einheitlichen Ganzen konstruieren will, muss das Objekt, mit dem er es zu tun hat, durch und durch kennen oder mit seiner Erkenntnis vollkommen umspannen. Das Objekt der Theologie aber ist, wie doch auch die moderne Theologie zugibt, Gott, der Gott, den auch die natürliche Vernunft noch als den unendlichen und unbegreiflichen Gott erkennt. Darum sollte auch vom Standpunkt der Ichtheologie aus die einheitliche Systembildung als ein titanisches Unternehmen erkannt und --- unterlassen werden.\par 4. Beachtenswertes über die Frage, in welchem Sinne und in welchem Sinn nicht die Theologie ein „System“ genannt werden könne, hat seinerzeit Kliesoth in der Beurteilung der Hofmannschen Systembildung gesagt. Kliesoth schrieb: \footnote{561} „Hofmann, ausgehend von der zur einfachsten Selbstaussage gebrachten Tatsache seines Christentums, aus derselben in unverbrüchlicher Notwendigkeit und unter völligem Absehen nicht allein von der Kirchenlehre, sondern auch von der Schrift das Ganze christlicher Lehre herleiten; oder mit andern Worten: besagte Tatsache, indem sie sich selbst zur Aussage bringt, sich selbst entfalten lassen zum Ganzen christlicher Lehre, und zwar in unverbrüchlicher Notwendigkeit, und ohne dass von anderswoher etwas aufgenommen werde. So habe ich v. Hofmanns Intentionen verstanden und habe, dass ich ihn darin richtig verstanden, an seinen eigenen Äußerungen nachgewiesen. Nun gibt es zwei Weisen der systematischen Behandlung. Die erste geht bloß darauf aus, ihren Stoff seiner Natur gemäß zusammenzuordnen. Sie ist die auf empirische [tatsächlich gegebene oder vorliegende] Stoffe anwendbar. Daher besteht ihr Charakteristisches in zweierlei: Erstens muss sie immer vis-a-vis dieses ihres empirischen\footnote{561) In der kirchl. Zeitschr., herausgegeben von Kliesoth und Meier, Jahrg. 6. Im Separatabdruck erschienen unter dem Titel „Der Schriftbeweis des D. von Hofmann“. Schwerin 1859, S. 174 ff.}