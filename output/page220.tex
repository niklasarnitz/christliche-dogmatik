Bekenntnisses, sondern des Evangeliums selbst sei. Sein Einfluß trug in hervorragender Weise dazu bei, dass die Wisconsinsynode ihre Verbindung mit dem Generalkonzil löste, und machte sich auch kräftig geltend, als im Jahre 1868 Verhandlungen mit der Missourisynode gepflogen wurden, die mit der gegenseitigen Anerkennung der beiden Körperschaften abschlossen. Darum gehört er auch mit zu denen, die als Gründer der Synodalkonferenz, welche im Jahre 1872 entstand, in gutem Gedächtnis gehalten zu werden verdienen.\footnotemark[628] Wir finden keine Lehrdifferenz zwischen Walther und Hönecke, worin wir mit Recht abermals ein Zeugnis für die einigende Kraft des Wortes Gottes sehen. Wie die Väter der Missourisynode untereinander, so waren auch Walther und Hönecke stark ausgeprägte und verschiedene Charaktere. Auch kamen sie aus verschiedenen kirchlichen Verhältnissen. Hönecke studierte Theologie in Halle unter Hupfeldt, Julius Müller und Tholuck, also zu einer Zeit, als in Halle die Herrschaft des Rationalismus bereits gebrochen war. Hönecke sagt in seiner Dogmatik\footnotemark[629] von Tholuck: „Tholuck hat viele von seinen Schülern auf den Weg des Lebens gewiesen, wurde ihnen aber gram, wenn sie strenger konfessionelle Bahnen einschlugen wie der Verfasser dieser Dogmatik, der ihn aber doch allzeit als einen Mann verehrt hat, dem er viel zu danken hatte.” Hönecke kam im Jahre 1863, also vierundzwanzig Jahre nach der Einwanderung der Sachsen, als Gesandter der Berliner Missionsgesellschaft nach den Vereinigten Staaten, um hier in der kirchlichen Versorgung der eingewanderten Deutschen tätig zu sein. Er wurde Glied der Wisconsinsynode und bald ihr theologischer Führer, wie bereits mit Prof. Schallers Worten oben berichtet wurde. Es wird hier wohl passend darauf hingewiesen, dass Hönecke in seiner Dogmatik\footnotemark[630} auch Walther als Theologen beschreibt. In dieser Schilderung Walthers beigeschreibt Hönecke zugleich sich selbst als Theologen, wie aus den beigefügten Urteilen hervorgeht. Er sagt über Walther: „Karl F. W. Walther war Schrifftheologe. Was der Nitzschianer Rattenbusch (Von Schleiermacher zu Nitzsch, S. 3) als Schwäche Walthers hinstellt, dass er für die Dogmatik wieder die Parole: Nur loci ausgegeben habe, da es die Signatur der Offenbarung sei, dass wir nur unzusammenhängende Stücke aus Gottes Geheimnissen erfahren, das muss Walther als Lob angerechnet\footnotetext[628]{}\footnotetext[629]{Im Vorwort zu Hönekes Dogmatik I.}\footnotetext[630]{I, 306.}