\noindent\makebox[\textwidth]
\bigskip
Der „glücklichen Inkonsequenz“ tritt die „unglückliche Konsequenz“. So ist Reformierten die Verwerfung der Taufe und des Abendmahls als Gnadenmittel dahin geraten, dass sie konsequent auch das äußere Wort des Evangeliums als Gnadenmittel verwarfen, sich auf eine eingebildete „unmittelbare innere Erleuchtung“ zurückzogen und dann dem „vollendeten Nationalismus“ anheimfielen. Andere Reformierte hat die Leugnung der Möglichkeit der Mitteilung der Eigenschaften in der Person Christi zur Leugnung der Menschwerdung des Sohnes Gottes, nämlich in den Sozinianismus, geführt.\footnote{374) Vgl. die weitere Darlegung II, 302 f., und das Beispiel Adam Neusers.} Innerhalb der lutherischen Kirche hat die schwankende Stellung des späteren Melanchthon auch darin ihren Grund, dass Melanchthon meinte zur Rettung der allgemeinen Gnade den Synergismus (das „verschiedene Verhalten“) in die christliche Heilsordnung einfügen zu müssen. Von diesem Irrtum aus wurde ihm der klare Blick für die christliche Wahrheit überhaupt so getrübt, dass er an dem Leipziger Interim mitarbeitete, an einem Schriftstück, das G. Plitt so charakterisiert: „Eine rechte Verhöhnung, ja eine Verleugnung der Reformation und der evangelischen Kirche. Tief verstimmt lehrte Melanchthon nach Wittenberg zurück.“\footnote{375) M. e. VI, 777.}

\medskip
\centering\textbf{Nichtfundamentale Lehren.}
\medskip

„Nichtfundamental“ im Unterschied von „fundamental“ werden passend solche Schriftlehren genannt, die zwar in der Schrift stehen, aber für den Glauben nicht Fundament oder Objekt sind, insofern der Glaube Vergebung der Sünden erlangt und zu Kindern Gottes macht. Es sind Lehren, in denen der Glaube derer, die Vergebung der Sünden bereits erlangt haben oder bereits Kinder Gottes geworden sind, in Erkenntnis sich betätigt und nach Gottes Willen auch betätigen soll. Solche Lehren sind z. B. die Lehren vom Antichrist und von den Engeln. Dass die Lehre vom Antichrist nicht zum Fundament der fides salvifica gehöre, wurde bereits dargelegt. Dasselbe ist in Bezug auf die Lehre von den Engeln zu sagen. Der Glaube, welcher die Vergebung der Sünden ergreift, ist nicht Glaube an die Engel, sondern lediglich Glaube an Christum. Dem Glauben an Christum kommt die rechte Erkenntnis auch dieser nichtfundamental Lehren insofern zugute, als sie die Gläubigen vor Gefahren warnen, wie die Lehre vom