159\\ \textit{Wesen und Begriff der Theologie.}\\ \noindent ist;\footnotemark[546] 2. insofern als bei der lediglich aus der heiligen Schrift geschöpften christlichen Lehre die Lehre von der Rechtfertigung \textgreek{δια πιστεως χωρις εργων}\footnotemark[547] so im Zentrum steht, dass alle anderen Lehren entweder Voraussetzungen (articuli antecedentes) oder Folgen (articuli consequentes) der Lehre von der Rechtfertigung sind.\footnotemark[548] Auch dieser Zusammenhang der christlichen Lehre ist nicht eine Konstruktion Luthers und der lutherischen Dogmatiker, sondern in der heiligen Schrift gelehrt. Wenn Paulus einerseits sagt, dass er den ganzen Rat Gottes verkündigt,\footnotemark[549] andererseits, dass er bei seiner Verkündigung nur den für die Sünden der Welt gekreuzigten Christum wusste,\footnotemark[550] so lehrt er damit, dass die Vergebung der Sünden um den Versöhnungstodes Christi willen das Zentrum der ganzen christlichen Lehre ist. Ebenso bezeichnet Petrus die Vergebung der Sünden durch den Glauben an Christum als die Zentralehre der ganzen Schrift Alten Testaments, wenn er sagt: „Von diesem [\textgreek{Ξριστοῦ}] zeugen alle Propheten, dass durch seinen Namen alle, die an ihn glauben, Vergebung der Sünden empfangen sollen.“\footnotemark[551] Der festgeschlossene innere Zusammenhang der christlichen Lehre tritt auch darin hervor, dass wir keinen Teil der christlichen Lehre erkennen können, ohne konsequenterweise das Ganze in Mitleidenschaft zu ziehen.\footnotemark[552] Stellen wir z.\,B. in Frage, dass die heilige Schrift Gottes eigenes unfehlbares Wort ist, so affiziiert das den Charakter der ganzen christlichen Lehre. Was christliche Lehre ist, entscheidet dann nicht Gott in seinem Wort, sondern das theologisierende menschliche Ich. Wird die metaphysische Gottheit Christi geleugnet, so fällt damit die satisfactio vicaria hin,\footnotemark[553] und wenn die satisfactio vicaria geleugnet wird, so gibt es keine Vergebung der Sünden durch den Glauben ohne des Gesetzes Werke, auch keine Gnadenmittel, die ex parte Dei die Vergebung der Sünden darbieten und ex parte hominis nur Glauben\\ \footnotetext[546]{Vgl. die ausführliche Darlegung unter dem Abschnitt „Nähere Beschreibung der Theologie als Lehre“, S.\,57 ff.}\ \footnotetext[547]{Vgl. die Kapitel „Die zentrale Stellung der Lehre von der Rechtfertigung“ II, 617 ff.; „Die Voraussetzungen der Rechtfertigung durch den Glauben ohne Werke“ II, 611 ff.; „Das Verhältnis zwischen Rechtfertigung und Heiligung im engeren Sinne“ III, 6 ff.}\ \footnotetext[548]{Mosb. 26, 27.}\ \footnotetext[549]{Apg. 20, 27.}\ \footnotetext[550]{1 Kor. 2, 2.}\ \footnotetext[551]{Apg. 10, 43.}\ \footnotetext[552]{Dies ist auch ausgesprochen Gal. 5, 9: \textgreek{μικρα ζυμη ολον το φυραμα ζυμοι}. Mehrere J. in „doktrineller Beziehung“ gesagt. Besonders Luther z.\,St. Gl. 5. IX, 642 ff.}\ \footnotetext[553]{Röm. 5, 10; 8, 32; 1 Joh. 1, 7.}