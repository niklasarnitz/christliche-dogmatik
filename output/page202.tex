\textsl{Wesen und Begriff der Theologie.}\hfill 191\par Walther deckt auch den Missbrauch auf, den die moderne Theologie mit dem sogenannten wiedergebornen Ich oder der erleuchteten Vernunft treibt. Er sagt: „Durch die Erleuchtung erhält ja die Vernunft nicht ein eigenes Licht neben der Schrift, vielmehr besteht ihre Erleuchtung eben darin, dass durch Wirkung des heiligen Geistes das Wort der Propheten und Apostel ihr einziges Licht in Sachen des Glaubens geworden ist. — Sofern sie aus ihren Prinzipien wider die Artikel des Glaubens disputieren will, insofern ist sie nicht wiedergeboren, weil die wiedergeborne Vernunft aus den Prinzipien des Wortes Gottes disputiert.“ Sehr eindringlich warnt Walther auch die christliche Kirche vor der Meinung, dass ihre Lehre dem Inhalte nach durch die Mittel der neueren Wissenschaft fortgebildet werden könnte. Er begründet diese Warnung also: „Wir halten die Heilige Schrift in Absicht auf die Gegenstände unseres Glaubens für so klar, dass wir nicht im entferntesten hoffen, dass uns durch die neueren größeren wissenschaftlichen Hilfsmittel ein neuer, der Kirche bis daher unbekannter und verschlossener gewesener Glaubensartikel werde aufgeschlossen werden oder schon aufgeschlossen worden sei. Wir glauben nicht an ein durch allmähliches Entstehen der Dogmen sich vollziehendes Wachstum der Kirche an Erkenntnis. Wir glauben vielmehr, dass schon die Kirche des ersten Jahrhunderts im Besitze aller derjenigen Dogmen war, die wirklich biblische Dogmen sind. Wir sehen die apostolische Kirche nicht für die Kirche in ihrer Kindheit an, die erst nach und nach durch die Arbeit wissenschaftlich gebildeter Theologen zum Mannesalter heranreife; wir sind vielmehr davon fest überzeugt, dass die Kirche in Absicht auf die Wahrheit und Reinheit ihrer Erkenntnis dem Monde gleich ist, der bald ab-, bald wieder zunimmt und selbst zuweilen traurige Eclipsen erfährt. Wir sagen uns nicht nur von solchen Zutaten der Wissenschaft zur Theologie los, welche der biblischen Wahrheit geradezu widersprechen, sondern kurzum von allem, was unsere biblische Theologie ergänzen soll; denn Gott verbietet ja nicht nur, seinem Worte etwas entgegenzustellen, sondern ebenso streng, etwas dazuzutun, 5 Mos. 12, 32.“\par Über die in unserer Zeit angestrebte Versöhnung (Synthese) zwischen Theologie und Wissenschaft äußert Walther sich so: „So gewiss uns ist, dass zwischen der christlichen Theologie und der wahren Wissenschaft, der Wissenschaft in abstracto, ein wirklicher Widerspruch nicht stattfinde und stattfinden könne, so halten wir es doch keineswegs weder für die Aufgabe eines Theologen noch für möglich, jemals unsere biblische Theologie und die Wissenschaft, wie sie in concreto vorhanden ist, miteinander zu versöhnen. Der Vorwurf, den man gegen uns erhebt, dass wir das gegenwärtige in Unglauben versunkene Geschlecht nicht dadurch auch an unserem Teile zum Glauben zurückzuführen suchen, dass wir der Welt die Harmonie des christlichen Glaubens und der Wissenschaft zeigen, dieser Vorwurf ist gegründet; aber wir achten denselben nicht für einen Vorwurf, sondern vielmehr für einen Ruhm, den wir uns durch Gottes Gnade nimmermehr nehmen lassen wollen. Denn wir sind des fest versichert, dass auch der jetzigen abgefallenen Welt nicht durch die Lüge, dass die göttlich offenbarte Wahrheit mit der Weisheit dieser Welt in dem schönsten Einklang stehe, sondern allein dadurch geholfen werden