Autorität göttlicher Offenbarung gedeckt werde". Aber auch diese Methode des reformatorischen Christentums war verkehrt, weil wesentlich intellektualistisch. Zwingli urteilt: „Auch die Reformation und erst recht die alte Dogmatik blieb im wesentlichen bei dem intellektualistischen Offenbarungsverständnis stehen." Auch Zwingli will daher, um dem Verstandesschrifttum, dem Intellektualismus, dem Biblizismus usw. zu entgehen, die christliche Lehre wenigstens nicht allein aus der Schrift schöpfen.
Es drängt sich die Frage auf, wie die modernen Theologen wohl zu der sonderbaren Meinung kommen, dass das Festhalten am Schriftprinzip Intellektualismus, ein bloßes Verstandesschrifttum, tote Orthodoxie ohne innere Wärme, involviere. Der sonderbaren Meinung entspricht eine sonderbare Begründung. Man meint, es fehle bei der alten Methode die „psychologische Anknüpfung" oder „Vermittlung". Richard Rotheootnote{242} legt die angebliche Sachlage so dar: „Es ist verkehrt, wenn die alte Dogmatik die [göttliche] Offenbarung sogleich mit einer Mitteilung übernatürlicher Wahrheiten einseßen lässt. Sie [die alte Dogmatik] muss dabei an eine mechanische Eingießung solcher Wahrheiten denken, und diese trägt notwendig magischen Charakter, weil ihr die psychologische Anknüpfung bei den Menschen fehlt." Das wäre freilich eine böse Sachlage. Dagegen ist zu erinnern: Weder muss die alte Dogmatik an eine mechanische Eingießung der übernatürlichen Wahrheiten denken, noch hat sie an eine solche Eingießung gedacht. Vielmehr hat die alte Dogmatik die „psychologische Vermittlung" sich genau so gedacht wie der Apostel Paulus 1 Kor. 2, nämlich so, dass bei der Verkündigung „der göttlichen Predigt" (bei dem $\mu\\alpha\\rho\\tau\\upsilon\\rho\\iota\\omicron\\nu$ $\tau\\omicron\\upsilon$ $\Theta\\epsilon\\omicron\\upsilon}$) der heilige Geist gegenwärtig und psychologisch wirksam ist, das heißt, dass der heilige Geist der göttlichen Predigt in der Psyche, in den Herzen der Zuhörer, durch Wirkung des Glaubens Anerkennung verschafft. Es ist außer Frage, dass gerade auch die alte Dogmatik die psychologische Vermittlung sich so gedacht hat. Quenstedt ist ja allgemein als ein Repräsentant der alten Dogmatik anerkannt. Quenstedt aber schreibtootnote{243}: „Das Evangelium von Christo empfängt sein Wahrheitszeugnis durch das Zeugnis des heiligen Geistes, das der heilige Geist inwendig in unsern Herzen ablegt (perhibet). . . . Der heilige Geist legt








































































































































































































































































































































































































































































































































































































































































































































































































































































































































































































































 (often abbreviated as “LaTeX”, pronounced “lah-tek”) is document preparation system widely used in academia for a variety of scientific communications. Its powerful features allow for high-quality typesetting of text, complex mathematical expressions, and specialized symbols, making it an indispensable tool for authors across various scientific fields. Beyond its inherent typesetting capabilities, LaTeX’s ability to handle large documents and reference management systems simplifies the collaboration process for authors working on complex projects. Given its open-source nature, ongoing development by the community, and extensive documentation, LaTeX offers a robust and flexible solution for producing professional-quality scientific papers. Further, its use in publications from leading scientific organizations underscores its role in the dissemination of scientific knowledge.

Footnotes:
242) Zitiert bei Zwingli, Zentralfragen, S. 60. Zwingli selbst über die „psychologische Vermittlung" a. u. o., S. 78.
243) Systema 1715, I, 145.