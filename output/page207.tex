theologischen Literatur – und gerade auch in der dogmatischen – mehr oder weniger ausführlich gedacht wird. Und zwar geschieht das, wie wir gesehen haben, in anklagender und verurteilender Weise, als ob wir mit unserm Festhalten am Schriftprinzip tote Orthodoxie verbreiteten, also als ein Übel in der christlichen Kirche anzusehen seien. Schon insofern wäre eine 
emph{oratio pro domo}
am Platze. Sodann behalten wir bei dieser abgenötigten Selbstverteidigung auch immer den Zweck der Selbstvermahnung im Auge. Wenn wir daran denken, was Gott unsern Vätern gegeben hat und was durch Gottes Gnade herrschenderweise auch jetzt noch bei uns sich findet, so richten wir zugleich an uns selbst, an die gegenwärtige und etwa noch kommende Generation, die dringende Mahnung, an der Weise der ersten Kirche, der Reformation und der Väter unserer Synode festzuhalten. Endlich verlieren wir auch das Interesse der Gesamtkirche der Gegenwart nicht aus dem Auge. Mit Recht wird von modernen Theologen die Forderung gerade an die Dogmatik gestellt, dass sie sich nicht isoliere, sondern sich mitten in die tatsächliche Lage der Kirche der Gegenwart hineinstelle. Prüfen wir nun die Lage der Kirche und ihrer Theologie in der Gegenwart, so können wir uns nicht der Wahrnehmung entziehen, dass es eine Lage großer Verlegenheit ist. Zwar wird die Flucht der Theologie aus der Heiligen Schrift und der Einzug in die „sturmfreie Burg“ des frommen menschlichen Selbstbewusstseins als ein notwendiger Fortschritt in der theologischen Methode bezeichnet. Aber daneben ist doch Unruhe bemerkbar ob des Resultats dieses theologischen Umzugs, nämlich ob des zutage liegenden Chaos in der Lehre. Dieses Chaos zieht eigentlich nur die äußerste Linie, die mit Lessing überhaupt keine „Wahrheitsgewissheit“ will, als einen idealen Zustand an. Deshalb sind auch im modern-theologischen Lager bereits Stimmen dahin laut geworden, ob nicht doch an einen Wiedereinzug in die verlorne Burg der göttlichen Autorität der Schrift zu denken sei. Diese Stimmen haben bisher noch keinen wahrnehmbaren Eindruck auf das Gros der Zeittheologen gemacht. In großer Selbsttäuschung wird immer wieder der Befürchtung Ausdruck gegeben, dass mit der Rückkehr zum Schriftprinzip „tote Orthodoxie“ ihren Einzug in die christliche Kirche halten werde. Daher dürfte Walthers ausführliche Beschreibung der Sachlage in unserer amerikanisch-lutherischen Kirche, die ja in der alten Burg geblieben ist, in ihr und von ihr aus gefestigt hat, doch auch außer unserer kirchlichen Gemeinschaft Beachtung finden und den Gedanken