Was die in Frage stehende Sache selbst, die \emph{„Selbstgewissheit“} der Theologie, betrifft, so ist festzuhalten: Eine Gewissheit, die grundsätzlich Gottes Wort als Basis abweist, ist 1. nicht christlich, 2. nicht Gewissheit, 3. nicht wissenschaftlich.

1. Christlich ist, was Christus lehrt und tut. Das wird allseitig zugegeben. Nun belehrt uns Christus, wie bereits zu Anfang dieses Abschnitts dargelegt wurde, sehr bestimmt auch über die Methode, wie es zur Erkenntnis der Wahrheit, also zur \emph{„Wahrheitsgewissheit“}, kommt, nämlich durch das Bleiben an seinem Wort. Und weil Christus bekanntlich stets recht hat, so sind Schleiermacher, Hofmann, Frank und alle, die ihnen in der Methode folgen und dem \emph{„christlichen Subjekt“} ganz oder halb \emph{„Selbstgewissheit“}, das ist, von Christi Wort unabhängige Gewissheit, zuschreiben, im Irrtum. Sie deuten nicht christlich, sondern unchristlich. Jeder Theologe, der dem Bewusstsein, das sich selbstständig etabliert, das Prädikat \emph{„christlich“}, \emph{„fromm“}, \emph{„wiedergeboren“} usw. zukommen lässt, täuscht damit sich selbst und andere. Alle Frömmigkeitsprädikate sind ohne sachliche Berechtigung.

2. Die behauptete \emph{„Selbstgewissheit“}, die sich vom Schriftwort, der einzigen Gewissheitsbasis, losgemacht hat, ist nicht Gewissheit, sondern Einbildung. Schon in der apostolischen Kirche zeigte sich die Meinung, dass man auch ohne Christi Wort in der Wahrheitsgewissheit stehen könne. Innerhalb der korinthischen Gemeinde gab es Leute, die sich selbst als \emph{„Propheten“} und \emph{„geistlich“} einschätzten, aber dabei das apostolische Wort beiseitesetzten. Paulus fordert jedoch von diesen Leuten ganz entschieden, dass sie ihre Selbstgewissheitsbasis fahren lassen und sich auf die Basis seines Apostelworts stellen, weil sein, Pauli, Wort Christi Wort sei. „Was ich euch schreibe, sind des HERRN Gebote.“ Und wenn jemand sich auf diese Basis nicht stellen wolle, so solle die Gemeinde ihn als infurabel seine Wege gehen lassen. Das ist der Sinn der scharfen Worte: „Ist jemand unwissend, der sei unwissend.“\footnote{462} Auch an andern Orten gab es \emph{„Selbstgewisse“} Lehrer. Aber der Apostel deckt auch diesen ihre Selbsttäuschung auf mit den Worten: „So jemand anders lehrt und bleibt nicht bei den gesunden Worten unsers HERRN JESU CHRISTI \dots der ist verblendet und weiß nichts, sondern ist süchtig (\emph{νοσῶν}, krank) in Fragen und Wortkriegen.“\footnote{463} Kurz, ohne Christi Wort oder, was dasselbe ist, ohne der Apostel Wort als Basis gibt es keine Wahrheitserkenntnis oder Wahrheitsgewissheit.