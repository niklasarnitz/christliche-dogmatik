selbst wohl den Überblick über den Gedankengang, durch den es bei ihm zur „Selbstgewissheit“ kommt, verloren habe. Soweit wir uns andere Frant verstehen, soll es in der Weise zur wissenschaftlichen Gewissheit kommen, dass das Subjekt sich selbst „verobjektiviert“, das ist, sich selbst als Objekt der Betrachtung setzt. Geschieht dies, dann empfängt das Subjekt durch das Objekt, das es selbst ist und das es selbst ist, „Eindrücke“, die es durch Denken zur „Erkenntnis“ erhebt und so sich selbst wissenschaftlich gewiss macht. Dabei ist nun aber immer vorausgesetzt, dass das Subjekt sich ja nicht nach einem außer ihm gelegenen Objekt, namentlich auch nicht nach der Heiligen Schrift und dem \emph{testimonium Spiritus Sancti}, umsehen darf. „Auf den Heiligen Geist“, sagt Frant, „kann ich mich dabei insofern nicht berufen, als ja erst in Frage steht, ob, was ich vernehme, Zeugnis des Heiligen Geistes sei, ganz ebenso wie ich mich nicht auf die heilige Schrift berufen kann, wenn in Frage steht, wie ich dazu komme, diese Schrift mir als heilige gelten zu lassen.“\footnote{System d. christl. Gewissheit 2, 1, 143.} Das Verfahren verfährt daher so, als wenn jemand, der nach einem Halt sucht (wenn die wissenschaftliche Gewissheit soll ja als „werdende“ dargestellt werden), sich selbst an irgendeinem Körperteil ergreift und damit den nötigen Stützpunkt gefunden zu haben meint. Die Theologie der Selbstgewissheit kann sich nicht beklagen, wenn einige grobe, zum Teil ungalante Bilder zur Charakterisierung ihrer wissenschaftlichen Methode gebraucht worden sind. Man hat nämlich gesagt, die Methode der Vergewisserung unter Ablehnung jedes „auswärtigen“ Stützpunktes durch Ergreifen des Ich als Objekt sei genau so „wissenschaftlich“ wie die Münchhausens, der sich selbst samt seinem Pferde an den eigenen Haaren aus dem Sumpf zog. Hierzulande hat man zur Illustration der Wissenschaftlichkeit der Methode an die Methode des Mannes erinnert, „who raised himself by his own bootstraps“. Und wenn nach der Begriff der „geschlossenen Einheit“ hinzugenommen werde, dem die Ich-Methode nachtrachte und der dieser Methode in so überschwänglichen Ausdrücken nachgerühmt werde („systematische Meisterschaft“, „wissenschaftliche Genialität“, „Höhe der wissenschaftlichen Entwicklung“ usw.), so ist ein etwas ungalantes Bild gebraucht worden. Die bewunderte „geschlossene Einheit“ ist verglichen worden mit der Einheit, die die Rähe dadurch herstellt, dass sie mit dem eigenen Schwanz spiele und diesen Körperteil als Objekt ergreifend, sich mehr oder minder schnell um sich selbst bewegt. Ohne Bild ist die Sachlage dahin richtig