\normalfont{56} Wesen und Begriff der Theologie. \par \textbf{10. Die nähere Beschreibung der Theologie, als Lehre gefaßt.} \par Weil die Theologie, subjektiv oder als Lehrthätigkeit gefaßt, die Tüchtigkeit (\textit{bearórns}) ist, nicht mehr und nicht weniger als Gottes Wort zu lehren, \footnote{200} dass die Kirche unserer Zeit in dem geschriebenen Wort der Apostel und Propheten besitzt, \footnote{201} so ist die Theologie, objektiv oder als Lehre (\textit{doctrina}) genommen, nichts anderes als die Darstellung der in der Heiligen Schrift vorliegenden Lehre. Was die Heilige Schrift an mehreren oder auch an vielen Orten über die einzelnen Lehren nach Text und Kontext aussagt, das stellt der Theologe an einen Ort zusammen. So entsteht die Lehre, die der christliche Theologe mündlich oder in Schriften vorträgt. Die allen lutherischen Theologen sagen von der Theologie, als Darstellung der christlichen Lehre gefasst (\textit{theologia positiva}), dass sie nichts anderes sei als die in die einzelnen Lehren zusammengedrängte Schrift selbst. Daher dürfe sich in dem Lehrleibe (\textit{corpus doctrinae}) kein Glied finden, auch wenn es das geringste wäre, das nicht in der recht aufgefassten Schrift seine Stütze habe. \footnote{202} Ebenso beschreibt Luther die Lehrthätigkeit aller Theologen nach der Apostelzeit. Er nennt die Theologen, sich selbst einschließend, „Katechumenen und Schüler der Propheten“, „als die wir nachsagen und predigen, was wir von den Propheten und Aposteln gehört und gelernt haben“. \footnote{203} Ein solcher Ernst ist es Luther mit dem bloßen „Nachsagen“ seitens der Theologen, dass er in negativer Bestimmung sagt: „Keine andere Lehre darf in der Kirche gelehrt und gehört werden als das reine Wort Gottes, das ist, die Heilige Schrift, oder Lehrer und Zuhörer sollen mit ihrer Lehre verflucht sein.“ \footnote{204} Dieselbe Wahrheit \par \hrule \par \textsuperscript{200)} 1 Petr. 4, 11: \textit{Εἴ τις λαλεῖ, ὡς λόγια θεοῦ}. \par \textsuperscript{201)} Die in der Kirche geltende Lehre ist \textit{ἡ διδασκαλία τοῦ σωτῆρος ἡμῶν θεοῦ}, Tit. 2, 10. \par \textsuperscript{202)} Was die Apostel mündlich gelehrt haben, dasselbe (\textit{ταὐτά}) haben sie auch geschrieben, 1 Joh. 1, 3. 4. \par \textsuperscript{203)} M. Pfaff, \textit{Thesaurus hermeneut.}, p. 5, auch zitiert in Walther-Baier I, 43. 46: \textit{Theologia positiva, si rem recte aestimemus, nihil aliud est \ldots quam ipsa Scriptura Sacra in certos locos concinno ordine et perspicua methodo redacta, unde ne unicum quidem membrum, quantulum etiam, in illo doctrinae corpore esse debet, quod non e Scriptura Sacra probe intellecta statuminetur}. \par \textsuperscript{204)} In der Auslegung der letzten Worte Davids, 2 Sam. 23, 1 ff. S. I, III, 1890. \par \textsuperscript{205)} Comment. ad Gal., ed. Erl., I, 91: \textit{Neque alia doctrina in ecclesia tradi et audiri debet quam purum Verbum Dei, hoc est, Sancta Scriptura, vel doctores et auditores cum sua doctrina anathema sunto}.