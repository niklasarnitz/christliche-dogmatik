Wesen und Begriff der Theologie. \hfill 180Gnadenmitteln, von der Kirche. Den Schluss bildet der Artikel von der Auferstehung der Toten. Die freudige und zuversichtliche Art, mit welcher Luther seinen Glauben „Stück für Stück“ bekennt, lässt gar nicht den Gedanken aufkommen, dass Luther um den Anfangspunkt verlegen gewesen sei. In seinen Katechismen, im Großen und im Kleinen, hat Luther einen andern Anfangspunkt. Er beginnt hier nicht mit dem Artikel von der Trinität, sondern mit den zehn Geboten, also mit dem Gesetz. Aber wiederum gewahren wir bei ihm keine Verlegenheit um den Anfangspunkt. Bei den alten Dogmatikern finden wir, weil sie teils die synthetische, teils die analytische Methode befolgen, sogar entgegengesetzte Anfangspunkte. Glacius sagt bei der Charakterisierung der synthetischen und der analytischen Methode ganz richtig: \emph{Methodus analytica prorsus contrarium cursum et ordinem synthesi tenet. Haec [die analytische Methode] incipit, ubi illa [die synthetische Methode] desinit, et contra, ibi desinit, ubi illa incoeperat}. Und obwohl Glacius die synthetische der analytischen Methode vorzieht, will er doch beide nebeneinander wachsen lassen: \emph{Licet theologia commodissime per synthesin tradatur, tamen et analyticam methodum aliquo modo recipere posset, ut adiuncta tabula testatur. Finis enim theologiae, de quo nos miseri homines maxime angimur, est vita aeterna, sicut ille [der reiche Jüngling] ex Christo quaerit, Matt. 19, 16.\footnote{591) Clavis II, 58.}} So werden auch wir es nicht wagen, dem einen oder andern Teil wegen der verschiedenen Gruppierung der einzelnen Lehren die dogmatische Erstehensordnung abzusprechen. Der Grund hierfür ist der, dass beide Teile, wie Kirn zugesteht, trotz der verschiedenen Anfangspunkte „eine im wesentlichen einheitliche Methode befolgen“, nämlich die einzelnen Lehren aus der Schrift nehmen. Tatsächlich steht es hinsichtlich des Anfangspunktes in der Darstellung der christlichen Lehre so, dass wir vorne oder hinten oder auch in der Mitte anfangen können. Geschieht der Anfang aus der Ewigkeit, so kommen wir stets sehr bald in medias res, in das Zentrum der christlichen Lehre, nämlich zur Lehre von der Vergebung der Sünden durch den Glauben an den menschgewordenen Sohn Gottes, der die Versöhnung ist für unsere und der ganzen Welt Sünde. Der Grund hierfür liegt in der festgeschlossenen inneren Einheit, die der aus der Schrift genommenen Theologie eigen ist. Wir könnten anfangen mit der Ewigkeit, bei dem ewigen Evangelium