gionen. Wären sie von ihren Vertretern stets unter ihrem richtigen\nPrädikat, „von Menschen gemacht“, „man-made“, auf den Markt\ngebracht worden, so würde die christliche Kirche vor viel Verwirrung,\nStreit und Parteibildung bewahrt geblieben sein.\n\nFragen wir noch, seit wann die Religion des Gnadenevangeliums\nin der Welt bekannt geworden sei, so ist zu sagen, dass sie\nunmittelbar nach dem Sündenfall in der Verheißung von dem\nWeibesamen, der der Schlange den Kopf zertreten sollte, geoffenbart\nwurde.\footnote{Val. Luther zu 1 Mos. 3, 15. St. V. I, 230 ff.; III, 650 ff.}\nDoch auch alle alttestamentlichen Propheten unisono die\nReligion des Evangeliums gelehrt und alle Kinder Gottes zur Zeit\ndes Alten Testamentes sie unanimiter geglaubt haben, bezeugt\nPetrus: „Von diesem [Jesu] zeugen alle Propheten, dass durch\nseinen Namen alle, die an ihn glauben, Vergebung der Sünden\nempfangen sollen.“\footnote{Apost. 10, 43.}\nEbenso sagt Paulus von der Gerechtigkeit,\ndie \textit{ἐκ πίστεως νόμον} durch den Glauben an Christum erlangt wird:\n\textit{μαρτυρουμένη ὑπὸ τοῦ νόμου καὶ τῶν προφητῶν}, Röm. 3. Er\nbringt dafür den historischen Nachweis im 4. Kapitel des Römerbriefes.\nLuthers Darlegungen über die Zweizahl der religiösen\nErkenntnisquellen ziehen sich durch alle seine Schriften hindurch.\footnote{Besonders ausführlich spricht sich Luther hierüber St. V. VII, 1704 bis 1712 aus.}\n\n\subsection*{5. Die Abfasse der Parteien innerhalb der äußeren Christenheit.}\n\nWeil die nichtchristlichen Religionen die Versöhnung mit Gott\nauf dem Wege der menschlichen Werke suchen, eigene Werke aber die\nGewissen nicht zur Ruhe bringen können, so ist es nicht befremdlich,\nsondern völlig naturgemäß, dass die nichtchristlichen Religionen in\nsich endlosen Gestalten auftreten. An diese Ursache der Vielgestaltigkeit\nder nichtchristlichen Religionen erinnert auch das lutherische\nBekenntnis. Die Apologie sagt: Quia nulla opera reddunt pacatam\nconscientiam, ideo subinde nova opera excogitantur praeter mandata\nDei.\footnote{M. 122, 87.}\nHingegen sollte man erwarten, dass innerhalb der Christenheit\nverschiedene Parteien gänzlich ausgeschlossen seien. Die christliche\nKirche hat ja für ihre Lehre nur eine Erkenntnisquelle, nämlich\nChristi Wort,\footnote{Joh. 8, 31. 32: „So ihr bleiben werdet an meiner Rede (\textit{ἐν τῷ λόγῳ τῷ\nἐμῷ}), so seid ihr meine rechten Jünger und werdet die Wahrheit erkennen.“}\ndas er der Kirche durch seine Apostel und Propheten