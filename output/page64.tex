\textbf{3.} Die theologische Tätigkeit schließt in sich die Tüchtigkeit, das ganze Wort Gottes, wie es in der Heiligen Schrift geoffenbart vorliegt, zu lehren. Der Apostel Paulus sagt von seinem Lehramt: „Ich habe euch nichts verhalten, das ich nicht verkündigt hätte alle den Rat Gottes“, \foreignlanguage{greek}{πᾶσαν τὴν βουλὴν τοῦ θεοῦ.}\footnote{185} Nur bei Verkündigung des ganzen Rates Gottes bleiben die öffentlichen Lehrer unschuldig am Verlorengehen der Zuhörer, wie Paulus von sich bezeugt: „Ich bin rein von aller Blut.“\footnote{186} Weil nach Gottes Ordnung die ganze christliche Lehre ohne Abkürzung und ohne Zusatz „öffentlich und sonderlich“ zu lehren ist, so mag auch in diesem Zusammenhang daran erinnert werden, dass großer Fleiß nötig ist sowohl auf seiten der Studenten der Theologie zur Erlangung der theologischen Tüchtigkeit als auch auf seiten der bereits im Amte stehenden Lehrer zur Bewahrung und Mehrung der theologischen Tüchtigkeit. Daher heißt es in der Ermahnung des Apostels an Timotheus: „Hab' acht auf dich selbst und auf die Lehre; beharre in diesen Stücken! Denn wo du solches tust, wirst du dich selbst selig machen und die dich hören.“\footnote{187}

\textbf{4.} Zur theologischen Tüchtigkeit gehört die Tüchtigkeit, die Irrlehrer zu widerlegen. In der Beschreibung der Eigenschaften, die sich an einem Ältesten oder Bischof finden sollen, heißt es Tit. 1, 9--11: \foreignlanguage{greek}{δυνατὸς ... τοὺς ἀντιλέγοντας ἐλέγχειν ... οὓς δεῖ ἐπιστομίζειν}. Das manchmal laut werdende Verlangen, dass der öffentliche Lehrer sich der Polemik enthalte, ist also wider die Schrift. Verboten ist in der Schrift der Streit um unnötige Dinge, z. B. um die Geschlechtsregister, Tit. 3, 9: „Der törichten Fragen aber, der Geschlechtsregister, des Zankes und Streites über dem Gesetz, entschlage dich (\foreignlanguage{greek}{παραιτοῦ}), denn sie sind unnütz und eitel.“\footnote{188} Verboten ist auch die Polemik aus fleischlichem Eifer, 2 Kor. 10, 3: \foreignlanguage{greek}{Οὐκ ἐν μαχίᾳ στρατευόμεθα κατὰ σάρκα.} Much ist wohl zu beachten, dass Tit. 1, 9 dem \foreignlanguage{greek}{δυνατὸς τοὺς ἀντιλέγοντας ἐλέγχειν} das \foreignlanguage{greek}{δυνατὸς παρακαλεῖν ἐν τῆι διδασκαλίᾳ τῆι ὑγιαινούσῃ} vorangeht. Dadurch kommt zum Ausdruck, dass der Widerlegung der falschen Lehre die Darlegung der rechten Lehre

\vspace{1em}
\footnotesize
\noindent 185) Apost. 20, 27.\\
186) Apost. 20, 26.\hfill 187) 1 Tim. 4, 16.\\
188) Auch Duendfeldt erinnert, Systema I, 14: In theologia polemica id praecipue cavendum, ne quaestiones otiose cumulentur, et lites ex litibus serantur, atque ita fiat theologia eristica et contentiosa, qua nimium altercando veritas amittitur.