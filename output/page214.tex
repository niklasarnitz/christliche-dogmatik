Wesen und Begriff der Theologie. \hfill 203\par\noindent „Kirche“ gewidmet „den evangelisch-lutherischen Pastoren Brohm, Bünger, Würger, Fürbringer, Geier, Gönner, Gruber, Keyl, Löber, Schieferdecker, Ferdinand Walther, Wege in Missouri, Illinois, Wisconsin und New York“. Wir legen einen Teil der Widmungsworte hierher, um den kirchlichen Zusammenhang Deutschlands mit der „streng konfessionellen Richtung“ der Missourisynode noch etwas weiter zu veranschaulichen. Die Worte tragen zugleich stark dogmatischen Charakter und sind daher auch in einer Dogmatik am Platze. Sie lauten:\par\begin{quote}„Mit dem Gruße alter, unüberwelklicher Liebe begrüße ich euch, die ihr vom Hause des Herrn seid, euch Genossen meiner ersten Liebe zu Christo, Genossen meiner ersten Freude an der Kirche des lauteren Bekenntnisses und des ungeschmälerten Haushalts Gottes, Genossen martervoller, nun durch Gottes Erbarmen beistandener Kämpfe. Du, mein Walther, weitest mich in den tiefen Ernst der göttlichen Gnadenordnung ein. Im Umgang mit dir und Bürger lernte ich die alten asketischen Schriften unserer Kirche zuerst kennen und lieben. In deiner Gemeinde, lieber Keyl, hielt ich meine erste Predigt; dort sah ich Wunder amtlicher Seelsorge, dort verlebte ich unter deinen Pfarrkindern paradiesische Tage. Welch tödlichen Hass der Welt die schlichte Predigt des Heilsweges erregt, sah ich an dir, lieber Bürger. An eurem Wort und Beispiel, ihr lieben Wege und Brohm, Löber und Fürbringer, ward ich des lutherischen Bekenntnisses recht gewiss und froh. Und in einer Zeit abermaligen Schwankens lernte ich von dir, teurer Gruber, die rechten Kennzeichen der wahren Kirche, an denen ich sie wiedersand, um sie hinfort nimmer wieder zu verlieren. \ldots Jeder eurer Namen, teure Brüder und Freunde, ist ein Stück meiner Lebensgeschichte, übersät mit unauslöschlichen Erinnerungen. Wir haben Jahre pfingstlicher Freude, blutigen Ringens, drückenden Bannes, gnädiger Befreiung miteinander durchlebt, und blicken wir zurück, so muss unser Mund voll Lachens und unsere Zunge voll Rühmens sein. Denn der Herr hat uns gerichtet mit Lindigkeit und regiert mit viel Verschonen. Noch ist in unsern Herzen seine süße Liebe, in unserm Munde das Wort seiner Wahrheit, und wir sind nicht irre geworden an seiner heiligen Kirche. So empfangt denn dieses geringe Angebinde als ein Lebenszeichen der Liebe eures Freundes zu dem Herrn, seinem Hause und euch, seinen Hausgenossen. \ldots Welch eine herrliche Zukunft wartet unserer Kirche, wenn sie die Lampe ihres guten Bekenntnisses mit Öl des Geistes versorgt und ohne Stillstand dem kommenden Herrn hurtig entgegengeht! Ihr geliebten Brüder alle jenseits des Meeres,“\end{quote}