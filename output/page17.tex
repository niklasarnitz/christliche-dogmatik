im Alten Testament der „Propheten Kinder“, die nichts Eigenes noch Neues sehen, wie die Propheten tun, sondern lehren, das sie von den Propheten haben.ootnote{11) Auslegung der letzten Worte Davids, 2 Sam. 23, 3. St. 3, III, 1890. G. A. 37, 12.} Von diesem Standpunkt aus wurde diese „Christliche Dogmatik“ geschrieben.

\subsection*{2. Der Religion im allgemeinen.}

Die Ableitung des lateinischen Wortes \textit{religio} von einem Stammwort (\textit{Etymologie}) ist bekanntlich bis auf den heutigen Tag streitig. Die sprachkundigen Lateiner selbst, ob Heiden oder Christen, vertreten verschiedene Ableitungen.ootnote{12) Der Heide Cicero will \textit{religio} von \textit{relegere} oder \textit{religere} im Sinne von fleißig oder sorgfältig betreiben (\textit{diligentia retractare}) ableiten. \textit{De Nat. Deorum} 2, 28. \textit{Qui omnia, quae ad cultum deorum pertinent, diligenter retractarunt et tanquam relegerent, sunt dicti religiosi ex relegendo, ut elegantes ex eligendo, tanquam a diligendo diligentes, ex intellegendo intellegentes.} Der Christ Lactantius tritt in ausdrücklichem Gegensatz zu Cicero für die Ableitung von \textit{religare} ein im Sinne von: an Gott binden, ihm verpflichten. \textit{Inst. Div.} 4, 28: \textit{Hac conditione gignimur, ut generanti nos deo iusta et debita obsequia praebeamus, hunc solum noverimus, hunc sequamur. Hoc vinculo pietatis obstricti deo et religati sumus, unde ipsa religio nomen accepit, non, ut Cicero interpretatus est, a religendo.} \textit{Augustinus} schwankt zwischen \textit{relegere} und \textit{religare}, wie sich aus einer Vergleichung von \textit{De Civ. Dei} 10, 4 und \textit{De Vera Relig.} c. 55, ergibt. Die meisten unserer lutherischen Theologen ziehen die Ableitung von \textit{religare} vor. Zinsdorf, Systema, 1715, I, 28; Hollaz, Examen Proleg. II, qu. 2. Ausführliche Aufzählung der verschiedenen Ableitungen bei Calov, Isag. 1, 275 sqq., zitiert in Bäler-Walther I, 14. Neuere Theologen und Philologen teilen sich in die schon genannten Ableitungen und fügen andere hinzu, über die man in den größeren Enzyklopädien nachlesen kann. Sehr ausführliches bei Voigt, Fundamentaldogmatik, S. 1—30.
}ootnote{13) O. Schilling, Wörterbuch zum V. L. III. Einl., zitiert als unerkanntes Axiom: „Die Etymologie muss ganz in der Regel einiges Licht auf das zu erklärende Wort, stets aber selten die sprachgeschichtliche Bedeutung desselben.“ Ebeling setzt sehr hinzu: „Die Grundbedeutung lässt sich selten ganz einwandfrei und unbestritten feststellen, und die geschichtliche Entwicklung der Bedeutungen und des Sprachgebrauchs ist unabhängig von Etymologie und Grundbedeutung. Auch Luther bemerkt über diesen Punkt (Opp. exeg. Lat. VIII, 89): Aliud} Wir können ohne Schaden für die Sache, nämlich ohne an unserer religiösen Erkenntnis eine Einbuße zu erleiden, die etymologische Frage unentschieden lassen, weil die Bedeutung eines Wortes in letzter Instanz nicht durch die Etymologie, sondern durch den Sprachgebrauch (\textit{usus loquendi}) bestimmt wird.