im alten Testament der „Propheten Kinder“, die nichts Eigenes noch Neues sehen, wie die Propheten tun, sondern lehren, das sie von den Propheten haben.\footnote{11)} Von diesem Standpunkt aus wurde diese „Christliche Dogmatik“ geschrieben.

\subsection*{2. Über Religion im allgemeinen.}

Die Ableitung des lateinischen Wortes religio von einem Stammwort (Etymologie) ist bekanntlich bis auf den heutigen Tag streitig. Die sprachkundigen Lateiner selbst, ob Heiden oder Christen, vertreten verschiedene Ableitungen.\footnote{12)} Wir können ohne Schaden für die Sache, nämlich ohne an unserer religiösen Erkenntnis eine Einbuße zu erleiden, die etymologische Frage unentschieden lassen, weil die Bedeutung eines Wortes in letzter Instanz nicht durch die Etymologie, sondern durch den Sprachgebrauch (\emph{usus loquendi}) bestimmt wird.\footnote{13)}

\footnotetext[11]{Auslegung der letzten Worte Davids, 2 Sam. 23, 3. St. z. III, 1890. G. W. 37, 12.}
\footnotetext[12]{Der Heide \emph{Cicero} will \emph{religio} von \emph{relegere} oder \emph{religere} im Sinne von fleißig oder sorgfältig betreiben (\emph{diligenter retractare}) ableiten. \emph{De Nat. Deorum} 2, 28. Qui omnia, quae ad cultum deorum pertinerent, diligenter retractarent et tanquam relegerent, sunt dicti religiosi ex \emph{relegendo}, ut elegantes ex \emph{eligendo}, tamquam a \emph{diligendo} diligentes, ex \emph{intelligendo} intelligentes. Der Christ \emph{Lactantius} tritt in ausdrücklichem Gegensatz zu \emph{Cicero} für die Ableitung von \emph{religare} ein im Sinne von: an Gott binden, ihm verpflichten. \emph{Inst. Div.} 4, 28: \emph{Hac conditione gignimur, ut generanti nos deo iusta et debita obsequia praebeamus, hunc solum noverimus, hunc sequamur. Hoc vinculo pietatis obstricti deo et religati sumus, unde ipsa religio nomen accepit, non, ut Cicero interpretatus est, a religendo. Augustinus} schwankt zwischen \emph{relegere} und \emph{religare}, wie sich aus einer Vergleichung von \emph{De Civ. Dei} 10, 4 und \emph{De Vera Relig.} c. 55, ergibt. Die meisten neuern lutherischen Theologen ziehen die Ableitung von \emph{religare} vor: \emph{Quenstedt, Systema,} 1715, I, 28; \emph{Hollaz, Examen Proleg.} II, qu. 2. Ausführliche Aufzählung der verschiedenen Ableitungen bei \emph{Calov, Isag.} I, 275 sqq., zitiert in \emph{Walch-Wassner} I, 14. Neuere Theologen und Philologen teilen sich in die schon genannten Ableitungen und fügen andere hinzu, über die man in den größeren Enzyklopädien nachlesen kann. Sehr ausführliches bei \emph{Voigt, Fundamentaldogmatik, S.} 1--30.}
\footnotetext[13]{O. Schilling, Wörterbuch zum V. T. III. Einl., zitiert als unerkanntes Axiom: „Die Etymologie muss, ganz in der Regel, einiges Licht auf das zu erhellende Wort, stets aber selten die Sprachgeschichte, Bedeutung desselben z. Erhellung selbst sehr hinzu.“ Die Grundbedeutung lässt sich selten ganz einwandfrei und unbestritten feststellen, und die geschichtliche Entwicklung der Bedeutungen und des Sprachgebrauchs ist unabhängig von Etymologie und Grundbedeutung. Auch \emph{Luther} bemerkt über diesen Punkt (\emph{Opp. exeg. Lat.} VIII, 89): \emph{Aliud}}