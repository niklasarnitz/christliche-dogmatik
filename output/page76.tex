Zeit sagt, dass sie nicht „Propheten" sind, sondern „der Propheten Kinder", „Katechumenen und Schüler der Propheten", weil sie nur nachsagen und predigen, was sie von den Propheten gehört und gelernt haben. Und wenn J. W. Scherzer die Theologie, sofern sie nicht dem in der Schrift vorliegenden \textit{ἀρχέτυπος} (theologia \textit{ἀρχέτυπος}) entspricht, unter die Kategorie mataeologia bringt und als häretisches, nichtiges Geschwätz bezeichnet,\footnote{\textit{Systema}, p. 2: In quantum [\textit{viatorium theologia}] illud \textit{ἀρχέτυπος} in Verbo nobis revelatum refert et exprimit, in tantum theologia vera est. Quae ab alio archetypo recedit, falsa et haeretica mataeologia est.} so ist das auch nichts Neues, sondern dasselbe, als wenn Luther alles, was ohne Schrift in der Kirche gelehrt wird, nicht Kirchenlehre nennen will, sondern als „Plappern" bezeichnet.\par Dieser Charakter der christlichen Lehre, dass sie im Gegensatz zu allen menschlichen Gedanken und Anschauungen doctrina divina sein müsse, ist, wie bereits eingangs bemerkt wurde, von der modernen Theologie prinzipiell völlig aufgegeben. Die Ursache hierfür ist ihre veränderte Stellung zur Heiligen Schrift. Wenn Luther es dem christlichen Theologen zur Pflicht macht, sich jeden Gedanken, der nicht den Schriftworten entnommen ist, wieder ausfallen zu lassen, und wenn die Dogmatiker einen Lehrer nur insofern christlich nennen, als seine Lehre theologia \textit{ἐκτύπος}, lediglich Wiedergabe der Schriftlehre ist, so hat das seinen Grund darin, dass sie --- Luther sowohl als die Dogmatiker --- die Heilige Schrift für Gottes eigenes Wort oder für „Gottes Mund" halten. Weil die modernen Theologen diese Stellung zur Schrift ablehnen\footnote{\textit{Nitsch: Stephan}, S. 258: „An der Gegenwart hat die orthodoxe Inspirationslehre kaum mehr dogmatische Bedeutung." Horst Stephan, Glaubenslehre, 1921, S. 52: „Heute ist die Inspirationslehre von der wissenschaftlichen Theologie aufgegeben; sie wirkt nur in der Laienorthodoxie \dots\ noch kräftig nach."} und es für die einzig wissenschaftlich richtige Methode halten, wenn sie sich aus „das fromme Bewusstsein des theologisierenden Subjekts" zurückziehen,\footnote{\textit{Nitsch: Stephan}, S. 12 ff. 15: „Niemand gründet seine Dogmatik in allprotestantischer Art auf die \textit{norma normans}" (Bibel).} so ist damit die Theologie anstatt auf die objektive göttliche Wahrheit auf die subjektive menschliche Anschauung eingestellt. Wir haben die Situation, die in den bekannten Worten Luthers, die Hase seinem Hutterus Redivivus als Motto vorauseilt, zum Ausdruck kommt: „Es will jedermann im Laden feil stehen, nicht dass er Christum"