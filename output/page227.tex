Im vorstehenden sind zwei theologische Vertreter der „streng konfessionellen Richtung“ der amerikanisch-lutherischen Kirche geschildert worden. Dass Walthers Wirkungskreis der größere war, hat mehrere Ursachen, ändert aber nichts an der Tatsache, dass beide Männer trotz der Verschiedenheit der Charaktere und der Lebensführungen durch das Band der völligen Einheit in der christlichen Lehre verbunden waren und bis an ihren Tod verbunden blieben. Beide waren „Repristinationstheologen“ im rechten, gottgesälligen Sinne des Worts und eine konkrete Illustration solcher Worte Luthers wie dieser: „Das Wort und die Lehre soll christliche Einigkeit oder Gemeinschaft machen; wo die gleich und einig ist, da wird das andere wohl folgen“; ferner: „Es folge eine Kirche der andern [in äusserlichen Dingen] freiwillig, oder man lasse eine jede bei ihren Gebräuchen; wenn nur die Einigkeit des Geistes im Glauben und im Worte erhalten wird, so schadet die Verschiedenheit und Mannigfaltigkeit in irdischen und sichtbaren Dingen nichts“; endlich: „Wir nicht des Friedens und Einigkeit, darüber man Gottes Wort verliert; denn damit wäre schon das ewige Leben und alles verloren. Es gilt hier nicht weichen noch etwas einräumen, dir oder einigem Menschen zuliebe, sondern dem Wort sollen alle Dinge weichen, es heiße Feind oder Freund. Denn es ist nicht um äusserlicher oder weltlicher Einigkeit und Friedens willen, sondern um des ewigen Lebens willen gegeben.“\footnote{644)} Schade, dass die amerikanisch-lutherische Kirche „streng konfessioneller Richtung“ und die Kirche in Deutschland auseinandergekommen sind! Um an die ursprüngliche Einigkeit des Geistes zwischen Walther und Franz Delitzsch zu erinnern: durch Walther ist die christliche und theologische Art zur vollen Entfaltung gekommen, die einst die Jugendfreunde in Leipzig verband und der Delitzsch – wir können uns des Eindrucks nicht erwehren – bis an sein Lebensende in einem gewissen Sinne nachgetrauert hat.\footnote{645)}

\footnotetext[644]{St. L. IX, 831; XVIII, 1985; IX, 831.}
\footnotetext[645]{Dies scheint uns auch hervorzugehen aus dem Beileidsschreiben, das Delitzsch anlässlich des Todes Walthers an die hinterlassene Familie richtete. Es ist datiert „Leipzig, Pfingstmontag 1887“ und enthält u. a. die folgenden Worte über Walther: „Es gibt kaum einen lebenden, der so wie ich mit ihm die lieber Eifer siehe zu dem desselben Heil und kann auch die Wahren, unter denen die Auwerwendung sich bewerkstelligt, durchlebt hat. Gott hat ihn in dem Feuer der Anfechtung geschält, so dass er für unsere lutherische Kirche eine eiserne Säule und eherne Mauer (Jer. 1, 18) geworden ist –, ein Wunder in meinen Augen, an welchem oft mein schwacher Glaube sich gestärkt hat. Zu manchen Dingen konnten wir, die beiden alten Freunde, uns in letzter Zeit nicht verständi-}
