\hfill 33\par\centerline{Wesen und Begriff der Theologie.}\par nicht die Vergehung seiner Sünden auf Grund des Wortes der Schrift und auf Grund der stellvertretenden Genugtuung Christi glaubt. Er gibt dann aber damit seine bisher eingenommene Parteistellung auf und kehrt zurück zur Glaubenseinheit mit der christlichen Kirche, die an Christi Worten bleibt und keinen anderen Grund der Zuversicht zur Gnade Gottes kennt als die Erlösung ($\dot{\alpha}\pi o\lambda\dot{u}\tau\varrho\omega\sigma\iota\varsigma$, Loskaufung), die durch Christum Jesum geschehen ist. So hätten wir uns durch die Vorführung der hauptsächlichsten kirchlichen Parteien davon überzeugt, dass Parteien innerhalb der christlichen Kirche ihren Grund in nichts anderem als in der Tatsache haben, dass die Schrift als einzige Quelle und Norm der christlichen Lehre verlassen wird und dann konsequenterweise in der einen oder anderen Form auch Irrlehre an die Stelle der christlichen Gnadenlehre tritt.\par In die Besprechung der Parteien innerhalb der äußeren Christenheit schließt sich die vielbehandelte Frage, ob auch die lutherische Kirche unter die Parteien eingereiht werden sollte. Zur sachgemäßen Beantwortung dieser Frage ist eine Verständigung darüber nötig, was wir unter „lutherischer Kirche“ und unter „Partei“ verstehen. Wir verstehen unter „lutherischer Kirche“ nicht alle Gemeinschaften, die sich zwar noch lutherisch nennen, sondern nur die, welche die lutherische Lehre, wie sie im Bekenntnis der lutherischen Kirche gelehrt und bekannt ist, tatsächlich lehren und bekennen. Unter Parteien verstehen wir solche kirchliche Gemeinschaften, die sich auf Grund schriftwidriger Lehren selbständig konstituiert haben. Bei diesem Verständnis von „lutherischer Kirche“ und „Partei“ ist zu sagen, dass die lutherische Kirche nicht eine Partei bildet, weil sie in ihrem Bekenntnis keine Sonderlehren vertritt, sondern nur die Lehre bekennt und lehrt, die nach Gottes Willen und Ordnung alle Christen bekennen und lehren sollen. Dies ist der ökumenische Charakter der Kirche der Reformation. Einerseits identifiziert sich die lutherische Kirche nicht mit der una sancta, sondern bekennt vielmehr, dass Kinder Gottes auch in solchen Gemeinschaften sich finden, in denen neben Menschenlehren noch so viel von dem Evangelium laut wird, dass dadurch der Glaube an Christum als den einzigen Sündentilger entstehen kann. Andererseits erhebt die lutherische Kirche den Anspruch, dass sie die Kirche der reinen Lehre sei, das heißt, dass ihre Lehre in allen Stücken mit der heiligen Schrift übereinstimme und nach Gottes Willen von allen Menschen zu glauben und anzunehmen sei. Der Beweis für diesen ökumenischen Charakter\index{Charakter, ökumenischer} der lutherischen Lehre ist natürlich auf dem Wege der Induktion zu führen. Wie Luther in seinem Glaubensbekenntnis vom Jahre 1529 sagt: \footnote{115} „Ob jemand nach meinem Tode würde sagen: Wo der Luther jetzt lebte, würde er diesen oder diesen Artikel anders lehren und halten, denn er hat ihn nicht genugsam bedacht usw.: darüber sage ich jetzt als dann und dann als jetzt, dass ich von Gottes Gnaden alle diese Artikel aufs fleißigste bedacht, durch die Schrift und wieder hierdurch oftmals gezogen und so gewiss dieselben wollte verfechten, als ich jetzt habe das Sakrament des Altars verfochten.“ – Hier sei nur noch darauf hingewiesen, dass die lutherische Kirche das Examen in Bezug auf die Schriftmäßigkeit ihrer Lehre auch an dem Punkte besteht, an welchem die große Mehrzahl der Theologen seit Augustin bis auf die neueste Zeit aus rationalistischen Erwägungen das Schriftprinzip verlassen hat. Wir meinen den Punkt, den man die \foreignlanguage{latin}{crux theologorum} genannt hat und den wir als die schwerste Belastungsprobe für das Festhalten am Schriftprinzip bezeichnen möchten. Man meint nämlich in Bezug auf die Gnade Gottes, dass die \foreignlanguage{latin}{universalis gratia} und die \foreignlanguage{latin}{sola gratia} nicht beide zumal feilgehalten werden könnten. Die Calvinisten behaupten, wir wie sehen, dass zur Rettung der \foreignlanguage{latin}{sola gratia} die \foreignlanguage{latin}{universalis gratia} preiszugeben sei; die Synergisten fordern, dass zur Rettung der \foreignlanguage{latin}{universalis gratia} die \foreignlanguage{latin}{sola gratia} geopfert werde. Beides zugleich festzuhalten, sei unmöglich. Die lutherische Kirche ist sich der Schwierigkeit, die hier für das menschliche Begreifen vorliegt, klar bewusst. Dennoch hält sie beides, sowohl die \foreignlanguage{latin}{universalis gratia} als auch die \foreignlanguage{latin}{sola gratia}, ohne Einschränkung fest, weil beide Lehren klar in der Schrift bezeugt sind. Sie erwartet die Lösung der Schwierigkeit, die hier für das menschliche Begreifen vorliegt, im ewigen Leben.\footnote{116}\par Zur Besprechung der Parteien innerhalb der äußeren Christenheit gehört auch ein Hinweis auf die Motive für die Abweichung von der Schriftlehre und die meistens sich anschließende Parteibildung. Die heilige Schrift kennt für diese abnorme Erscheinung innerhalb der christlichen Kirche keine edlen, sondern nur fleischliche Motive.\par\vfill\noindent\rule{\textwidth}{0.4pt}\vspace{0.2ex}\noindent\hspace*{0cm}\footnotesize{115) St. L. XX, 1004 ff. Erl. 30, 303 ff.}\par\hspace*{0cm}\footnotesize{116) P.C. 709, 28, 29; 557, 17--19 (Bekenntnis zur gratia universalis). -- F.O. 623, 9--11; 716, 57--64 (Bekenntnis zur sola gratia und Verzicht auf die Lösung der hier für das menschliche Begreifen vorliegenden Schwierigkeit in diesem Leben). Ebenso Luther, De Servo Arbitrio, opp. v. a. VII, 365, Erl. 2. XVIII, 1965 f.}