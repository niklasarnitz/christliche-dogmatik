\section*{Wesen und Begriff der Theologie. \hfill 134}

christbrüderliche Gemeinschaft gibt: „So jemand zu euch kommt und bringet diese Lehre [nämlich die Lehre Christi] nicht, den nehmet nicht zu Hause und grüßet ihn auch nicht; denn wer ihn grüßet, der macht sich teilhaftig seiner bösen Werke.“\footnote{473)}
Auf Ungewissheit weist auch die Tatsache hin, dass die moderne Theologie aus den christlichen Lehren „Probleme“ macht. Wir sahen unter dem Abschnitt „Offene Fragen und theologische Probleme“\footnote{474)} dass es allerdings sano sensu offene Fragen und Probleme in der Theologie gibt. Das sind Fragen, die bei der Erwägung der in der heiligen Schrift geoffenbarten Lehre auftauchen (quaestiones adnatae), für deren Beantwortung das Schriftzeugnis entweder ganz fehlt oder doch nicht in solcher Klarheit ausgesprochen vorliegt, dass vorsichtige Theologen zu sagen wagten: \foreignlanguage{greek}{$\nu\alpha\iota \tau\tilde{\eta} \delta\iota\alpha\tau\alpha\rho\alpha\chi\tilde{\eta}$}. Vorsichtige Theologen nämlich tragen, wie Luther, eine große Scheu davor, der klar in der Schrift geoffenbarten göttlichen Lehre eine menschliche Meinung an die Stelle zu stellen, selbst wenn diese Meinung auch eine Wahrscheinlichkeit für sich hat, wie z. B. der Tradizionismus. Luther sagt gelegentlich: „Wes ich selbst nicht gewiß bin [nämlich aus der Schrift nicht gewiß bin], das will ich niemand lehren.“\footnote{475)} Solche Theologen sind freilich nicht zu tadeln, sondern zu loben. Durch ihre Weigerung, das zu entscheiden, was die Schrift nicht klar entscheidet, stellen sie die Majestät der heiligen Schrift, ihre einzigartige göttliche Autorität, ins Licht, der sich keine menschliche Autorität an die Stelle stellen darf. Sie reden und entscheiden, wo die Schrift redet und entscheidet; wo die Schrift nicht redet und entscheidet, da treten sie in Demut zurück und schweigen. In diesem Sinne reden christliche Theologen von offenen Fragen und Problemen. Aber bei den modernen Theologen tritt uns die Tatsache entgegen, dass sie gerade aus den Lehren, die in der Schrift klar gelehrt vorliegen, „Probleme“ machen. Sie reden von einem Problem der Schöpfung und Erhaltung der Welt, der Person und des Werkes Christi, der Bekehrung und Rechtfertigung, der Inspiration der Schrift, des Verhältnisses zwischen den christlichen und den nichtchristlichen Religionen usw. Und von ihrem Standpunkt aus haben sie es wahrlich mit „Problemen“ zu tun. Erstlich ist das menschliche Subjekt, das sie an Stelle der heiligen Schrift als Quelle und Norm der Theologie eingesetzt haben, fehlbar. Das geben sie – freilich im Widerspruch mit der Behaup.

\vfill
\noindent
\footnote{473) Matth. 7; Röm. 16, 17; 2 Joh. 10. 11.}
\footnote{474) E. 104 ff.}
\footnote{475) St. L. XX, 1062.}