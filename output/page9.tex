\begin{center}
\textbf{XI}

\textbf{Inhaltsangabe.}
\end{center}

\subsection*{\textbf{Die Lehre von Gott. (De Deo.)}}
\begin{enumerate}
  \item Die natürliche Gotteserkenntnis, S. 445.
  \item Die christliche Gotteserkenntnis, S. 451.
  \item Der Kampf der Kirche um die christliche Gotteserkenntnis, S. 457 (der Kampf gegen die Leugner der drei Personen, S. 459).
  \item Der Kampf gegen die Leugner des einen Gottes, S. 461.
  \item Einwände gegen die Homousie oder die Einheit Gottes, S. 466.
  \item Die Lehre von der heiligen Dreieinigkeit im Alten Testament, S. 474.
  \item Die Unbegreiflichkeit der Dreieinigkeit für die menschliche Vernunft, S. 480.
  \item Die kirchliche Terminologie im Dienst der christlichen Gotteserkenntnis, S. 490.
\end{enumerate}

Nähere Darlegung der Schriftlehre von Gottes Wesen und Eigenschaften (De essentia et attributis divinis).

A. Das Verhältnis des göttlichen Wesens zu den göttlichen Eigenschaften und der Eigenschaften zueinander, S. 524.

B. Verschiedene Einteilungen der göttlichen Eigenschaften, S. 533.

Negative Eigenschaften, wodurch Unvollkommenheit, die sich bei den Kreaturen finden, von Gott negiert werden: die Einheit, S. 536; Einfachheit, S. 538; Unveränderlichkeit, S. 540; Unendlichkeit, S. 542; Allgegenwart, S. 543; Ewigkeit, S. 547.

Positive Eigenschaften, die sich auch an Kreaturen finden, aber Gott in absoluter Vollkommenheit zukommen: Leben, S. 549; Wissen, S. 549; Weisheit, S. 551; Verstand und Wille in Gott, S. 557; die Heiligkeit Gottes, S. 561; die Gerechtigkeit, S. 561; die Wahrhaftigkeit, S. 563; die Macht, S. 564; Gottes Güte, Barmherzigkeit, Liebe, Gnade, Sanftmut, S. 565.

\subsection*{\textbf{Die Schöpfung der Welt und des Menschen. (De Creatione.)}}
\begin{enumerate}
  \item Die Erkenntnisquelle der Lehre von der Schöpfung, S. 570.
  \item Wesen und Begriff der Schöpfung, S. 571.
  \item Der Zeitraum der Schöpfung, S. 572.
  \item Die Ordnung im Schöpfungswerk, S. 572.
  \item Das Schöpfungswerk im einzelnen nach den Tagen, S. 574.
  \item Dichotomie und Trichotomie, S. 581.
  \item Die Einheit des Menschengeschlechts, S. 582.
  \item Einzelnes zum biblischen Schöpfungsbericht, S. 583.
  \item Der Endzweck der Welt, S. 585.
  \item Schlussbemerkungen, S. 586.
\end{enumerate}

\subsection*{\textbf{Die göttliche Vorsehung oder die Erhaltung und Regierung der Welt. (De Providentia Dei.)}}
\begin{enumerate}
  \item Der Begriff der göttlichen Vorsehung und Einwände dagegen, S. 587.
  \item Das Verhältnis der göttlichen Vorsehung zu den causae secundae, S. 592.
  \item Die göttliche Providenz und die Sünde, S. 595.
  \item Die göttliche Zulassung der Sünde, S. 596.
  \item Die göttliche Providenz und die menschliche Freiheit, S. 597.
\end{enumerate}