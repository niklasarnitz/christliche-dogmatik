{\centering 55 Wesen und Begriff der Theologie.\\}

Wohl aber ist der Kirche befohlen, die falschen Lehrer nicht in der Kirche zu dulden, sondern sie mit Gottes Wort zu bekämpfen. Die gebotene Bekämpfung mit Gottes Wort schließt in sich, die falschen Lehrer als solche, nämlich als solche, die vom Apostelwort abweichen, zu erkennen,\footnotemark[190] sie zu widerlegen,\footnotemark[191] sie zu isolieren, das ist, mit ihnen keine kirchliche Gemeinschaft zu halten\footnotemark[192] und sie eventuell, wenn sie nicht selbst kirchenflüchtig werden, von der kirchlichen Gemeinschaft förmlich auszuschließen.\footnotemark[193]

\subsection*{5. Die theologische Tüchtigkeit}
Die theologische Tüchtigkeit schließt endlich auch die Tüchtigkeit in sich, um der christlichen Lehre willen zu leiden. Auf diesen Teil der theologischen Tüchtigkeit weist die Schrift ebenfalls reichlich hin. Der Apostel Paulus ruft Timotheus zu: „$\varkappa \alpha \varkappa \mathrm{o} \pi \alpha \dot{\vartheta} \eta \sigma o \nu$, und zwar $\mu \varepsilon \gamma \alpha \dot{\lambda} \omega s$ δεισμῶν.“\footnotemark[194] Die Leidensfähigkeit ist deshalb eine notwendige Eigenschaft eines christlichen Lehrers, weil das Evangelium von des Seligwerden durch den Glauben an den gekreuzigten Christus ohne des Gesetzes Werke so gar nicht nach dem Geschmack der Welt, sondern „$\mathcal{I} o \upsilon \delta \alpha i o \iota s \mu \varepsilon \dot{v} \sigma \varkappa \dot{\alpha} \nu \delta \alpha \lambda o \nu$, $\mathcal{E} \lambda \lambda \eta \sigma \iota$ δέ $\mu \omega \rho i \alpha$ ist.“\footnotemark[195] Christus stellt daher den Christen für das Leben in dieser Welt das Prognostikon: „$\mathcal{E} \nu \tau \tilde{\omega} \varkappa \sigma \mu \omega \omega$ μισούμενοι ὑπὸ πάντων τῶν ἐθνῶν διὰ τὸ ὄνομά μου.“\footnotemark[196] Dieser Hass wendet sich naturgemäß und erfahrungsgemäß besonders gegen die im öffentlichen Amt stehenden Lehrer.\footnotemark[197] Wenn diese nicht leidenssüchtig sind, nicht auf Gut, Ehre, Stellung, auch Leib und Leben verzichten können, so werden sie Kompromisse mit dem Irrtum schließen oder ganz abfallen und damit die Seligkeit verlieren.\footnotemark[198] Daher die Ermahnung Pauli an Timotheus: „So sei nun stark, mein Sohn, durch die Gnade in Christo Jesu!“\footnotemark[199]

\begingroup\small
\footnotetext[190]{Röm. 16, 17: „σκοπεῖν τοὺς τὰς διχοστασίας καὶ τὰ σκάνδαλα παρὰ τὴν διδαχὴν ἣν ὑμεῖς ἐμάθετε, ποιοῦντας.“}
\footnotetext[191]{Tit. 1, 9. 11: „ἐλέγχειν, ἐπιστομίζειν.“}
\footnotetext[192]{Röm. 16, 17: „ἐκκλίνατε ἀπ᾽ αὐτῶν.“ 2 Joh. 10: „χαίρειν αὐτῷ (dem falschen Lehrer) μὴ λέγετε.“}
\footnotetext[193]{Des Apostels Verfahren in Bezug auf Hymenaeus und Alexander, 1 Tim. 1, 20; vgl. 2 Tim. 2, 17; 4, 14.}
\footnotetext[194]{2 Tim. 2, 3, 9.}
\footnotetext[195]{1 Kor. 1, 23.}
\footnotetext[196]{Matth. 24, 9.}
\footnotetext[197]{Apok. 9, 16. „schalt mich ihm [Paulus] zeigen, wieviel er leiden muss um meines Namens willen.“ Apok. 26, 21 berichtet Paulus: „Sie hatten sich unterstanden (ἐπεχείρησαν), mich zu läutern.“}
\footnotetext[198]{2 Tim. 2, 12: „εἰ ἀρνούμεθα κἀκεῖνος ἀρνήσεται ἡμᾶς.“}
\footnotetext[199]{2 Tim. 2, 1. 8.}
\endgroup