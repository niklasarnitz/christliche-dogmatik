Antichrist, oder die Güte Gottes noch mehr ins Licht stellen, wie die Lehre von den Engeln. Welche Güte und Gnade Gottes leuchtet daraus hervor, wenn es Hebr. 1, 14 von den Engeln heißt: „Sind sie nicht allzumal dienstbare Geister, ausgesandt zum Dienst um derer willen, die ererben sollen die Seligkeit?“! So gilt auch in Bezug auf die articuli non-fundamentales 2 Tim. 3, 16: \textit{πᾶσα γραφὴ θεόπνευστος καὶ ὠφέλιμος}. Aber auch bei der Leugnung von nichtfundamentalsten Lehren liegt die Gefahr vor, dass die göttliche Autorität der Schrift geleugnet wird. Wer nicht glauben will, dass es Engel gibt, wiewohl ihm bekannt ist, dass die Schrift die Existenz und die Tätigkeit der Engel lehrt, der leugnet die göttliche Autorität der Schrift und gibt damit das Erkenntnisprinzip der ganzen christlichen Religion preis. Baier sagt deshalb sehr richtig: \textit{Interim etiam in his cavendum est, ne errorem amplectendo aut profitendo in revelationem divinam Deumque ipsum temere peccetur}.\footnote{Compendium 1, 65.} Dies gilt natürlich auch in Bezug auf die historischen, geographischen, archäologischen usw. Angaben der Schrift. Diese sind freilich nicht Objekt des Glaubens, insofern der Glaube der Vergebung der Sünden teilhaftig macht. Es ist ein schwerer Irrtum Bellarmins, wenn er sagt: \textit{Catholici tam late patere volunt obiectum fidei iustificantis, quam late patet Verbum Dei}.\footnote{Lib. 1 De Iustif., c. 4; bei Quenstedt II, 1362.} Dadurch wird aus dem Glauben ein Werk gemacht. Es ist durchaus festzuhalten, dass Objekt der \textit{fides iustificans} nur die evangelische Verheißung ist, die Vergebung der Sünden um Christi willen darbietet.\footnote{Dies ist ausführlich dargelegt II, 505 unter dem Abschnitt „Der seligmachende Glaube hat nur das Evangelium zum Objekt“.} Wer aber der Schrift in den kleinen Dingen der historischen, geographischen usw. Angaben nicht glaubt, wie wird er unter den \textit{terrores conscientiæ} der Schrift in den großen Dingen glauben, die von des Sohnes Gottes Inkarnation und \textit{satisfactio vicaria} handeln und gegen die alle religiösen Begriffe sprechen, die der Mensch von Natur bei sich beherbergt?\footnote{1 Kor. 1, 23; 2, 14.} Deshalb hat Philippi sich gedrungen gefühlt, in der dritten Auflage seiner Dogmatik die Zweifel zurückzunehmen, die er früher in Bezug auf die Zuverlässigkeit der historischen usw. Angaben der Schrift geäußert hatte.\footnote{Vgl. Zusatz zur 3. Aufl. der Glaubenslehre I, 279. Vgl. Bericht d. Synodalconf. 1886, S. 35.}