schweige er seiner Worte und lasse sie in weltlichem und Hausregiment gelten; allhier in der Kirche soll er nichts reden denn dieses reichen Hauswirts Wort; sonst ist es nicht die wahre Kirche. Darum soll es heißen: „Gott redet.“ Welche Abnormität liegt daher darin vor, wenn solche, die Lehre in der Christenheit sein wollen, „Lehrfreiheit“ für sich in Anspruch nehmen! Wer Lehrfreiheit fordert, stellt sich neben und eo ipso wider Christum. Daher heißen alle falschen Lehrer Antichristen, \textit{1 Joh. 2, 18}: $\alpha\nu\tau\acute{\iota}\chi\rho\iota\sigma\tau o\iota$ $\pi o\lambda\lambda o\acute{\iota}$ $\gamma\varepsilon\gamma\acute{o}\nu\alpha\nu$. Dieser Punkt ist ausführlich unter dem Abschnitt behandelt: „Christi Ausrichtung des prophetischen Amtes im Stande der Erhöhung.“\footnotemark[532]\par\begin{enumerate}\item[2.] Die Christen haben in der Schrift den klaren und oft wiederholten Befehl, nur solche Prediger zu hören, die aus des HErrn Munde --- das ist für unsere Zeit: aus der Heiligen Schrift --- lehren und nicht aus dem eigenen Innern.\footnotemark[533] Allen, die nicht Christi Lehre bringen, sollen sie den christlichen Bruderguß verweigern,\footnotemark[534] sie für ausgeblasene Ignoranten halten\footnotemark[535] und von ihnen weichen.\footnotemark[536] Groß ist daher der Unverstand der Prediger, die keine Verpflichtung auf das Bekenntnis der christlichen Gemeinde wollen, mit der Begründung, dass dadurch ihre Lehrfreiheit ungehörig eingeschränkt werde. Schon die bloße Forderung der Lehrfreiheit bringt an den Tag, dass solche nicht wissen, was göttliche Ordnung ist in bezug auf die Lehre, die in der christlichen Kirche gelehrt werden soll. Noch größer aber ist der Unverstand der \textit{theologischen Professoren}, wenn diese „Lehrfreiheit“ als ein ihnen zukommendes Privilegium in Anspruch nehmen. Auf sie findet alles den Predigern Gesagte verschärfte Anwendung, weil sie die zukünftigen Prediger der Kirche heranbilden sollen. Der Einwand, dass ihnen als Vertretern der „theologischen Wissenschaft“ Lehrfreiheit zukomme, ist nicht zutreffend, weil Christus sehr klar lehrt, dass alle Erkenntnis oder alles Wissen der christlichen Kirche nur durch das Bleiben an Christi Wort vermittelt. Schon die bloße Forderung der Lehrfreiheit seitens theologischer Lehrer stellt klar die Tatsache ins Licht, dass ihnen die Qualifikation für das theologische Lehramt abgeht.\end{enumerate}\par Ein Blick in die Geschichte der Kirche zeigt uns die Tatsache, dass die sogenannte wissenschaftliche Theologie in den Landeskirchen der verschiedenen Länder schon ziemlich lange und ziemlich allgemein Lehrfreiheit genoß. Aber das ist auch eine Hauptursache des gegen=\par\vspace{1em}\par{\small\footnotetext[532]{II, 400 f.}\par\footnotetext[533]{Jer. 23, 16. 31.}\par\footnotetext[534]{2 Joh. 7---11.}\par\footnotetext[535]{1 Tim. 6, 3 ff.}\par\footnotetext[536]{Röm. 16, 17.}\par}