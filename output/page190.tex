\section*{Wesen und Begriff der Theologie.}
\setcounter{page}{179}
klasse gehören. Wir würden das theologische Studium unnötig belastet und zugleich verwirrend wirken, wenn wir z. B. den Studierenden als theologisch wichtig darlegen wollten, dass ein bestimmter Theologe, der sich prinzipiell von der Schrift als Gottes Wort losgesagt hat und ein Jahrhundert liefern will, mehr nach links nach Ritschl oder mehr nach rechts nach Frankreich bestimmt sei.
Um die christliche Kirche von der Schriftmethode abzudrängen und auf die Sachmethode einzustellen, erhebt die moderne Theologie den Einwand, dass der Theologe bei der Darstellung der christlichen Lehre aus der Schrift gar nicht wisse, wo er anfangen solle. Wir sahen schon, dass z. B. Frank so argumentiert. Ebenso argumentiert auch Ishmels, wenn er sagt:\footnote{589} „Sie [die Dogmatik] darf sich nicht damit begnügen, lediglich die biblischen Lehraussagen zusammenzustellen. Ohne weiteres müsste auch jemandem, der noch so sehr zu einer derartigen Auffassung neigte, klargemacht werden können, dass die Frage ihn schon in Verlegenheit bringen müsste, an welchem Punkte er nun mit der Darstellung der Schriftaussagen zu beginnen habe und wie er sie zu verknüpfen habe.“ Das wäre allerdings eine fatale Situation, wenn die, welche nur das Schriftprinzip gelten lassen, zwar eine Dogmatik schreiben wollten oder sollten, aber gar nicht wüssten, wo sie anfangen sollten. Ohne Anfang in der Dogmatik gäbe es auch keine Mitte der Dogmatik und kein Ende der Dogmatik. Kurz, alle, die in der Dogmatik am Schriftprinzip festhalten, wären tatsächlich minus Dogmatik. Wenigstens sollten sie „von Rechts wegen“ ohne Dogmatik sein. Dagegen verweisen wir zunächst auf die Tatsache, dass wir bei Luther und den alten Theologen, obwohl sie hartnäckig am Schriftprinzip festhalten, gar keine „Verlegenheit“ um den Anfang wahrnehmen. Bekanntlich erzählt Luther am Ende seiner Schrift „Bekenntnis vom Abendmahl Christi“ (1528) „vor Gott und aller Welt“ seinen Glauben \emph{„Stück für Stück“}\footnote{590} Er beginnt mit dem „hohen Artikel der göttlichen Majestät, dass Vater, Sohn und Heiliger Geist drei unterschiedliche Personen, ein rechter einiger, natürlicher, wahrhaftiger Gott ist“, also mit dem Artikel von der Trinität. Dann folgen die Artikel von der Person Christi, vom Werk Christi, von der Sünde, von der Rechtfertigung (durch den Glauben an Christum im Gegensatz zu aller Werkslehre), von den
\vspace{1em}
\begin{footnotesize}
\noindent\textsuperscript{589)} Aus der Kirche, ihrem Lehren und Leben unter dem Abschnitt „Aufgabe und Bedeutung der Dogmatik“, S. 121.
\noindent\textsuperscript{590)} St. L. XX, 1094 ff.
\end{footnotesize}