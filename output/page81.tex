\noindent 70\hfill Wesen und Begriff der Theologie.\par\noindent reformierten Schwärmer, die sich wider das Schriftwort auf ihren unmittelbaren erleuchteten „Geist“ beriefen, schreibt Luther: „Grund und Ursach’ solches ihres Dünkels ist, dass man diese Worte, „Das ist mein Leib“ müsse aus den Augen tun und zuvor durch den Geist die Sachen bedenken. . . . Dieser Teufel geht frei daher ohne Larve und lehrt uns öffentlich die Schrift nicht ansehen.\footnote{240) St. 2. XX, 1022 f. Vgl. Zwinglis Mahnung, „dass sich niemand der Irren lassen in den ängstlichen Versuchungen der Worte [die Abendmahlsworte sind gemeint]; denn wir sehen unsern Grund nicht darein“. Abgedruckt in Luthers Werken, St. 2. XX, 477.} gleich wie der Münzer und Carlstadt auch taten, welche hatten auch ihre Kunst aus dem Zeugnis ihrer Inwendigkeit und durften der Heiligen Schrift nicht für sich selbst, sondern für die andern zu lehren als ein äußerlich Zeugnis des Zeugnisses in ihrer Inwendigkeit.“\par\noindent Wie allgemein die Raththeologie gegen das Schriftprinzip auf den Plan tritt und den modern-theologischen Markt beherrscht, geht daraus hervor, dass nicht nur die ausgesprochen liberalen Theologen, sondern auch die als positiv geltenden von „Intellektualismus“ usw. reden, falls die christliche Lehre aus der Schrift anstatt aus dem Innern des Theologen geschöpft und normiert werde. Auch Zwingli z. B. beschuldigt sowohl die erste Kirche als auch die Kirche der Reformation und insonderheit die Dogmatiker einer intellektualistischen Schriftauffassung, weil sie sich für die christliche Lehre direkt auf die Schrift beriefen.\footnote{241) Zentralfragen 2, S. 56 ff.} Zwingli hat zwar eine Entschuldigung für die erste Kirche. Die „junge Theologie“ habe es mit einem Gegensatz zu tun gehabt, der „zum guten Teil selbst religiös orientiert war“. Deshalb habe es nahe gelegen, sich auf eine einzigartige übernatürliche Offenbarung zu berufen. Gemeint ist die Berufung auf das Wort der Propheten und Apostel als Gottes Wort und Lehre. Aber verkehrt sei diese Methode doch gewesen, weil „wesentlich intellektualistisch“. Ebenso hat Zwingli eine Entschuldigung für das „reformatorische Christentum“. Die Kirche der Reformation hatte es ebenfalls mit einem Gegensatz zu tun, der stark religiös orientiert war. Die römische Kirche nahm für die in ihr geltende traditionelle Lehre göttliche Autorität in Anspruch.