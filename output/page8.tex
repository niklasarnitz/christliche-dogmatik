\section*{Inhaltsangabe.}
\subsection*{Wesen und Begriff der Theologie.}
\textit{(De Natura et Constitutione Theologiae.)}
\begin{enumerate}
\item Die Verständigung über den Standpunkt, S. 1. —
\item Über Religion im allgemeinen, S. 6. —
\item Die Zahl der Religionen in der Welt, S. 8. —
\item Die zwei Erkenntnisquellen der tatsächlich bestehenden Religionen, S. 10. —
\item Die Ursache der Passion in der wahren Christenheit, S. 29. —
\item Das Christentum als absolute Religion, S. 36. —
\item Christliche Religion und christliche Theologie, S. 42. —
\item Die christliche Theologie, S. 44. —
\item Die nähere Beschreibung der Theologie, als Tüchtigkeit gefasst, S. 50. —
\item Die nähere Beschreibung der Theologie, als Lehre gefasst, S. 56. —
\item Einteilungen der Theologie, als Lehre gefasst, S. 84. Gesetz und Evangelium, S. 84. Fundamentale und Nichtfundamentale Lehren, S. 89. Primäre und sekundäre Fundamentallehren, S. 95. Nichtfundamentale Lehren, S. 102. Offene Fragen und theologische Probleme, S. 104. —
\item Die Kirche und die christlichen Dogmen, S. 108. —
\item Der Zweck der Theologie, den sie an den Menschen erreichen will, S. 116. —
\item Die näheren Mittel der Theologie, wodurch sie ihr Ziel an den Menschen erreicht, S. 118. —
\item Theologie und Wissenschaft, S. 119. —
\item Theologie und Gewissheit, S. 123. —
\item Theologie und Lehrfortbildung, S. 147. —
\item Theologie und Lehrfreiheit, S. 154. —
\item Theologie und System, S. 158. —
\item Theologie und Methode, S. 172. —
\item Die Erlangung der theologischen Tüchtigkeit, S. 228.
\end{enumerate}

\subsection*{Die Heilige Schrift.}
\textit{(De Scriptura Sacra.)}
\begin{enumerate}
\item Die heilige Schrift ist für die Kirche unserer Zeit die einzige Quelle und Norm der christlichen Lehre, S. 233. —
\item Die heilige Schrift ist im Unterschiede von allen andern Schriften Gottes Wort, S. 256. —
\item Die heilige Schrift ist Gottes Wort, weil sie von Gott eingegeben oder inspiriert ist, S. 262. —
\item Das Verhältnis des heiligen Geistes zu den Schreibern der heiligen Schrift, S. 275. —
\item Die Einwände gegen die Inspiration der heiligen Schrift, S. 280 (vier verschiedene Stile in den einzelnen Büchern der Schrift; die Verteilung auf historische Forschung; Die verschiedenen Lesarten; ungenaue Widersprüche und irrige Angaben; ungenaue Zitate der neutestamentlichen Schreiber aus dem Alten Testament; geringe und dem heiligen Geist nicht anständige Dinge; Solözismen, Barbarismen, versehrte Satzkonstruktionen). —
\item Geschichtliches zur Lehre von der Inspiration, S. 320. —
\item Luther und die Inspiration der Schrift, S. 334. —
\item Zusammenfassende Charakteristik der neueren Theologie, sofern sie die Inspiration der Schrift leugnet, S. 360. —
\item Die Folgen der Leugnung der Inspiration, S. 367. —
\item Die Eigenschaften der heiligen Schrift, S. 371 (die göttliche Autorität, S. 371; die göttliche Kraft, S. 381; die Vollkommenheit, S. 383; die Deutlichkeit, S. 386). —
\item Die geschichtliche Bezeugung der Schrift, S. 398 (Homologumena und Antilegomena). —
\item Die Integrität der biblischen Texte, S. 408. —
\item Die Schrift im Grundtext und die Übersetzungen, S. 419. —
\item Der Gebrauch der Schrift zur Entscheidung von Streitigkeiten, S. 422. —
\item Die Autorität der Schrift und die Symbole, S. 427. —
\item Schrift und Exegese, S. 434.
\end{enumerate}