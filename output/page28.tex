\noindent\makebox[\textwidth]{\normalfont\itshape Wesen und Begriff der Theologie.\normalfont\hfill 17}\par\vspace{1em}\noindent fähig nicht als etwas Fertiges, Unantastbares, schlechthin von obenher Gegebenes verehre. Kurz, Voraussetzung für die Aufstellung eines „echt“ oder „rein“ philosophischen Religionsbegriffs ist die Beiseitesetzung der göttlichen Autorität der hei-ligen Schrift. Einen solchen rein menschlichen Religionsbegriff gibt es.\par Aber wir stehen hier sofort wieder vor dem heidnischen Religionsbegriff mit dem Inhalt: „salvation by works“, wie Martin Luther es ausdrückt. Und das ist ganz in Ordnung. Wir müssen uns immer wieder daran erinnern, dass der Inhalt der christlichen Religion, wonach die Menschheit durch Christi satisfactio vicaria mit Gott versöhnt ist und der Mensch daher ohne eigene Werke durch den Glauben an Christum einen gnädigen Gott hat, für jeden Menschen, die Philosophen eingeschlossen, terra incognita ist. Der Inhalt der christlichen Religion ist nie in eines Menschen Herz gekommen, εὐκαίρων ἀδρόνων οὐκ ἀνίφῃ.\textsuperscript{49)} Dagegen eignet allen Menschen, die Philosophen eingeschlossen, ein Wissen um das Gesetz Gottes. Des Gesetzes Werk zeigt auch in den Herzen der Philosophen geschrieben.\textsuperscript{50)} Daher bewegen sich auch die religiösen Gedanken der Menschen, die in die Abteilung „Philosophen“ gehören, auf dem Gebiet des Gesetzes und der Menschenwerke. Sokrates will in seiner Todesstunde dem Askulay noch einen Hahn geopfert haben, und Kant, den manche für den ersten wirklichen Religionsphilosophen erklärt haben, hat das Wesen der Religion mit Verwerfung der christlichen Versöhnungslehre in die menschliche Sittlichkeit umgesetzt.\textsuperscript{51)} Luther hat den Religionsbegriff der Philosophen sehr klar erkannt und herausgestellt. Er sagt z. B.\textsuperscript{52)} „Aus dieser natürlichen Erkenntnis [des Gesetzes] haben ihren Ursprung alle Bücher der Philosophen, die vor andern etwas vernünftiger gewesen sind, als des Khop, des Aristoteles, des Plato, des Xenophon, des Cicero, des Cato... Aber wenn du fragst vom Gewissen, wie das zufriedenstellen sei, und von der Hoffnung des ewigen Lebens, so sind sie in Wahrheit wie der Rabe, der hier [1 Mos. 8,7] um den skaften herum fliegt und draußen nicht Frieden findet, innen im Kasten aber ihn nicht sucht, wie Paulus von den Juden sagt Röm. 9: „Israel hat dem Gesetz der Gerechtigkeit nachgestanden und das Gesetz der Gerechtigkeit nicht überkommen.“ Die Ursache ist: Das Gesetz ist wie der Rabe, ist ein Amt des Todes und der Sünden und macht Heuchler.\textsuperscript{52)}“ Kurz, je „reiner“ wir\vfill\small\noindent 49) 1 Cor. 2, 9.\par 50) Röm. 2, 15.\par 51) M. Heinze in M. S. V. I. 613 f.\par 52) St. 2, 1, 621.\par 53) Nach ausführlicher zu Ref. 9, 2 St. V. VI, 102 ff.\par P. Bieber, Dogmatik. I.