23\n\textbf{Wesen und Begriff der Theologie.}\n\ngegeben hat.\footnote{76} In diesem Wort Christi aber ist die fruchtbare Mutter der Parteibildungen, nämlich die Werkreligion, aufs Entschiedenste verworfen\footnote{77} und hingegen die göttliche Sündenvergebung ohne des Gesetzes Werke, durch den Glauben an Christum einmal vollbrachtes und vollkommenes Versöhnungswerk aufs klarste gelehrt.\footnote{78} Zudem steht auch durch Erfahrung fest, dass durch den Glauben an die Versöhnung, die durch Christum geschehen ist, die menschlichen Gewissen zur Ruhe kommen und also keine Veranlassung vorliegt, nach andern Versöhnungsmethoden Umschau zu halten.\footnote{79} So sollte man wirklich erwarten, dass verschiedene Parteien und Spaltungen in der christlichen Kirche völlig ausgeschlossen seien, wie sie denn auch in der Schrift ausdrücklich verboten sind.\footnote{80} Aber in der Geschichte der christlichen Kirche tritt uns ein ganz anderes Bild entgegen. Schon die apostolische Kirche hatte unter Parteibildungen zu leiden.\n\nWoher diese Parteibildungen? Sie haben ihren Grund weder in der Verschiedenheit des Klimas, wie die einen gemeint haben, noch auch in dem Klassenunterschied, wie andere gesagt haben,\footnote{81} sondern in der Tatsache, dass innerhalb der Kirche Lehrer auftraten und Anhang fanden, die nicht bei dem Wort der Apostel und Propheten Christi blieben, sondern ihr eigenes Wort verkündigten und damit konsequenterweise auch die differentia specifica der christlichen Religion, die Rechtfertigung aus dem Glauben ohne des Gesetzes Werke, schädigten oder geradezu leugneten. Dass schon unter den Augen der Apostel Leute auftraten, die das apostolische Wort nicht als Gottes Wort gelten lassen wollten, sondern dagegen ihre menschliche Meinung setzten, erhellt klar aus Paulus’ Mahnung an die römische Gemeinde: „Ich ermahne euch, liebe Brüder, dass ihr aufsehet auf die, die da\n\n\n\n\footnote{76} Joh. 17, 20: \textgreek{Διὰ τοῦ λόγου αὐτῶν}. nämlich der Apostel, werden an Christum glauben die, die bis an den Jüngsten Tag gläubig werden. Eph. 2, 20: \textgreek{ἐποικοδομηθέντες ἐπὶ τῷ θεμελίῳ τῶν ἀποστόλων καὶ προφητῶν}.\n\footnote{77} Gal. 2, 16: \textgreek{Οὐ δικαιωθήσεται ἐξ ἔργων νόμου πᾶσα σάρξ}. Gal. 3, 10: \textgreek{Ὅσοι ἐξ ἔργων νόμου εἰσίν, ὑπὸ κατάραν εἰσίν}.\n\footnote{78} Röm. 3, 28: \textgreek{Λογιζόμεθα οὖν πίστιει δικαιοῦσθαι ἄνθρωπον χωρίς ἔργων νόμου}. Gal. 2, 16: \textgreek{Εἰδότες ὅτι οὐ δικαιοῦται ἄνθρωπος ἐξ ἔργων νόμου ἐὰν μὴ διὰ πίστεως Ἰησοῦ Χριστοῦ}.\n\footnote{79} Röm. 5, 1: \textgreek{Δικαιωθέντες οὖν ἐκ πίστεως, εἰρήνην ἔχομεν πρὸς τὸν θεὸν διὰ τοῦ κυρίου ἡμῶν Ἰησοῦ Χριστοῦ}. Kol. 2, 10: „Eoit ir ad’O [in Christo] περιπληρωμένοι.“\n\footnote{80} 1 Kor. 1, 10: \textgreek{μὴ ᾖ ἐν ὑμῖν σχίσματα}.\n\footnote{81} Vgl. hierüber Ritschl-Stephan, Ev. Dogmatik, S. 270. Hase, Hutterus Redivivus 10, S. 12, Anm. 1.