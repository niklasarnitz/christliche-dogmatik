\noindent\makebox[\textwidth][s]{78 \quad Wesen und Begriff der Theologie.}\[1em]ausgezeichnet und allen Ernstes behauptet, nicht „Lehre“, sondern „Glaube“ sei zu predigen, mit der hinzugefügten Begründung, dass nur auf diese Weise „lebendiges Christentum entstehe und der toten Orthodoxie“, dem „Intellektualismus“, gewehrt werde. Die Sache liegt doch so, dass die christliche Religion von allem Anfang an als Lehre oder Lehrmitteilung in die Welt getreten ist. Lehre sind schon die Worte von dem Weibesamen, der der Schlange den Kopf zertreten wird, 1. Mose 3, 15. Ja, der Apostel Paulus versichert uns, dass das ganze Alte Testament uns zur Lehre, $\epsilon\iota\varsigma \tau\eta\nu \eta\mu\epsilon\tau\epsilon\rho\alpha\nu$ $\delta\iota\delta\alpha\sigma\kappa\alpha\lambda\iota\alpha\nu$, $\pi\rho\acute{\omicron}\varsigma$ $\delta\iota\delta\alpha\sigma\kappa\alpha\lambda\iota\alpha\nu$, geschrieben sei, Röm. 15, 4; 2. Tim. 3, 16. Und als die Zeit erfüllt, der Sohn Gottes im Fleisch erschienen war und hier auf Erden wandelte, da übte er selbst die Lehrtätigkeit aus. Er lehrt aus dem Schiff (Luk. 5, 3), vom Berge (Matth. 5, 2), in den Synagogen (Luk. 4, 15), zieht lehrend durch das Land (Matth. 4, 23). Auch die vierzig Tage zwischen Auferstehung und Himmelfahrt benutzt Christus noch zum Lehren (Apost. 1, 3), und vor seiner Himmelfahrt gibt er seiner Kirche den Auftrag, die Völker bis an den jüngsten Tag zu lehren: „Lehret sie halten alles, was ich euch befohlen habe!“ Dem sind die Apostel nachgekommen. Der Apostel Paulus hat nicht abgelassen, öffentlich und sonderlich ($\kappa\alpha\tau\text{' } \omicron\acute{\iota}\kappa\omicron\upsilon\varsigma$) den ganzen Rat Gottes zu lehren, Apost. 20, 20. 27. Und wie Paulus seine eigene Lehrtätigkeit betont, so gebietet er auch Timotheus und Titus, an der Lehre festzuhalten, die sie von ihm gelernt haben, 2. Tim. 1, 13; Tit. 1, 9; 2. Tim. 2, 2. Ein Bischof soll Lehrhaftig, $\delta\iota\delta\alpha\kappa\tau\iota\kappa\acute{\omicron}\varsigma$, sein (1. Tim. 3, 2), und der erste Gebrauch der Schrift ist der Gebrauch zur Lehre, $\pi\rho\acute{\omicron}\varsigma$ $\delta\iota\delta\alpha\sigma\kappa\alpha\lambda\iota\alpha\nu$ (2. Tim. 3, 16). Und wie die Lehrer in den Gemeinden, so sollen auch die Gemeinden an der Lehre bleiben und die Lehre unter sich in Übung halten, Kol. 3, 16; „Lasset das Wort Christi unter euch reichlich wohnen in aller Weisheit; Lehret und ermahnet euch selbst!“ Und 2. Thess. 2, 15: „So stehet nun, liebe Brüder, und haltet an den Satzungen, die ihr gelehret seid, es sei durch unser Wort oder Epistel.“ An den Christen zu Jerusalem wird gelobt, dass sie beständig blieben in der Lehre ($\delta\iota\delta\alpha\chi\acute{\eta}$) der Apostel, Apost. 2, 42; und der Apostel Johannes hält das Bleiben an der Lehre Christi für so wichtig, dass er die Weisung erteilt, jedem, der die Lehre Christi nicht bringt, die christliche Bruderschaft zu verweigern, 2. Joh. 9--11. Wenn nun trotzdem moderne Theologen behaupten, dass die Heilige Schrift, das geschriebene Wort der Apostel und Propheten, nicht als „Lehre“ oder