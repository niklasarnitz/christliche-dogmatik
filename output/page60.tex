\hfill 49 \hfill Wesen und Begriff der Theologie.

Selbstverständlich ist die Theologie, als Tüchtigkeit (\textit{ἱκανότης}) gefasst, der erste und eigentliche Begriff des Wortes, während Theologie, als Lehre gefasst, erst im zweiten und abgeleiteten Sinne des Worts Theologie genannt werden kann, weil „die Theologie erst in der Seele des Menschen sein muss, ehe sie von ihm gelehrt, in Rede oder Schrift dargestellt werden kann“.% 
\footnote{177}{So Walther, L. u. W. 14, 9. Müßäus, Introd. in Theol. 1679, p. 2. „Dogma: die Theologie \textit{habitus} moralis aut \textit{habitus} Theologiae estoque eine \textit{affectuar}-Lehrart, kompt.“ S. 4 will zwar die alten lutherischen Theologen entschuldigen, wenn sie dennoch die Theologie als persönliche Tüchtigkeit fassten, nämlich als die Tüchtigkeit, Gottes Wort zu lehren und Sünder zur Seligkeit zu führen; aber er fügt doch tadelnd hinzu: „in jener Definition ist aber sowohl die unmittelbare Bezeichnung der Theologie zur Seligkeit als auch ihre Fassung als einer persönlichen Eigenschaft zwar im besten Sinne religiösen Ernstes gemeint, aber wissenschaftlich nicht richtig.“ Diese Kritik hat auch von Luthards Standpunkt aus keinen erkennbaren Sinn. Wenn er mit Kahnis die Theologie als „das wissenschaftliche Selbstbewusstsein der Kirche“ darstellt, so fasst er ebenfalls die Theologie als „persönlich Eigenschaft“, da dieses „Selbstbewusstsein“, auch das „wissenschaftliche“, Menschen voraussetzt, denen es als Wissenschaft anhaftet. Ein unpersönliches Selbstbewusstsein ist ein Widerspruch in sich selbst. Tatsächlich will Luthardt auch an Personen innerhalb der Kirche gedacht haben, nämlich an die Theologen, die im Unterschiede von den gewöhnlichen Christen ein wissenschaftliches Selbstbewusstsein besäßen. Nur stimmt das dann wieder nicht mit der Behauptung, dass die Theologie „das wissenschaftliche Selbstbewusstsein der Kirche“ sei, weil die Theologen, namentlich auch die mit wissenschaftlichem Selbstbewusstsein ausgestatteten, Gott sei Dank, nicht die Kirche sind.}

„dogmatisch“ in der Form von Hutters Compendium Theologicum corrigiertem Friede. Aus 2 Tim. 1, 13 ist so viel klar: 1. dass Timotheus von dem Apostel Paulus \textit{ὑγιαίνοντες λόγοι}, das ist, die reine göttliche Wahrheit ausdrückende, durch Menschenanordnung nicht verderbte Worte, \textit{ἀγαθή παραθήκη}, 2. auch den Apostel die \textit{ἀγαθή παραθήκη} mit dem Timotheus nicht nur unterordnung der gleichen Erziehung vorgetragen hat, sondern als \textit{ἱκανότης}, Abbild, Vorbild, Muster, norma sanorum vorbordem, nach denen Timotheus in seinem Lehramt sich richten sollte, was doch besonders durch das Eze. „halte!“ „halte fest!“ ausgedrückt wird. Es wird also doch wohl, wenn wir die Worte nicht umdeuten, der Sache nach auf eine gehörte „Normaldogmatik“ hinauskommen. Plitt z. St.: „Was ich dir gegeben habe, sei dein Original, die gesunden Worte, die gesunde Lehre, wie Tit. 1, 9; 1 Tim. 3, 9, im Gegensatz gegen die verkehrten.“ Matthäs z. St.: „\textit{ἱκανότης}, wie 1 Tim. 1, 13 die ausgewählte Grundform das ihr nur Vorbild, die dargestellte Norm.“ Hutter im Weberschen Kommentar will nur „Bild“ übersehen und fügt hinzu „Wenn der Worte Wiesinger u. a. „\textit{ἱκανότης} geradezu durch Vorbild“, so auch Luther, übersetzen, so ist dies ungenau, da die hierin ausgedrückte Bezeichnung nicht im Worte selbst liegt.“ Aber die Beziehung ist durch den Zusammenhang gefordert. Ein Bild, nach dem man sich richten soll, ist so ipso ein Vorbild.

\footnotetext[178]{5. Bierbaum, Dogmatik I.}