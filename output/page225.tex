heimnis wäre auch dann völlig gelöst, wenn wir als Grund der Nichtbekehrung mit Calvin an die Stelle der gratia universalis die gratia particularis setzen dürften. Aber auch diese Lösung ist verboten, weil sie uns in Widerspruch zu all den Schriftauslagen bringt, die so klar und gewaltig die gratia universalis, seria et efficax lehren und die auf Bekehrung abzielende Wirkung des Heiligen Geistes ausdrücklich auch auf die Menschen ausdehnen, welche nicht bekehrt und selig werden. Darum ist angesichts der gleichen Gnade Gottes und angesichts des gleichen gänzlichen Verderbens der Menschen auf eine Vernunftgemäße („Erkenntnismäßige“) Beantwortung der Frage: Cur alii, alii non? in diesem Leben zu verzichten. Mit andern Worten: Es ist die einzig rechte gottgewollte Theologie, an diesem Punkte ein in diesem Leben unlösbares Geheimnis anzuerkennen. So Luther und Chemnitz. So auch sehr ausgesprochen die Konkordienformel, die nach Angabe der Tatsache: „Einer wird verstockt, verblendet, in verkehrten Sinn gegeben, ein anderer, so wohl in gleicher Schuld, wird wiederum bekehrt“ ausdrücklich hinzufügt, dass hier eine „Frage“ vorliege, deren Beantwortung in diesem Leben unmöglich sei, weil die Schriftsoffenbarung über Hos. 13,9 nicht hinausführe („Israel, dass du verdirbst, die Schuld ist dein; dass dir geholfen wird, ist lauter meine Gnade“), und sehr bestimmt darlegt, dass wir nach der Schrift bei einer Vergleichung der Seligwerdenden mit den Verlorengehenden bei den ersteren die selige Schuld und das gleich üble Verhalten lehren müssten wie bei den letzteren. Dagegen ist aus den vorhin genannten lutherischen Synoden sonderbarerweise behauptet worden: „Es ist nicht wahr, dass die lutherische Kirche die Frage, warum bei dem einen Menschen Tod und Widerstreben weggenommen wird, bei dem andern nicht, unbeantwortet lässt. Es ist nicht wahr, dass die Lutheraner diese Frage niederschlagen.“ „Dass von zwei Menschen, welche das Evangelium hören, bei dem einen Widerstreben und Tod weggenommen wird, bei dem andern nicht – das hat seinen Grund in dem Willen des Menschen, es hat seinen Grund darin, dass der eine der Gnade Gottes beharrlich, hartnäckig und mutwillig widerstrebt, während der andere sein natürliches Widerstreben vom heiligen Geist überwinden lässt. Es hat seinen Grund in der freien Selbstentscheidung des Menschen.“ Gott „lässt es von der Entscheidung des Menschen abhängen, wessen er sich erbarmen und wen er verstocken wird“. „Wir werden den Grund davon, dass bei dem einen Menschen Widerstreben und Tod weggenommen wird,