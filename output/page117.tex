\noindent Wesen und Begriff der Theologie. \hfill 106
baren Problemen gehört auch die sogenannte crux theologorum, nämlich die Frage, die sich, wie die Konkordienformel erinnert, angesichts der Tatsache erhebt: „Einer wird verstockt, verblendet, in verkehrten Sinn gegeben, ein anderer, so wohl in gleicher Schuld (\textit{in eadem culpa}), wird wiederum bekehrt.“\textsuperscript{388} Die Konkordienformel warnt davor, diese Frage in diesem Leben beantworten zu wollen. Sie verweist die Beantwortung in das ewige Leben.\textsuperscript{389} Das rechte Urteil wird also dahin lauten: „Die Theologen, die wirklich offene Fragen in der Theologie schließen oder wirklich theologische Probleme lösen wollen, handeln a. schriftwidrig, weil sie nicht bei 1 Petr. 4, 11 bleiben: \foreignlanguage{greek}{ἐν ταῖς λαλῇ, ὡς λόγια Θεοῦ}; b. unwissenschaftlich, weil sie ein Wissen vorgeben, das sie nicht besitzen. Nach Joh. 8, 31. 32 vermittelt sich alle christliche Wahrheitserkenntnis durch das Bleiben an Christi Wort. Was ohne das Bleiben an den gesunden Worten unsers Herrn Jesu Christi für Wahrheitserkenntnis ausgegeben wird, gehört nach 1 Tim. 6, 3 in das Gebiet der Einbildung oder des Nichtwissens (\foreignlanguage{greek}{κενοδοξία, μὴ ἐπισταμένος}).“

Sive, primos parentes lapsu suo meritos esse, ut quales ipsi erant post lapsum et corpora et anima, tales progerminentur omnes posteri. Quomodo autem malum illud contineat minime, salva fide pessi ignorari, quia Spiritus Sanctus non voluit hoc vertere et perspicuis Scripturae testimoniis antefacere. Agit auch Baiers \textit{Juris dogmatici}sche Bemerkungen I, 67, nota c; ferner Lutgardt, \textit{Dogmata} II, S. 168 f.

\vspace{1em}

\noindent 388) M.~716, 57.

\noindent 389) Der Versuch, diese Frage zu beantworten, hat einerseits zum Calvinismus (Leugnung der \textit{universalis gratia}), andererseits zum Semipelagianismus und Synergismus (Leugnung der \textit{sola gratia}) geführt. Semipelagianisch ist, dass nicht die Frage a) ob \textit{in de me} (ob der Mensch) vom \textit{cur alii alii non} (von wem; ein anderer nicht; warum nicht) und so ziemlich in allen Zeiten (auch von Luther und den alten lutherischen Theologen) gestellt worden ist, eine Sünde und einen Irrtum involviere. Gefährlich und Irrtum geleugnet haben, die, welche mit Melanchthon, dem Vater des Synergismus innerhalb der lutherischen Kirche, die Frage vermittelt des „verschiedenen Verhaltens“ beantwortet haben. Auch die Konkordienformel reiht ihre Aussage: „Einer wird verstockt usw., ein anderer, so wohl in gleicher Schuld, wird wiederum bekehrt“ und die Kraken ein, wenn sie sagt: „In diesen und dergleichen Fragen“ ein. Sie warnt aber vor der Beantwortung der Frage, wenn sie hinwegsetzt: „An diesen und dergleichen Fragen setzt uns Paulus ein gewisses Ziel (\textit{certas metas}), wie fern wir gehen sollen, nämlich nur so weit, dass wir bei dem einen Teil erkennen sollen Gottes Gericht, \ldots was wir alle wohl verdient hätten“, und dass wir, der andere Teil, „Gottes Güte ohne und wider unser Verdienst \ldots erkennen und preisen, weil Gott uns, die wir doch in gleicher Schuld sind, „nicht verstockt und verworfen“. Vgl. die ausführliche Darlegung bei der Lehre von der Bekehrung (II, 285 ff.) und bei der Lehre von der Gnadenwahl (III, 566 ff.).