\section*{Wesen und Begriff der Theologie.}

Ein Teil der Zitate, die Walther seiner Ausgabe von Baier eingefügt hat, sind auch als Korrekturen des Baier'schen Textes gemeint. Was \glqq die milde Orthodoxie des Muzäus\grqq{} betrifft, die bei Risch-Stephan Baier zugeschrieben wird und die sich namentlich in synergieistischen Redeweisen Baires zeigt, pflegte Walther damit zu entschuldigen, dass Baier Muzäus' Schwiegersohn war.\footnote{614) Z\textit{ur} Einigung der amerikanisch-lutherischen Kirche 2, S. 38.}

Ein Verständnis ist in Walthers Worten der kirchliche Zustand unserer amerikanisch-lutherischen Kirche, die angeblich einer bloßen \glqq Repristinationstheologie\grqq{} samt \glqq mechanischer Christauffassung\grqq{} und \glqq toter Orthodoxie\grqq{} verfallen ist, den Tatsachen entsprechend beschrieben. Der Verfasser dieser Dogmatik konnte nicht umhin, sich davon zu überzeugen, weil er in einem Zeitraum von mehr als vierzig Jahren Gelegenheit hatte, Hunderten von Gemeindeversammlungen, Pastoralkonferenzen und Synodalversammlungen beizuwohnen. Dass während der ganzen Zeit mehr oder weniger auch stets Schwächen und Gebrechen zutage traten und auch jetzt zutage treten, versteht sich in der \textit{ecclesia militans} von selbst. Wir möchten noch auf einen Punkt hinweisen, der sich auf die Einigkeit in der Lehre bezieht. Diese völlige Übereinstimmung in der Lehre hat hierzulande und auch in Deutschland Anstoß erregt und ist oft recht unsachlich kommentiert und sogar als Resultat der Beugung unter die Autorität eines Mannes dargestellt worden. Nichts kann verkehrter sein. Wir haben die meisten Väter der Synode noch persönlich gekannt. Es waren nicht nur grundverschiedene, sondern zum Teil auch sehr starke und selbstständige Charaktere, so dass man, menschlich zu reden, erwarten konnte, sie würden sehr bald in verschiedenen Richtungen auseinanderfahren. Dass dies nicht geschah, ist uns je länger, je mehr als ein Zeugnis für die einigende Kraft des Wortes Gottes erschienen. Auch die verschiedenen politischen Ansichten zur Zeit des amerikanischen Bürgerkrieges, die sich hie und da stark bemerklich machten und auch in öffentlichen Versammlungen hervortraten, konnte die durch Wirkung des Heiligen Geistes auf der Schrift beruhende Einigkeit des Glaubens nicht zerstören. Man rief sich gegenseitig zu: \glqq Die Politik hat uns nicht zusammengeführt, sie soll uns auch nicht auseinandertreiben.\grqq{}

Es möchte als nicht ganz passend erscheinen, dass in einer dogmatischen Schrift die kirchliche Lage in unserer amerikanisch-lutherischen Kirche etwas ausführlich geschildert wird. Indessen ist erstlich zu beheben, dass unserer Kirche in der neueren und neuesten