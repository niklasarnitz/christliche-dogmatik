\centerline{\large Inhaltsangabe.}
\centerline{XI}

\bigskip

\centerline{\textbf{Die Engel. (De Angelis.)}}
\noindent 1. Die Existenz der Engel und die Zeit ihrer Erschaffung, S. 603. --- 2. Der Name der Engel, S. 603. --- 3. Beschaffenheit und Fähigkeiten der Engel, S. 604. --- 4. Zahl der Engel und Unterschiede unter denselben, S. 609. --- 5. Gute und böse Engel, S. 610. --- 6. Die guten Engel und ihre Verrichtungen, S. 611. --- 7. Die bösen Engel, ihre Verrichtungen und ihre ewige Strafe, S. 613.

\bigskip

\centerline{\textbf{Die Lehre vom Menschen. (Anthropologia.)}}

\subsubsection*{\textbf{A. Der Mensch vor dem Fall (De statu hominis ante lapsum):}}
\begin{enumerate}
    \item Die Erschaffung nach dem göttlichen Ebenbilde, S. 617.
    \item Der Inhalt des göttlichen Ebenbildes, S. 618.
    \item Ebenbild Gottes im weiteren und eigentlichen Sinne, S. 621.
    \item Das Verhältnis des göttlichen Ebenbildes zur menschlichen Natur, S. 622.
    \item Unmittelbare Folgen des göttlichen Ebenbildes im Menschen, S. 624.
    \item Der Endzweck des göttlichen Ebenbildes, S. 625.
    \item Das Weib und das göttliche Ebenbild, S. 626.
\end{enumerate}

\subsubsection*{\textbf{B. Der Mensch nach dem Fall (De statu peccati).}}
\paragraph*{\textit{Die Sünde im allgemeinen (De peccato in genere):}}
\begin{enumerate}
    \item Der Begriff der Sünde, S. 631.
    \item Gesetz und Sünde, S. 633.
    \item Die Erkenntnis des göttlichen Gesetzes, das alle Menschen verbindet, S. 635.
    \item Die Ursache der Sünde, S. 638.
    \item Die Folgen der Sünde, S. 641.
\end{enumerate}

\subsubsection*{\textbf{C. Die Erbsünde (De peccato originali):}}
\begin{enumerate}
    \item Der Begriff der Erbsünde, S. 645.
    \item Die Wirkung der Erbsünde auf den Verstand und Willen des Menschen, S. 652.
    \item Die negative und positive Seite des Erbsündenerbens, S. 656.
    \item Das Subjekt des Erbsündenerbens, S. 659.
    \item Die Folgen des erbsündlichen Verderbens, S. 661.
\end{enumerate}

\subsubsection*{\textbf{D. Die Tatsünden:}}
\begin{enumerate}
    \item Name und Begriff der Tatsünden, S. 669.
    \item Die Ursachen der Tatsünden: Causae peccati actualis intra hominem, S. 670; causae peccati actualis extra hominem, S. 671.
    \item Die Schriftlehre vom Ärgernis, S. 672.
    \item Die Schriftlehre von der Versuchung, S. 674.
    \item Einteilungen und Benennungen der Tatsünden, S. 675 (u. Untertheilung der Tatsünden nach der verschiedenen Besitzung des menschlichen Willens, S. 676);
    \begin{itemize}
        \item[f.] Die peccata actualia im Verhältnis zum Gewissen, S. 677;
        \item[c.] Einteilung der Sünden nach dem Objekt, S. 678;
        \item[d.] Einteilung der Sünden nach dem Grad, S. 678;
        \item[e.] peccata mortalia, et venialia, S. 680;
        \item[f.] herrschende und nicht herrschende Sünde, S. 681;
        \item[g.] Die Teilnahme an fremden Sünden, S. 681;
        \item[h.] himmelschreiende Sünden [peccata clamantia], S. 682;
        \item[i.] Die Sünde wider den heiligen Geist, S. 683.
    \end{itemize}
\end{enumerate}