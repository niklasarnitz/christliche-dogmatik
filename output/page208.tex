197\n\n\begin{center}\nWesen und Begriff der Theologie.\n\end{center}\n\nnahelegen, es möchte das, was sich in der amerikanisch-lutherischen\nKirche bewährt hat, auch in andern Ländern als das rechte Kirchbaumittel\nsich bewähren. Experto crede Ruperto, pflegt Luther zu sagen, wenn er\ndie Kraft des Wortes Gottes preist. Gesetz den Fall, dass z. B. in\nDeutschland die theologischen Lehrer und die Pastoren, anstatt Luther\nund die Dogmatiker anzuklagen, durch Gottes Gnade zu deren\ntheologischer Methode zurückkehren würden (sola Scriptura und\nsola gratia, resp. satisfactio vicaria), so würde der göttlichen Verheißung\ngemäß auch in Deutschland wieder ein wahres lutherisches Kirchenwesen\nentstehen, und an die Stelle der Lehrverwirrung würde Übereinstimmung\nin der Lehre treten; denn: „Εὰν ὑμεῖς μείνῃτε ἐν τῷ λόγω τῷ ἐμῷ, ...\nγνώσεσθε τὴν ἀλήθειαν.“ Zurzeit ist ja auch in Deutschland die Trennung\nvon Kirche und Staat offiziell ausgesprochen worden. Dies hat die Frage\nin den Vordergrund gedrängt, wie sich die Kirche Deutschlands zur\nSicherung ihres Lebens unter den veränderten Umständen neu einrichten\nhabe. Man hat an die bischöfliche Verfassung gedacht und sie zum Teil\nbereits eingeführt. Gegen die bischöfliche Verfassung an sich ist ja\nnichts einzuwenden; aber ohne Rückkehr zu dem Wort der Apostel und\nPropheten als Gottes unfehlbarem Wort fehlt der Grund, auf dem\ndie christliche Kirche erbaut ist, und sind auch die „Bischöfe“ nur\nein Dekorationsstück, das die traurige Sachlage in der Kirche verdeckt.\nGegenwärtig geht durch lutherische Länder und lutherische Landesteile\ndie Klage, dass nicht nur die große römische Sekte, sondern auch die\nverschiedenen reformierten Sekten sich besonders eifrig der Propaganda\nwidmen. Ohne Rückkehr zur Schrift ist auch die Kirche Deutschlands\nnicht nur der Propaganda Roms gegenüber machtlos, sondern auch der\nPropaganda solcher reformierten Sekten nicht gewachsen, die neben\nIrrtümern noch die Schrift als Gottes Wort gelten lassen und – auch\nnoch die satisfactio vicaria lehren. Durch Preisgebung der Schrift als\ndes Wortes Gottes und durch die damit verbundene Preisgebung der\nsatisfactio Christi vicaria haben die modernen Theologen Deutschlands\ndie Waffen der christlichen Kirche den Feinden ausgeliefert und sind so\nipso ebenso machtlos gegen Rom und die Sekten, wie das politische\nDeutschland nach Auslieferung der Waffen ein Spielball der Willkür\nseiner Feinde ist. Die Theologie Deutschlands muss zu der Theologie\nzurückkehren, die sie an der „streng konfessionellen amerikanisch-lutherischen\nKirche“ als Repräsentationstheologie verwarf. Übrigens sollten wir in\ndiesem Zusammenhang daran erinnern, dass diese