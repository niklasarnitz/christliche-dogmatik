Theologen rät, sich alle Gedanken, die ihm \glqq ohne Schrift\grqq{} in irgendeiner Lehre eingefallen sind, sich möglichst schnell wieder ausfallen zu lassen, sondern das sagt er gerade auch dort, wo er die christliche Lehre mit einer \glqq goldenen Kette\grqq{} und mit einem \glqq geschlossenen Ringe\grqq{} vergleicht. Dieser Beschreibung der christlichen Lehre bei Luther liegt stets die Voraussetzung zugrunde, dass jedes einzelne Glied der \glqq goldenen Kette\grqq{} direkt durch das Schriftwort gegeben ist. Luther sagt nämlich hinzu, dass jeder, der irgendeinen Artikel der christlichen Lehre leugne, damit Gott in seinem Wort leugne und zum Lügner mache. Luther schreibt:\footnote{565) St. \OE. XX, 1781.} \glqq Gewiss ist's, wer einen Artikel nicht recht glaubt oder nicht will (nachdem er vermahnet und unterrichtet ist), der glaubt gewisslich keinen mit Ernst und rechtem Glauben. Und wer so kühn ist, dass er darf Gott leugnen oder Lügen strafen in einem Wort [der Schrift], und tut solches mutwilliglich wider und über das, so er eins oder zweimal vermahnet oder unterweiset ist, der darf auch (tut's auch gewisslich) Gott in allen seinen Worten leugnen und Lügen strafen. Darum heißt's, rund und rein, ganz und alles geglaubt oder nichts geglaubt. Der heilige Geist [das die ganze Schrift ist, III, 1890] lässt sich nicht trennen noch teilen, dass er ein Stück sollte wahrhaftig und das andere falsch lehren oder glauben lassen. Ohne wo Schwache sind, die bereit sind, sich unterrichten zu lassen und nicht halsstarrig zu widersprechen.\grqq{} Luther kannte genau auch die Hofmannsche Konstruktionsmethode, deren Eigentümlichkeit, wie wir gesehen haben, darin besteht, erst die Lehre unter sorgfältiger Beiseitelegung der Heiligen Schrift aus dem eigenen Innern zu konstruieren, um dann nachträglich das Lehrprodukt in der Schrift aufzusuchen und nach der Schrift zurechtzustellen. So lauteten auch reformierte Sakramentsschwärmer im Streit um die Abendmahlslehre Luther die theologische Methode zu, die Schriftworte vom Abendmahl zunächst ganz aus den Augen zu tun, die Abendmahlslehre aus dem \glqq Glauben\grqq{} zu nehmen und danach dann die Schriftworte vom Abendmahl \glqq auszulegen\grqq{}. Luther äußert sich etwas kräftig über diese Konstruktionsmethode. Er sagt:\footnote{566) St. \OE. XX, 780. 782.} \glqq Es ist der Übermut des leidigen Teufels, der uns spottet durch solche Schwärmer in dieser großen Sache, dass er vorgibt, er wolle sich mit der Schrift weisen lassen so ferne, dass er die Schrift zuvor aus dem Wege tue.\grqq{} Luther seinerseits besteht darauf,