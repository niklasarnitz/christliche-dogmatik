innerhalb der kirchlichen Gemeinschaft, die die offiziell das äußere Wort Christi als Medium der Sündenvergebung verwerfen, dennoch gläubige Kinder Gottes vorhanden sind, kommt daher, dass es immer eine Anzahl Seelen gibt, die im Widerspruch mit der offiziellen Lehre ihren Glauben auf das äußere Wort gründen.\footnote{338}

5. Die Schrift spricht den christlichen Glauben allein ab, die leibliche Auferstehung der Toten und das ewige Leben leugnen. Bekanntlich wollen neuerdings auch Führer des Interchurch World Movement Kirche und Welt bereden, es sei genug, in diesem Leben an Christum zu glauben; das hereafter, die Auferstehung samt Himmel und Hölle, könne man auf sich beruhen lassen.\footnote{339} So im wesentlichen auch die liberal-protestantische Theologie, die die Loslösung der christlichen Lehre von der heiligen Schrift konsequenter vollzieht als die „positive“ Richtung. Horst Stephan meint vom Auferstehungsglauben: „Er ist von einem Interesse am menschlichen Leid und seinen verklärten Eingehnen in die Ewigkeit erfüllt, das mit dem christlichen Glauben keinen unbedingt notwendigen Zusammenhang hat, sondern manchmal eher als eine Nachwirkung des jüdischen Vergeltungsglaubens erscheint.“\footnote{340} Die Heilige Schrift sagt von denen, die, wie Humenäus, Alexander und Philetus, die zukünftige leibliche Auferstehung der Toten ablehnten und „vergeistigen“ wollten (\textgreek{λέγοντες τὴν ἀνάστασιν ἤδη γεγονέναι}), dass sie am Glauben Schiffbruch erlitten und der Wahrheit geschifft hätten (\textgreek{περὶ τὴν πίστιν ἐναυάγησαν} --- \textgreek{περὶ τὴν ἀλήθειαν ἠστόχησαν}).\footnote{341} Und was ihr Verhältnis zur christlichen Kirche betrifft, so spricht der Apostel ihnen die Gliedschaft in der Kirche ab mit den Worten: \textgreek{οὐδὲ παρέδωκα τῷ σατανᾷ}, damit sie gezüchtigt werden und nicht

substantielle (Christus) und dem Fundamentum organicum (das Wort des Evangeliums von Christus) unterscheiden, so lehren sie damit nicht ein doppeltes Fundament des Glaubens. Dollaz (Examen, Proleg., c. 2, qu. 19) weist ausdrücklich darauf hin, dass Christus, das Fundamentum substantiale, \emph{unvermittelt} des Wortes von Christus, des Fundamentum organicum, zum Fundament des Glaubens wird. Anders ist neuerdings Zeugen. Weil sie das Wort der Apostel und Propheten Christi als Gottes Wort ablehnen, so wollen sie den Glauben auf „Christi Person“, auf „den lebendigen Christus“ und „gründen unter Verleihung des Wortes der Schrift. Aber wer an Christi Wort vorbeiglaubt, glaubt eo ipso auch an den „lebendigen Christus“ vorbei.

\footnote{338} Die nähere Darlegung II, 535 ff. Über die „glückliche Antikonsequenz“ in den reformierten Gemeinschaften III, 188 ff.
\footnote{339} Die Belege in v. u. Bl. 67, 1 ff.
\footnote{340} Glaubenslehre, 1921, S. 119.
\footnote{341} I Tim. 1, 19. 20; 2 Tim. 2, 17. 18.