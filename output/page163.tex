zu hölzern und äußerlich sein; in etwas verdeckter Weise zieht man wieder die Werke heran. Gesetz und Evangelium mengt man wieder zusammen. Das Wort Gottes und die Predigt wird so zurückgestellt, als wenn die Sakramente die Hauptsache tun, jedenfalls erst Leben in die Kirche bringen müssten. Die sichtbare Kirche kommt wieder zu solcher Wichtigkeit, als wenn sie die wahre Kirche, die Inhaberin aller Verheißungen Gottes wäre. Und was soll ich zu dem Verhältnis der Kirchen, von Amt und Regiment, von Chiliasmus und ewigem Leben sagen? Die Streitfragen liegen vor jedermanns Augen, und wenn der Streit nicht etwas auf sich hätte, so würde er nicht so heftig sein. 	extdots Ich sehe den Fall, dass wir in allen diesen aufgezählten oder nicht aufgezählten Abweichungen und Veränderungen einig wären, würde das noch lutherische Lehre heißen können, oder würde man den Mut haben, das Fortbildung der lutherischen Lehre zu nennen, was die wesentlichsten Stücke der lutherischen Lehre wie alten Schutt hinausfegt? Ich wenigstens würde nicht das Herz haben, mich einen Lutheraner zu nennen, und würde offen gestehen: Wir sind allesamt abgewichen. 	extdots Es lässt sich freilich ziemlich sogar erwarten, dass die wissenschaftliche Theologie in nicht gar ferner Zeit ihren Kredit verlieren wird. Während die übrigen Wissenschaften sich mit ihren wahren und unleugbaren Fortschritten die Achtung der Welt erringen, weist die Theologie die grenzenloseste Verwirrung auf, und indem sie fortschreitet, weiß niemand recht, worin der Fortschritt besteht, da einer des andern Fortschritte als Rückschritte bezeichnet und die Kirche von allen Fortschritten nicht nur keinen Gewinn, sondern nur Streit und Beulen und Wunden aufzuweisen hat. So ist es gekommen, dass die übrigen Wissenschaften ein gemeinsames Band um alle gebildeten Völker geschlungen haben und alle Kräfte in ihren Dienst nehmen, indes die Theologie aller Art zersplittert und zerteilt, die doch ihrem Beruf und ihrem Stoffe nach einigen sollte in dem einen Heile, welches allen Völkern bestimmt ist. Das ist ein sehr kläglicher und niederschlagender Anblick, der wahrlich nicht dazu ermutigen wird, sich den erregenden theologischen Wissenschaft anzuvertrauen. Im Jahre 1870 urteilte die Berliner, von Hengstenberg gegründete, vom Tauscher fortgesetzte „Ev. Kirchenzeitung“	extsuperscript{555)}, „Wir müssen sagen, dass die gegenwärtige Zeit mit ihrer theologischen und kirchlichen Zerrissenheit zur Entwicklung der kirchlichen Lehre am wenigsten geeignet ist. Sind doch sogar die kirchlichsten [!] Theologen derootnote{	extsuperscript{518)} Zitiert in S. u. W. 1875, S. 70.}