nach dem Sinn des Papstes, ausgelegt werde.ootnote{85) So das Tridentinum. Sess. IV, Decretum de editione et usu sacrorum librorum. Schmetz, S. 15.} Bei dieser Deutung der Schrift nach dem Sinn der „heiligen Mutter Kirche“, resp. des Papstes, kommt die römische Partei dahin, dass sie die Zentrallehre der christlichen Religion, die Rechtfertigung des Menschen aus dem Glauben an das Gnaden evangelium ohne des Ge-setzes Werke, ausdrücklich und sehr entschieden mit dem Fluch belegt.ootnote{86) Tridentinum, Sess. VI, can. 11. 12. 20. Die weitere Darlegung unter dem Abschnitt „Die Hauptsache und die Lehre von der Rechtfertigung“, II, 667 ff.} Dass es trotzdem innerhalb der römischen Partei noch Christen gibt, kommt daher, dass immer einzelne Seelen wider der „Kirche“ Verbot, namentlich in der Anfechtung, das Vertrauen auf eigene Werke fahren lassen und ihre Zuversicht vor Gott allein auf die Gnade Gottes in Christo setzen.ootnote{87) Apologie 151, 271: Etiamsi in ecclesia pontificiis aut nonnulli theologi ac monachi docuerunt, remissionem peccatorum, gratiam et iustitiam per nostra opera et novos cultus quaerere, ... mansit tamen apud aliquos pios semper cognitio Christi.} Aber dabei bleibt die Tatsache bestehen: Dass die römische Kirche als solche in der äußeren Christenheit eine besondere Partei bildet, ist erstlich darin begründet, dass sie Christi Wort, das Wort der Propheten und Apostel, die Heilige Schrift, nicht zur Geltung kommen lässt, sondern an die Stelle des christlichen Erkenntnisprinzips tatsächlich die Lehrbestimmungen der sancta mater ecclesia, das ist, des Papstes, setzt. Oder wie Luther es in den schmalkaldischen Artikeln adäquat ausdrückt:ootnote{88) M. 321, 4.} „Der Papst rühmet, alle Rechte sind im Schrein seines Herzens (in scrinio sui pectoris), und was er mit seiner Kirche urteilt und heißt, das soll Geist und Recht sein, wenn’s gleich über und wider die Schrift oder das mündliche Wort ist.“ Mit der tatsächlichen Beiseitelegung der Schrift ist dann auch die Proskription der christlichen Gnadenlehre gegeben. Die ganze große Maschinerie der römischen Partei ist auf Werklehre und die Autorität des Papstes eingestellt. Den Fall gesetzt, dass diese beiden Faktoren aufgegeben würden, so würde die römische Partei aus der äußeren Christenheit verschwinden.
ewline Die reformierten Kirchengemeinschaften geben in thesi ebenfalls die göttliche Autorität der Schrift zu, und zwar mit starker Betonung der Inspiration der Schrift, namentlich bei den älteren, aber auch bei neueren Reformierten.ootnote{89) Gaufen, Kunze, Vöhl, Thedd, Hodge.} Es ist weithin Mode