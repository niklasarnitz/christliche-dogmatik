Wesen und Begriff der Theologie. \\hfill 58\\n\\nsagt wird, davon nimmt der Theologe, weil es sein Beruf in der Gegenwart mit sich bringt, sorgfältig Notiz, aber um es als wertlose menschliche Anschauung abzuweisen, sofern es nicht mit Gottes eigenem Schöpfungsbericht, den wir in der Schrift haben, stimmt. Der christliche Theologe lehrt über den \\emph{Sündenfall} und die Sünde nicht mehr und nicht weniger, als was Gott darüber in der Heiligen Schrift berichtet, lehrt und urteilt. Der christliche Theologe hat auch an diesem Punkte von einer ganzen Anzahl menschlicher Ansichten über die Entstehung der Sünde und über den Charakter der Sünde und ihrer Folgen Kenntnis zu nehmen. Aber er, seinerseits, bringt alles unter die Antithese, was Gottes Tiefe in der Schrift, die nicht gebrochen werden kann, widerspricht. Der christliche Theologe lehrt von der \\emph{Erlösung} in der Sünde gefallenen und der Gott schuldig gewordenen Menschheit, nämlich von des ewigen Sohnes Gottes Menschwerdung, Person und Werk, nur das als christliche Lehre, was Gott selbst über diese großen Dinge ($\\tau\\dot{\\alpha}$ $\\mu\\varepsilon\\gamma\\dot{\\alpha}\\lambda\\alpha$ τοῦ $\\theta\\varepsilon oῦ$) lehrt, die ja nie in eines Menschen Herz kamen ($\varepsilon\\nu$ καρδίᾳ ἀνθρώπου οὐκ ἀνέβη)\\footnote{205) 1 Kor. 2, 9; Röm. 1, 18.}, sondern von der Welt her verschwiegen waren, nun aber offenbart sind durch der Propheten Schriften auf Befehl des ewigen Gottes (διὰ τε γραφῶν προφητικῶν κατ’ ἐπιταγὴν τοῦ αἰωνίου θεοῦ)\\footnote{206) Röm. 16, 25. 26. Noch ausführlicher Eph. 3, 7--12.}. Sonderlich bei der Lehre von der Erlösung sieht sich der christliche Theologe der Gegenwart vor die Tatsache gestellt, dass die moderne Theologie die göttliche Erlösungsmethode, insonderheit die stellvertretende Genugtuung Christi (\\emph{satisfactio vicaria}), kritisiert, für „zu juristisch“ erklärt und ablehnt. Aber diese von der menschlichen Anschauung aus an der göttlichen Erlösungsmethode geübte Kritik hat auf den christlichen Theologen nur die Wirkung, dass er um so entschiedener die Erlösung, die nach der Schrift Alten und Neuen Testaments durch die stellvertretende Genugtuung Christi geschehen ist,\\footnote{207) Die ausführlichste Darstellung unter dem Abschnitt „Die stellvertretende Genugtuung“ II, 407--454.} aus der Schrift vorträgt. Und was den articulus stantis et cadentis ecclesiae, die Lehre von der Erlangung der \\emph{Rechtfertigung} vor Gott, betrifft, so lehrt der christliche Theologe, dass der Mensch die Vergebung seiner Sünden erlangt durch den Glauben ($\pi\\acute{\\iota}\\sigma\\tau\\varepsilon\\iota$), das ist, durch den Glauben an das \\emph{Evangelium}, welches die Vergebung der Sünden um Christi stellvertretender Ge-