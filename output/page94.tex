{\hfill 83\par}\section*{Wesen und Begriff der Theologie.}\par{}Sein Apostel als Gottes eigenes, unfehlbares Wort gelten lässt.\footnote{Joh. 10, 35; Matth. 4, 4. 7. 10; Luk. 24, 25. 44—46; Joh. 17, 12; Matth. 26, 54; Röm. 16, 25. 26; 2 Tim. 3, 15. 16; 2 Thess. 2, 15.}\par{}Aus dieser Erkenntnis ergibt sich dann von selbst die einzig richtige theologische Methode, nämlich die Methode, nach welcher wir die christliche Lehre weder aus dem Ich anderer Menschen noch aus dem eigenen Ich beziehen, sondern die Schrift voll und ganz als \textquotedblleft göttliches Religionslehrbuch\textquotedblright{} anerkennen und behandeln und von Herzen dem Diktum zustimmen: Quod non est biblicum, non est theologicum. Es verschwinden dann auch die hässlichen Scheltworte von Buchstabentheologie, Intellektualismus, Biblizismus, vom papierenen Papst samt dem vom Himmel gefallenen Lehrgesetz usw. An die Stelle dieser Scheltreden tritt die Erkenntnis und das Bekenntnis, dass die aus der Schrift geschöpfte Lehre, weil sie Gottes eigene Lehre, \textit{doctrina divina}, ist, auch sicherlich instand sein wird, sich innerliche Anerkennung zu verschaffen und so anstatt \textquotedblleft sole orthodoxie\textquotedblright{} leidendiger, \textquotedblleft lebenswarmes\textquotedblright{} Christentum zu vermitteln. Ebenso verschwindet der in mehrfacher Form auftretende Spott über \textquotedblleft reine Lehre\textquotedblright{}. An die Stelle des Spottes tritt die Erkenntnis, dass reine Lehre (\textgreek{ὑγιαινούσῃ διδασκαλίᾳ})\footnote{2 Tim. 1, 13: \textgreek{ὑποτύπωσιν ἔχε ὑγιαινόντων λόγων, ὧν παρ’ ἐμοῦ ἤκουσας}; Tit. 1, 9: \textgreek{δύνατος \ldots παρακαλεῖν ἐν τῇ διδασκαλίᾳ τῇ ὑγιαινούσῃ}.}\par{}die einzige Art von Lehre ist, die nach göttlicher Ordnung einem christlichen Lehrer anständig ist.\par{}Diese Erkenntnis schafft vornehmlich drei Tugenden im Theologen:\begin{enumerate}\item Er wird an einer eigenen Weisheit verzagen und in der demütigen Gesinnung an die Schrift herantreten, die durch das Wort \textquotedblleft Mede, Hörr, dein Knecht höret!\textquotedblright{} zum Ausdruck kommt.\item Er wird, weil die Schrift auch für den Theologen das einzige Lehrbuch der christlichen Lehre ist, die in der Schrift geoffenbarte göttliche Lehre durch den \textit{Glauben} als das einzige \textit{medium cognoscendi} in sich aufnehmen und treu wiedergeben. Er wird Gott bitten, ihn davor zu bewahren, dass er das Stroh eigener Einfälle in den Weizen der göttlichen Gedanken menge (Jer. 23, 28).\item Er wird durch Gottes Gnade Mut und Freudigkeit gewinnen, für die aus der Schrift geschöpfte Lehre, weil sie Gottes Lehre ist, \textit{Alleinberechtigung} zu beanspruchen und so an keinem Teil den Indifferentismus und damit den Chaos in der Lehre zu wehren. Was den letzten Punkt, den Mut und die Freudigkeit, betrifft, so hat Theodor Kaftan\footnote{Moderne Theologie des Alten Glaubens, 1906. S. 120 f.} gelegentlich den Leuten, die er noch dem \textquotedblleft altteologischen\textquotedblright{} Lager zu=\end{enumerate}