\hfill VII\n\n\section*{Vorwort.}\nIm ersten Studienjahre an wurde die gelebt Klingende theologische Phrase und alle ins Kraut schießende Rhetorik unbarmherzig ausgeschieden und weggeschnitten mit der Begründung, daß der usus didacticus der heiligen Schrift an erster Stelle stehe. Es gelte, stets so zu lehren und zu predigen, daß, soweit der Pastor in Betracht kommt, durch die unverkürzte Predigt des Gesetzes die Sicheren aus ihrer fleischlichen Sicherheit aufgeschreckt und die erschrockenen Gewissen durch das unverklausulierte Evangelium (satisfactio vicaria) der Gnade Gottes und der Seligkeit gewiß werden. Zum besten dienen mußte uns auch der Umstand, daß wir zu allen Zeiten Feinde ringsum hatten, von Rom, den schwärmerischen Sekten und untreuen Lutheranern an bis zu den Unitariern und den christusfeindlichen Logen herab. Dieser Kampf zwang uns zu fortgehender intensiver Beschäftigung mit der christlichen Lehre in den einzelnen Gemeinden, in den Pastoralkonferenzen und bei den Synodalversammlungen. Freilich, wir müßten blind sein, wenn wir nicht auch die Schwächen sehen sollten, die unserer kirchlichen Gemeinschaft stets anhaften. Wir hatten und haben Mühe, in einzelnen Gemeinden die rechte Praxis durchzuführen, resp. aufrechtzuerhalten. Wir haben auch Sezessionen erlebt, die uns tief demütigten. Andererseits sind wir durch Gottes Gnade gewiß, daß die unter uns im Schwange gehende Lehre die in der Schrift geoffenbarte und im lutherischen Bekenntnis bezeugte christliche Lehre ist und daher auf Alleinberechtigung Anspruch machen muß. Von diesem Gesichtspunkt aus will auch diese \emph{Christliche Dogmatik} sowohl in ihren thetischen als auch in ihren antithetischen Darlegungen beurteilt sein.\n\n\vspace{1em}\n\centering\nSOLI DEO GLORIA!\n\par\n\n\vspace{1em}\nSt. Louis, Mo., im April 1924.\n\hfill J. Pieper.