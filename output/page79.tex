\section*{Wesen und Begriff der Theologie. \hfill 68}
sondern eine norma normata, die von dem Ich des Theologen zu
examinieren und zu korrigieren ist. Überhaupt steht es ja so, dass
sich die Funktionen der Schrift, wonach sie beides, Quelle und
Norm der christlichen Lehre, ist, nicht voneinander trennen lassen.
Die Schrift ist nur dadurch und deshalb die Norm der christlichen
Lehre, weil sie deren einzige Quelle ist. Und wer von uns in
der abnormen theologischen Verfassung sich befindet, dass er meint,
zum Zweck der Darstellung der christlichen Lehre die Bibel zunächst
gänzlich aus den Augen tun und statt dessen in das eigene Ich
hineinschauen zu müssen, der wird schwerlich hinterher die Bibel
wieder zu dem Zweck öffnen, sie als korrigierende Norm gegen sein
Ichprodukt zu verwenden. Vielmehr wird er kraft der einmal ein-
geschlagenen falschen Richtung die Bibel hinterher zu dem Zweck
öffnen, sie auf seine Menschengedanken zu sieben. So hat denn
auch Hofmann die versprochene nachträgliche Revision an dem, was
er unter Beiseitesetzung der Schrift „rein für sich, in abgeschlossener
Selbstständigkeit“ in seinem Ich „gefunden“ hatte, nicht vollzogen.
Er hat vielmehr auch solche Teile des Ichprodukts wie die Leugnung
der \emph{satisfactio vicaria} auch gegen warnende Freunde in seinen „Schuss-
schriften“ in geradezu fanatischer Weise verteidigt, wie man
damals sagte: mit dem Zorn der Löwenmutter, der man ihr Junges
rauben will. Es sollte doch jedem Theologen völlig klar sein, dass
wir an diesem Punkte vor einem \emph{aut -- aut} stehen. Entweder lassen
wir die Schrift Gottes eigenes Wort sein und lehren aus ihr, als
der einzigen Quelle und Norm der Theologie, \emph{doctrinam divinam},
oder wir leugnen, dass die Schrift Gottes unfehlbares Wort ist,
unterscheiden in ihr zwischen Wahrheit und Irrtum und lehren aus
unserm Ich heraus in Gottes Kirche unsern eigenen „Herzens-Geist“.
Die göttliche Autorität, die wir der Schrift absprechen, sprechen wir
notwendig unserm eigenen menschlichen Geist zu. Wir schwimmen
im Meer des Subjektivismus. Die menschliche Meinung wird in
der Kirche auf den Lehrstuhl gesetzt. Die Theologie ist nicht mehr
theozentrisch, sondern anthropozentrisch orientiert.

Wie ernstlich die Losagung von der Schrift und damit von
der \emph{doctrina divina} seitens der modernen Theologen gemeint ist, geht
daraus hervor, dass sie nicht bloß defensiv auftreten (z.B. durch die
Behauptung, dass sie nur „alte Wahrheit“ in „neuer Weise“ lehren),
sondern auch zu einer stark ausgeprägten Offensive übergehen.
Sie belegen das Beziehen der christlichen Lehre aus der Schrift
mit einer ganzen Reihe von bösen Namen wie: „Intellektualismus“,