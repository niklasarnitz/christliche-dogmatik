(worden), was über eine bestimmte Lehre in der ganzen Schrift geoffenbart vorliegt. Und dies ist der Grundgedanke der synthetischen Methode. Neuere Theologen zogen die Dogmatiker, die die Lokalmethode befolgen, nach, dass sie den dogmatischen Stoff in Stücke zerschnitten und den gliedlichen Zusammenhang sowie den Fortschritt des Systems nicht zur Anschauung gebracht haben.\textsuperscript{580}
Hingegen lassen sie die von den späteren Theologen befolgte analytische Methode als die Mehr der Wissenschaft entsprechende gelten. Dorner sagt: „Georg Calixt hat die analytische Methode aufgestellt, die bald Anfang, auch bei seinen Gegnern, wie Calov, Dannhauer und Hülsemann, fand. Sie [die analytische Methode] sucht aus einer obersten Wahrheit, dem höchsten Gut des sterblichen Menschen, die einzelnen dogmatischen Sätze als Glieder und Vermittlungen des obersten Zweckes abzuleiten. Dieser oberste Zweck ist die Seligkeit des Menschen im Genusse Gottes.“\textsuperscript{581} Es muss zugegeben werden, dass die analytische Methode formell mehr den Forderungen der modernen Theologie entspricht, die ja ihren wissenschaftlichen Charakter darin sieht, dass die christliche Lehre nicht aus der Heiligen Schrift genommen, sondern aus dem Ich des Theologen entwickelt wird. So erweckt die analytische Methode, auf die Theologie angewandt, allerdings den Schein, als ob sie, anstatt allein aus der Schrift zu schöpfen, die christlichen Lehren durch Vernunftschlüsse aus dem Endzweck ableiten, also finden wollte. Walch pflegte in Vorlesungen zu sagen: „Die analytische Methode sucht die Sache gleichsam zu finden. Wir können aber in der Theologie nicht aus dem Zweck die Mittel erschließen.“ Zur Ehrenrettung der späteren Theologen ist jedoch zu sagen, dass sie die analytische Methode zumeist nur äußerlich befolgen, das heißt, diese Methode nur zur äußerlichen Gruppierung der Lehren verwenden, tatsächlich aber bei der Darlegung der einzelnen Lehren das Schriftprinzip festhalten. Wie dies Baier (1, 79) ausdrücklich betont: \textit{Finis cognitio ex revelatione divina petita natura prior est cognitione mediorum iisdem ex divina rerelatione petita}. Wir will daher auch unter den „altprotestantischen“ Dogmatikern, weil sie das Schriftprinzip festhalten, keinen Unterschied gelten lassen. Er sagt: „Die altprotestantische Dogmatik verfährt trotz der wechselnden Bevorzugung der synthetischen und analytischen Anordnung nach einer
\vspace{1.5em}
\noindent\parbox{\linewidth}{\raggedright\footnotesize
581) So Dorner, Geschichte der protestantischen Theologie, S. 551.\\
582) A. a. O.\\
583) Pieper, Dogmatik, I.
}
