35 Wesen und Begriff der Theologie. Das Urteil der Schrift steht in scharfem Gegensatz zu dem Urteil neuerer Theologen aller Schattierungen, die für die Abweichung von der Schriftlehre nicht nur eine Reihe edler Motive, wie wissenschaftlichen Sinn, Wahrheitssinn usw., annehmen, sondern auch „verschiedene Richtungen in der Kirche für von Gott intendiert und der Kirche nützlich“ erklären. Aber die Schriftaussagen, welche das gegenteilige Urteil zum Ausdruck bringen, sind zahlreich und sehr bestimmt. Wie die Schrift oft und ernst von der Abweichung von der Lehre der Apostel, die christl. eigene Lehre ist, warnt,ootnote{117) Röm. 16, 17.} so weiß sie auch reichlich auf die Beweggründe für dieses Tun hin. In allgemeiner Bezeichnung nennt sie als Beweggrund das eigene Interesse: „Solche dienen nicht dem Herrn Jesu Christo, sondern ihrem Bauche, und durch süße Worte und prächtige Reden verführen sie die unschuldigen Herzen.“ootnote{118) Röm. 16, 18.} Spezifizierend nennt sie: Aufgabenheit in eigener Weisheit,ootnote{119) 1 Tim. 6, 3.} Ehre vor Menschen,ootnote{120) Joh. 5, 44.} Kreuzesscheuen,ootnote{121) Gal. 6, 12.} Neid.ootnote{122) Matth. 27, 18.} Die mildeste Bezeichnung des Beweggrundes ist „Unwissenheit“.ootnote{123) 1 Tim. 6, 3; Joh. 16, 3; 1 Tim. 1, 13.} Die Dualität der Motive bestätigt auch die Kirchengeschichte. Novatian wäre wohl nicht der Vater des Novatianismus und der novatianischen Spaltung geworden, wenn er, und nicht Cornelius, zum Bischof von Rom erwählt worden wäre.ootnote{124) Seeberg, Dogmengeschichte I, 138. So auch F. G. Foster in Concise Dictionary von Jackson, Chambers und Foster: „The theoretical difference grew out of a personal one.“} Auch Zwingli wäre schwerlich als Reformator neben und wider Luther aufgetreten, wenn er nicht gemeint hätte, Luther wäre nur „ein redlicher Ajax oder Diomedes unter viel Nestoren, Ulysses, Menelaen“.ootnote{125) St. G. XX, 1134.} Es liegt nicht außerhalb des Rahmens einer Dogmatik, nachdrücklich auch darauf hinzuweisen, dass die böse Art, welche dem Abfall vom Schriftwort und der Parteibildung zugrunde liegt, nicht bloß Novatian, Zwingli und den Parteistiftern der apostolischen Zeit eigen war, sondern in uns allen sich findet und auch unter uns fortwährend wirksam ist. Wir erfahren es in den einzelnen Gemeinden, in der Synode und in den Verbindungen von Synoden, wie Ehrgeiz, Neid, persönliche Zu- und Abneigungen sich regen und in Parteibildung und Trennung auszuarten drohen. \ 	extit{Multae in ecclesiis haereses ortae sunt tantum odio doctorum.}ootnote{126) Apol. 128, 121.} Es steht