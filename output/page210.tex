verlebte, für die glücklichste Zeit seines Lebens erklärt. Auch das berichtet A. Köhler a.

a.
D.: „Die drei letzten Jahre seines akademischen Studiums, 1832–34, nennt Delitzsch selbst
die glücklichsten seines Lebens: „sie waren die Zeit meiner ersten Liebe, die Frühlingszeit meines
geistlichen Lebens.“ Auch ist darauf hinzuweisen, dass Delitzsch noch längere Zeit nach der Beendigung
seines akademischen Studiums der lutherischen Wahrheit Zeugnis gegeben hat. Sein Biograph in der
Realenzyklopädie berichtet, dass Delitzsch, nachdem er 1835 in Leipzig als Doktor der Philosophie
promoviert hatte, bis zum Jahre 1842 die „gottesdienstlichen Übungen“ der „Stillen im Lande“ leitete,
und dass „die religiöse Richtung dieser Kreise die eines in den Bahnen streng lutherischer Bekenntnis-
treue wandelnden gesunden Pietismus war“. Dies geht auch klar hervor aus einer Schrift, die Delitzsch
im Jahre 1839 zum dreihundertjährigen Reformationsjubiläum der Stadt Leipzig herausgab unter dem
Titel „Luthertum und Lügentum“.

An dieser Schrift ist uns immer interessant und lehrreich gewesen, dass hier Delitzsch in allen Haupt-
punkten die Stellung hochstreitet, die eine wahrhaft christliche Theologie der modernen Theologie gegenüber
einzunehmen hat. Es ist dies aber die Stellung, die unsere amerikanisch-lutherische Kirche „streng konfes-
sioneller Richtung“ von allem Anfang an charakterisierte und dann auch gegen die Beschuldigung der
„Repristination“ in zusammenfassender Weise von Walther in „Lehre und Wehre“ 1875 geschildert
wird. Delitzsch’ Festschrift ist als „mehr praktisch und erbaulich“ klassifiziert worden; sie kann aber auch
zugleich als dogmatisch und als dogmatisch lehrreich bezeichnet werden. Zudem zeigt sie in den Haupt-
punkten eine sachliche Übereinstimmung mit den Vätern der Missouri-Synode, wenn die letzteren auch in
der Regel dieselben Sachen in ruhigerem Ton behandelt. Doch Delitzsch’ Schrift ist ja eine Festschrift.
Der Autor redet die lutherischen Gemeinden Leipzigs an: „Evangelisch-lutherische Gemeinden meiner
teuren geliebten Vaterstadt, nehmt zur bevorstehenden Jubelfeier der in unserer Mitte eingeführten
Reformation auch meinen mit der innigsten Fürbitte verbundenen Festgruß.“ In dem Vorwort sagt
Delitzsch gegen die Anklage, dass er Repristinations-Theologie treibe: „Ich bekenne, ohne mich zu schämen,
dass ich in Sachen des Glaubens um dreihundert Jahre zurück bin, weil ich nach langem

\footnote{619}{Der Gesamttitel lautet: Luthertum und Lügentum. Ein offenes Bekenntnis beim Reformationsjubiläum der Stadt Leipzig. Von Franz Delitzsch. Grimma 1839.}