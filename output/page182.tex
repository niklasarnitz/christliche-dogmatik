Systembildung, die von dem „Reformator der Kirche des 19. Jahrhunderts“, von Schleiermacher, der Theologie des 19. Jahrhunderts eingeimpft worden ist, genau besehen, ein Laienspielerkunststück darstelle. Jene auf der Linie Schleiermacher-Hofmann-Frank usw. gelegene Systembildung komme tatsächlich darauf hinaus, dass Unbegreifliche, das nun einmal von der christlichen Religion nicht getrennt werden könne, entweder ganz zu streichen oder doch zu vermindern, um auf diese Weise den christlichen Glauben einem ungläubigen Geschlecht annehmbar zu machen. Jene allgemein bewunderte Systembildung ist auch mit dem Profkulttheos verglichen worden. Dagegen haben nun auch moderne Theologen zur notwendigen Rettung des Unbegreiflichen im Christentum eine andere Weise empfohlen. Es müsse als Tatsache anerkannt werden, dass alles, was der Glaube in diesem Ton in bezug auf die christliche Lehre aussage, notwendig in der Form des „Irrationalen“ oder des „Paradoxons“ verlaufe. Das klingt gut und scheint mit 1~Kor.~13,9 im Einklang zu stehen: „Unser Wissen [in diesem Leben] ist Stückwerk, und unser Weislagen ist Stückwerk.“ Aber solange die Vertreter der Theologie des „Irrationalen“ nicht zur Heiligen Schrift als Gottes Wort und als der einzigen Quelle und Norm der Theologie zurückkehren, bestimmen sie das „Irrationale“ oder das Überrationale, das sich im Christentum findet, nicht aus der Schrift, sondern aus dem Ich des theologisierenden Subjekts. Es liegt daher nur eine anders benannte Form der menschlichen Systembildung vor. Der \emph{Subjektivismus} ist nicht überwunden, sondern bleibt prinzipiell in Geltung. Das „christliche Ich“ bleibt bei der Methode, sich um die eigene Achse zu drehen. Und wir wollen nicht vergessen, dass dies der Fall ist bei jeder Form des Subjektivismus. Solange wir an die Stelle der sola gratia die Selbstbestimmung, die Selbstliebung, die Selbstentscheidung, das menschliche Verhalten, die facultas se applicandi ad gratiam usw. setzen, und ferner: solange wir an die Stelle der sola Scriptura das „fromme“ Selbstbewusstsein, das religiöse Erlebnis, das religiöse Erkennen, den „Glauben“ usw. des theologisierenden Subjekts setzen, so lange bleiben wir --- auf den wissenschaftlichen Charakter unsers Verfahrens gesehen --- in dem großen Reigen derer, die sich, wie gesagt, od zar \textgreek{ἔννοιαν} um die eigene Achse bewegen. Und das Resultat dieser Zirkelbewegung um das eigene Ich ist Heilsungewissheit und Wahrheitsungewissheit. Daher der