glauben werden. Es gilt also in der Theologie nicht, aus dem Wort des Apostels und Propheten in das theologische Ich zu flüchten, sondern es kommt in der Theologie alles darauf an, dass das theologisierende menschliche Ich, von sich selbst loskomme. Und das geschieht nur in der Weise, dass der Theologe alle eigenen Gedanken und Anschauungen, die sich bei ihm melden, sorgfältig unterdrückt und lediglich solchen Gedanken, Reden und Lehren Heimatsrecht bei sich gestattet, die in Christi Wort ausgedrückt vorliegen. Und das ist nicht unwürdige \glqq{}Knechtschaft\grqq{} und \glqq{}Buchstabendienst\grqq{}, wie man gemeint hat, sondern das ist unsere herrliche Freiheit, die wir als christliche Theologen genießen dürfen. Christus belehrt uns auch darüber Joh. 8, wenn er sagt: \glqq{}So ihr bleiben werdet an meiner Rede \dots{} so werdet ihr die Wahrheit erkennen, und die Wahrheit wird euch freimachen\grqq{}, \foreignlanguage{greek}{ή ἀλήθεια ἐλευθερώσει ὑμάς}. Die schmählichste Menschenknechtschaft, die es in der Welt gibt, ist das Gefangensein in den eigenen irrigen Gedanken. Die Befreiung von dieser Knechtschaft der eigenen irrigen Gedanken in den Dingen, die unsere eigene und aller Menschen Seligkeit betreffen, ist der Zweck, zu dem uns Christus sein eigenes Wort durch seine Apostel und Propheten gegeben hat. Also nicht los von der Schrift, sondern hin zur Schrift und zu ihr allein als Quelle und Norm der Theologie! Luther dankt Gott, dass er ihm die Gnade verliehen habe, sich alle Gedanken, die ihm \glqq{}ohne Schrift\grqq{} gekommen waren, wieder ausfallen zu lassen.\par Wie schlecht es um eine Theologie bestellt ist, die von der Schrift losgekommen ist und sich auf dem Gebiet des \glqq{}frommen Glaubensbewusstseins\grqq{} angesiedelt hat, liegt auch in ihren Resultaten klar zutage. Ein trauriges Produkt dieser Theologie ist die Leugnung der satisfactio Christi vicaria. Auch Hofmann, den man den Vater der Ichtheologie unter den konservativen lutherischen Theologen des neunzehnten Jahrhunderts genannt hat, hat sehr bestimmt die stellvertretende Genugtuung Christi geleugnet. Und jetzt ist die Leugnung der satisfactio vicaria fast so allgemein verbreitet wie die Leugnung der Inspiration der heiligen Schrift. Und hier liegt der tiefste Grund für die Tatsache, dass die heilige Schrift nicht als Christi Wort erkannt wird. Wer die satisfactio vicaria leugnet, der kennt den Christus nicht, den die Schrift lehrt;\footnote{Joh. 1, 29; Matth. 20, 28 usw.} und insofern jemand Christentum nicht kennt, kann er auch Christi Wort nicht erkennen, wie Christus selbst sagt.\footnote{Joh. 8, 43. 47.} Wir sprechen nicht jedem Theologen, der vom