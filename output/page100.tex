\begingroup\centering 89 \par Wesen und Begriff der Theologie. \par\endgroup\n\n\subsection*{2. Fundamentale und nichtfundamentale Lehren.}\n\nSelbstverständlich kann die Unterscheidung zwischen fundamentalen und nichtfundamentalen Lehren nicht den Zweck haben, von der Annahme gewisser Lehren, die in der Schrift stehen, zu dispensieren. Dies steht keinem Menschen zu und ist in der Schrift ausdrücklich verboten. Dies gilt sowohl von der Unterscheidung zwischen fundamentalen und nichtfundamentalen Lehren als auch von der weiteren Unterscheidung zwischen primären und sekundären Fundamentalartikeln. Christi Auftrag an seine Kirche lautet dahin: „Lehret sie halten alles, was ich euch befohlen habe, \foreignlanguage{greek}{διδάσκοντες αὐτοὺς τηρεῖν πάντα ὅσα ἐνετειλάμην ὑμῖν.“}\footnote{310} Ebenso war es im Alten Testament Menschen verboten, zu dem geschriebenen Wort hinzuzutun oder davon abzutun.\footnote{311} Damit ist auch zugleich gesagt, dass nichts in der Schrift für überflüssig oder unnütz zu erklären ist; denn \foreignlanguage{greek}{ὅσα προεγράφη, εἰς τὴν ἡμετέραν διδασκαλίαν ἐγράφη.}\footnote{312} Dennoch ist die Unterscheidung von fundamentalen und nichtfundamentalen Lehren nach der Schrift berechtigt. Auch hat diese Unterscheidung großen praktischen Wert, wie aus dieser ganzen Darlegung hervorgehen wird.\n\nIn welchem Sinne die Unterscheidung zwischen fundamentalen und nichtfundamentalen Lehren schriftgemäß und praktisch wichtig ist, wird erkannt, wenn wir z. B. die Lehren von Christo und vom Antichrist nebeneinanderstellen. Beide Lehren stehen in der Schrift. Sie stehen aber in einem ganz verschiedenen Verhältnis zur Entstehung des seligmachenden Glaubens. Die Lehre von Christo bildet das \textit{Fundament} dieses Glaubens, weil der seligmachende Glaube Christum in seiner stellvertretenden Genugtuung zum \textit{Objekt} hat oder Glaube an Christum ist, Gal. 3, 26: „Ihr seid alle Gottes Kinder durch den Glauben an Christus Jesus.“ Die Lehre vom Antichrist dagegen steht nicht in diesem fundamentalen Verhältnis zum christlichen Glauben. Die Schrift sagt nicht, dass ein Mensch durch die Erkenntnis des Antichrists die Vergebung der Sünden und die Seligkeit erlangt, wie sie dies durchweg von der Erkenntnis Christi oder vom Glauben an Christum sowohl im Alten als im Neuen Testament aussagt. Das trotzdem auch die Lehre vom Antichrist nicht vergeblich in der Schrift steht, sondern dem seligmachenden Glauben insofern dient, als sie vor Gefahren warnt, die dem christ-\n\n\vspace{1.5em}\n\footnotesize\n\begin{enumerate}\n    \item[310)] Matth. 28, 20. Demgemäß Paulus Apost. 20, 27.\n    \item[311)] 5 Mos. 4, 1; 8 und 5 Mos. 17.\n    \item[312)] Röm. 15, 4 und 2 Tim. 3, 16; 1 Kor. 10, 11; Röm. 4, 23. 24.\n\end{enumerate}