Eine Selbsttäuschung liegt ferner vor, wenn die moderne Theologie an die Stelle der Schrift den „Glauben“ oder das christliche „Glaubensbewusstsein“ als Bezugsquelle der christlichen Lehre setzt, wie dies nicht nur die links-, sondern auch die rechtsstehenden neueren Theologen tun.\footnote{Ritschl-Stephan, S.~7; Am schärfsten vertreten diese Ansicht Julius Haftmann und Herrmann; aber auch Ahmeß betont kräftig, dass es immer nur „auf eine scharfe begriffliche Fixierung dessen abgesehen sein kann, was dem Glauben unmittelbar gewiss ist“. (Zentralfragen, S.~101.)} Es gibt freilich ein christliches Glaubensbewusstsein und auch ein Reden oder Lehren aus diesem Bewusstsein. „Ich glaube, darum rede ich“, $\epsilon\pi\acute{\iota}\sigma\tau\epsilon\upsilon\sigma\alpha, \delta\iota\grave{o} \kappa\alpha\acute{\iota} \epsilon\lambda\acute{\alpha}\lambda\eta\sigma\alpha$.\footnote{2 Kor.~4, 13; Ps.~116, 10.} Aber dieser christliche Glaube vermittelt sich in der Christenheit ebenfalls nur durch den Glauben an der Apostel Wort, wie Christus Joh.~17, 20 ausdrücklich erklärt, dass alle Gläubigen bis an den Jüngsten Tag durch der Apostel Wort an ihn glauben werden. Der Glaube, welcher nicht Glaube an der Apostel Wort ist, nicht der Apostel Wort zur Quelle und Norm hat, sondern sich von diesem Wort losmacht, ist ex toto, seinem ganzen Umfange nach, menschliche Einbildung, wie der Apostel Paulus 1 Tim.~6, 3 ausdrücklich erklärt, indem er jedem Lehrer, der nicht bei den gesunden Worten Christi bleibt, Hypnose und Unwissenheit zuschreibt. Auch Luther sagt: „Der Glaube lehrt und hält die Wahrheit“, fügt aber sofort hinzu; „Denn er haftet an der Schrift, die lügt und trügt nicht.“ „Glaube“ und „Gottes Wort“ sind allerdings unzertrennlich miteinander verbunden. Aber nicht in der Weise, dass der Glaube das erste und die Lehre das zweite wäre, so dass der Glaube die Lehre legte, sondern umgekehrt so, dass Gottes Wort das erste ist, das den Glauben setzt und bestimmt. Wie Luther sagt: „Das Wort Gottes ist das erste von allem; dem folgt der Glaube.“ „Was seine Ankunft aus der Schrift nicht hat, ist gewisslich vom Teufel selbst.“ Dadurch, sagt Luther, dass die Theologen von der Schrift abgenommen sind, „die allein aller Weisheit Quelle [in der Theologie] ist“, sind „Ungeheuer“ (portenta) von Theologen geworden, „wie Thomas, Scotus und andere.“\footnote{Ev.~I.~XI, 162; XIX, 34; XIX, 1080; I, 1289 f. Exeg.~opp.~Lat.~Erl.~IV, 328.} Dieses scharfe Urteil Luthers trifft voll und ganz die moderne Theologie, sofern sie nicht Gottes Wort, die heilige Schrift, Erkenntnisquelle und Objekt des Glaubens sein lassen will, sondern den Glauben