205 Wesen und Begriff der Theologie.\n\ndes Volkes fast einmütig an der heiligen Schrift als Gottes Wort und an der aus der Schrift geschöpften christlichen Lehre fest. Aus diesem Glauben heraus sangen die gottbegnadeten Sänger der lutherischen Kirche die schönen Kirchenlieder, die noch jetzt unser Herz erquicken. Jetzt steht es leider anders, nämlich so, dass die öffentlichen Lehrer der Kirche fast einstimmig die unfehlbare göttliche Autorität der Schrift leugnen und die christliche Lehre, die in jenen herrlichen Kirchenliedern besungen werden, als Repristinationstheologie verwerfen. Dies geschieht gerade auch in derselben Nummer der \"Neuen kirchlichen Zeitschrift\" von seiten eines Theologen, der mit bedeutender sachlicher Unkenntnis von einem \"unwiederbringlichen Fall der aldogmatischen Inspirationslehre\" redet und mit einem entsprechenden Wandel an Wahrheitssinn anklagend auf ein \"lutherisches Judentum\" hinweist, \"wo man in mechanischer, äußerlich vollzogener Aneignung der Autorität der Schrift wie der Bekenntnisschriften seine Aufgabe schon erfüllt zu haben glaubt\". Wir Theologen sind erfahrungsgemäß sehr schwer zu bessern, wenn wir einmal gründlich vom rechten Wege abgekommen sind, z. B. nicht mehr wissen, ob wir aus der heiligen Schrift oder aus dem eigenen Inneren lehren sollen. Darum wird in Deutschland, auch diesmal, wie im vorigen Jahrhundert, das Heil vornehmlich aus den \"Laienkreisen\" kommen müssen, vielleicht unter der Führung bisher wenig beachteter Pastoren. Gerade wie bei uns in Amerika gegenwärtig die Laien unter den Baptisten eine durch das ganze Land sich erstreckende Vereinigung zustande zu bringen suchen, die den Zweck hat, Kirche und Welt vor einer Generation von ungläubigen Pastoren zu schützen, die das Produkt der Universitäten und Seminare sind, auf denen \"für die göttliche Schöpfung die Evolution eingesetzt wird, für die göttliche Autorität der heiligen Schrift das Glaubensbewusstsein des Individuums, für Christum, den Sohn Gottes, der Idealmensch Jesu, für den Glauben an die stellvertretende Genugtuung Christi moralische Bestrebungen nach dem Vorbilde des Idealmenschen Jesu, für den Himmel und die ewige Seligkeit irdische Glückseligkeit (social gospel).\"\footnote{620}\n\n\"Was in längerer Ausführung von der Darstellung der Missourisynode gefragt ist, gilt auch von den Synoden, die mit ihr in\n\n\footnotetext{620} Siehe S. 146 f. Die Mitteilungen aus \emph{The Fundamentalist}, Vol. II, No. 1.