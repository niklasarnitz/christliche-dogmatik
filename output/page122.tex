\noindent\parbox[t]{0.5\textwidth}{Wesen und Begriff der Theologie.}\hfill 111\n\n\medskip\n\n\noindent Letualismus, Biblizismus, tote Orthodoxie usw. resultiere, falls die Schrift als unfehlbares Werk anerkannt und als einzige Quelle und Norm der christlichen Lehre verwendet werde. Dass dies kein kirchliches, sondern ein den Grund der Kirche umstoßendes Dogma sei, wurde in dem vorhergehenden Abschnitt ausführlich nachgewiesen. Wir haben es also mit einer ungenügenden Definition von „Dogma“ zu tun, wenn eine Lehre, „die kirchliche Anerkennung beansprucht“, unter die kirchlichen Dogmen“ eingereiht wird.\n\n\medskip\n\n\noindent In positiver Darlegung ist zu sagen: Jedes Dogma ist kirchlich, das aus dem „Lehrbuch“ der christlichen Kirche, der heiligen Schrift, geschöpft ist; und jedes Dogma ist unkirchlich, das seine „Ankunft“ (Luthers Ausdruck) nicht aus der Schrift hat. Die Sachlage ist ja die, wie ebenfalls in dem vorhergehenden Abschnitt ausführlich dargelegt wurde, dass die christliche Kirche gar keine eigene, sondern nur Christi Lehre hat, lehrt und bekennt. \textit{Luther:} Ecclesia Dei non habet potestatem condendi ullum articulum fidei, sicut nec ullum unquam condidit, nec condet in perpetuum. Freilich lehrt, bekennt und billigt (approbat) die christliche Kirche articulos fidei seu Scriptura, aber nicht als Oberherr (more maioris sive auctoritate iudiciali), sondern als Untertan (more minoris), wie ein Knecht (servus) das Siegel seines Herrn.\footnote{409} Und das gilt nicht nur von Lokalgemeinden, sondern auch von allen größeren kirchlichen Versammlungen, von Synoden, Konziliien usw.\footnote{410} Es ist auch die Frage abgehandelt worden, ob Lehrbestimmungen, die nur dem Sinne nach, aber nicht dem Ausdruck nach in der Schrift gegeben sind, mit Recht kirchliche Dogmen genannt werden. In concreto handelt es sich um die Frage, ob wir z.\,B. von einem Dogma der „Trinität“, der „Homousie“ usw. reden könnten. Wir werden Luther recht geben, wenn er in Bezug auf das \foreignlanguage{greek}{ὁμοούσιος} sagt: „Es ist ja wahr, man soll außer der Schrift nichts lehren in göttlichen Sachen, wie St. Hilarius schreibt, I. De Trin. Das meinet sich nicht anders denn, man soll nichts anderes [als die Schrift] lehren. Aber dass man nicht sollte brauchen mehr und andere Worte, das kann man nicht halten, sonderlich im Zank, und wenn die Ketzer die Sachen mit bösen Griffen wollen falsch machen und der Schrift Worte verkehren; da war vonnöten, dass man die Meinung der Schrift, mit so vielen Sprüchen gesezt, in ein kurz und Summarienwort fassete und\n\n\vfill\n\footnotesize\n\noindent 409) Opp. v. a. IV, 373. St. Q. XIX, 952.\n\noindent 410) Vgl. die weitere Darlegung unter dem Abschnitt Ecclesia Repraesentativa III, 496 ff.