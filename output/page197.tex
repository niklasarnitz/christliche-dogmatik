186 \quad Wesen und Begriff der Theologie.legenem Gesicht gegenübersteht trotz der gelegentlichen Äußerungen der Freude ob „der Vielheit der Glaubensindividualitäten“! Dass es hingegen in der Periode von Luther bis auf Hollaz an der vermischten Fülle von Lehrverschiedenheiten fehlt, erklärt sich leicht und ungezwungen aus der Tatsache, auf die Mirns uns hinwies, wenn er sagte, dass „die altprotestantische Dogmatik trotz der wechselnden Bevorzugung der synthetischen und der analytischen Methode nach einer im wesentlichen einheitlichen Methode verfährt“, nämlich nach der Methode, die christliche Lehre aus der Heiligen Schrift zu nehmen. Diese einheitliche Methode ergibt naturgemäß ein einheitliches Resultat, d. i. Einheit in der Lehre. Gott hat die heilige Schrift so eingerichtet, dass die Erkenntnis der Wahrheit nicht bloß möglich, sondern das Irren von der Wahrheit ausgeschlossen ist, solange wir bei den Worten der Schrift bleiben, wie Christus dies sehr klar bezeugt, wenn er uns Joh. 8 die Erkenntnis der Wahrheit zusagt, falls wir an seiner Rede bleiben. — So viel hier über die zufällige Kritik, die die moderne Theologie an den Personen und Schriften der alten lutherischen Dogmatiker übt.Die verurteilende Kritik, die die moderne Theologie an der altprotestantischen Theologie von Luther an bis auf Hollaz übt, wird konsequenterweise auch auf die Theologen und Kirchengemeinschaften der Gegenwart ausgedehnt, die die Schrift noch für Gottes Wort halten und daher auch darauf bestehen, dass die christliche Lehre allein aus der Schrift genommen und beurteilt werde. Die Kritik fällt um so schärfer aus, als sie mit der Furcht verbunden ist, die Schrift möchte auch in der Kirche der Gegenwart wieder für Gottes Wort gehalten und damit die Theologie des frommen Selbstbewusstseins außer Kurs gesetzt werden. Einerseits führt die moderne Theologie, wie wir wiederholt hörten, eine sehr zuversichtliche Sprache. Zum Beispiel: „Die dogmatische Methode ist heutzutage verhältnismäßig einheitlich.“ „Niemand gründet seine Dogmatik in altprotestantischer Art auf die \emph{norma normans}, die \emph{Bibel}.“ „In der Gegenwart hat die orthodoxe Inspirationslehre kaum mehr dogmatische Bedeutung.“ Die wenigen Theologen, die sie noch vertreten, sind „Nachzügler“, „ihre Zahl ist gering, ihre Bemühungen fruchtlos, ihr Unwille auf die Genossen, welche sich den Weg nach vornehin neu bahnen, eindruckslos“. Andererseits fehlt es jedoch nicht an Äußerungen der Furcht, es möchte das, was gründlich abgetan ist und am Boden liegt, zu neuem Leben erwachen. So heißt es von den wenigen Theologen, die als „Nachzügler“ auf ihre Standes-