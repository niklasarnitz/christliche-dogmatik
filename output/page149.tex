\noindent\begin{minipage}[t]{0.5\textwidth}\raggedright Wesen und Begriff der Theologie.\end{minipage}%\begin{minipage}[t]{0.5\textwidth}\raggedleft 138\end{minipage}\par\bigskip\setcounter{footnote}{481}allerdings ein fataler Einwand gegen die „\emph{Selbstgewissheit}“. Es ist das der Punkt, an dem die Ichtheologie -- sie datiert ja nicht erst seit Schleiermacher, sondern ist reichlich vorher dagewesen -- in eine Verlegenheit kommt, die sie nicht überwinden kann. Als die Schwärmer im 16. Jahrhundert behaupteten, sie hätten den „\emph{Geist}“ und natürlich auch die Wahrheitsgewissheit unabhängig vom „\emph{äußeren Wort}“, da legte Luther ihnen nahe, sie möchten doch auch ihr eigenes Reden und Schreiben einstellen, es wäre denn, dass sie in der hochmütigen Meinung ständen, „der \emph{Geist} könnte durch die \emph{Schrift} oder mündlich \emph{Wort} der apostel kommen, aber durch ihre [der Schwärmer] \emph{Schrift} und \emph{Wort} müsste er kommen“.\footnote{Schmalt. Art. M. 322, 6.}\setcounter{footnote}{482}Frank erkennt die Berechtigung dieses Einwurfs an. Er gibt zu, dass sein „\emph{System der christlichen Gewissheit}“ allerdings niemand zur Erlangung der Gewissheit dienen könne noch auch solle. Gleichzeitig macht er aber darauf aufmerksam, dass es noch andere Interessen als die christliche Gewissheit in der Welt gebe, nämlich das wissenschaftliche Interesse, und diesem Interesse solle sein „\emph{System}“ dienen. Frank schreibt wörtlich in der zweiten Auflage seines Buches:\footnote{I, 119 f.} „Wendett man ein, wie ein hervorragender, vor Jahren hingeschiedener Theologe getan, wenn es so gemeint sei, so dürfte der Nutzen gering sein; denn wer solche Erfahrung gemacht habe und in der Gewissheit stehe, brauche jenen Nachweis nicht, und wer sie nicht gemacht habe und nicht darin stehe, dem helfe er nichts, so antworte ich: Nichts Weiteres begehrte ich, als einigermaßen zu verstehen das wirklich Vorhandene, die tatsächlich gegebene Gewissheit. \dots Wohl eine geringe Aufgabe, aber doch eine Aufgabe, nämlich eine wissenschaftliche: wer, in der Gewissheit stehend, kein Verlangen danach trägt, der lasse davon, und wer nicht darin steht, der lasse auch davon!“Aber auch gegen diese Behauptung Franks, dass die Aufgabe, die er sich gestellt hat, wenigstens wissenschaftlichen Charakter trage, muss im Interesse der Wissenschaft Verwahrung eingelegt werden. Es ist nicht leicht, eine einigermaßen klare Vorstellung von dem Gedankengang zu gewinnen, durch den Frank den wissenschaftlichen Beweis für die christliche Selbstgewissheit erbringen will. Franks Klagen, dass man seine „\emph{Fragestellung}“ nicht verstanden habe, waren berechtigt. Demgegenüber haben aber jene Beurteiler nicht nur auf die Tatsache hingewiesen, dass Frank eine unnötig schwere Sprache rede, sondern auch an die Möglichkeit erinnert, dass Frank