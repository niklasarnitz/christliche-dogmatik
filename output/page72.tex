\begin{flushright} 61 \end{flushright}\begin{flushleft} Wesen und Begriff der Theologie. \end{flushleft}\par reden denn dieses reichen Hauswirts [Gottes] Wort; sonst ist es nicht die wahre Kirche. Darum soll es heißen: „Gott redet.“\footnote{216) St. v. XII, 1413 f.} Von hier aus verstehen wir Luthers sonderbar klingendes Wort, dass die christliche Lehre nicht „ins Vaterunser gehöre“. Luther will damit die Regel einschärfen, dass der Prediger nicht nötig haben solle, für die von ihm vorgetragene Lehre Gott um Vergebung der Sünden zu bitten.\footnote{217) St. v. XVII, 1343 f.} Vielmehr müsse der Prediger gewiss sein, er habe nicht sein, sondern Gottes Wort gepredigt, das ihm „Gott nicht vergeben soll noch kann, sondern bestätigen, loben, krönen und sagen: Du hast recht gelehret, denn ich habe durch dich geredet, und das Wort ist mein.“ So ernst ist es Luther mit der Forderung, dass die innerhalb der Kirche vorgetragene Lehre Gottes eigene Lehre sein müsse, dass er hinzufügt: „Wer solches nicht rühmen kann von seiner Predigt, der lasse das Predigen nur anstehen, denn er leuget gewisslich und lästert Gott.“\footnote{218) St. v. VIII, 37.} Dasselbe Thema behandelt Luther bei der Beschreibung der Autorität der christlichen Kirche. Er spricht der Kirche jede Autorität ab, christliche Lehre zu machen oder Artikel des Glaubens zu setzen, und begründet dies mit der Tatsache, dass die Kirche gar kein \emph{eigenes}, sondern nur \emph{Christi} Wort habe und verkündige. Von jeder Lehre, die nicht Christi Wort ist, sagt er: „Ob man gleich auch viel Geschwätz macht außerhalb Gottes Wort, noch ist die \emph{Kirche} in dem Plaudern nicht. . . . Lass sie sich toll derwegen ‚Kirche, Kirche!‘ schreien; ohne Gottes Wort ist es nichts.“\footnote{219) St. v. XII, 1414.} Und dies will Luther auch auf die Lehrer in der Kirche, die in dem eminenten Sinne Theologen, Theologen κατ' ἐξοχήν, nennt, angewendet wissen, nämlich auf die Professoren der Theologie. Ein Teil der neueren Theologen hat die Stellung vertreten, dass wohl der schlichte Prediger sich damit begnügen könne, ja solle, die in der Schrift vorliegende Lehre vorzutragen, dass aber den Universitätstheologen, die die Wissenschaft zu vertreten haben, diese Restriktionen nicht aufzuerlegen seien.\footnote{220) So z. B. Kahnis in sehr ausgesprochener Weise in „Zeugnis von den Grundwahrheiten des Protestantismus gegen D. Schenkelberg“, Leipzig 1862, S. 133: „Ich habe meine Dogmatik nicht für schriftliche Welt, nicht für gebildete Nicht-Theologen, sondern nur für wissenschaftliche Theologen geschrieben. Nachdem aber D. Kengtenberg und D. Münkel in durchaus unberechtigter Weise die Sache in weitere Kreise gebracht haben, habe ich in dieser Schrift wenigstens mich an weitere“} Dagegen fordert Luther auch von den Universitäts-