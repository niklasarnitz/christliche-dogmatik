% Page 147\n% Wesen und Begriff der Theologie.\ndem Vorbilde des Idealmenschen Christus gesetzt werden.\footnote{\textbf{501} Wir sehen einige Worte aus der Zeitschrift \emph{The Fundamentalist}, Vol. II, No. 1, hierher: „The Radicals are set on substituting ‘evolution’ for creation, ‘the principle animating the cosmos’ for the living God, consciousness of the individual for the authority of the Bible, reason for revelation, sight for faith, ‘social service’ for salvation, reform for regeneration, the priest for the prophet, ecclesiasticism for evangelism, the human Jesus for the divine Christ, a man-made ‘ideal society’ for the divinely promised kingdom of God, and humanitarian efforts in this poor world for an eternity of joy in God’s bright home.“ Sonderlich wird auf D. Hoedicks fürstlich in New York gehaltene und im Lande weitverbreitete Predigt hingewiesen: „Dr. Harry Emerson Fosdick, for example, not only preached his now famous sermon here in New York on the question, ‘Shall The Fundamentalists Win?’ in which he repudiated the inspiration of the Scriptures; the virgin birth; the vicarious atonement; and the second coming of our Lord; but this sermon was then put into pamphlet form, and has been broadcasted throughout the nation.“} Wie weit diese „Laienorganisation“ einen Damm gegen die Verderbensflut bilden wird, steht dahin. In unserer Kirchengemeinschaft, die sich unter dem Namen Synodalkonferenz zusammengeschlossen hat, ist bisher, Gott sei Dank, noch keine Organisation der Laien gegen die Prediger nötig. Wir wissen unter den Tausenden von Predigern keinen einzigen, der die Inspiration der Schrift antastete und infolgedessen auf den Standpunkt der Irrtheologie gedrängt würde. Aber es gilt, die Gefahr im Auge zu behalten, die uns auch durch unsere amerikanische Umgebung droht.\n\n\subsection*{17. Theologie und Lehrfortbildung.}\nBekanntlich wird gerade auch zu unserer Zeit die Fortbildung der christlichen Lehre für nötig und nützlich gehalten, und zwar nicht nur von den verschieden abgestuften liberalen Theologen, sondern auch von den „konfessionell“ gerichteten neueren Lutheranern. Beide werden im Folgenden beizeiten gebraucht werden. Die Theologen, welche sich gegen Lehrfortbildung ablehnend verhalten, haben den Namen „Repristinationstheologen“ bekommen, sonderlich die sogenannten Missourier;\footnote{\textbf{502} Vgl. das Vorwort zu L. u. W. 1875, S. 1 ff., 33 ff., besonders S. 65 ff.} aber auch andere, z.\,B. Philippi.\footnote{\textbf{503} Hofmann in „Schriftstufen“, I. Heft. S. 2, gegen und über Philippi: „Mag immerhin fortschlafen, wer es gern bequem hat.“} Auch innerhalb der lutherischen Kirche Amerikas wurde die Fortbildung der lutherischen Lehre aus dem Programm gesetzt.\footnote{\textbf{504} Hierher gehört der Artikel „Die falschen Stützen der modernen Theologie von den offenen Fragen“, L. u. W. 1868, S. 97 ff.; ferner „Die moderne Lehrentwicklungslehre“, 1877, S. 129 ff.}