geliums Gottes den Glauben an das Evangelium in den Herzen erzeugt und so ipso den Wahrheitsbeweis für das Evangelium führt. Nehmen wir doch die Sache, wie sie tatsächlich liegt! Im Zustande der durch das Gesetz Gottes gewirkten \emph{contritio} verschwindet das Interesse an Vernunftgründen, weil der Mensch „in einen Haufen“ geschlagen ist.\par Und dann durch die Verkündigung des Evangeliums der Glaube an den Sünderheiland entstanden ist, freut sich der Mensch der rettenden göttlichen Wahrheit, ohne nach Vernunftgründen für dieselbe Umschau zu halten. In diesem Sinne ist das Axiom gemeint: „Die beste Apologie der christlichen Religion ist ihre Verkündigung.“ Meint, dass der Mensch genügend auf das Verständnis des Evangeliums vorbereitet sei, wenn er „Gott sucht und nach sittlicher Vollkommenheit strebt“.\footnote{442) Eb. Dogmatik, S. 37.} Gerade die Verweisung an aller sittlichen Vollkommenheit oder die persönliche Erkenntnis der Verdammungswürdigkeit (\emph{terrores conscientiae, contritio}) ist die unumgänglich nötige, aber auch genügende Vorbereitung auf das „Verständnis“ des Evangeliums. Was die Vernunftbeweise für die christliche Religion betrifft, so können wir freilich dem natürlich-vernünftigen Menschen, insonderheit auch dem gebildeten, dartun, dass es doch wohl vernünftiger sei, die christliche Religion gelten zu lassen, als sie zu verwerfen. Dies gehört in das Gebiet der Apologetik. Aber der Apologet muss sich bewusst bleiben, dass es nicht seine Aufgabe ist, dem Ungläubigen die Wahrheit der christlichen Religion zu demonstrieren, sondern ihm die Unwahrhaftigkeit des Unglaubens aufzudecken, weil der Unglaube, ob bewusst oder unbewusst, die Intelligenz vorschiebt, während ihm doch die böse Richtung des Willens zugrunde liegt, wie Christus bezeugt: „Ο φαῦλα πράσσων μισεῖ τὸ φῶς καὶ οὐκ ἔρχεται πρὸς τὸ φῶς, ἵνα μὴ ἐλεγχθῇ τὰ ἔργα αὐτοῦ.“\footnote{443) Joh. 3, 19. 20.} Es gibt keine wissenschaftlichen Gründe gegen das Christentum. Dieser Gegenstand wird unter dem Abschnitt „Die göttliche Autorität der Heiligen Schrift“, speziell bei der „\emph{fides humana}“, weiter behandelt.\par\subsection*{16. Theologie und Gewissheit.}\par Bekanntlich wird zu unserer Zeit die „\emph{erkenntnis-theoretische Frage}“ viel erörtert. Das ist die Frage, wie ein Theologe zur subjektiven oder \emph{persönlichen Gewissheit} (\emph{personal assurance}) der christlichen Lehre komme. Es fehlt nicht an Geständnissen aus