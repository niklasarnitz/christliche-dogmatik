oder sein Geheimnis wolle offenbaren, sondern sein eigen Geheimnis und schöne Gedanken, die er über Christi Geheimnis hält, nicht will umsonst gehabt haben, damit er horst auch die Teufel zu bestreben, so er doch nie eine Mücke betretet hat oder bekehren kann, wo nicht das Verkehren das Ärgste daran wäre.“\footnote{231} Das dieses Diktum Luthers die Situation in der Kirche der Gegenwart beschreibt, dafür wurde schon ein Zeugnis aus dem modern-theologischen Lager angeführt. Es wird zugestanden, dass die große Übereinstim-mung in dem „Prinzip“, nach welchem die christliche Lehre nicht aus der Bibel, sondern aus dem frommen Selbstbewusstsein zu beziehen ist, eine „schier unendliche Fülle“ von verschiedenen Richtungen in der Lehre zur Begleiterscheinung habe.\footnote{232} So tief ist die moderne Theologie durch die Beiseitesetzung des Schriftprinzips unter das christliche Niveau herabgegangen, dass sie „verschiedene Richtungen“ in der Lehre sogar als Schönheiten der christlichen Kirche ein-schätzt und die Übereinstimmung in Lehre und Glauben, die doch in der Schrift so klar gefordert ist,\footnote{233} als eine Abnormität und als „Repristination“ eines „überwundenen“ theologischen Standpunktes bezeichnet.

Unter den neueren lutherischen Theologen „konservativer“ Rich-tung ist Hofmann-Erlangen besonders entschieden für die Ich-theologie eingetreten. Man hat ihn deshalb auch wohl den Vater der Ichtheologie innerhalb der lutherischen Kirche des neunzehnten Jahrhunderts genannt. Kürzlich hieß es im Leipziger „Theologischen Literaturblatt“, „dass Hofmann und erst recht Braut bewusst und grundsätzlich die volle Selbstgewissheit des Christentums und seiner Theologie in sich selbst vertraten“.\footnote{234} Hofmann nämlich er-teilt dem Theologen die Weisung, bei der Darstellung der christlichen Lehre nicht nur das, was die Kirche gelehrt hat, sondern auch das, was in der heiligen Schrift über die christliche Lehre aus-

\footnote{231} St. V. XIV, 397. Srf. 63, 371.
\footnote{232} Risch-Stephan, Lehrs. d. ev. Dogmatik, S. 16, und Einleitg., S. IX.
\footnote{233} Die Schrift fordert 1 Kor. 1, 10 \textgreek{τὸ αὐτὸ λέγειν} = Ihr abero voll wai \textgreek{ἐν τῇ ἀύτῇ} προqvn Eph. 4, 3 die \textgreek{μία πίστις}, 2 Tim. 1, 13 \textgreek{ὑγιαινόντων λόγων}. Der Ausdruck „gesunde Lehre“ (Luther: „heilsame Lehre“), \textgreek{ἢ διδασκαλία, ἢ γραιμονεα}, Tit. 1, 9; 1 Tim. 6, 3; 2 Tim. 1, 13; 3, 1; 1 Tim. 4, 1. 10, ist reine Lehre in dem Sinne, dass die in der Kirche vorzutragende Lehre Gottes eigene Lehre sein müsste, ohne Beimischung von menschlichen Gedanken, wie der Ausdruck 1 Tim. 6, 3 vom Apostel selbst erklärt wird.
\footnote{234} Theol. Literaturblatt, herausgegeben von Ihmels, Leipzig, 8. Dez. 1922, S. 395.