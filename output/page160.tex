\noindent Weisen und Begriff der Theologie.\hfill 149\par Daß man die von den Aposteln verkündigte Lehre Christi glauben, dass sie dadurch in Christo vollkommen sind, \textgreek{ἐν ὧν ἐστε πεπληρωμένοι}.\footnote[511]{Kol. 2, 8-9. 16–20. Beate meint z. St. (in \emph{The Expositor's Greek Testament}): “There is no condemnation of philosophy in itself, but simply of the empty, but plausible, sham that went by that name at Colossae.” Ähnlich Meyer und wohl die meisten neueren Exegeten. Nach Text und Kontext steht es so, dass der Apostel vor aller Philosophie warnt, sofern sie in Sachen der christlichen Lehre mitreden und die christliche Lehre ergänzen oder vollkommener machen will. Dahin, sagt der Apostel, liegt ein leerer Betrug, weil die Philosophie \textgreek{κενὸς ἀπάτη} ist, nur ein trübseliger Scheinbetrug, während die Christen \textgreek{χριστὸν ἔχετε} selbst und sind nicht ein bloßer Mensch, sondern in dem die ganze Fülle der Gottheit wohnt, so dass die christliche Lehre im Unterschied von aller Menschenlehre \textgreek{πᾶσαν ἀνθρωπίνην διδασκαλίαν} ist, von der Theologie \textgreek{πᾶσα θεότης κατοικεῖ} ausgeschlossen. Meyer meint, „Luthers häufige Verwechslungsurteile über die Philosophie“ hätten darin ihren Grund, dass Luther „die Entartung derselben in der aristotelischen Scholastik vorschweifte“. Von daher sei das S. 17 mitgeteilte Zitat aus Luther „Luther lobt die Philosophen, sofern in Sache des Welt- (z. B. auf der natürlichen Vernunft lehren. Dabei hält er aber fest: „Die Philosophen sind nicht Theologen; darum erinnert Paulus nicht vergebens (Kol. 2, 8), dass wir uns in nicht wehren sollen vor der Philosophie, das ist, vor aller Philosophie, weil ein solcher nichts hat als Worte menschlicher Weisheit, welche sicherlich mit dem Evangelio nicht übereinkommen und damit auch nicht übereinstimmen können.“ St. V. XXIII, 1932. Ferner St. V. I, 484 f., zu Röm. 6, 6: „Die Philosophen disputieren an etlichen Orten nicht so die Furcht (timere) von Gott, von Gottes Verheißung, darnach Gott aller regieret; und wüssten solcher Christen vertrauten weren, zu welchen sie Lehre aus Socrates, Xenophon, Plato usw. Propheten machen. Wer sie aber also davon disputieren, dass sie nicht wissen, dass Gott seinen Sohn Christum gesandt habe zur Seligkeit der Sünder, so sind diese köstlichen und schönen Disputationen die höchste Unwissenheit von Gott und eitel Gotteslästerung nach der Meinung dieses Textes, welcher kurz und rund ein solch Urteil fällt, dass alles Dichten und Trachten und Vornehmen des menschlichen Herzens durchaus böse ist.“}\par Dagegen ist von Vertretern der Reformlehrbildung eingewendet worden, dass die Kirche im Laufe der Zeit sich genötigt gesehen habe, dem sich erhebenden Irrtum gegenüber die christliche Lehre in besonderen Formulierungen, wie \textgreek{ὁμοούσιος}, \textgreek{θεοτόκος}, \textgreek{homo mere passive} se habe usw., zum Ausdruck zu bringen. Das ist Tatsache. Aber durch den Gegenstoß hervorgerufene neue Formulierungen der Schriftlehre wie die ebengenannten sind kein Beweis für Lehrfortbildung, sondern vielmehr ein Beweis für das Gegenteil, nämlich für das Bleiben bei der Schriftlehre (\textgreek{μένειν}