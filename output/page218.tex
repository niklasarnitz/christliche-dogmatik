\vspace*{12pt}
\noindent\hspace*{\fill}\textrm{Wesen und Begriff der Theologie.}\hspace*{\fill}\textrm{207}\par

\vspace{12pt}
\noindent
eingeben und aus göttlicher Eingebung geschrieben sein soll. Weil Schrift und Gottes Wort Wechselbegriffe sind, so ist die Schrift auch die einzige Quelle und Norm der Theologie.\footnote{624) 1, 320 ff.} Was in der Dogmatik gehört, entscheidet die Schrift und nicht etwa ein systematischster Aufbau. Die Systemmacherei führt auf Abwege. Man denke nur an die Gnadenwahllehre des Calvinismus oder die Wahl- und Berufungslehre der synergistischen Lutheraner hüben und drüben.\footnote{625) 1, 250 f.} Versteht man unter dogmatischer Methode die äußere Gruppierung der Lehren, so darf die Methode eine unterschiedliche sein. Methodus ist arbitraria. Aber keine Methode darf so angewandt werden, dass dadurch irgend etwas in der Schrift Gegebenes hinausmethodisiert wird.\footnote{626) 1, 325 ff.} Hönecke beschreibt die synthetische und analytische Methode, erwähnt auch Coccejus' Föderalmethode und die biblisch-historische Methode, die die biblische Geschichtsgrundlage legt und daran das Dogmatische anschließt. Keine ist schlechthin zu verwerfen. \glqq Zu verwerfen ist aber die von Neueren (Krant usw.) angewandte Methode der Entwicklung aus dem christlichen Bewusstsein usw.\grqq Mit dieser Methode wollen die Neueren \glqq um die Schrift herumkommen. Sie öffnet dem Subjektivismus Tür und Tor\grqq. \glqq Die Dogmatik ist lediglich systematische, das ist, wohlgeordnete Darstellung der aus Gottes Wort im Glauben geschöpften theologischen Erkenntnis. Sie ist also beständig damit beschäftigt, sich aus Gottes Wort zu begründen und zu beweisen. Das Wort Gottes ist wie ihre Quelle (\emph{principium cognoscendi, norma causativa}), auch ihre Richtschnur, nach der sie beständig gemessen wird.\grqq Über die Symbole sagt Hönecke: \glqq unsere Kirchen hat ihre Erkenntnis aus Gottes Wort in den Symbolen niedergelegt, und so muss eine lutherische Dogmatik auch vor dem Richtmaß der Symbole bestehen. Doch ist damit nicht ein zweites Richtmaß gesetzt; denn die Symbole haben selbst ihr Richtmaß an der Schrift, sie sind selbst eine normierte Norm (\emph{norma normata}).\grqq Auch der altprotestantischen Dogmatiker nimmt Hönecke sich an: \glqq Durch die ganze neuere positive Dogmatik geht ein anderer Zug als durch die alte orthodoxe Dogmatik. Die alte Dogmatik ist theozentrisch, die neuere positive anthropozentrisch\grqq.\footnote{627) 1, 315.} Auch den vielgeschmähten Calov tritt Hönecke mit diesen Worten ein: \glqq Der bedeutendste Theologe unter denen, die die Methode des Calixt anwenden, ist unstreitig Abraham Calov. Leiblich rüstig bis ins