\begin{center}\begin{minipage}{\textwidth}\begin{flushright}2\end{flushright}\end{minipage}\end{center}\vspace{-2em} \begin{center}\textit{Wesen und Begriff der Theologie.}\end{center}\vspace{1em} \noindent Schrift, sondern aus dem „frommen Selbstbewusstsein des dogmatisie-renden Subjekts“ beziehen und normieren. Wie im Papsttum von der Schrift nur so viel gilt, als der Papst anerkennt und be-stätigt, so will die neuere protestantische Theologie in der Schrift nur das gelten lassen, was das fromme theologisierende Subjekt für der Annahme würdig erklärt. Dies ist eine genaue Beschreibung der Sachlage, wenn wir auf das Gros der neueren Theologen sehen, die Schrift und Gottes Wort nicht „identifizieren“ und daher auch die christliche Lehre nicht aus der Schrift, sondern aus dem eigenen Innern schöpfen und normieren wollen. Damit ist die Ordnung der Dinge in der christlichen Kirche nicht bloß verschoben, sondern auf den Kopf gestellt. Wir haben es mit einer richtigen Revolution gegen die göttliche Ordnung in der christlichen Kirche zu tun.\par Demgegenüber halten wir in vollem Umfange den Standpunkt fest, dass die heilige Schrift durch den einzigartigen göttlichen Akt der Inspiration Gottes eigenes unfehlbares Wort ist, „Gottes Buch“ \textsuperscript{2)} aus dem allein bis an den jüngsten Tag die christ-liche Lehre in allen ihren Teilen zu schöpfen und zu normieren ist. Und für diesen Standpunkt bitten wir nicht um Entschuldigung, son-dern machen ihn als den einzig richtigen geltend. Dieser Standpunkt hat große Vorbilder für sich. Erstlich das normative Vorbild Christi und seiner heiligen Apostel. Denn diese haben, wie bei der Lehre von der heiligen Schrift ausführlich darzulegen ist, durchweg Schrift und Gottes Wort „identifiziert“: \textit{Ἱεράντεα}, Scriptura sacra locuta, res decissa est. Für diesen Standpunkt haben wir auch das normierte Vorbild des Reformators der Kirche, Luthers. Wenn Luther sagt: „Das Wort sie sollen lassen stahn“, so meint er das Wort der heiligen Schrift. Dabei Luthers Erinnerung an alle Lehrer der Schrift, die Theologen — und sie sonderlich — eingeschlossen: „Du sollst also mit der Schrift handeln, dass du denkst, wie es Gott selbst rede.“ \textsuperscript{4)}\par\vspace{1em} \fontsize{9pt}{11pt}\selectfont \begin{enumerate}\item[\textsuperscript{2)}] Ausdruck bei Nitzsch: Stephan, Lehrbuch d. Ch. Dogmatik \S 12, S. 13.\item[\textsuperscript{3)}] Luthers Benennung der Schrift. St. V. IX, 1071.\item[\textsuperscript{4)}] Predigten über das 1. Buch Moses, 1527. St. V. III, 21.\item[\textsuperscript{5)}] Exeg. opp. Lat. Ed. Erl. IV, 328. St. V. I, 1289 f.\end{enumerate}