\noindent Wesen und Begriff der Theologie.\hfill 185
\par
anführen, werden von Gerhard nach dem Zusammenhang, in
dem sie stehen, und nach dem Sprachgebrauch so ausführlich behandelt,
dass der Sache eben so viel als zu wenig geschehen ist. Nun sei
die Arbeit in seinem großen dogmatischen Werk in den Noten zur
I. B. v. dem großen Buchhalter der lutherischen Theologie genannt
Theologie sowie unter den Abschnitten $\beta\epsilon\beta\alpha\iota\omicron\varsigma$ und $\epsilon\kappa\delta o\chi\tilde{\omega}\varsigma$
vorge-
nehmlich Schriftdarlegung. Dass Calov ein großer Schrifttheologe
war, geht nicht nur aus seiner \emph{Biblia Illustrata}, sondern
aus allen seinen Hauptschriften hervor. Calov war es auch, der den
Studenten der Theologie immer wieder einschärfte, dass ihnen die
Kenntnis des hebräischen und griechischen viel nötiger sei als das
Studium der Väter und der Scholastiker.\footnote{603) Calov, Theol. Antisyncretistica, th. 4: Theologiae studioso longe
magis necessaria est linguae Hebraicae et Graecae notitia quam theologiae
scholasticae aut patrum studium aut philosophiae.} Trefflich schreibt jemand
im Meuselschen Lexikon gegen die beliebte Herabsetzung der alten
Theologen als Schrifttheologen: „Der Schriftbeweis nimmt in der
altlutherischen Dogmatik eine hervorragende Stelle und breiten
Raum ein. Besonders die Verfasser der Loci, aber auch die der
späteren Systemata machen vollen Ernst mit der heiligen Schrift
nicht bloß als nachträglicher Norm der anderswoher geschöpften
Lehre, sondern als ihrer prinzipiellen Quelle und erheben
aus ihr die dogmatischen Wahrheiten, welche sie logisch begründen
und gegen die Einwendungen der Widersacher verteidigen, so dass
z. B. die dogmatischen Werke eines Chemnitz und Johann Gerhard
zugleich eine Fundgrube gründlichster exegetischer
Erörterungen sind.\footnote{604) Sub voce ‚Schriftbeweis‘ VI, 93.} Die Heimat der „dogmatischen Er-
egese“ ist nicht bei der Theologie des 16. und 17. Jahrhunderts, wie
die Anklage lautet, zu suchen, sondern findet sich bei der Anklägerin,
im Lager der modernen Theologie. Diese Theologie lehnt es ja
entschieden ab, ihren Glauben durch die Lehre der Schrift entstehen
und normieren zu lassen. Das wäre nach ihrer eigenen ausdrücklichen
Erklärung „intellektualistischer“ Schriftgebrauch. Vielmehr will sie
die Schrift nach der eigenen Anschauung, nach dem frommen Selbst-
bewusstsein des theologisierenden Subjekts, auslegen und normieren.
Das ist wahrlich „dogmatische Exegese“ im eminentesten Sinne des
Worts. Und aus dieser eminent dogmatischen Exegese erklärt sich
auch leicht und ungezwungen „die schier endlose Fülle“ der Lehr-
verschiedenheiten, denen die moderne Theologie doch mit etwas ver-