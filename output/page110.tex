\hfill 99 \hfill Wesen und Begriff der Theologie. \par als die bloße Gnade". \footnote{360) St. 8. XVIII, 1730. Erl. Opp. v. a. VII, 166.} Ähnliche Aussprachen finden wir auch bei späteren lutherischen Theologen. So sagt Hülsemann, dass nicht jede falsche Lehre, die ihrer Beschaffenheit nach den Glaubensgrund zerstört, diese Wirkung auch bei jedem irrenden Individuum hat, weil eben eine „glückliche Inkonsequenz" vorliegen kann. \footnote{361) Calvinismus Irreconciliabilis, p. 432; zitiert Baier-Walther I, 62: Non sunt dogmata, quod ex sua natura aliquod fidei necessario praesuppositum aut eam consequens astruit vel destruit, idem in hominis cuiusque mente illud efficit.} \par Mit dieser Konfession wurde und wird im Interesse des Indifferentismus viel Missbrauch getrieben. Es gilt daher, klar zu erkennen und festzuhalten, dass die „glückliche Inkonsequenz", vermöge welcher durch Gottes besondere Bewahrung ein Irrender für seine Person nicht aus dem Glauben fällt, dem Irrtum selbst nimmermehr Existenzberechtigung in der christlichen Kirche verschaffen kann. Diese irrige Folgerung erlaubten sich römische Theologen im Kampf gegen Luther. Wenn Luther darauf bestand, dass die römischen Irrtümer abzutun seien, so wurde ihm entgegengehalten, dass die von Luther als Irrtümer bezeichneten Lehren auch von „Heiligen" vorgetragen worden seien, und zwar auch von solchen Heiligen, die Luther selbst unter die wahren Kinder Gottes zähle. Diese römische Weise zu argumentieren wiederholt sich in der Kirche, sooft eine Berufung auf die Väter der lutherischen Kirche zu dem Zweck stattfindet, Abirrungen der Väter gegen das Zeugnis der Schrift eine Berechtigung in der Kirche zu sichern. Die sich zum Schutze irriger Lehren auf den Vorgang frommer irrender Väter berufen, werden von Luther auf die Möglichkeit hingewiesen, dass sie den frommen Vätern zwar nachfolgen, „aber zu ihnen werden sie nicht kommen". \footnote{362) St. 9. XIX, 1133.} Es ist ein überaus ernstes Ding um das Lehren in Gottes Hause, der christlichen Kirche. Die in diesem Amte stehen, sollten nie vergessen: \textbf{1.} In der Schrift wird nirgends und niemand Lizenz erteilt, in irgendeinem Punkte von Gottes Wort abzuweichen. Vielmehr gilt für die christliche Kirche bis an den Jüngsten Tag als Hausordnung: „Behret sie halten alles, was ich euch befohlen habe!" \footnote{363) Matth. 28, 20.} \textbf{2.} Jede Abweichung von Christi Wort, das die Kirche im Hort seiner Apostel hat, wird ausdrücklich als ein Ärgernis ($\alpha\nu\chi\alpha\nu\delta\alpha\lambda\omicron\nu$ $\pi\nu\omicron\iota\epsilon\iota\nu$) bezeichnet. \footnote{364) Röm. 16, 17.} Der Irrtum, welcher dem ersten Irrenden durch Gottes besondere Bewahrung unschädlich sein