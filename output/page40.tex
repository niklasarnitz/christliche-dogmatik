gemacht wird. Calvin akzeptiert Luthers Feststellung des status controversiae: „All ihr Ding steht darauf, dass Christus' Leib müsse allein an einem Ort sein, leiblicher- und begreiflicherweise.“\footnote{98}\par 3. Die Frage, ob die Gnade Gottes in Christo allgemein sei (gratia universalis) oder partikular (gratia particularis), meinen die calvinistischen Reformierten nicht aus den Schriftauslagen, die auf die universale Gnade lauten,\footnote{99} sondern aus dem geschichtlichen „Erfolg“ oder der geschichtlichen „Erfahrung“ beurteilt zu sollen. „We must assume that the result is the interpretation of the purposes of God.“\footnote{100} Das reformierte Argument, wodurch die auf die allgemeine Gnade lautenden Schriftauslagen beseitigt werden, verläuft bekanntlich so: Weil tatsächlich nicht alle Menschen selig werden, so ist der Schluss geboten, dass Christi Verdienst und Gottes Gnadenwille nicht auf alle Menschen sich erstrecke. Die Annahme eines allgemeinen Gnadenwillens bei einem partikularen Resultat wäre eine Beleidigung der göttlichen Weisheit, Macht und Majestät.\footnote{101} Die schärfste Verwerfung des Schriftprinzips\par\par\noindent\hrulefill\par\par\footnote{98} Ausführliche Darlegung des Motivs für die reformierte Abendlehre III, 376 ff.\footnote{99} Joh. 1, 29; 3, 16 ff.; 1 Joh. 2, 1; 2 Tim. 2, 4--6 usw. Die ausführlichste Darlegung II, 21 ff.\footnote{100} Charles Hodge, Systematic Theol. II, 323. Ebenso Calvin, Inst. III, 24, 17, 18. Quamlibet enim universales sint salutis promissiones nihil tamen e reproborum praedestinatione discrepant, modo in earum effectum mentem dirigamus. \emph{Experientia docet, ita [Deum] velle resipiscere quos ad se invitat, ut non tangat omnium corda.}\footnote{101} Calvin beruft sich Inst. III, 24, 16 zur Widerlegung des allgemeinen Gnadenwillens auf Gottes Allmacht: Si tenacius urgeant, quod dicitur [Deum] velle misereri omnium, ego contra excipiam, quod alibi scribitur, Deum nostrum esse in coelo, ubi faciat quaecunque velit, Ps. 115, 3. Hodge, a. a. O.: \"It cannot be supposed that God intends what is never accom-plished -- that the adoptive means for an end which is never to be attained. This cannot be affirmed of any rational being who has the wisdom and power to secure the execution of his purposes. Much less can it be said of Him whose power and wisdom are infinite.\" Übrigens hat sich Calvin bei Anführung von Ps. 115, 3 eine Änderung im Wortlaut erlaubt. Die Worte: „Aber unser Gott ist im Himmel; alles, was ihm beliebt (\emph{PJD}), macht er“, sind nur eine Beschreibung der Allmacht Gottes im Gegensatz zur Ohnmacht der Götzen der Heiden, wie sofort hervorgehoben wird, V. 4: „Jener Götzen sind Silber und Gold, von Menschenhänden gemacht.“ Durch Einschiebung eines ubi: Unser Gott ist im Himmel, „ubi faciat quaecunque velit“, verkehrt Calvin den Gedanken dahin, dass Gott im Himmel anderes wolle und handle als auf Erden.