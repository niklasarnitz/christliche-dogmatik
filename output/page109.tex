\section*{Wesen und Begriff der Theologie.}
\setcounter{page}{98}
zogen. Luther schreibt Ende des Jahres 1521 in seiner Schrift „Vom Missbrauch der Messe“: „Es geschieht noch ohne Zweifel viel frommen Christen, dass sie in einem einfältigen Glauben ihres Herzens Messe halten und achten, es sei ein Opfer. Aber dieweil sie sich auf das Opfer [vor Gott] nicht verlassen, ja, sie halten’s dafür, dass alles, was sie tun, Sünde sei, und hangen allein an der lautern Barmherzigkeit Gottes, werden sie erhalten, dass sie in diesem Irrtum nicht verderben.“\footnote{357} Luther schreibt ferner im Jahre 1539 in seiner Schrift „Von den Konzilien und Kirchen“: „Jetzt sind viel großer Herren und gelehrter Leute, die bekennen frei und fest, dass unsere Lehre vom Glauben, der ohne Verdienst gemacht aus lauter Gnade, recht sei; aber dass man darum sollte Klosterei und Heiligendienst oder dergleichen lassen und verachten, das stößt sie vor den Kopf, so es doch die Folge und Konsequenz erzwinget. Denn es kann ja niemand gerecht werden ohne durch den Glauben; daraus folget, dass man durch Klosterleben nicht könne gerecht werden.“ „Ja, Luther „nimmt sich selbst bei der Nase“ und führt seine eigene Person als ein Beispiel der Inkonsequenz an. Er habe schon vor zwanzig Jahren gelehrt, „dass allein der Glaube ohne Werke gerecht mache“, und dabei doch noch hart an Möncherei und Nonnerei gehalten. Zur Begründung fügt er hinzu: „Ach unbedächtiger Narr konnte nicht sehen die Folge, die ich müsste nachgeben, dass, wo es der Glaube allein täte, so könnte es die Möncherei und Messe nicht tun.“\footnote{358} Vom Synergismus, wie er ihm in Erasmus’ facultas se applicandi ad gratiam entgegentrat, urteilt Luther, dass er den Menschen im Vertrauen auf eigenes Können in Sachen der Erlangung der Seligkeit bestärke und dadurch den christlichen Glauben an die Vergebung der Sünden ohne des Gesetzes Werke unmöglich mache.\footnote{359} Zugleich gibt Luther die Möglichkeit einer glücklichen Inkonsequenz bei einzelnen Personen zu, nämlich dass sie zwar in der Theorie, in Schriften und Disputationen, dem Menschen in geistlichen Dingen noch ein Vermögen zuschreiben, aber in der Praxis, „sooft sie vor Gott treten, um zu beten und beten oder mit ihm zu handeln, geben sie einher in gänzlicher Vergessenheit ihres freien Willens, verzweifeln an sich selbst und erbitten für sich nichts anderes
\footnote{357}{St. V. XIX, 1131.}
\footnote{358}{St. V. XVI, 2238.}
\footnote{359}{Opp. v. a. VII, 154: Quamdiu [homo] persuasus fuerit, sese vel tantumtum posse pro salute sua, manet in fiducia sui, nec de hominis cuiusque mente illud efficit.}