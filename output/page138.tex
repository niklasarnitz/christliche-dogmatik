127
Wesen und Begriff der Theologie.

halten Burg zu unserer Zeit endgültig gestürmt sei. Dem durch moderne wissenschaftliche Methoden scharf entwickelten „Wirklichkeitssinn“ sei es unmöglich, die Schrift für Gottes eigenes Wort zu halten. Wir hören: „In der Gegenwart hat die orthodoxe Inspirationslehre kaum mehr dogmatische Bedeutung. Doch wird sie von einzelnen, wie Kölling und Nösgen, noch immer mit einigen Abwandlungen behauptet. Ein durchaus positiver Theologe sagt von solchen Nachzüglern: Ihre Zahl ist gering, ihre Bemühungen fruchtlos, ihr Unwille auf die Genossen, welche sich den Weg nach vornehin neu bahnen, eindruckslos.“\ldots Die übrigen Theologen --- auch die konservativen --- verwerfen die alte Lehre.\footnote{452} Steht es so, dann gilt es allerdings, sich nach einer Gewissheitsbasis umzusehen, die gewisser und zuverlässiger ist als die heilige Schrift. Die modernen Theologen meinen eine solche Basis im Menschen selbst, im christlichen Ich, im Selbstbewusstsein des theologisierenden Subjekts, gefunden zu haben. Sie geben zwar dieser „sturmfreien Burg“ verschiedene Namen: frommes Selbstbewusstsein des theologisierenden Subjekts, wiedergeborenes Ich, christliches Glaubensbewusstsein, christliche Erfahrung usw. Sie stritten und streiten auch darüber, an welchem Punkt auf dem Territorium des Ich der eigentliche Sitz der Gewissheit zu finden sei, ob im Gefühl oder im Denken oder im Wollen oder auch in einer Kombination der genannten Faktoren. Auch darüber wurde und wird gestritten, ob die Gewissheit sich auf die moralische („ethische“) Beschaffenheit oder auf den „Glauben“ des christlichen Ich gründe. Alle aber stimmen, weil sie die Schrift als Gottes Wort aufgegeben haben, naturgemäß darin überein, dass die Gewissheitburg nicht außerhalb, sondern innerhalb des „christlichen Subjekts“ zu finden sei.\par

Diese Theologie der „Selbstgewissheit“ hat Schleiermacher im ersten Viertel des vorigen Jahrhunderts in seiner „Glaubenslehre“ auf den theologischen Markt geworfen\footnote{453} und damit allgemeine und anhaltende Bewunderung erregt, und zwar nicht nur im liberalen, sondern auch im positiven, sonderlich auch im „lutherisch-konfessionellen“ Lager. Bei Nitzsch-Stephan\footnote{454} ist Schleiermachers „Glaubenslehre“ eine reformatorische Tat“, eine „innerste Großtat“, „die\par

\vspace{\baselineskip}
\begin{footnotesize}
\noindent 452) Nitzsch-Stephan, S. 258.\par
\noindent 453) Der gründliche Glaube, nach den Grundsätzen der evangelischen Kirche im Zusammenhange dargestellt von D. Friedrich Schleiermacher. Das Vorwort zum ersten Druck ist datiert: Berlin, am Sonnabend vor Trinitatis des Jahres 1821.\par
\noindent 454) Ev. Glaubenslehre, S. 43 ff.\par
\end{footnotesize}