beiden ersten Kapitel. Bei der Lehre von Gott muss die Unterscheidung zwischen der natürlichen und der christlichen Gotteserkenntnis ausführlicher dargestellt werden, weil die moderne Theologie, bis in lutherisch sich nennende Kreise hinein, dynamistisch-unitarisch geworden ist. Bei der Lehre vom Menschen erforderte die Lehre von der Sünde an mehreren Punkten längere Darlegungen, weil die moderne Theologie von ihrem Ich-Standpunkt aus in römisch-zwinglicher Weise auf den Begriff der \textquotedblleft schuldlosen Sünde\textquotedblright gekommen ist. Um in dem erforderlichen Kontakt mit der Gegenwart zu bleiben, mussten daher gewisse Partien in diesem Bande besonders betont werden.\par Dagegen bedarf es einer besonderen Erklärung, resp. Entschuldigung, weshalb S. 182 ff. eine längere Darlegung eingefügt ist, die eigentlich nicht in eine Dogmatik gehört. Es handelt sich um die namentlich von Deutschland aus auch in dogmatischen Schriften erhobene Anklage, dass innerhalb der Missourisynode eine \textquotedblleft Repristinationstheologie\textquotedblright gepflegt werde, die als ein Übel in der christlichen Kirche angesehen werden müsse. Unsere Theologie, so wird behauptet, verleite infolge der \textquotedblleft Identifizierung\textquotedblright von Schrift und Gottes Wort zu einem \textquotedblleft Intellektualismus\textquotedblright, bei dem lebendiges \textquotedblleft Herzenschristentum\textquotedblright nicht recht aufkommen könne. Im Anschluss an diese Kritik, und um, womöglich, den Schreck vor der \textquotedblleft Repristinationstheologie\textquotedblright zu beseitigen, musste ich in längerer Ausführung darstellen, wie es in unserer kirchlichen, der \textquotedblleft Repristinationstheologie\textquotedblright ergebenen Gemeinschaft aussieht. Um historisch korrekt zu bleiben, durfte ich die weitere Tatsache nicht verschweigen, dass die an der Missourisynode beklagte Theologie mit klarem Bewusstsein auch in anderen kirchlichen Gemeinschaften gepflegt wird. Ich weise auf D. Hänekes sehr ausführliche \textquotedblleft Ev.-Luth. Dogmatik\textquotedblright hin, aus der hervorgeht, dass die Lehrstellung der Synode von Wisconsin u. a. St. völlig mit der Lehrstellung der Missourisynode deckt. In diesem Exkurs finden sich ferner