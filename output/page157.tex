der Segung der Objekte durch das Subjekt keine Verwirrung anrichten, weil die vernunftlose Kreatur sich nicht nach den Ideen Fichtes, Platos oder irgendeines andern Ich richtet. Anders steht es auf geistlichem Gebiet. Da sehen wir, dass die Segnung der Objekte durch das menschliche Subjekt bei den vernünftigen Kreaturen zahlreiche Bewunderer und Abnehmer findet. Hieraus ergibt sich die Pflicht aller derer, die durch Gottes Gnade offene Augen haben, den Trug der Selbstbewusstseinstheologie aufzudecken und zu bekämpfen.\par Wir haben im vorstehenden den unchristlichen und verderblichen Charakter der Bewusstseinstheologie vornehmlich an der Hand deutschländischer Schriften nachgewiesen. Aber diese Theologie beherrscht auch bei uns in den Vereinigten Staaten so ziemlich die ganze protestantische Theologie. Ja, sie ist bei uns, als im Lande der reformierten Sekten, recht eigentlich beheimatet. Zwingli und Calvin waren mit ihrer Lehre von einer unmittelbaren Wirksamkeit des heiligen Geistes prinzipielle Vertreter der Ichtheologie. Dass das falsche Prinzip sich damals nicht vollständig entfalten konnte, lag in dem gewaltigen Einfluss Luthers. Weil dieser Einfluss Luthers zu Anfang des 19. Jahrhunderts fehlte, so braucht es uns nicht zu befremden, dass Schleiermacher mit seiner reformiert-pantheistischen Theologie hierzulande Bewunderer und Anhänger fand, wenn man auch im einzelnen Ausstellungen zu machen hatte.\footnote{\textsuperscript{499} Anknüpfung in Bezug auf den Schleiermacherschen Einfluss hierzulande ist Strongs langer Artikel "The Theology of Schleiermacher as Illustrated by His Life and Correspondence" in seinen \emph{Miscellanies}, Vol. I, 1--57.}\par Gegenwärtig steht es in den Vereinigten Staaten so, dass unsere alten und neuen großen Universitäten mit einer teilweisen Ausnahme von Princeton die Selbstbewusstseinstheologie vertreten, soweit sie sich mit Theologie befassen. Wir berichten kürzlich in "Lehre und Wehre"\footnote{\textsuperscript{500} 1923, S. 89 f.} von einer "Organisation der Laien" gegen das von den christlichen Grundwahrheiten (fundamentals) abgefallene Predigergeschlecht, das in den Universitäten und Predigerseminaren großgezogen worden ist und nun das Land überwuchert. Die Organisation wird damit begründet, dass in den Universitäten und den meisten Predigerseminaren ein Predigergeschlecht großgezogen worden ist, das die christlichen Grundwahrheiten (fundamentals) leugnet. Besonders wird darauf hingewiesen, dass an die Stelle der göttlichen Autorität das Ich-Bewusstsein (consciousness of the individual) und an die Stelle der stellvertretenden Genugtuung Christi moralische Bestrebungen nach