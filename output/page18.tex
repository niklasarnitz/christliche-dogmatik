\pagestyle{fancy}\fancyhead[L]{Wesen und Begriff der Theologie.}\fancyhead[R]{7}\fancyfoot[C]{}\renewcommand{\headrulewidth}{0pt}Aber auch der Sprachgebrauch verhilft uns nicht zu dem in unserer Zeit so eifrig gesuchten allgemeinen Religionsbegriff, der das Christentum und die nichtchristlichen Religionen unter ein gemeinsames genus bringen soll. Freilich ist der Gebrauch des Wortes „Religion“ Heiden und Christen gemeinsam. Naturgemäß verbinden aber die Heiden heidnische, die Christen christliche Begriffe mit dem Wort, und diese Begriffe stellen sich bei näherer Betrachtung sofort als völlig entgegengesetzte heraus.\par Weil die Heiden das Evangelium von Christo nicht kennen,\textsuperscript{14} wohl aber noch eine Kenntnis von Gottes Gesetz haben,\textsuperscript{15} so bewegen sich alle religiösen Gedanken der Heiden auf dem Gebiet des Gesetzes. Sie verstehen unter Religion die menschliche Bemühung, sich durch eigenes Tun oder eigene Werke (Gottesdienste, Opfer, moralische Bestrebungen, Askese usw.) die Gottheit gnädig zu stimmen, das ist, sie verstehen unter Religion eine Religion des Gesetzes. In Bezug auf diese Weichenstimmung der heidnischen Religionen herrscht in alter und neuer Zeit ziemlich allgemeine Übereinstimmung.\textsuperscript{16} Die Christen hingegen verstehen unter Religion das gerade Gegenteil, nämlich den Glauben an das Evangelium, das ist, den Glauben an die göttliche Botschaft, dass Gott durch Christi stellvertretende Genugtuung (\emph{satisfactio vicaria}) mit allen Menschen\par ist grammaticas. [dazu rechnet Luther auch die etymologische Betrachtung eines Wortes], \emph{aliud Latine loqui}. \emph{Ideo non tam sermonis grammatici et regulati quam phrasium} [Sprachgebrauch] \emph{habenda est ratio.} ... \emph{In Latina lingua multa vocabula usu in alienam a grammaticis regulis significationem degenerarunt.}\par\vspace{1em}\noindent\textsuperscript{14)} Kor. 2, 6-10: ἔτι καιρὸς ἀνθρώπων οὐκ ἀνέβη. κ.τ.λ.\par\textsuperscript{15)} Röm. 1, 32: τῷ δικαίωμα τοῦ θεοῦ ἐπιγνόντες; Röm. 2, 15: ἐνδεικνυνται τὸ ἔργον τοῦ νόμου γραπτόν ἐν ταῖς καρδίαις αὑτῶν.\par\textsuperscript{16)} Karl Stade, \emph{Modern Problem}, 1916, S. 183 f.: „Die heidnische Religion hat bereits ihre Eigentümlichkeit, so sehr unmenschliche Verunstaltungen zur Verhöhnung Gottes kennt.“ „Der normale Weg der heidnischen Religion ist immer der, dass der Mensch das Bewusstsein der Sünde zu überwinden sucht, indem er sich bemüht, seine Sünde wieder gutzumachen.“ Lothart (Glaubenslehre, 1898, S. 467): „Das ist das Charakteristische des Heidnischen, dass hier alles Verhältnis von Gott und Mensch leistungsmäßig, also nach dem Gesichtspunkte der Werktätigkeit, betrachtet wird.“ So richtig auch Zhmels, Aus der Kirche, S. 52. Ebenso das lutherische Bekenntnis. \emph{Apologia} (M. 434, 144): \emph{Opera inarrnant hominibus in regulos. Haec naturaliter nihilum humana ratio, et quia tam, tum opera comni, ideoin non intelliquit neque considerat, haec cominiau, haec opera mereri remissionem peccatorum et iustificare. Haec opinio legis haeret naturaliter in animis hominum, neque executi potest, nisi quum di vinitus docemur.}