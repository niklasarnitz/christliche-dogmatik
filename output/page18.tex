\pagenumbering{arabic}
\setcounter{page}{7}
\section*{\centering{Wesen und Begriff der Theologie.}}

Aber auch der Sprachgebrauch verhilft uns nicht zu dem in unserer Zeit so eifrig gesuchten allgemeinen Religionsbegriff, der das Christentum und die nichtchristlichen Religionen unter ein gemeinsames genus bringen soll. Freilich ist der Gebrauch des Wortes „Religion“ Heiden und Christen gemeinsam. Naturgemäß verbinden aber die Heiden heidnische, die Christen christliche Begriffe mit dem Wort, und diese Begriffe stellen sich bei näherer Betrachtung sofort als völlig entgegengesetzte heraus.

Weil die Heiden das Evangelium von Christo nicht kennen,\footnote{14) Kor. 2, 6—10: \textalpha\textnu\textiota\ \textkappa\textalpha\textiota\textrho\textiota\textomega\ \textalpha\textnu\texttheta\textrho\textomega\textpi\textomega\textnu\ \textomicron\textupsilon\textkappa\ \textalpha\textnu\textepsilon\textbeta\texteta.}
wohl aber noch eine Kenntnis von Gottes Gesetz haben,\footnote{15) Röm. 1, 32: \texttau\textomicron\ \textdelta\textiota\textkappa\textalpha\textiota\textomega\textmu\textalpha\ \texttau\textomicron\textupsilon\ \textTheta\textepsilon\textomicron\textupsilon\ \textepsilon\textpi\textiota\textgamma\textnu\textomega\textnu\texttau\textepsilon\textsigmaf; Röm. 2, 15: \textepsilon\textnu\textdelta\textepsilon\textiota\textxi\textiota\textnu\ \texttau\textomicron\textupsilon\ \texttau\textomicron\textmu\textomicron\textupsilon\ \textnu\textomicron\textmu\textomega\textnu\ \textgamma\textrho\textalpha\textpi\texttau\textomicron\textupsilon\ \textepsilon\textnu\ \texttau\textalpha\textiota\textsigmaf\ \textkappa\textalpha\textrho\textdelta\textiota\textalpha\textiota\textsigmaf\ \textalpha\textupsilon\texttau\textomega\textnu.}
so bewegen sich alle religiösen Gedanken der Heiden auf dem Gebiet des Gesetzes. Sie verstehen unter Religion die menschliche Bemühung, sich durch eigenes Tun oder eigene Werke (Gottesdienste, Opfer, moralische Bestrebungen, Askese usw.) die Gottheit gnädig zu stimmen, das ist, sie verstehen unter Religion eine Religion des Gesetzes. In Bezug auf diese Wesensbestimmung der heidnischen Religionen herrscht in alter und neuer Zeit ziemlich allgemeine Übereinstimmung.\footnote{16) Karl Stäudtlin, Moderne Problem, 1816, S. 183 f.: „Die heidnische Religion hat darin ihre Eigentümlichkeit, dass sie nur menschliche Veranstaltungen zur Versöhnung Gottes kennt.“ „Der normale Weg der heidnischen Religion ist immer der, dass der Mensch das Bewusstsein der Sünde zu überwinden sucht, indem er sich bemüht, seine Sünde wieder gutzumachen.“ Luthardt (Glaubenslehre, 1898, S. 467): „Das ist das Charakteristische des Heidnischen, dass hier alles Verhältnis von Gott und Mensch leistungsmäßig, also nach dem Gesichtspunkte der Werktätigkeit, betrachtet wird.“ So richtig auch Themelis, Aus der Kirche, S. 52. Ebenso das lutherische Bekenntnis, Apologie (M. 184, 144): Opera inarrant hominibus in seculos. Haec naturaliter minatur humanna ratio, et quia tam, cum opera coernit, idonea non intelligit neque considerat, haec comiatio, haec opera mereri remissionem peccatorum et iustificare. Haec opinio legis haeret naturaliter in animis hominum, neque executi potest, nisi quum divinibus docemur.}
Die Christen hingegen verstehen unter Religion das gerade Gegenteil, nämlich den Glauben an das Evangelium, das ist, den Glauben an die göttliche Botschaft, dass Gott durch Christi stellvertretende Genugtuung (satisfactio vicaria) mit allen Menschen