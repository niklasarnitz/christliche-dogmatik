\begin{center}
  77 \hfill Wesen und Begriff der Theologie.
\end{center}

zu seiner eigenen Quelle und zu seinem eigenen Objekt macht. Manche Vertreter der Zschtheologie sind zu der abenteuerlichen Behauptung fortgeschritten, dass die christliche Religion es überhaupt „nicht eigentlich“ mit Lehre zu tun habe, und dass daher auch die heilige Schrift nicht als „ein göttliches Religionslehrbuch“\footnote{Nitsch, Stephan, S. 249.} auffaßen sei. Diese Auffassung, die sich bei den alten Dogmatikern und auch noch bei Luther finde, sei als ein Überbleibsel aus dem Papsttum anzusehen. Auch in Meusels „Kirchlichem Handlexikon“ hat dieser moderne sowohl schriftwidrige als auch unübernünftige Gegensatz zwischen Lehre und christlicher Religion Aufnahme gefunden, wenn es dort heißt: „Es handelt sich innerhalb des Neuen Testaments, auch bei dem Apostel Paulus, zunächst nicht um Lehre, sondern um Offenbarung und Religion. Was Grau (in Jöcher, Handb. d. theol. Wissensch. I, 561) für den Paulinismus bemerkt, der Inhalt desselben sei Religion und Leben, nicht Lehrbegriff oder Lehrsystem, gilt dem ganzen Neuen Testament.“\footnote{Kirchl. Handlexikon IV, 209, sub „Lehrbegriff“.} Auch bei Ihmels finden sich wiederholt Äußerungen wie diese: „Es muss deutlich werden, wie gerade der evangelische Glaube ein Verständnis der Offenbarung aufdrängt, das nicht in einer Lehrmitteilung das Wesentliche sieht, sondern in einem tatsächlichen Aussichheraustreten Gottes.“\footnote{Ihmels, Aus der Kirche usw., 1914, S. 18. Vgl. a. a. O., S. 110. 137. 144 u. öfter.} Als ob nicht Christi Wort, das wir im Wort seiner Apostel haben, ein „Aussichheraustreten“ Gottes und eine „Lehrmitteilung“ wäre, auf welche allein der Glaube, der die Wahrheit erkennt, sich gründet, wie Christus Joh. 8 ausdrücklich erklärt: „So ihr bleiben werdet an meiner Rede (\emph{logos}), \dots so werdet ihr die Wahrheit erkennen.“ Mit Recht hat in neuerer Zeit Eduard König\footnote{In der Schrift „Der Glaubensakt des Christen, nach Begriff und Fundament untersucht“, 1891, S. 119. Auch in seiner neuesten Schrift, „Theologie des N. Ts.“, 1921, S. 313, kommt König auf diesen Punkt zurück.} vor dem Unterfangen, den Glauben sein Objekt, nämlich die in der Schrift vorliegende Lehre, zu entziehen, gewarnt, weil dadurch der biblische Begriff von „Glauben“ aufgegeben und die Art der christlichen Religion als einer positiven Religion umgestoßen werde. Es liegt in der Tat eine fast unbegreifliche Geistesverwirrung vor, wenn man von der christlichen Religion, selbst auch nur „zunächst“, die „Lehre“ oder die „Lehrmitteilung“