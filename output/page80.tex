\section*{\textsl{Wesen und Begriff der Theologie.}\hfill 69}

Liberalismus, Buchstabenlehre, mechanische Schriftauffassung, Auffassung der Schrift, als ob sie ein Lehrgesetzbuch, ein vom Himmel gefallener Gesetzeskodex, ein papierner Papst usw. wäre. Wir stoßen an diesem Punkte bei den modernen Theologen so ziemlich auf das ganze Vokabular von Scheltworten, das sowohl die Römischen als die reformierten Schwärmer gegen Luther und die lutherische Kirche verwendeten. Auch römische Theologen haben darüber gespottet, dass die christliche Kirche hinsichtlich ihrer Lehre von „Papier“ und „Pergament“ abhängig sein sollte. Sie hatten dabei das Interesse, das des Papstes als Quelle und Norm der christlichen Lehre festzuhalten. Ebenso halten die reformierten Schwärmer Luthers festhalten am Schriftwort tote Buchstabenheologie und unevangelisches Christentum. Sie hatten dabei das Interesse, dem „heiligen Geist“, dem ein „Wagen“ (vehiculum, plaustrum) weder nötig noch anständig sei,\footnote{Mit überaus trichterer Berufung auf 2. Kor. 3, 3; „nicht mit Tinte, sondern mit dem Geist des lebendigen Gottes“. Franz Coster behauptet in seinem Enchiridion Controversiarum Praecipuarum, c. 1, p. 43: Christum nec ecclesiam suam a chartacivis Scriptura pendere nec membranis mysteria sua committere voluisse. (Verl. Gutachten, I, 90. 92.)} freie Bahn in der Kirche zu schaffen. Weil aber Gottes heiliger Geist die Weise hat, einen „Wagen“, nämlich die Gnadenmittel, zu gebrauchen, so war das Interesse bewusst oder unbewusst – tatsächlich kein anderes, als den angeblich unmittelbar erleuchteten eigenen Geist in der Kirche Gottes auf den Herrscherthron zu setzen. Und wenn nun die modernen Theologen das Beziehen und Normieren der christlichen Lehre aus der heiligen Schrift als Intellektualismus, Buchstabenheologie usw. bezeichnen und vom papiernen Papst reden und an Stelle der Schrift das „Erlebnis“ des Theologen zur Quelle und Norm der christlichen Lehre machen, so ist das Interesse – bewusst oder unbewusst oder halbbewusst – kein anderes, als das Produkt des eigenen Geistes in Gottes Kirche als oberste Autorität zu etablieren. Die göttliche Autorität, welche der Schrift abgesprochen wird, wird tatsächlich dem Ich des Theologen zugesprochen. Wir haben das Resultat, das Luther als Folge der papistischen Distredierungen der Schrift so beschreibt: „Sie reden solch Ding nur darum, dass sie uns aus der Schrift führen und sich selbst zu Meistern über uns erheben, dass ihre Traumpredigten glauben sollen.“\footnote{Vgl. die Zitate unter dem Abschnitt „Die Ursache der Parteien innerhalb der äußeren Christenheit“, S. 26.}\footnote{Sl. G. V, 334 f.} Und gegen die