weiß, ein Manko an Freudigkeit und Kraft“ zum Vorwurf gemacht. Er sagt: „Auf das Ganze gesehen, macht sich im alttheologischen Lager vielfach eine gewisse Mutlosigkeit breit, eine gewisse Verzagt-heit, ein Gefühl der Schwäche: wir können nicht.“ Wenn dies wirklich der Fall ist, so ist den verzagten Leuten im alttheologischen Lager die Überzeugung, dass die Schrift durch Inspiration Gottes eigenes unfehlbares Wort ist, unter dem Druck des Zeitgeistes in den Hintergrund getreten. Sobald sie sich aber darauf besinnen, worauf ihr Glaube und der Glaube der ganzen Christenheit auf Erden sich gründet, Eph. 2, 20, und was sie in der Anfechtung von innen und außen unumgänglich nötig haben, werden sie nach der Anweisung Christi und seiner Apostel wieder freudig das \foreignlanguage{greek}{ἑββαρραν}, „Es steht geschrieben!“ sprechen und Luthers Wort: „Das Wort sie sollen lassen stahn!“ nicht bloß äußerlich, sondern von des Herzens Grund nachsagen. Sie werden dann auch die „größere Freudigkeit“, die Theodor Kaftan bei sich und andern modernen Theologen wahr-nimmt, richtig einzuschätzen und zu klassifizieren wissen, nämlich als die 1 Tim. 6, 3 beschriebene Typhose, die in ihrer geschichtlichen Erscheinung durch den Rabbi und die Schwärmer aller Zeiten exemplifiziert wird. Omnis fidei \textit{vana} est, quae non nititur Verbo Dei,\footnotemark[276] und: Deus solo suo Verbo voluit suam voluntatem, sua \textit{consilia} deformari nobis, non nostris conceptibus et imaginationibus.\footnotemark[277]

\section*{11. Einteilungen der Theologie, als Lehre gefaßt.}

Unter diesem Abschnitt behandeln wir die Kapitel: 1. Gesetz und Evangelium. 2. Fundamentale und nichtfundamentale Lehren. 3. Offene Fragen und theologische Probleme.

\subsection*{1. Gesetz und Evangelium.}

Frank hat darauf hingewiesen, dass eine von Luther und den alten Theologen sehr sorgfältig behandelte Lehre aus der modernen Theologie so ziemlich verschwunden ist und geradezu abgewiesen wird.\footnotemark[278] Das ist die Lehre vom Gesetz und Evangelium und speziell von dem Unterschied von Gesetz und Evangelium. „Nichts hat es in bestimmtester Weise ausgesprochen, dass die hergebrachte Unterscheidung von Gesetz und Evangelium mit allen ihren Konsequenzen

\begin{footnotesize}
\footnotetext[276]{Luther zu Jel. 7, 9. Erl. Lat., XXII, 83; St. L. VI, 70.}
\footnotetext[277]{Luther zu 3 Mos. 4, 12. Erl. Lat., XIII, 137. St. L. III, 1417.}
\footnotetext[278]{Dogmatische Studien. Erlangen u. Leipzig, 1892, S. 104--135.}
\end{footnotesize}