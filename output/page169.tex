\textbf{Wesen und Begriff der Theologie. 158}

der ganzen Schrift durch die Autorität dessen gedeckt ist,\footnote{Röm. 10, 35; 17, 14. 17.} den er im Glauben als seinen Gott und Erlöser anbetet. Das ist die Stellung zur Schrift, die Luther jedem Christen dringend empfiehlt: „Kannst du es nicht verwechseln, wie es [bei der Schöpfung der Welt] sechs Tage sind gewesen, so tu dem Heiligen Geist die Ehre, dass er gelehrter sei denn du. Denn du sollst also mit der Schrift handeln, dass du denkst, wie es Gott selbst rede.“\footnote{Vorrede auf die Predigten über das erste Buch Mosis. St. 3, III, 21.} Und das ist nicht eine des Christen unwürdige oder „unfreie“ Unterwerfung unter eine „äußere Autorität“, sondern eine kindliche, willige, freie, herrliche Unterwerfung, die der Christ sein ganzes Leben hindurch in Bezug auf Gottes Walten übt, sofern er dies Walten nicht versteht. Dieser Gegenstand wird wieder aufgenommen bei der Lehre von der Heiligen Schrift unter dem Abschnitt „Die göttliche Autorität der Heiligen Schrift“, speziell bei der Frage, wie die Heilige Schrift uns Menschen eine göttliche Autorität wird. Dort ist auch über den Nutzen und den Gebrauch solcher Argumente gehandelt, die menschlichen Glauben an die Göttlichkeit der Heiligen Schrift erzeugen können.

\subsection*{19. Theologie und System.}

Auch das Wort System wird nicht immer in demselben Sinne gebraucht.\footnote{Vgl. Preisschneider, Systematische Enzyklopädie S. 39. Meier-Magister, Kompendium I, 76. Irgendein größeres Reallexikon, z. B. Century Dictionary, VII, 6142, unter „System“. Kiefroth, Der Schriftbeweis des Dr. J. Chr. K. von Hofmann. Schwerin 1859, S. 173--190. Nitsch-Stephan, Ev. Dogmatik, S. 10 f. 18.} Es ist daher herauszustellen, in welchem Sinne, und in welchem Sinne nicht, die Theologie ein System genannt werden könne.

Verstehen wir unter System ein in sich zusammenhängendes Ganzes, so ist die christliche Lehre ein System. Die christliche Lehre nämlich, die lediglich aus der Heiligen Schrift genommen wird, bildet ein in sich zusammenhängendes Ganzes in doppelter Hinsicht: 1. insofern als die Schrift ihrem Inhalt nach nicht differierende menschliche Lehrbegriffe (einen mosaistischen, johannäischen, petrinischen, paulinischen usw. Lehrbegriff) vorlegt, sondern den einheitlichen Lehrbegriff Gottes (\textit{doctrinam divinam}) darbietet, weil alle Schrift von Gott eingegeben und völlig irrtumslos