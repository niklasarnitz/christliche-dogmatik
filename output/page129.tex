\subsection*{Wesen und Begriff der Theologie.\quad 118}

Theologie auch Heiligung und gute Werke. Titus soll die Gläubigen lehren, \textit{\foreignlanguage{greek}{καλῶν ἔργων προΐστασθαι}}.\footnote{424) Tit. 3, 8.} Aber nicht als Ursache oder Vorbedingung oder Mittel der Erlangung der Vergebung der Sünden und Seligkeit, was ein Charakteristikum der Lehrer ist, die der Apostel Paulus mit dem Fluch belegt,\footnote{425) Gal. 1, 8.} sondern als Folge und Wirkung der ohne Werke durch den Glauben bereits erlangten Vergebung der Sünden und Seligkeit. Auf diese Weise erzielt die Theologie sowohl die rechte Qualität als auch eine erfreuliche Quantität der guten Werke.\footnote{426) Die nähere Darlegung III, 56 ff.}

\subsubsection*{14. Die äußeren Mittel der Theologie, wodurch sie ihr Ziel an den Menschen erreicht.}

Wie der Theologe den schriftmäßigen Zweck der Theologie, der in der Erzeugung des Glaubens an Christum und in der Führung zur Seligkeit besteht, nicht aus dem Auge verlieren darf, so darf er sich auch nicht den Blick trüben lassen in Bezug auf die Mittel, durch welche dieser Zweck erreicht wird. Solche Mittel sind nicht: weltliche Gewalt, äußerer Zwang, Staatshilfe, social affairs usw. Die Versuchung, auf solche unchristliche Mittel zu verfallen, ist nicht gering. Sofern der Theologe noch das Fleisch an sich hat, ist er geneigt, seine Stellung als Pastor einer Gemeinde oder auch als Professor der Theologie für ehrenvoller und gesicherter zu halten, wenn der Staat mit seiner Autorität und Gewalt hinter ihm steht. Dies hat früher und jetzt gar manche veranlasst, für staatskirchliche Verhältnisse einzutreten und die „Freikirche“, die einzige von Gott geordnete äußere Gestalt der Kirche Christi, abzulehnen. Aber auch die Theologen, welche Freikirchen angehören, sind der Versuchung ausgesetzt, zum Bau der Kirche zu unkirchlichen Mitteln zu greifen, wie wir dies hierzulande in den social affairs, community churches, in dem Dringen auf ein „klartes Kirchenregiment“ usw. vor Augen haben. Daher gehört der \textit{\foreignlanguage{greek}{ἐξ ἐναντίας ἐκ τοῦ θεοῦ}}, die einem Theologen eigen sein soll,\footnote{427) 2 Kor. 3, 5.} eine herzliche Zuversicht zu den von Gott geordneten, vor der Welt und dem eigenen Fleisch unscheinbaren Gnadenmitteln, zu dem \textit{\foreignlanguage{greek}{εὐαγγέλιον τῆς χάριτος τοῦ θεοῦ}}.\footnote{428) Apost. 20, 24.} Das hat Christus seiner Kirche zu lehren und zu predigen geboten, und als sein Apostel von den Pastoren und Gemeinden von Ephesus für dieses Leben ab-