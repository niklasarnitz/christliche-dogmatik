\begin{flushright}32\end{flushright}\n\n\begin{center}\textbf{Wesen und Begriff der Theologie.}\end{center}\n\naufgegeben haben. Sie leugnen nämlich, dass die Heilige Schrift Gottes eigenes und unfehlbares Wort sei. Sie geben damit die heilige Schrift als Quelle und Norm der christlichen Lehre und eo ipso das Prinzip der Einigkeit in der christlichen Kirche auf. Denn nur die, welche an Christi Wort bleiben, erkennen die Wahrheit, wie Christus selbst sagt.\footnote{110) Joh. 8, 31. 32.} Und Christi Apostel, Paulus, versichert uns, dass jeder, der nicht an den gesunden Worten unsers Herrn JEsu Christi bleibt, verdürrt ist und nichts weiß.\footnote{111) 1 Tim. 6, 8 ff.} Die moderne Theologie, auch die sogenannte positive, setzt an die Stelle der heiligen Schrift als theologisches Erkenntnisprinzip die „Erfahrung“ oder das „Erlebnis“ des „theologisierenden Subjekts“, auch „Glaubensbewusstsein“, „christliches Bewusstsein“, „wiedergeborenes Ich“ usw. genannt. Hinsichtlich dieser „theologischen Methode“ herrscht in der modernen Theologie allerdings eine große Übereinstimmung. Wir lesen sogar: „Niemand gründet seine Dogmatik in altprotestantischer Art auf die norma normans, die Bibel.“\footnote{112) Ritschl-Stoypjan, Ev. Dogmatik, S. 15.} Über diese allgemeine Übereinstimmung in der theologischen Methode ist – wir fürchten die Worte schon früher an – „verbunden mit einer schier endlosen Fülle von Verschiedenheiten in der Anwendung dieser Grundsätze, wie sie bald mehr durch die religiöse Individualität des Dogmatikers, bald mehr durch den Grad seiner wissenschaftlichen Konsequenz verursacht wird“.\footnote{113) W. A. c., Vorwort, IX.} Und wie von den modernen Theologen in großer Übereinstimmung die heilige Schrift als einzige Quelle und Norm der christlichen Lehre aufgegeben ist, so wird von ihnen auch in großer Übereinstimmung die Schriftlehre von Christi satisfactio vicaria und damit die Schriftlehre von der Rechtfertigung durch den Glauben „ohne des Gesetzes Werke“ ($\textit{oúx ἐξ ἔργων}$) abgelehnt.\footnote{114) Thieme in R.E.3 XXI, 120.}\n\nAuf die Frage, ob bei der Leugnung der Schrift als des Wortes Gottes und bei der Leugnung der satisfactio Christi vicaria noch der christliche Glaube möglich sei, ist zu antworten: Konsequenterweise nicht. Wer Christo und seinen Aposteln Joh. 10, 35; 2 Tim. 3, 16; 1 Petr. 1, 10—12 nicht glaubt, sollte ihnen konsequenterweise auch Joh. 3, 16; Matth. 20, 28; Joh. 1, 29; 1 Joh. 1, 9; Röm. 3, 28 usw. nicht glauben. Aber es kann geschehen und ist geschehen, dass jemand, der theoretisch die Inspiration der heiligen Schrift und die stellvertretende Genugtuung Christi leugnete, in der Anfechtung und Todes-