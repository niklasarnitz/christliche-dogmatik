kommt noch kürzer in dem bekannten Axiom zum Ausdruck: Quod non est biblicum, non est theologicum.

Daher gehört zur näheren Charakterisierung der Lehre, die in der christlichen Kirche heimatsberechtigt ist, dass der christliche Theologe nicht wandelbare menschliche Meinungen und Ansichten, sondern die unwandelbare göttliche Wahrheit oder Gottes eigene Lehre (\emph{doctrina divinam}) zu lehren hat. Diese Beschaffenheit der christlichen Lehre ist mit der Beschaffenheit der Quelle, das ist, mit der Beschaffenheit der heiligen Schrift, gegeben, aus welcher der christliche Theologe die Lehre, welche er vorträgt, schöpft. Weil die Heilige Schrift nach dem Zeugnis Christi und seiner Apostel und auch nach ihrer Selbstbezeugung im Herzen der Christen Gottes eigenes unfehlbares Wort ist, so ist auch die der Schrift entnommene Lehre nicht \textgreek{κατὰ τὸν παράδοσιν τῶν ἀνθρώπων}, das ist, nicht Menschenlehre (Kol. 2, 8), sondern Gottes eigene Lehre, \textgreek{ἡ διδασκαλία τοῦ ἀληθοῦς ἡμῶν Θεοῦ} (Tit. 2, 10). Weil die moderne Theologie die Schrift nicht als Gottes Wort gelten lässt und daher aus der nach ihrer Meinung unzuverlässigen heiligen Schrift in das eigene Innere, in das theologische Ich, flüchtet und dieses Ich zur Bezugsquelle der christlichen Lehre macht, so ist damit prinzipiell die Lehre, die sich in der Kirche Gottes zur Annahme darbietet, von dem Gebiet der objektiven göttlichen Wahrheit abgerückt und auf das Gebiet der subjektiven menschlichen Ansicht verlegt. Bei dieser Sachlage erscheint es angezeigt, den Punkt von der Beschaffenheit der in der christlichen Kirche berechtigten Lehre ausführlicher darzulegen.

Wir halten fest: Weil die heilige Schrift nicht Menschenwort, sondern Gottes Wort ist, so gehört zur näheren Beschreibung der Theologie, objektiv oder als Lehre gefasst, dass die vom Theologen aus der Schrift geschöpfte Lehre (\emph{doctrina e Scriptura sacra hausta}) göttliche Lehre, \emph{doctrina divina}, ist. Und zwar nicht bloß in dem Sinne, dass sie von Gott und göttlichen Dingen handelt, sondern gerade auch und primo loco in dem Sinne, dass sie im Gegensatz zu allen menschlichen Lehren, Anschauungen und Urteilen über Gott und göttliche Dinge Gottes eigene Lehre, Anschauung und Urteil ist.

Um zu illustrieren: Der christliche Theologe lehrt von der Schöpfung der Welt und des Menschen, was Gott darüber