\section*{Wesen und Begriff der Theologie.}
\emph{(De natura et constitutione theologiae.)}

\subsection*{1. Die Verständigung über den Standpunkt.}

Bei der Sachlage in der Kirche der Gegenwart ist eine Verständigung über den theologischen Standpunkt nötig. Der Standpunkt, von welchem aus diese Dogmatik geschrieben wurde, ist die Überzeugung, dass die Heilige Schrift in spezifischstem Unterschied von allen andern Büchern, die es sonst noch in der Welt gibt, Gottes eigenes unfehlbares Wort und deshalb die einzige Quelle und Norm der Lehre ist, die eine christliche Dogmatik darzustellen hat. Es gab eine Zeit, in der innerhalb der christlichen Kirche dieser Standpunkt, wenige Ausnahmen abgerechnet, gar nicht in Frage gestellt wurde. Diese Zeit reicht bis in die erste Hälfte des achtzehnten Jahrhunderts hinein. Seitdem und sonderlich in der Gegenwart hat sich die Sachlage in dem Maße geändert, dass das, was früher Regel war, nun zur Ausnahme geworden ist, soweit die öffentlichen Lehrer in Betracht kommen. Die öffentlichen Lehrer, die in weiteren Kreisen bekannt sind und als Vertreter der protestantischen Theologie der Gegenwart angesehen werden, leugnen fast ohne Ausnahme, dass die Heilige Schrift durch Inspiration Gottes eigenes Wort ist. Sie lehnen es daher auch ab, die Heilige Schrift als die einzige Quelle und Norm der Theologie anzusehen und zu verwenden. Es hat eine allgemeine Flucht aus der angeblich unzuverlässigen Heiligen Schrift in das eigene menschliche Ich eingesetzt, das man euphemistisch „christliches Glaubensbewusstsein“, „wiedergeborenes Ich“, „Erlebnis“ usw. nennt. Durch diese Los-von-der-Schrift-Bewegung ist innerhalb des modernen Protestantismus ein Stand der Dinge eingetreten, der sein Analogon in der römischen Kirche hat. Wie in der römischen Kirche nicht die Heilige Schrift, sondern das Ich des Papstes tatsächlich die einzige Quelle und Norm der Lehre ist, als der „alle Rechte im Schrein seines Herzens“ hat,\footnote{Schmall. Art. M. 321, 4.\par F. Rieper, Dogmatik. 1.} so wollen auch die modernen protestantischen Theologen die christliche Lehre nicht aus der Heiligen