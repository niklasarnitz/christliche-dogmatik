\noindent 124 \hfill Wesen und Begriff der Theologie.\n\ndem modern-theologischen Lager \„positiver\“ und \„liberaler\“ Richtung, dass die Behandlung dieser Frage Schwierigkeiten darbiete.\footnote{444) Frank, System der christlichen Gewissheit I, 128. Zwecks. die christliche Wahrheitsgewissheit, 1901, S. 8. Horst Stephan, Glaubenslehre, 1921, S. 66.} Die Schwierigkeiten sind selbstgemacht. Sie haben ihren Grund in dem Abfall von der Schrift als Gottes Wort.\n\nIn der Schrift ist die \„erkenntnis-theoretische Frage\“ sehr klar und allgemeinverständlich beantwortet. Christus belehrt alle Christen, die Theologen eingeschlossen, dahin: \„So ihr bleiben werdet an meiner Rede . . . so werdet ihr die Wahrheit erkennen.\“\footnote{445) Joh. 8, 31. 32.} In diesen Worten kommt ein Doppeltes zur Aussage: 1. dass eine christliche Gewissheit gibt, \textgreek{γνωθεῖν τὴν ἀλήθειαν}, ihr werdet die Wahrheit erkennen; 2. dass die christliche Gewissheit im Bleiben, das ist, im Glauben an Christi Wort besteht, \textgreek{ἐὰν ὑμεῖς μείνητε ἐν τῷ λόγῳ τῷ ἐμῷ}, wenn ihr an meinem Wort bleibt. In dieser Aussage Christi ist klar gelehrt, dass die christliche \„Wahrheitsgewissheit\“ mit dem Glauben an Christi Wort zusammenfällt. Und wenn weiter gefragt wird, wie es zu dem Glauben, der an Christi Wort bleibt, komme, so lehrt uns auch darüber die Schrift nicht im Zweifel. Christi Wort hat die Eigenschaft, dass es selbst den Glauben wirkt.\footnote{446) Röm. 10, 17: \„\textgreek{Ἡ πίσις ἐξ ἀκοῆς}, \textgreek{ἡ δὲ ἀκοὴ διὰ ῥήματος Θεοῦ}.\“} Der Grund hierfür ist der, dass mit dem Wort Christi, wenn es durch Hören oder Lesen in den menschlichen Geist aufgenommen wird, die Wirkung des heiligen Geistes verbunden ist, wie Paulus 1 Kor. 2, 5 vom christlichen Glauben sagt, dass er nicht \textgreek{ἐν σοφίᾳ ἀνθρώπων}, sondern \textgreek{ἐν δυνάμει θεοῦ} sein Entstehen und sein Bestehen habe. Luther bringt diese Tatsache durch das Axiom zum Ausdruck: Der Mensch ist certus passive, sicut Verbum Domini certum est active. Luther erklärt dies näher so: \„Wo dieses Wort [Gottes Wort] ins Herz kommt mit rechtem Glauben, da macht\textquotesingle{}s das Herz ihm gleich, auch fest, gewiss und sicher, dass es so steif aufrecht und hart wird wider alle Anfechtung, Teufel, Tod und wie es heißen mag, dass es tröstlich und hochmütiglich alles verachtet und spottet, was zweifeln, sagen, böse und zornig sein will, denn es weiß, dass ihm Gottes Wort nicht lügen kann.\“\footnote{447) St. L. III, 1887. Erl. 37, 8.} Aber die Schrift erteilt in Bezug auf die \„erkenntnis-theoretische Frage\“ noch weiteren wichtigen und nötigen Unterricht. Sie schärft nämlich mit besonderer Beziehung auf die Lehrer der Kirche sehr