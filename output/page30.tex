im Wort hinausreiche. Diese Meinung ist so gewiss eine irrige, als Christus alle religiöse Wahrheitserkenntnis nur durch den Glauben an sein Wort vermittelt sein lässt und der Apostel Paulus jedem Menschen, insonderheit auch dem Lehrer in der Kirche, Aufgeblasenheit und Ignoranz zuschreibt ($\\kappa\\epsilon\\nu\\omicron\\phi\\rho\\omicron\\nu\\iota\\alpha$, $\\mu\\eta\\ \\epsilon\\pi\\iota\\sigma\\tau\\alpha\\mu\\epsilon\\nu\\omicron\\varsigma$)\footnote{1 Tim. 6, 3 ff.}, der nicht bei Christi Worten bleibt.\footnote{Cuenstedt, I, 57.} Das Resultat der Theologie, die den Glauben zum Wissen erheben will, ist, wie alte lutherische Theologen es derb ausdrücken, ein Monstrum, nämlich ein mixtum compositum aus Theologie und Philosophie, ähnlich „dem zweigestaltigen Geschlecht der Kentauren“\footnote{Vgl. das Zitat aus Anselms Cur Deus Homo, II, Notz 1050.}. Nach dieser theologischen Methode hat schon Anselm den stellvertretenden Charakter der Gesetzeserfüllung Christi, die obedientia activa, aus dem Versöhnungswerk Christi gestrichen\footnote{Also aus Abälards Auslegung des Römerbriefs, II, Note 1005.}, und Abälard nach derselben Methode\footnote{Man weiß nicht auch R. Seberg darauf hin, dass Anselm und Abälard auf demselben rationalistischen Boden sich bewegen. Beide stellen neben die fides die ratio. Dagenstedt, II, 41 f.} die stellvertretende Genugtuung Christi (satisfactio vicaria) aus der christlichen Lehre beseitigt\footnote{Ausführliche Darlegung unter dem Abschnitt „Nähere Beschreibung moderner Versöhnungstheorien“ II, 429 ff.}. Bei den neueren Theologen hat der Versuch, den Glauben zum Wissen zu erheben, zu dem Resultat geführt, dass bei ihnen die Leugnung der satisfactio vicaria und die Leugnung der Schrift als der einzigen Quelle und Norm der christlichen Lehre zugestandenermaßen ganz allgemein geworden ist. Dieser Gegenstand muss unter mehreren der folgenden Abschnitte, insonderheit auch unter dem Kapitel „Theologie und Wissenschaft“, wieder aufgenommen werden.

\\subsection*{4. Die zwei Erkenntnisquellen der tatsächlich bestehenden Religionen.}

Wie es auf den Inhalt gesehen, nur zwei wesentlich verschiedene Religionen gibt, die Religion des Gesetzes oder der eigenen Werke und die Religion des Evangeliums oder des Glaubens an Christum, so gibt es auch nur zwei verschiedene Erkenntnisquellen (principia cognoscendi), aus denen die tatsächlich bestehenden Religionen geschöpft werden. Die Religion des Gesetzes in ihren verschiedenen Gestalten außerhalb und innerhalb der äußeren