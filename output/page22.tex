11\hfill Wesen und Begriff der Theologie.\par\par söhnen wollen, stoßen wir auf das persönliche Verhältnis des bösen Gewissens vor Gott oder, was dasselbe ist, auf die Empfindung des göttlichen Zorns und somit auch auf das Verhältnis der Hoffnungslosigkeit. Der Grund hierfür ist der, dass die Menschen, welche auf dem Wege des eigenen Tuns Gott versöhnen wollen, sich vergeblich bemühen; denn: „Aus des Gesetzes Werken wird kein Fleisch vor Gott gerecht.“\footnote{Röm. 3, 20; Gal. 2, 16.} Damit stimmt auch die Erfahrung. Die Schrift berichtet Eph. 2, 12 nicht bloß von einigen, sondern von allen Heiden ohne Unterschied, dass sie „zu jener Zeit“, als sie Heiden waren, „keine Hoffnung hatten und ohne Gott in der Welt waren, \textgreek{ἐλπίδα μὴ ἔχοντες καὶ ἄθεοι ἐν τῷ κόσμῳ}.“ Auch ihre reichlich dargebrachten Opfer ändern ihr „persönliches Verhältnis zu Gott“ nicht im Geringsten, weil ihre Opfer nicht Gott, sondern den Dämonen dargebracht wurden, „\textgreek{ἢ θύει τὰ ἔθνη, δαιμονίοις θύει καὶ οὐ θεῷ}.“\footnote{1 Kor. 10, 20.} Das persönliche Verhältnis der Heiden ist und bleibt also bei allen religiösen Bestrebungen ein Verhältnis des bösen Gewissens und der Hoffnungslosigkeit. Dasselbe gilt natürlich auch von allen denen, die innerhalb der äußeren Christenheit ihr Verhältnis zu Gott auf dem Wege der eigenen Werke regeln wollen. Auch von ihnen gilt: „Die mit des Gesetzes Werken umgehen (so \textgreek{ἐκ ἔργων νόμου}), die sind unter dem Fluch.“\footnote{Gal. 3, 10.} Bei den Christen hingegen ist durch den Glauben an die durch Christum gewirkte Versöhnung das „persönliche Verhältnis zu Gott“ das Verhältnis des guten Gewissens oder der Gnadengewissheit und somit auch das persönliche Verhältnis der Hoffnung des ewigen Lebens, das Gott allen an Christum Gläubigen zu geben verheißen hat. Diesen status quo bei den Christen berichtet erfahrungsmäßig der Apostel Paulus in den Worten: „Nun wir denn sind gerecht worden durch den Glauben, so haben wir Frieden mit Gott durch unsern HERRN JESUM Christum ...“ und rühmen uns der zukünftigen Herrlichkeit, die Gott geben soll“ und: „Wir rühmen uns auch Gottes durch unsern HERRN JESUM Christum, durch welchen wir nun die Versöhnung empfangen haben.“\footnote{Röm. 5, 1. 2. 11.} Es bleibt also bei der Zweizahl der wesentlich verschiedenen Religionen, wenn wir auch „Religion im allgemeinen“ als „das persönliche Verhältnis des Menschen zu Gott“ definieren.\par\par Dasselbe gilt auch von der vielgebrauchten Formel, wonach „Religion im allgemeinen“ als die Art und Weise der Gottes-