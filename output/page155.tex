Deshalb hält derselbe Luther in Bezug auf die „Wahrheitsgewissheit“ an dem Axiom fest, dass der Mensch sich nicht selbst gewiss macht, sondern durch Gottes Wort gewiss gemacht werde; homo certus est passive, sicut Verbum Dei est certum active. So fern ist Luther von allem Subjektivismus, zu dessen Beschützer man ihn machen möchte, dass er auch davor warnt, den wahren, vom heiligen Geist gewirkten Glauben zum Fundament der Gewissheit zu machen und so den Glauben auf den Glauben zu gründen. Luther schreibt:\footnote{496}{S. z. XVII, 2213.} „Wahr ist's, dass man glauben soll zur Taufe [und das bezieht Luther auch auf das äußere objektive Wort]; aber auf den Glauben soll man sich nicht taufen lassen. Es ist gar viel ein ander Ding, den Glauben haben und sich auf den Glauben verlassen und also sich darauf lassen. Wer sich auf den Glauben taufen lässt, der ist nicht allein ungewiss, sondern auch ein abgöttischer, verleugneter Christ, denn er traut und baut auf das Seine, nämlich auf eine Gabe, die ihm Gott gegeben hat, und nicht auf Gottes Wort alleine, gleichwie ein anderer traut und baut auf seine Stärke, Reichtum, Gewalt, Weisheit, Heiligkeit, welches doch auch Gaben sind, von Gott ihm gegeben.“\par Wir haben in dem vorstehenden Abschnitt scharf über die „Selbstbewusstseinstheologie“ geurteilt. Aber es ist nicht unser, sondern Gottes Urteil, wie es klar in der heiligen Schrift geoffenbart vorliegt. Dazu denken wir bei diesem scharfen Urteil auch an uns selbst. Die böse Art, welche sich in der Theologie des „menschlichen Selbstbewusstseins“ geltend macht, wohnt noch in allen Christen, sofern sie noch das böse Fleisch an sich haben. Auch das Fleisch der Christen repräsentiert noch den von Gott abgefallenen Menschen, der, weil er von Gott, seinem Zentrum, abgefallen ist, sich selbst zum Zentrum der Dinge macht, selbstbewusst ist, sich selbst setzt, auch über Gottes Wort sich setzt, also den „Urmenschen“ spielt. Wer von dieser Art herrschenderweise los ist, der preise Gottes Gnade, die ihn ohne sein Verdienst in kirchliche Verhältnisse geführt hat, die der wahren Theologie nicht hinderlich, sondern nur förderlich waren. Wer in hochmütiger Gesinnung bei sich sagen wollte: „Ich danke dir, Gott, dass ich nicht bin wie die andern Leute“, würde eo ipso in die Theologie der Selbstgewissheit zurückfallen, auch mitten in der äußeren Gemein-\par\setcounter{footnote}{494}\addtocounter{footnote}{1}\footnotetext[\thefootnote]{Luth. VI, 1967. Opp. v. A. VII, 367.}