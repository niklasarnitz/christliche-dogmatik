\textbf{V}
\centering\textbf{Vorwort.}

(S. 199 ff.) einige Zitate aus einer Schrift, die Franz Delitzsch
im Jahre 1839 zum dreihundertjährigen Reformationsjubiläum
der Stadt Leipzig herausgab. Der Zweck dieser Zitate ist der
Nachweis, dass die amerikanisch-lutherische Kirche "streng konfes-
sioneller Richtung" das bewahrt, zu klarer Darstellung gebracht
und praktisch angewendet hat, was Gott vor nun beinahe hundert
Jahren auch in Deutschland gab. Delitzsch sagt -- um einige
seiner Sätze in dies Vorwort herüberzunehmen --: "Ich bekenne,
ohne mich zu schämen, dass ich in Sachen des Glaubens um drei-
hundert Jahre zurück bin, weil ich nach langem Irrsal erkannt
habe, dass die Wahrheit nur eine, und zwar eine ewige, un-
veränderliche und, weil von Gott gesondert, keiner Sichtung und
Besserung bedürftig ist." "Ich predige euch Rückschritt,
nämlich zum Worte Gottes, von dem ihr gefallen seid." "Was
ich ausgesprochen und zu verteidigen gesucht habe, das ist nichts
anderes als der Glaube der altlutherischen Kirche, zu dem unsere
Vorfahren vor dreihundert Jahren am heiligen Pfingstfest unter
brünstigem Dankgebet sich bekannt haben." Und Delitzsch stand
nicht allein da. Der Verfasser dieser Dogmatik hat schon als
Student, später als Pastor und auch noch als Lehrer der Theo-
logie mit großem Interesse und wahrer Herzensfreude einige
kleinere Schriften von Ernst Sartorius gelesen. Es sind dies
Schriften "Die Religion außerhalb der Grenzen der bloßen Ver-
nunft" (1822), "Die Unwissenschaftlichkeit und innere Verwandt-
schaft des Nationalismus und Romanismus" (1825), "Von dem
religiösen Erkenntnisprinzip" (1826). In diesen Schriften ist
dogmatisch noch klarer als bei Delitzsch auf die rechte Art der
christlichen Theologie trefflich hingewiesen. Von dem Lesen dieser
und anderer Schriften, die aus Deutschlands Erweckungszeit vor
hundert Jahren stammen, sollte sich die moderne deutschländische
Theologie nicht durch die Tatsache abhalten lassen, dass die Ver-
fasser derselben unter dem Druck einer unwissenschaftlichen Theo-