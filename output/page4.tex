\noindent\hfill V
\section*{Vorwort.}

(S.~199~ff.) einige Zitate aus einer Schrift, die Franz Delitzsch im Jahre 1839 zum dreihundertjährigen Reformationsjubiläum der Stadt Leipzig herausgab. Der Zweck dieser Zitate ist der Nachweis, dass die amerikanisch-lutherische Kirche „streng konsessioneller Richtung“ das bewahrt, zu klarer Darstellung gebracht und praktisch angewendet hat, was Gott vor nun beinahe hundert Jahren auch in Deutschland gab. Delitzsch sagt --- um einige seiner Sätze in dies Vorwort herüberzunehmen ---: „Ich bekenne, ohne mich zu schämen, dass ich in Sachen des Glaubens um dreihundert Jahre zurück bin, weil ich nach langem Irrsal erkannt habe, dass die Wahrheit nur eine, und zwar eine ewige, unveränderliche und, weil von Gott gesondert, keiner Sichtung und Besserung bedürftig ist.“ „Ich predige euch Rückschritt, nämlich zum Worte Gottes, von dem ihr gefallen seid.“ „Was ich ausgesprochen und zu verteidigen gesucht habe, das ist nichts anderes als der Glaube der altlutherischen Kirche, zu dem unsere Vorfahren vor dreihundert Jahren am heiligen Pfingstfest unter brünstigem Dankgebet sich bekannt haben.“ Und Delitzsch stand nicht allein da. Der Verfasser dieser Dogmatik hat schon als Student, später als Pastor und auch noch als Lehrer der Theologie mit großem Interesse und wahrer Herzensfreude einige kleinere Schriften von Ernst Sartorius gelesen. Es sind dies Schriften „Die Religion außerhalb der Grenzen der bloßen Vernunft“ (1822), „Die Unwissenschaftlichkeit und innere Verwandtschaft des Nationalismus und Romanismus“ (1825), „Von dem religiösen Erkenntnisprinzip“ (1826). In diesen Schriften ist dogmatisch noch klarer als bei Delitzsch auf die rechte Art der christlichen Theologie trefflich hingewiesen. Von dem Wesen dieser und anderer Schriften, die aus Deutschlands Erweckungszeit vor hundert Jahren stammen, sollte sich die moderne deutschländische Theologie nicht durch die Tatsache abhalten lassen, dass die Verfasser derselben unter dem Druck einer unwissenschaftlichen Theo-