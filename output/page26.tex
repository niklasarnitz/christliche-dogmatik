15\\Wesen und Begriff der Theologie.\\Logischen Zustände und Erscheinungen auf die Tatsache, dass beide abgesondert, Nichtchristen und Christen, ihrem gemeinsamen menschlichen Wesen entsprechend, eine menschliche Seele und seelische Bewegungen haben, also auf eine rein formale Gleichartigkeit. Was aber die Richtung und die damit angegebene Qualität der Bewegungen betrifft, so ist nicht Gleichartigkeit, sondern ein völliger Gegensatz vorhanden. Nicht zu vergessen ist auch bei dem Studium der Religionspsychologie, dass die Seelen der Nichtchristen nach Christi zuverlässiger Aussage Wohn- und Wirkungsstätten des scharfen Gewappneten sind, der seinen Palast sicher bewahrt,ootnote{Luc. 11, 21.} während die Seelen der Christen vom Geist Gottes bewohnt und getrieben werden.ootnote{1 Kor. 3, 16; Röm. 8, 11, 14.} Wofür sich der Apostel Paulus auch auf die Erfahrung der früheren Heiden und Juden, die beide psychologische Studien durchgemacht haben, beruft.ootnote{Eph. 2, 11, 12 (	extit{μνημονεύετε}); 1 Kor. 12, 2 (	extit{οἰδάτε}); Eph. 2, 2, 3: 	extit{Περιπατήσατε... κατὰ τὸν ἄρχοντα τῆς ἐξουσίας τοῦ ἀέρος, τοῦ πνεύματος τοῦ νῦν ἐνεργοῦντος ἐν τοῖς υἱοῖς τῆς ἀπειθείας}.} Da nun der Fürst dieser Welt und der heilige Geist nicht wesentlich dieselben, sondern zwei verschiedene, einander gerade entgegengesetzte psychologische Phänomene in den Seelen hervorrufen, so führt uns auch die auf die Religion angewandte Psychologie nicht auf einen einheitlichen Religionsbegriff, sondern im Gegenteil auf zwei wesentlich verschiedene Religionen.\par Aber auch die geschichtliche Betrachtung der Religionen führt uns nicht über die Zweizahl derselben hinaus. Wenn wir die ,,religiösen Erscheinungen'' in den nichtchristlichen Religionen uns vorführen und mit denen der christlichen Religion vergleichen (Comparative Religion, Vergleichende Religionsforschung), so stehen wir abermals dem Resultat gegenüber, dass die nichtchristlichen Religionen ihr Verhältnis zur Gottheit, einerlei ob sie sich die Gottheit monotheistisch denken oder polytheistisch oder sonstwie verkehren, auf dem Wege des menschlichen Tuns regeln wollen, während die christliche Religion gerade in dem 	extit{oiov} ἔλ 	extit{ἔργων}, ihr Wesen hat.ootnote{Röm. 3, 28, 4, 5; Eph. 2, 8.} Die wirklich geschichtliche Betrachtung der Religionen führt zu dem Resultat, das Max Müller in einer glücklichen Stunde als Ertrag seiner vergleichenden Religionsforschung so zusammenfasst:ootnote{Dieselben Worte sind 11, 2, Note 8, in deutscher Übersetzung mitgeteilt. Wir geben hier das englische Original.} "In the discharge of my duties for forty years as professor of Sanskrit