\mbox{}\hfill 128\n\par\n\centering \textbf{Wesen und Begriff der Theologie.}\n\par\nweitaus bedeutendste Dogmatik der ganzen neueren Theologie":\n„ihre Leistung ist, dass sie das menschliche Selbstbewusstsein zur Höhe seiner Entwicklung führt“. Auch aber N. Seeberg nennt Schleiermacher den „Reformator der Theologie unsers Jahrhunderts“ und seine „Glaubenslehre“, „das vollkommenste und großartigste dogmatische Werk, das die evangelische Kirche bisher hervorgebracht hat“; „dies Buch hat das 19. Jahrhundert Theologie gelehrt“.\footnote{R. Seeberg, Die Kirche Deutschlands im 19. Jahrhundert, 1903, S. 90. 84.} Diese Bewunderung ist auch, wie bereits erwähnt wurde, in das „lutherisch-konfessionelle“ benannte Lager eingedrungen. N. Seeberg berichtet: „Man kann sagen, dass die gesamte dogmatische Arbeit der Kirche im 19. Jahrhundert ihre Ziele und Bahnen durch dieses Werk Schleiermachers erhalten hat.“\footnote{A. a. O., S. 84.} Namentlich hat auch die sogenannte „Erlanger Theologie“ sich auf die Theologie der Selbstgewissheit festgelegt. Es ist neuerdings von den Erlanger Theologen Hofmann und Frank gezeigt worden,\footnote{Bachmann=Erlangen im Theol. Literaturblatt. Ihmels, Leipzig 1922, S. 395.} „Hofmann und erst recht Frank haben bewusst und grundsätzlich die volle Selbstgewissheit des Christentums und seiner Theologie vertreten.“ Hofmanns ent-schiedene Erklärung über die Selbstgewissheit wurde schon mitgeteilt. Hofmann sagt vom christlichen Bewusstsein, dass es „nicht von der Kirche abhängt noch auf der Schrift, die auf sich die Kirche beruht, auch nicht an jener oder dieser die eigentliche und nächste Verbürgung seiner Wahrheit hat, sondern in sich selbst ruht und unmittelsbar gewisse Wahrheit ist, von dem ihm selbst einwohnenden Geiste Gottes getragen und verbürgt“.\footnote{Schriftbeweis II, S. 11.} Was Frank betrifft, so hat er zum Erweis der „Selbstgewissheit“ des Christentums und der Theologie sein „System der christlichen Gewissheit“ geschrieben, das in erster Auflage 893 Seiten und in zweiter Auflage 954 Seiten umfasst. Frank sagt:\footnote{System der christl. Gewissheit? I, 49.} „Wir haben es hier mit dem zentralen und spezifischen Wesen der christlichen Gewissheit zu tun, wo keine irgendwie von außen kommende Autorität für sich, sondern das christliche Subjekt selbst und persönlich über den Grund und das Recht seiner Gewissheit entscheidet.“ Wie entschieden Frank gerade auch die Heilige Schrift als Fundament der von ihm gemeinten „Gewissheit“ ausgeschieden haben will, sagt er an einer