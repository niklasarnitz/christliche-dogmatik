in the University of Oxford, I have devoted as much time as any man living to the study of the Sacred Books of the East, and I have found the one key-note, the one diapason, so to speak, of all these so-called sacred books, whether it be the Veda of the Brahmans, the Puranas of Siva and Vishnu, the Koran of the Mohammedans, the Zend-Avesta of the Parsees, the Tripitaka of the Buddhists,— the one refrain through all — salvation by works. They all say that salvation must be purchased, must be bought with a price, and that the sole price, the sole purchase-money, must be our own works and deservings. Our own holy Bible, our sacred Book of the East, is from beginning to end a protest against this doctrine. Good works are, indeed, enjoined upon us in that sacred Book of the East far more strongly than in any other sacred book of the East; but they are only the outcome of a grateful heart — they are only a thank-offering, the fruits of our faith. They are never the ransom-money of the true disciples of Christ. Let us not shut our eyes to what is excellent and true and of good report in these sacred books, but let us teach Hindus, Buddhists, Mohammedans, that there is only one sacred Book of the East that can be their main-stay in that awful hour when they pass all alone into the unseen world. It is the sacred Book which contains that faithful saying, worthy to be received of all men, women, and children, and not merely of us Christians — that Christ Jesus came into the world to save sinners.

Endlich führt auch der \textit{philosophische Religionsbegriff} hier über die Zweizahl der Religionen hinaus. Wir stoßen hier auf die Schwierigkeit, dass über den Sinn und Inhalt eines \textit{philosophischen Religionsbegriffs} seine Vertreter keineswegs einig sind. Am verständlichsten reden nach die Religionsphilosophen, welche den \textit{philosophischen Religionsbegriff „rein“} auffassen, das heißt, bei der Feststellung „des Wesens der Religion“ von der heiligen Schrift als Gottes Wort und als Quelle und Norm der christlichen Religion gänzlich absehen wollen. Wird so der \textit{philosophische Religionsbegiff „rein“} gefasst, so ergibt sich allerdings ein Religionsbegriff, welcher der „menschlichen Idee“ von Religion entspricht. Sehr richtig wird von diesem Standpunkt aus gesagt, dass es eine Religionsphilosophie, „streng genommen“, erst dann geben könne, wenn das menschliche Bewusstsein über den Autoritätsglauben und die Vorstellung von einer wunderbaren Belehrung der Menschen durch göttliche Offenbarung hinausgeschritten sei und die religiösen Glaubenssätze