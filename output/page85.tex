\noindent\makebox[\linewidth]{\hfill 74}\vspace{-1.5\baselineskip}\section*{Wesen und Begriff der Theologie.}\vspace{\baselineskip}gelehrt haben, dadurch kann in keinem Menschen contritio und fides, nämlich fides in Christum crucifixum, gewirkt oder „erlebt“ werden. Es ist ziemlich allgemein zugestanden, dass Schleiermacher von seinem reformiert-pantheistischen Standpunkt aus den Begriff der Sünde überhaupt nicht kennt. Und wenn Hofmann aus seinem der Schrift gegenüber „selbständigen“ Glaubensbewusstsein die Erbsünde leugnet,\footnote{Schriftbeweis 2, 562.} so ist auch er konsequenterweise ein schlechter Bußprediger. Beide, Schleiermacher und Hofmann, leugnen ferner die satisfactio vicaria, und damit liegt auf der Hand, dass sie, soweit die aus ihrem Ich geschöpfte Lehre in Betracht kommt, die fides an den gekreuzigten Sünderheiland nicht wirken oder „erleben“ lassen können. Es gilt, klar zu erkennen und festzuhalten, dass erst die contritio nicht aus menschlichen, speziell auch nicht aus „wissenschaftlich vermittelten“ Anschauungen von der Sünde zu lehren ist, sondern aus Gottes eigenem Gesetzesurteil, das die Kirche ebenfalls bis an den Jüngsten Tag in dem in der Schrift geschriebenen Gesetz befiehlt, gelehrt werden muss, und zwar ohne Zusatz und ohne Abtun.\footnote{Matth. 5, 17--19. Gal. 3, 10. 12.} Das ist dann „die Donnerart Gottes, damit er beide die offenbarlichen Sünder und falschen Heiligen in einen Haufen schlägt und lässt keinen recht haben, treibt sie allesamt in das Erschrecken und Verzagen. \ldots Das ist nicht activa contritio, eine gemachte Reu', sondern passiva contritio, das rechte Herzleid, Leiden und Fühlen des Todes. Und heißt die rechte Buße anfangen, und muss der Mensch hie hören [nämlich aus Gottes Wort] solch Urteil: Es ist nichts mit euch allen; ihr seid öffentliche Sünder oder Heilige, ihr müsst alle anders werden und tun, weder ihr seid seid und tut, ihr seid, wer und wie groß, weise, mächtig und heilig, als ihr wollt; sie ist niemand fromm“ \textnormal{(Röm. 15, 16.)}\footnote{Schmalt. Urt., 312, 1--3.} Und in Bezug auf den Glauben, der zur contritio, zu den terrores conscientiae, hinzukommen muss, ist festzuhalten, dass er ebenfalls nicht aus menschlichen Anschauungen --- auch nicht aus „wissenschaftlich vermittelten Anschauungen“ --- von der Vergebung der Sünden zu lehren ist, sondern lediglich aus Gottes Anschauung, das ist, aus Gottes Wort, das die Kirche, Gott sei Dank, in dem in der Schrift geschriebenen Evangelium befiehlt. Das ist dann das Evangelium Gottes, \foreignlanguage{greek}{τὸ εὐαγγέλιον τοῦ θεοῦ}, das dem Paulus ausgesondert war,\footnote{Röm. 1, 1.} dass er tatsächlich verkündigte\footnote{Röm. 10, 14. 17.} und das er sich weder durch theologisierende