\pagenumbering{arabic}
\setcounter{page}{96}
\pagestyle{empty}
\noindent 96\hfill Wesen und Begriff der Theologie.\par
\vspace{\baselineskip}

mente dem Glauben an die Vergebung der Sünden um Christi willen zum Fundament gegeben sind. Die Taufe geschieht ja als δέματος ἁμαρτιῶν\footnote[349]{Apost. 2, 38.}, und im Abendmahl wird Leib und Blut Christi dargereicht als ὑπὲρ δεδομένον und ὑπὲρ ἐκχυνόμενον εἰς ἄφεσιν ἁμαρτιῶν\footnote[350]{Luc. 22, 19 ff.; Matth. 26, 26 ff.}. Liegt somit in beiden Sakramenten eine Zulage oder Darbietung der Vergebung der Sünden vor, so soll sich der Glaube auch auf diese Sakramente gründen\footnote[351]{Die ausführliche Darlegung bei der Lehre von der Taufe unter dem Abschnitt „Der Gnadenmittelcharakter der Taufe (Baptismal Grace)“, 111, 308 ff.}, und Taufe und Abendmahl werden deshalb mit Recht zu den Fundamentallehren gerechnet. Es kann aber jemand in Bezug auf Taufe und Abendmahl aus Schwachheit in der Erkenntnis irren und doch im Glauben an die Vergebung der Sünden stehen, wenn er sich an das gehörte oder gelesene Wort des Evangeliums hält. Der Grund ist der, dass das Wort des Evangeliums die ganze von Christus erworbene Vergebung der Sünden darreicht und Taufe und Abendmahl dieselbe Gnade nur in anderer und besonders tröstlicher Form (\emph{verbum visibile} – \emph{applicatio individualis}) darbieten. Es steht demnach so: Wer Taufe und Abendmahl nicht als Gnadenmittel erkennt und braucht, hat weniger Stützen für seinen Glauben an den um Christi willen gnädigen Gott, als Gott ihm zugedacht hat. Trotzdem hat er, wenn er das Wort des Evangeliums glaubt, durch diesen Glauben die ganze Vergebung der Sünden und somit auch die Seligkeit. So steht es bei allen Kindern Gottes in den reformierten Kirchengemeinschaften, die unter Anleitung ihrer gestorbenen und lebenden Lehrer aus Schwachheit in der Erkenntnis Taufe und Abendmahl nicht als von Gott geordnete Rechtfertigungsmedien erkennen und gebrauchen. Hierauf macht sowohl Luther\footnote[352]{St. V. XVII, 2212.} als auch die Vorrede zum Konkordienbuch\footnote[353]{M., S. 17 ff.} aufmerksam. Solche Lehren nun wie Taufe und Abendmahl, die ihrer Beschaffenheit nach dem Glauben an die Vergebung der Sünden auch zugrunde liegen sollen, aber doch nicht schlechthin zu diesem Glauben nötig sind, weil derselbe schon unabweisbar (durch das Wort des Evangeliums) die unbedingt nötige Stütze hat, sind nicht umfassend \emph{articuli fundamentales secundarii} genannt worden. So rechnet auch Luenstedt\footnote[354]{Systema I, 355.} die Lehre von den Sakramenten zu den Artikeln, \emph{qui non simpliciter fundamentales seu causa salutis sunt, ad fundamentum tamen pertinent.}\footnotemark[355]
