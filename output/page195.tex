Staltung unantastbarer Satzungen bestehen könnte''. ,,Noch bei Duenstedt finde sich Neues ,,fast gar nicht''.\footnote{600}{Nitsch-Stephan, S. 26 ff.} Wir können es verstehen, dass dieser Mangel an Uneinigkeit die moderne Theologie befremdet. Nimmt sie doch in ihrer eigenen Mitte eine ,,schier endlose Fülle von Verschiedenheiten'' wahr, nachdem sie sich auf eine ,,einheitliche Methode'' geeinigt hat, nämlich auf die Methode, die christliche Lehre nicht aus der einen heiligen Schrift, sondern aus dem ,,religiösen Erlebnis'' der vielen theologisierenden Individuen zu beziehen. Selbst die Positiven unter den Erlebnistheologen sehen die Übereinstimmung in der christlichen Lehre als eine Abnormität an. ,,Meine Lehre'' wird mit einem ,,sogenannt'' eingeführt. Dabei ist es allerdings verständlich, dass die Tatsache der ,,wesentlichen Übereinstimmung'' der alten Theologen zwei Jahrhunderte hindurch den modernen Theologen unheimlich vorkommt. Sie suchen daher und finden auch eine Erklärung für diese Tatsache. Aber ihre Erklärung ist historisch unrichtig. Sie behaupten nämlich, die alten Dogmatiker hätten sich ganz allgemein der ,,dogmatischen Exegese'' schuldig gemacht, die Schrift nach dem ,,kirchlichen Lehrbegriff'', insonderheit nach der Konkordienformel, ausgelegt. Nicht nur Duenstedt wird ,,gesunde Schriftforschung'' abgesprochen, sondern auch Chemnitz wird nachgesagt, dass er Melanchthons Loci ,,nach dem Kanon der Konkordienformel auslege''. Dieselbe falsche Beschuldigung wird von der großen Mehrzahl der neueren Theologen gegen die ganze ,,altprotestantische Dogmatik'' erhoben.\footnote{601}{Kirm, Grundriss 3, S. 4. Nitsch-Stephan, S. 25 ff. Dorner, Geschichte der protestantischen Theologie, S. 559. Gas, Geschichte der protestantischen Dogmatik I. 338.} Wir sehen hierher, was wir anderswo\footnote{602}{Zur Einigung der amerikanischen-lutherischen Kirche 2, S. 654.} ausführlicher dargelegt haben: ,,Man muss es als eine Sage bezeichnen, die ungeprüft von Mund zu Mund geht, dass die lutherischen Dogmatiker Leute waren, die eigentlich nur Dogmen nach der Lehrtradition und nach Anleitung der Symbole registrierten, um den Schriftbeweis und die Erhebung der Lehre aus der Schrift sich aber wenig kümmerten. Wer sich die Mühe gemacht hat, die großen Dogmatiker, z.\,B. Gerhard, Quenstedt, Calov, nur einigermaßen kennen zu lernen --- schon die Prüfung auch nur eines locus würde genügen --- wird ganz anderer Ansicht. Bei Gerhard ist die Darlegung der Lehre aus der Schrift Anfang, Mitte und Ende. Die Schriftstellen, welche die Irrlehrer für sich