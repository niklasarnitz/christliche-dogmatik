\text{Wesen und Begriff der Theologie.}\hfill \text{73}\par Gleichwie auch unsere Enthusiasten das äußerliche Wort verdammen, und doch sie selbst nicht schweigen, sondern die Welt voll plaudern und schreiben, gerade als könnte der Geist durch die Schrift oder mündlich Wort der Apostel nicht kommen, aber durch ihre Schrift und Wort müsste er kommen!''\footnote{246) M. 322, 5. 6.}\par Überhaupt bewegt sich die ganze Terminologie der neueren Theologen, die die christliche Lehre aus dem eigenen Innern beziehen wollen, auf dem Gebiet der Selbsttäuschung und damit auf dem Gebiet der Unwahrheit. Das ist klar erkennbar, wenn wir uns das hierher gehörende Vokabular in seinen Hauptpunkten vergegenwärtigen. Die theologischen Lehrer unserer Zeit haben die unabweisbare Pflicht, der studierenden Jugend die hier vorliegenden Selbsttäuschungen aufzudecken.\par Eine Selbsttäuschung liegt vor in der Berufung auf das christliche „Erlebnis''. Es gibt freilich ein christliches Erlebnis. Ohne persönliches christliches Erlebnis gibt es kein Christentum. Jeder Mensch, der ein Christ ist, erlebte und erlebt noch immerfort ersichtlich seine persönliche Verwandlungswürdigkeit vor Gott und sodann das Vertrauen des Herzens auf die Vergebung der Sünden, die Christus durch seine satisfactio vicaria erworben hat. Aber dieses doppelte Erlebnis vermittelt sich nur durch die Verkündigung oder das Lehren des Wortes Gottes, erstlich des Gesetzes und dann des Evangeliums. Wie Christus in seinem Namen unter allen Völkern zu predigen befiehlt Buße (\textnormal{μετάνοιαν}) und Vergebung der Sünden (\textnormal{ἀγαπᾶν ἀγαπᾶσθαι})\footnote{247) Luk. 24, 46. 47.}), denselben Auftrag erhielt in einer Spezialerscheinung Christi der Apostel Paulus, und erhaltenem Auftrage gemäß verkündigte Paulus Juden und Heiden Buße und Bekehrung zu Gott (\textnormal{μετανοεῖν καὶ ἐπιστρέφειν ἐπὶ τὸν θεόν})\footnote{248) Apg. 26, 20.}.\par Beiderlei Wort, das Wort des Gesetzes und das Wort des Evangeliums, hat nun die christliche Kirche bis an den jüngsten Tag in dem schriftlich fixierten Wort der Apostel. Und aus diesem Wort, das Gottes eigenes Wort ist und Gottes eigenes Urteil in re „Sünde'' und „Vergebung der Sünden'' zum Ausdruck bringt, und nicht aus dem Erlebnis der Theologen, unter Beiseitestellung des Schriftworts, wie die Zchthheologie uns zumutet, ist Buße (\textit{contritio}) und Vergebung der Sünden (\textit{remissio peccatorum sive fides in Christum}) in der christlichen Kirche zu lehren. Was „Großvater'' und „Vater'' der Zchthheologie im neunzehnten Jahrhundert, Schleiermacher und Hofmann, samt ihren Nachfolgern aus ihrem Innern über die Sünde