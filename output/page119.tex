108\hfill Wesen und Begriff der Theologie.\par
dass man sagen solle, er sei gelehrter denn andere: da platzt denn der Pöbel mit Haufen zu, sperrt Augen, Ohren und Maul auf. So wird des Glaubens und der Liebe geschwiegen; denn sie meinen, es sei täglich Brot, das sie alle genug gehört haben und wissen; [es] sei verdrießlich, immerdar ein Ding zu hören.\par
In welchem besonderen Sinne neuere Theologen, weil sie die Inspiration der Schrift leugnen und die christliche Lehre aus dem eigenen Innern beziehen wollen, von „Problemen" reden, wird unter dem Abschnitt „Theologie und Gewissheit" behandelt werden.\par
\subsection*{12. Die Kirche und die kirchlichen Dogmen.}\par
Nicht nur die in der heiligen Schrift vorliegende \emph{doctrina divina} in der christlichen Kirche berechtigt, wie im vorhergehenden Abschnitt dargestellt wurde, so ist damit bereits der Sache nach die Frage, was kirchliche Dogmen seien und welcher Wert ihnen zukomme, beantwortet. Wir fügen hierüber noch einen besonderen Abschnitt bei, weil die Dogmenfrage ein in der Kirche der Gegenwart viel behandeltes Thema ist. Die einen treten hier entschieden für ein „undogmatisches" (creedless) Christentum ein. Sie reduzieren das Christentum und die „eigentliche" Aufgabe der christlichen Kirche auf das „soziale Evangelium" (the social gospel). Das „soziale Evangelium" ist so gemeint, dass die Kirche das „Jenseits", inklusive Himmel und Hölle, vergesse oder doch in den Hintergrund treten lasse und statt dessen auf das „Diesseits", die Beglückung der Menschheit in dieser Welt, sich einstelle. Das „soziale Evangelium" betrachtet „Boston as of equal importance with the New Jerusalem, because it takes, almost literally, the vision of St. John, who saw the 'New Jerusalem coming down out of heaven' to occupy this earth".\footnote{394) Hinscher Donald, The Expansion of Religion, 1896, S. 125. Die ausführliche Darlegung u. s. w., 1920, S. 270 ff.; „Die moderne Diesseitigkeitstheologie." Auch L. u. W., 1921, S. 2 ff.; „Das Christentum als Jenseitsreligion." Hier finden sich auch die Literaturangaben. Auf denselben Gegenstand gehen ein die Lehrerhandlungen des Michigan-Distrikts, Bericht 1919, S. 44 ff.}\par
Anders halten dafür, dass die Kirche ohne Dogma nicht wohl auskommen könne. Aber an die Stelle der alten Dogmen, die sich überlebt hätten, müsse ein neues, der Gegenwart angepasstes Dogma treten, über das die Verhandlungen freilich noch nicht abgeschlossen seien. Manche haben sich auch dahin ausgesprochen, dass