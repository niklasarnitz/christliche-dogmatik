\hfill 153\n\n\centerline{Wesen und Begriff der Theologie.}\n\par\nNeuzeit, ein v. Hofmann, ein Thomasius, ein Hengstenberg sogar, mit ihren Versuchen, diese und jene Lehre der Kirche weiter zu entwickeln, so ziemlich -- \emph{sit venia verbo} -- verunglimpft.“ Weiter unten heißt es: „Der in der Tat neue Anstoß, den die christliche Lehrentwicklung durch Schleiermacher erhalten hat, ist ebenso verderblich für die Lehre wie für das Leben geworden.“\n\par\nDen ausführlichsten Nachweis, dass die von der modernen Theologie versuchte Lehrfortbildung nicht eine Fortbildung der christlichen Lehre, sondern einen Abfall von derselben darstellt, hat wohl D. Walther in einer Reihe von Artikeln geliefert, die sich unter der Überschrift „Was ist es um den Fortschritt der modernen lutherischen Theologie in der Lehre?“ in drei Jahrgängen von „Lehre und Wehre“ finden.\footnote{Es sind dies die Jahrgänge 1875, 1876, 1878.} Walther nennt die Theorie, wonach die Dogmen sich erst nach und nach bilden, „eine protestantisch maskierte Schwester des Romanismus“ und eine Umbildung der Kirche „zu einer Philosophenschule, deren Arbeit es ist, die Wahrheit ewig zu suchen, während die Kirche nach Gottes Wort die „Hauslehre“ ist, welcher die Wahrheit als ihr köstlichster Schatz, als ihre gute Beilage vertraut ist, dass sie sie bewahre durch den heiligen Geist, 2 Tim. 1, 13. 14; 1 Tim. 6, 20.“ Walther erinnert auch mit den alten lutherischen Theologen daran, dass die Kirche, geschichtlich betrachtet, nicht eine fortschreitende und aufwärts gerichtete Lehrentwicklung an sich wahrnehmen lasse, sondern vielmehr, was die öffentliche reine Lehre anlange, dem zunehmenden und abnehmenden Monde gleiche, „bald Zeiten sonderlicher Gnadenheimsuchungen, bald Ellipsen erfahre“.\footnote{a. u. W. 1868, S. 136 f.}\n\par\nWir weisen schließlich noch zurück auf das, was wir unter dem Abschnitt „Das Christentum als absolute Religion“ gesagt haben.\footnote{S. 36--42.} Fortbildungsversuche machen wir nur so lange und nur insofern, als wir die christliche Lehre noch nicht kennen. Nachdem wir sie uns insofern wir sie durch Gottes Gnade im Glauben an Gottes Wort erkannt haben, stehen wir anbetend vor ihrer unüberänderlichen göttlichen Größe. Bekanntlich ist das auch die Stellung der heiligen Engel den Dingen gegenüber, die uns die Apostel, durch den heiligen Geist vom Himmel gesandt, verkündigt haben: \foreignlanguage{greek}{εἷς ὃ ἐπιθυμοῦσιν ἄγγελοι παρακύψαι}.\footnote{1 Petr. 1, 12.} Die „Neprinstions-Theologie“ ist die einzige Theologie, die in der christlichen Kirche existenzberechtigt ist.\footnote{Joh. 8, 31. 32; 17, 20; 1 Tim. 6, 3 ff.; Eph. 2, 20.}