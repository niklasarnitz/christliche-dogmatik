178 Wesen und Begriff der Theologie.\par im wesentlichen einheitlichen Methode. Diese lässt sich kurz als die Methode der formalen oder schriftmäßigen Schriftautorität bezeichnen. Sie besteht darin, dass die einzelnen biblischen Aussagen, die als Lehrenoffenbarung verstanden werden, Inhalt und Form der dogmatischen Sätze unmittelbar bestimmen. Ihren Halt findet diese Methode in der Theorie der Verbalinspiration der Schrift.\footnote{588) Grundriss der ev. Dogmatik 3, 2. 4.}\par Ganz richtig sagt Kirn, dass die einheitliche Methode der altprotestantischen Dogmatik in der Verbalinspiration der Schrift ihren Halt findet. Die alten Dogmatiker hätten nicht so hart ob dem sola Scriptura gehalten auch gegen Dissentierende in der eigenen Mitte, z. B. gegen Calixis consensus quinquasecularis wenn sie nicht überzeugt gewesen wären, dass die heilige Schrift Gottes inspiriertes Wort ist. Weil nun aber die Verbalinspiration nicht, wie Kirn meint, eine „Theorie“ ist, das ist, eine nicht unmaßgebliche Prädikantenlehre der altprotestantischen Dogmatiker, sondern die maßgebliche Ansicht Christi und seiner Apostel, so dient es der dringend nötigen Klarheit im theologischen Unterricht, wenn wir die protestantischen Theologen der Gegenwart nach ihrer prinzipiellen Stellung zur Heiligen Schrift in zwei Klassen einteilen. Wir fragen nicht danach, ob sie die synthetische oder analytische Methode befolgen, auch nicht danach, ob sie innerhalb dieser Methoden nebenbei noch die Definitions- oder Kausalmethode oder beide verwenden. Wir fragen vielmehr bei allen Theologen, einerlei in welchem Lande sie leben, lediglich danach, ob sie die heilige Schrift noch für Gottes Wort halten und aus ihr allein die christliche Lehre schöpfen und normieren wollen, oder ob sie bereits so „modern“ geworden sind, dass die Schrift nicht mehr für Gottes Wort und die einzige Quelle und Norm der Theologie halten, sondern sich vorgenommen haben, der christlichen Kirche ein Lehrprodukt darzubieten. Mit den ersteren ist eine Verständigung möglich, auch bei vorliegenden Irrtümern. Mit den letzteren ist die Verständigung unmöglich, weil es an dem gemeinsamen Boden fehlt. \emph{Contra principium negantem disputari non potest.} Dabei können wir der Liebe nach gewissen modernen Theologen, die vom Schriftprinzip abgekommen sind, gutschreiben, was sie inkonsequenterweise von der christlichen Lehre noch stehen lassen. Zugleich ist es aber unsere Pflicht, die Studierenden stets nachdrücklich daran zu erinnern, dass alle Theologen, die das Schriftprinzip aufgegeben haben, auf Grund der Schrift in eine