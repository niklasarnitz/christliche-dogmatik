\begin{center}\textit{Wesen und Begriff der Theologie.}\hfill 52\end{center}\vspace{1em}

2. Die theologische Tüchtigkeit begreift in sich die Tüchtigkeit, nur Gottes Wort zu lehren, das ist, die Tüchtigkeit, sich aller eigenen und anderer Menschen Gedanken über Gott und göttliche Dinge zu enthalten und die vorzutragende Lehre nur aus Gottes Wort, also für unsere Zeit aus der heiligen Schrift, zu nehmen. Von allen, die diese Tätigkeit nicht besitzen, heißt es 1. Tim. 6, 3: „So jemand anders lehret und bleibet nicht bei den gesunden Worten unsers HERRN JEsU Christi, der ist verdüstert und weiß nichts“, \greektext tauphuqeis mhden epistasthai\greektext . Christi Worte aber haben wir in den Worten seiner Apostel und Propheten.\footnote{183} Es widerspricht daher der Tüchtigkeit zum christlichen Lehramt und disqualifiziert für dasselbe, wenn jemand die Lehre, die in der christlichen Kirche zu lehren ist, nicht allein der Heiligen Schrift entnehmen, sondern auch z. B. aus angeblichen unmittelbaren Offenbarungen (Schwärmer) oder aus dem sogenannten „christlichen Bewusstsein“, „Glaubensbewusstsein“, dem „wiedergeborenen Ich“, dem „christlichen Erlebnis“ (moderne Theologen) oder aus den Dekreten des Papstes und der „Kirche“ (römische und romanisierende Protestanten) oder aus der „Geschichte“ usw. schöpfen will. Luther: „Jeremias hat ein ganz Kapitel von den falschen Propheten geschrieben, Jer. 23. Unter andern Worten sagt er also (V. 16): „So spricht Gott, der HERR der Heerscharen: Ihr sollt nicht hören auf der Propheten Wort, die euch predigen; sie betrügen euch und predigen ihres eigenen Herzens Gesicht oder Dünkel und nicht aus dem Mund Gottes.“ Siehe da, alle Propheten, die nicht aus dem Mund Gottes predigen, die betrügen, und Gott verbeut, man soll sie nicht hören. Ist der Spruch nicht klar, dass, wo nicht Gottes Wort gepredigt, da soll niemand zuhören, auch bei der göttlichen Majestät Gebot und Ungnaden, und sei eitel Trügerei? O Papst, o Bischöfe, o Pfaffen, o Mönche, o Theologen, wo wollt ihr hie vorüber? Meint ihr, dass ein gering Ding sei, wenn die hohe Majestät verbeut, was nicht aus Gottes Munde geht und etwas anders denn Gottes Wort ist? Es hat solches nicht ein Drechsler oder Hirt gesagt. Wenn du von deinem Herrn hörtest sagen zu dir: Wer hat dich das geheißen? Das habe ich dir nicht befohlen, ich achte, du dürftest daraus so viel vernehmen, du solltest es nicht getan haben und als Verbot gemieden haben.\footnote{184}

\begingroup\renewcommand\thefootnote{\arabic{footnote}}
\footnote{183} Eph. 17, 20; 1 Petr. 1, 10–12; Eph. 2, 20.\footnote{184} St. 2. IX, 821 f.\endgroup