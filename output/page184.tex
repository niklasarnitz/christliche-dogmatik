Flacius\footnote{569} meint: Theologia per synthesin commodissime traditur,\footnote{569} sagt Baier ziemlich bestimmt: Partes Theologiae revelatae juxta ordinem analyticum collocandae sunt.\footnote{570} In allgemeiner Beschreibung versteht man unter \emph{synthetischer} Methode die Anordnung der Gedanken oder des zu behandelnden Stoffes, wobei wir von den Ursachen zu den Wirkungen fortschreiten oder aus den gegebenen Bestandteilen das Ganze zusammensetzen.\footnote{571} Die \emph{analytische} Methode befolgt die entgegengesetzte Ordnung. Sie geht von den Wirkungen auf die Ursachen zurück oder sie sucht aus einem besondern Teil, z.\,B. dem Endzweck (\emph{finis}), das Ganze abzuleiten.\footnote{572}Auf die Theologie angewandt, handelt die \emph{synthetische} Methode zuerst von Gott als dem Ursprung, wie aller Dinge, so auch der Seligkeit des Menschen, dann von den Ursachen und Mitteln, durch welche der sündig gewordene Mensch zur Seligkeit geführt wird, und schließlich von den letzten Dingen, abschließend mit der ewigen Seligkeit. Nach der \emph{analytischen} Methode wird zuerst von den letzten Dingen, also von der Seligkeit des Menschen, gehandelt, um von dort aus nach Betrachtung des Menschen, der zur Seligkeit geführt werden soll, auf die Ursachen und Mittel (Patris gratiosa voluntas, Filii redemptio, Spiritus Sancti gratia applicatrix, media gratiae etc.) zurückzugehen. Die späteren Theologen meinen, die analytische Methode befolgen zu sollen, weil die Theologie eine praktische Tätigkeit sei, in der man zuerst das Ziel (\emph{finis}: Seligkeit) erkennen müsse, um darauf den Gegenstand, an dem das Ziel erreicht werden soll (\emph{subjectum operationis}, den Menschen) zu untersuchen und endlich mit der Erwägung der Ursachen und Mittel, durch welche der Endzweck an dem \emph{subjectum operationis} erreicht wird, zu schließen.\footnote{573}Die synthetische Methode befolgen im allgemeinen die Dogmatiker von Melanchthon an bis Gerhard inklusive, also, um die Haupt-\footnotetext[569]{Clavius Scripturae, 5. Ausg., vom Jahre 1674, II, 56. Flacius fügt auch eine Tabelle bei, welche zeigt, wie die einzelnen Lehren nach den verschiedenen Methoden zu stehen kommen.}\footnotetext[570]{Compendium I, 76.}\footnotetext[571]{Flacius, Clavius, Ausg. Jena 1674, p. 58.}\footnotetext[572]{Flacius, a.\,a.\,O. Baier-Walther I, 29.}\footnotetext[573]{Cuenstedt I, 25.}