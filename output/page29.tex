\noindent 18\hfill Wesen und Begriff der Theologie.\par den philosophischen, der „menschlichen Idee“ entsprechenden Religionsbegriff fassen, desto sicherer führt er an der christlichen Religion vorbei. Er schließt die christliche Religion nicht in sich, sondern tritt zu ihr in diametralem Gegensatz. So kommen wir auch vermittelst der psychologischen, geschichtlichen und philosophischen Betrachtung der Religion nicht über die Urzahl der wesentlich verschiedenen Religionen hinaus.\par Es gibt, worauf wir hier wenigstens im Vorbeigehen hinweisen sollten, auch solche Vertreter eines „philosophischen“ Religionsbegriffs, die von der Offenbarung der Heiligen Schrift nicht absehen wollen, sondern im Gegenteil die Schrift offenbarung voraussetzen und als Objekt der Betrachtung fordern. Sie geben zu, dass die christliche Religion über alle menschlichen Ideen von Religion hinausliege. Sie meinen aber die in der Schrift geoffenbarten und zunächst auf die Autorität der Schrift hin geglaubten christlichen Wahrheiten nachträglich für den menschlichen Denkprozess so darlegen zu können, dass sie, auch abgesehen von der Offenbarung und Autorität der Schrift, von dem denkenden Menschheitsgeist als Wahrheit erkannt und begriffen werden könnten. So war Anselm, des „Vaters der Schöpfung“, Credo, \emph{ut intelligam} gemeint. Anselm eifert einerseits gegen die „modernen Dialektiker“, die das Wissen vor den Glauben stellen wollen und daher von vornherein das ablehnen, was sie nicht verstehen (\emph{intelligere}) können. Andererseits stellt Anselm den Christen die Aufgabe, vom Glauben zum Wissen fortzuschreiten (\emph{proficere}).\footnote{54) Mansi XX, 742; Christianus per fidem debet ad intellectum proficere, non per intellectum ad fidem accedere, aut, si intelligere non valet, a fide recedere. Sed cum ad intellectum valet pertingere, delectatur.} Ganz ähnlich solche neuere Theologen, die es als die eigentliche Aufgabe der „wissenschaftlichen Theologie“ unserer Zeit ansehen, den Glauben zum Wissen zu erheben, das intellektuelle Bedürfnis der Christen zu befriedigen oder — was dasselbe ist — die christliche Religion als „absolute“ Wahrheit zu erweisen, das heißt, als eine Wahrheit, die, auch abgesehen von der Schrift offenbarung, als Wahrheit erkannt werden könne.\footnote{55) Vgl. Huthardt, Dogmatik 10, S. 5 ff., unter dem Abschnitt „Die Berechtigung der Theologie“. Auch schon Harleß, Theol. Enzyklopädie, 1837, S. 27.} Diesem Versuch der Standeserhöhung des Glaubens liegt die Meinung zugrunde, dass, wenn auch nicht allen Christen, so doch dem „Theologen“ schon in diesem Leben eine Erkenntnis der christlichen Religion eignen könne, die über den Glauben an Gottes Offenbarung hinausgeht.