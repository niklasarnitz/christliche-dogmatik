\text{113} \hfill \text{Wesen und Begriff der Theologie.}

Das „kirchliche Dogma“, das ist, die aus der heiligen Schrift
geschöpfte Lehre, ist auch der Faktor, der den inneren Zusam\-menhang
zwischen den theologischen Disziplinen wahrt und ihnen
den theologischen Charakter sichert, wenn wir die Theologie z. B. in
dogmatische, historische, exegetische und praktische Theologie einteilen.
Die \emph{historische Theologie} ist die vom heiligen Geist gewirkte
Tüchtigkeit, nicht nur die Ereignisse historisch genau darzustellen, son\-dern
auch die dokumentarisch festgestellten Ereignisse und Zustände nach
der heiligen Schrift zu beurteilen, das heißt, unter Gottes eigenes
Urteil zu stellen, das wir in der heiligen Schrift besitzen. Diese
Beurteilung nach der Schrift macht die Kirchengeschichte zu einer
theologischen Disziplin. Die Beurteilung der Ereignisse nach
der subjektiven Anschauung der mit Kirchengeschichte beschäftig\-ten
Dogmatiker oder nach einer andern außerbiblischen Norm
zerstört den theologischen Charakter der Kirchengeschichte. Eine christ\-liche
Kirchengeschichte berichtet uns, „wie es dem lieben Evangelium
in der Welt ergangen ist“, wie es Luther gelegentlich ausdrückt. Wo
es daher in der Kirche recht zugeht, da wird bei der Wahl eines Pro\-fessors
der Kirchengeschichte darauf gesehen, dass der zu Erwählende
die schriftgemäße Lehre in allen ihren Teilen wohl innehabe, damit
der kirchengeschichtliche Unterricht nicht verwirrend, sondern christlich
belehrend wirke. Der Lehrer der Kirchengeschichte soll nicht „Ehr\-furcht
vor der Geschichte“ bezwecken, wie wir kürzlich lasen, son\-dern
wie jede Fachtheologie die Ehrfurcht vor Gottes Wort ver\-mitteln
und stärken. – Die \emph{exegetische Theologie} ist die vom
heiligen Geist verliehene Tüchtigkeit, die Studierenden bei dem in
den Worten der Schrift ausgedrückten Sinn festhalten und falsche Aus\-legungen
als Text und Kontext widersprechend aufzeigen. Damit
die Exegese nicht ihren theologischen Charakter gefährde, hat der
Exeget das \emph{Scriptura Scripturam interpretatur} und \emph{Scriptura sua
luce radiat} durchweg festzuhalten. Alles außerbiblische Material,
mag es die Sprache oder die historischen Umstände betreffen, darf in
der Exegese nicht entscheidend sein. Dies ist namentlich auch
in bezug auf die historischen Umstände. Jede Auslegung ist abzu\-weisen,
die die Worte der Schrift nach einem „historischen Hinter\-grund“
deutet, der nicht in der Schrift selbst gegeben, sondern ganz
oder teilweise zeitgenössischen Prosaschriftenten entnommen ist. Aller
historische Hintergrund, der zum Verständnis der Schrift nötig ist, ist
in der Schrift selbst gegeben. Dieser Gegenstand ist bei der Lehre
von der heiligen Schrift unter dem Abschnitt „Schrift und Exegese“

\quad 7. Pieper, Dogmatik. 1.

\quad 8