\section*{Wesen und Begriff der Theologie.}
\hfill 54
\vspace{1em}

vorhergehen muss, damit der Zuhörer befähigt wird, die Polemik als berechtigt zu erkennen und innerlich mitzumachen.

Bei Nichtbeachtung dieser Ordnung bringt sich der Lehrer leicht in den Verdacht der Streitsucht und des ungerechten Richtens. Aber die Polemik vom öffentlichen Lehramt ausschließen zu wollen, ist wider die Schrift. Sie ist den Lehrern ausdrücklich geboten, wie wir bereits aus Tit. 1, 9--11 sahen und durch die ganze Schrift so reichlich belehrt werden. Alle Propheten und Apostel und Christus selbst haben mit der Verkündigung der rechten Lehre auch die Verwerfung der falschen Lehre verbunden. Walther geht nicht zu weit, wenn er schreibt: „Wer zwar die reine Lehre vorträgt, aber die derselben entgegenstehende falsche Lehre nicht straft und widerlegt, vor den Wölfen in Schafskleidern, das ist, vor den falschen Propheten, nicht warnt und sie nicht entlarvt, der ist kein treuer Hauswalter über Gottes Geheimnisse, kein treuer Hirte der ihm anvertrauten Schafe, kein treuer Wächter auf den Zions, sondern nach Gottes Wort ein Schafsknecht, ein stummer Hund, ein Verräter. Wie viele Seelen dadurch verloren gehen, und wie sehr dadurch die Kirche Schaden leidet, dass die Leugnung nicht geübt wird, liegt zu klar am Tage, als dass es eines Beweises bedürfte. Nicht nur wird die rechte Lehre meist erst dann recht gefasst, wenn zugleich der Gegensatz klar geworden ist, die falschen Lehrer suchen auch ihren Irrtum so listig mit dem Schein der Wahrheit zu umgehen, dass Einfältige ohne vorher erfahrene Warnung trotz ihrer Liebe zur Wahrheit nur zu leicht betrogen werden. Vergebens versucht der Prediger seine Hände in Unschuld zu waschen, weil er die Wahrheit gepredigt habe, wenn er nicht zugleich vor dem Irrtum, und zwar unter Umständen auch mit Nennung des Namens der Irrgeister, gewarnt hat, wenn seine Schafe entweder noch während seiner Amtsverwaltung oder doch, nachdem er sie verlassen musste, eine Beute reißender Wölfe in Schafskleidern werden.“\footnote{189} Was den „duldsamen Geist“ betrifft, so muss auch an dieser Stelle an den Unterschied zwischen Staat und Kirche erinnert werden. Wir müssen zwischen dem Dulden falscher Lehrer im Staat und in der Kirche unterscheiden. Der christlichen Kirche des Neuen Testaments ist nicht geboten, falsche Lehrer aus dem Staat oder der bürgerlichen Gesellschaft zu vertreiben, wozu die Anwendung äußerer Gewalt nötig wäre, die der Kirche verboten ist.

\vspace{1em}
\hrule
\vspace{0.5em}
\footnotesize
189) Pastoral, S. 82 f.