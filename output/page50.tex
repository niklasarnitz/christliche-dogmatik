der Gebote Gottes und der Kirche fordern\footnote{141} und mit den neueren Protestanten, die \textquotedblleft die Umprägung des menschlichen Lebens in seine göttliche Gestalt\textquotedblright (also die Heiligung und die guten Werke) für den Wert des Erlösungswerkes Christi vor Gott \textquotedblleft mitbegründend\textquotedblright sein lassen.\footnote{142} So würden wir die christliche Religion ihres wesentlichen Charakters berauben, nämlich sie auf das Niveau der Gesetzesreligion herabdrücken und damit an die Stelle der Gewissheit der Gnade und der Gotteskindschaft das monstrum incertitudinis setzen.

Halten wir hingegen an Christi satisfactio vicaria fest und damit auch an Röm. 3, 28 und 5, 1 ff., dann ist uns so ipso das Christentum die absolute Religion, über welche hinaus es keine Entwicklung und kein höheres geben kann. Wir theologischen Lehrer unserer Zeit haben die Pflicht, die Studierenden mit großem Ernst vor allen neueren \textquotedblleft Sühnetheorien\textquotedblright zu warnen, die Christi satisfactio vicaria als zu \textquotedblleft juridisch\textquotedblright und \textquotedblleft äußerlich\textquotedblright teils ausdrücklich verwerfen, teils doch als verbesserungsbedürftig bezeichnen.\footnote{143}

Die christliche Religion ist zum andern deshalb vollkommen und unüberbietbar, weil sie nicht Menschenwort, sondern Gottes eigenes Wort, das über alle menschliche Kritik erhaben ist, zur einzigen Quelle und Norm hat. Für die Kirche unserer Zeit ist dies das geschriebene Wort Gottes, die Heilige Schrift (sola Scriptura). In Bezug auf die Heilige Schrift aber müssen wir nach dem normativen Beispiel Christi und seiner Apostel\footnote{144} festhalten, dass sie, obwohl durch Menschen geschrieben, dennoch nicht eine Mischung von Menschen- und Gotteswort, sondern Gottes eigenes und daher unverbrüchliches Wort ist, wie unter dem Abschnitt von der Inspiration der Schrift näher dargelegt wird. Wollten wir die Heilige Schrift nicht für Gottes eigenes, unverbrüchliches Wort, sondern für eine Mischung von Gottes- und Menschenwort halten, so würde sie notwendig Objekt der menschlichen Kritik, und mit der absoluten Weisheit der christlichen Religion wäre es aus. Die menschliche Ansicht überbietet sich im Laufe der Zeit, so dass sie nach fünfhundertzwanzig Jahren oder noch früher das Verwirft, was sie heute noch als zum Wesen der christlichen Religion gehörend

\footnote{141}{Tridentinum, Sess. VI, can. 11. 12. 20.}
\footnote{142}{So z. B. Kirn, Grundriß, S. 118. Hiermit wird die \textquotedblleft Bürgschaftstheorie\textquotedblright an die Stelle der satisfactio Christi vicaria gesetzt.}
\footnote{143}{Vgl. II, 420 ff. den Abschnitt \textquotedblleft Nähere Beschreibung moderner Versöhnungstheorien\textquotedblright.}
\footnote{144}{Joh. 10, 35; 2 Tim. 3, 16. 17; 1 Petr. 1, 10–12; Eph. 2, 20.}