\setcounter{footnote}{89}%geworden, den Unterschied zwischen der reformierten und der lutherischen Kirche dahin zu bestimmen, dass die reformierte Kirche „mehr ausschließlich“ die Schrift Quelle der christlichen Lehre sein lasse, während die lutherische Kirche, weil mehr „historisch“ und „konservativ“ geartet, neben der Schrift auch die Tradition zur Geltung kommen lasse.\footnote{Eine Zusammenstellung von Aussagen über den Charakter der lutherischen und der reformierten Kirche bei \textsc{Luthardt}, \textit{Dogmatik} II, S. 26 f.} Dies dogmengeschichtliche Auffassung ist sachlich unzutreffend. Die Sachlage ist klar erkennbar diese: Die reformierte Kirche, sofern sie in Zwinglis und Calvins Bahnen wandelt, setzt in den Lehren, durch welche sie sich von der lutherischen Kirche unterscheidet und als selbstständige Partei in der äußeren Christenheit sich etabliert hat, das Schriftprinzip beiseite und setzt an dessen Stelle klar ausgesprochene und sehr nachdrücklich festgehaltene rationalistische Axiome.\par Es geschieht dies 1. in Bezug auf die von Gott geordneten Gnadenmittel.\footnote{Siehe die Abschnitte „Die Gnadenmittel im allgemeinen“ und „Alle Gnadenmittel haben denselben Zweck und dieselbe Wirkung“, III, 122 ff.} Obwohl die Aussagen der Heiligen Schrift dahin lauten, dass die göttliche Darbietung der von Christo erworbenen Vergebung der Sünden und die Hervorbringung und Stärkung des Glaubens durch die von Gott geordneten äußeren Mittel (durch das Wort des Evangeliums und durch die Taufe und das Heilige Abendmahl) sich vollzieht,\footnote{So \textsc{Zwingli}, \textit{Fidei Ratio}, \textsc{Klemeier} S. 24: „Duae vel vehiculum Spiritui non est necessarium, spee enim est virtus et latio, qua cuneata feruntur, non qui ferri opus habet.“ Ebenso \textsc{Calvin}, \textit{Inst.} IV, 14, 17: Der Genfer Katechismus (\textsc{Klemeier}, S. 161) schärft ein: „Non esse visibilibus signis inhaerendum, ut salutem inde petamus.“ \textsc{Charles Hodge}, \textit{Syst. Theol.}, II, 684: „Efficacious grace acts immediately.“ So beschränkt auch \textsc{Böhl}, \textit{Dogmatik}, S. 447 f., die Geltung und Wirkung des Wortes auf die vorher und unmittelbar Wiedergeborenen.} so behaupten doch Zwingli und Calvin und auch neuere Reformierte, es sei für den heiligen Geist nicht anständig, seine Gnadenoffenbarung und seine Gnadenwirksamkeit an die äußeren, von Gott geordneten Mittel zu binden, und tatsächlich bediene sich auch der heilige Geist, wo er zur Seligkeit wirke, nicht dieser äußeren Mittel.\footnotemark[93] Es war dieser von den Gnadenmitteln losgelöste „heilige Geist“, der zur Zeit der Reformation im protestantischen Lager Trennung anrichtete und die Beschuldigung erhob, dass Luther das Evangelium nicht recht erkannt habe, sondern im „Fleisch“ – darunter wurde Luthers Festhalten an den Gnadenmitteln verstanden...