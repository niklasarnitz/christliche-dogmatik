\phantomsection\begin{center}\small{\textit{Wesen und Begriff der Theologie.}} \hfill 210\end{center}werden. Er stellt sich damit in die Reihen der genuinen lutherischen Theologen, während die sogenannten konfessionellen Theologen drüben, da sie wirkliche Systeme erstreben, unter dem Einfluss Schleiermachers stehen, wie ihnen das Rattenbusch auch erklärt. --- Systeme machen, scheinbar widersprechende Lehren zu reimen, ist nach Walther nicht Aufgabe des Theologen. Im Gegenteil, alle Systemmacherei hält er für Schaden und nicht für Gewinn in der Theologie; sie bringt nicht Vertiefung, sondern nur Auflösung der Lehren. Er schließt sich da dem Worte Luthers an: „Wenn es soll Reimen gelten, so werden wir keinen Artikel im Glauben behalten.“ Ebensowenig wie Systematisieren und Lehren reimen, hält er es für Aufgabe des Theologen, was auch die Neueren so viel wollen, Schrift und Wissenschaft, Glauben und Wissen zu versöhnen. Dabei muss nach ihm Schrift und Glaube leiden.\par --- Bei aller Achtung vor wirklicher Wissenschaft (\textit{L. u. V. 21, Vorwort}) ist ihm wissenschaftliche Theologie im Sinne der Neueren etwas Fremdes. Die Wissenschaft soll in der Theologie nur als Magd dienen; will sie mehr sein, so muss sie hinaus. Es verdrießt schon die Schrifttheologie, wo man meint, mit wissenschaftlichen Beweisen dem Schriftwort nachhelfen zu wollen.\par --- Das einzige Erkenntnisprinzip der Theologie ist Walther die Schrift. Was nicht aus der Schrift ist, gehört nicht in die Theologie. „Nicht weniger stimmen wir“, schreibt er, „daher auch mit Johann Gerhard: Das einzige Prinzip der Theologie ist das Wort Gottes; darum ist, was nicht in Gottes Wort geoffenbart ist, nicht theologisch.“ Er verwarf daher absolut alles Theologisieren auf Grund der erleuchteten Vernunft (\textit{L. u. V. 21, 225 ff.})\footnote{L. u. V. 21, 225 ff.}. „Alle solche Apologetik“, sagt er (\textit{L. u. V. 34, 326}), „hassen wir von ganzem Herzen, denn sie setzt voraus, dass es noch etwas Gewisseres gebe als Gottes Wort, aus welchem Gewisseren sich der geheimnisvolle Inhalt der Offenbarung herleiten lasse.“\par --- Während die neueren konfessionellen Theologen drüben die Theologie als „kirchliche Wissenschaft vom Christentum“ (so Luthardt, Komp., S. 2) definieren und von ihrem Verwandtschaftsverhältnis zur „Philosophie“ (\textit{Sttngen, Dogm. I, 411})\footnote{Sttngen, Dogm. I, 411. (a. a. D., S. 397).} reden, bezeichnet sie Walther, was bei Stöckingen als primitiver Standpunkt gilt (a. a. D., S. 397), mit Chemnitz und den unsern Alten als habitus practicus. Er sagt: „Was Zweck des Amts ist, ist auch Zweck der Theologie. Es ist dies aber der wahre Glaube, die Erkenntnis der Wahrheit zur Gottseligkeit und endlich das ewige Leben“ (\textit{L. u. V. 14, 73}).\par --- „Ein Theologe wird man nach Walther nur durch den Heiligen Geist aus dem Worte Gottes. Ein rechter