{\large 10\hfill Wesen und Begriff der Theologie.}

Um diesen diametralen Gegensatz zwischen dem Christentum und allen andern Religionen und damit die Zweizahl der wesentlich ver-schiedenen Religionen zu beseitigen, sucht man sonderlich in neuerer Zeit gelegentlich nach einem „allgemeinen Religionsbegriff“. Man versteht darunter, wie bereits gesagt wurde, einen Religions-begriff, der so weit und umfassend ist, dass er das Wesen nicht nur der heidnischen Religionen, sondern auch der christlichen Religionen zum Ausdruck bringt, also die nichtchristlichen Religionen und die christliche Religion unter ein genus befasst. Aber eine nähere Prüfung der Definitionen von Religion, in denen man einen allgemeinen, Christentum und Heidentum umfassenden Religionsbegriff ausge-drückt findet, lässt uns klar erkennen, dass man sich nur eines gemein-samen Ausdrucks bedient, während die bezeichnete Sache völlig verschieden bleibt, solange man die Grundtatsache des Christentums festhält, nämlich die Weltversöhnung durch Christi satisfactio vicaria. Mit Recht hat Karl Hase an „Verbaldefinitionen“ von Religion erinnert, in denen das Wesen der christlichen Religion übersehen wird.\footnote{23}

Dies ist an einigen Beispielen darzulegen. Wir kommen über die Zweizahl der Religionen nicht hinaus, wenn wir „Religion im allgemeinen“ als das „persönliche Verhältnis des Menschen zu Gott“ definieren. Diese Definition ist gegenwärtig ziemlich allgemein an-genommen. So sagt Macpherson: „The common element in all religions is the recognition of a relation between men and God.“\footnote{24} Ebenso Luthardt: „So verschieden die Bezeichnungen für das, was wir Religion nennen, sein mögen, in allen spricht sich doch ein Verhältnis zur Gottheit, wenn auch ein mehr oder minder innerliches und persönliches, aus. Und das dürfen wir wohl als den allgemeinen Begriff der Religion bezeichnen.“\footnote{25} Aber „Verhältnis“ ist eine bloße Abstraktion. Sobald wir konkret werden, das ist, sobald wir das tatsächlich oder geschichtlich vorliegende Ver-hältnis des Menschen zu Gott nach seiner Qualität untersuchen, sehen wir uns sofort der Tatsache gegenüber, dass das „Verhältnis“ ein Zweifaches ist. Bei allen Menschen, die durch eigenes Tun Gott versühnen

\tiny
\begin{enumerate}
    \setcounter{enumi}{22}
    \item Hutterus redivivus 10, S. 11.
    \item Christian Dogmatics; Edinburgh 1898, S. 10.
    \item Glaubenslehre, 1898, S. 34.
\end{enumerate}
