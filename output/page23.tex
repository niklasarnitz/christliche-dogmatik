\hfill\textbf{12}\section*{Wesen und Begriff der Theologie.}
Verehrung bestimmt wird (ratio Deum colendi sive Deo serviendi). Sobald wir daran gehen, die tatsächlich vorliegenden Wesen der Gottesverehrung auf ihr Wesen zu prüfen, stellt sich sofort ihr wesentlicher Unterschied heraus. Die Christen verehren Gott als den Gott, der ihnen ohne des Gesetzes Werke um Christi stellvertretender Genugtuung willen gnädig ist und dem sie daher ihre Werke nicht als Lösegeld für ihre Sünden, sondern als Dankopfer für ihre Erlösung, die durch Christum geschehen ist, darbringen. Wie Paulus von sich sagt: „Was ich jetzt lebe im Fleisch, das lebe ich im Glauben des Sohnes Gottes, der mich geliebt hat und sich selbst für mich dargegeben.“\footnote{30} Und das allein ist ein Gott wohlgefälliger und vernünftiger Gottesdienst.\footnote{31}
Alle Nichtchristen hingegen, weil sie noch ein böses Gewissen haben, meinen ihre religiösen Bestrebungen, soweit sie noch vorhanden sind, darauf richten zu müssen, Gott durch eigenes Tun zu versöhnen. Und dieser modus Deum colendi atque Deo serviendi gefällt Gott so wenig, dass er vielmehr unter Gottes Fluch liegt. „Denn die mit des Gesetzes Werken umgehen, die sind unter dem Fluch“, Gal. 3, 10.
Als das allen Religionen Gemeinsame ist auch das Streben nach Sicherung des Lebens mit Hilfe einer höheren Macht bezeichnet worden. \textit{Kirn z. B.} meint sagen zu können: „Was wir in allen Religionen wiederfinden, ist das Streben nach Sicherung, Ergänzung und Vollendung des persönlichen und gemeinschaftlichen Lebens mit Hilfe einer höheren, übermenschlichen Macht.“\footnote{32} Aber dieses „Streben“, sich durch eigenes Tun das Leben zu sichern, passt nur auf die nichtchristlichen Religionen, weil allen Nichtchristen die Werkreligion, die opinio legis, angeboren ist. Was aber die christliche Religion betrifft, die im Glauben an den für die Sünden der Welt gekreuzigten Christus besteht, so wird sie von keinem Menschen „erstrebt“. Sie ist ja nie in eines Menschen Herz gekommen,\footnote{33} und wenn sie ihm im Werk der Verkündigung entgegentritt, wird sie seinerseits, solange er ein natürlicher Mensch ist, als ein Ärgernis und eine Torheit gewertet, die nicht zu erstreben, sondern zu verwerfen sei.\footnote{34} Auch neuere Theologen geben zu, dass ein allgemeiner Religionsbegriff, der als genus auch die nichtchristlichen Religionen in sich befasse, in der heiligen Schrift sich nicht finde. So heißt es bei \textit{Nitsch}. \textit{Stephan}:\footnote{35} „Im Alten Testament wird ein allgemeiner Begriff der
{\small
\begin{enumerate}
\item[30)] Gal. 2, 20.\hfill 31) Röm. 12, 1.
\item[32)] Grundriß 8, S. 10.\hfill 33) 1 Kor. 2, 9.
\item[34)] 1 Kor. 1, 23; 2, 14.\hfill 35) Ev. Dogmatik, 1912, S. 112.
\end{enumerate}
}