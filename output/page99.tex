\centerline{88 \hspace{4cm} Wesen und Begriff der Theologie.}
\vspace{0.5cm}
2. In bezug auf gute Werke soll der Theologe aus dem Gesetz lehren, was gute, das ist, Gott wohlgefällige, von Gott ge-botene Werke sind, wie Christus die Frage nach guten Werken aus dem Gesetz beantwortet hat\footnote{302}. Dies ist auch noch bei Christen nötig, insofern sie den \textgreek{παλαιὸς ἄνθρωπος} an sich haben und Speise- und Trankgebote, Ehelosigkeit usw. für von Gott gebotene Werke halten\footnote{303}. Was aber das Tun der von Gott gebotenen Werke betrifft, so muss der Theologe wissen und sich stets gegenwärtig halten, dass er die Lust und Kraft zum Tun solcher Werke, die vor Gott gut sind, nur durch das Evangelium bewirken kann. Dies ist des Apostels Praxis: „Ich ermahne euch, liebe Brüder, durch die Barmherzigkeit Gottes, dass ihr eure Leiber begebet zum Opfer, das da lebendig, heilig und Gott wohlgefällig sei.“\footnote{304} \emph{Axiom: Lex praescribit, Evangelium inscribit.}\footnote{305} Luther sagt in bezug auf die theologische Kunst, gute Werke zustande zu bringen: „Ein Gesetzschreiber dringet mit Drohen und Strafen; ein Gnaden-prediger locket und reizet mit erzeigter göttlicher Güte und Barmherzigkeit.“\footnote{306}
\vspace{0.5cm}
3. Auch was die Bekämpfung der Sünde betrifft, so muss der Theologe wissen, dass er durch die Lehre des Gesetzes im besten Falle der Sünde äußerlich wehrt, innerlich aber die Sünde mobil macht und mehrt. Paulus berichtet aus der Erfahrung: „\textgreek{ὅτε ἦμεν ἐν τῇ σαρκὶ [das ist, ὑπὸ τὸν νόμον], τὰ παθήματα τῶν ἁμαρτιῶν τὰ διὰ τοῦ νόμου ἐνήργειτο ἐν τοῖς μέλεσιν ἡμῶν εἰς τὸ καρποφορῆσαι τῷ θανάτῳ}.“\footnote{307} Nur durch das Evangelium wird die Sünde im Menschen getötet: \textgreek{κατηργήθημεν ἀπὸ τοῦ νόμου, ἀποθανόντες ἐν ᾧ κατειχόμεθα, ὥστε δουλεύειν ἡμᾶς ἐν καινότητι πνεύματος}.\footnote{308} Ebenso Röm. 6, 14: „Die Sünde wird nicht herrschen können über euch, tintenmal ihr nicht unter dem Gesetz seid, sondern unter der Gnade.“ \emph{Axiom: Lex necat peccatorem, non peccatum; evangelium necat peccatum, non peccatorem.} „Darum, welcher diese Kunst, das Gesetz vom Evangelium zu scheiden, wohl kann, den sehe obenan und heiße ihn einen Doktor der Heiligen Schrift.“\footnote{309}
\vspace{0.5cm}
\begin{footnotesize}
\noindent
302) Matth. 15, 1 ff.; 22, 35 ff.; 19, 16 ff.\\
303) Kol. 2, 16--23; 1 Tim. 4, 1 ff.\\
304) Röm. 12, 1. \hspace{3.5cm} 305) Jer. 31, 31--34.\\
306) St. E. XII, 318. \hspace{3.5cm} 307) Röm. 7, 5.\\
308) Röm. 7, 6. \hspace{3.5cm} 309) Luther, St. E. IX, 802.
\end{footnotesize}