202 \quad Wesen und Begriff der Theologie.

„Zu euch wende ich mich zurück, geliebte Mitgenossen derselben Stadt und derselben Kirche, denen ich gleich an der Schwelle meines Büchleins mit freundlichem Festgruße entgegentrat. Ich habe mich unterwunden, euer Lehrer zu sein; denn ich weiß, ihr habt Lehrer, die berufen sind, euch zu lehren und zu weiden. Aber das Wort Gottes gebietet uns auch: \emph{„Kasset uns untereinander an uns ermahnen, und das so viel mehr, soviel ihr sehet, dass sich der Tag nahet!“} Hebr. 10, 25. Diesem göttlichen Aufrufe habe ich Folge zu leisten gesucht, denn ich bin ja einer aus eurer Mitte, erfüllt mit herzlicher Liebe zu unserm vielgeliebten Leipzig, der Stadt der Menschenfreundlichkeit und Milde, und zu dem feuerwerten Sachsenlande, dem Lande des Wiedersinns und der Treue. Was ich ausgesprochen und zu verteidigen gesucht habe, das ist nichts anderes als der Glaube der altlutherischen Kirche, zu dem unsere Vorfahren vor dreihundert Jahren am heiligen Pfingstfest unter brünstigem Dankgebet sich bekannten. \emph{„Forcht in der Schrift; ihr werdet erfahren und erkennen, dass dieser Glaube der Lutherische, das er der christliche ist, gegründet auf das unwandelbare und unvergängliche Wort der ewigen Wahrheit.} Dieser Glaube hat nichts zu schaffen mit wirrem Zweifel, brütendem Trübsinn und kränkelndem Siechtum, wie viele meinen; o nein, er bringt helle Augen, getrosten Mut und kernige Frische. Die erleuchtete Vernunft erkennt seine unumstößliche Wahrheit, das wiedergeborene Herz findet in ihm himmlischen Trost, seligen Frieden und reiche Erquickung. Dieser Glaube überwindet die Pforten der Hölle und hält durch die Tore des Todes einen ewigen Triumphzug. \emph{„Sollen wir, meine Geliebten, einen solchen bewährten und festen und freudigen und sieghaften Glauben dahinbegeben für ein halbiertes Christentum, das auf zwei Seiten sinkt und Christum und Belial zu vereinigen sucht, oder gar für eine dummtölzige Aufklärung, die Gottes Wort Lügen straft und die Vernunft vergöttert, die uns im Leben zu belustigen, aber im Sterben nicht zu trösten vermag? Wir würden töricht an uns selber und unverantwortlich an unsern Nachkommen handeln.“}

Aber auch noch zehn Jahre später (1849), als er Professor der Theologie in Rostock war, hat Delitzsch seinen amerikanischen Freunden \emph{„streng konfessioneller Richtung“} nicht nur seinen Gruß entboten, sondern auch sein Bekenntnis zum lutherischen Bekenntnis erneuert und die Mahnung hinzugefügt, an diesem Bekenntnis festzuhalten, weil darin die „Zukunft“ der lutherischen Kirche beschlossen sei. Delitzsch hat nämlich seine Schrift \emph{„Vom Hause Gottes oder der}