da{\ss} die Schwärmer die Abendmahlslehre nicht aus dem \glqq{}Glauben\grqq{} konstruieren, sondern sie in den Schriftworten, die vom Abendmahl handeln, ausgedrückt aufzeigen. \glqq{}So ist das die Summa davon, dass wir die helle, dürre Schrift für uns haben, die also lautet: \glq{}Nehmet, esset; das ist mein Leib.\grq{}\grqq{} So sollen auch der Zwingel und der Ökolampad \glqq{}Schrift ausbringen, die also laute: \glq{}Das bedeutet meinen Leib\grq{} oder: \glq{}Das ist meines Leibes Zeichen.\grq{}\grqq{} So gänzlich fern liegt Luther alle Lehrkonstruktion unter Beiseitelassung der Schriftworte, die von den einzelnen Lehren handeln. Luthers gewaltige \glqq{}Predigt von den Christen garnisch und waffen\grqq{}\footnote{St. A. IX, 810 ff.} ist dem Hauptgedanken nach eine Warnung vor jeder Lehrkonstruktion, nämlich eine Warnung vor einem \glqq{}Glauben\grqq{}, der nicht in allen Teilen auf dem ausgedrückten Schriftwort ruht. \glqq{}Wir haben die Artikel unsers Glaubens in der Schrift genugsam gegründet, da halte dich an und lass dir es nicht mit Glossen drehen und nach der Vernunft deuten, wie sich\textquoteright{}s reime oder nicht.\grqq{} Auf den Einwand, dass nicht eine Einheit, sondern Widersprüche herauskommen möchten, wenn man sich lediglich an die Schriftworte halte, antwortet Luther: \glqq{}Die Schrift wird nicht wider sich selbst noch einigen Artikel des Glaubens sein, ob es wohl in deinem Kopf widereinander ist und sich nicht reimt.\grqq{} Kurz, nach Luther kommt der christlichen Lehre freilich \glqq{}Einheit\grqq{} zu, aber nicht eine Einheit \glqq{}im Kopfe\grqq{} des dogmatisierenden menschlichen Subjekts, nach der Konstruktionsmethode, sondern eine Einheit, die darin begründet ist, dass die christliche Lehre in allen ihren Teilen direkt aus der Schrift genommen wird, die Gottes Wort und deshalb nicht wider sich selbst, sondern völlig einheitlich ist. \glqq{}Das hat den guten Mann Ökolampad betrogen, dass Schrift, so seinem Kopf widereinander sind, freilich müssen vertragen werden und ein Teil einen Verstand nehmen, der sich mit dem andern leidet. . . . Wenn sie aber sich bedächten zuvor und sähen zu, wie sie nichts reden wollten denn Gottes Wort, wie St. Petrus lehrt, und ließen ihr Sagen und Segen daheim, so richteten sie nicht so viel Unglücks an.\grqq{}\footnote{St. A. XX, 798.} Es sollte noch erwähnt werden, dass die Systembildung, welche auf eine \glqq{}lückenlose\grqq{}, \glqq{}widerspruchslose\grqq{} Einheit oder auf ein festgeschlossenes \glqq{}logisches Ganzes\grqq{} eingestellt ist, auch in der Gegenwart scharfe Erfahrung hat. Es ist gesagt worden, dass die ganze