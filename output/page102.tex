\pagenumbering{arabic}\n\setcounter{page}{91}\n\markboth{Wesen und Begriff der Theologie.}{Wesen und Begriff der Theologie.}\n\n[nämlich doctrinae justificationis] tenent, sunt vel Iudaei vel Turcae vel Papistae vel haeretici.\footnote{318) Ad Gal. Ep. I, 20; Ep. I, IX, 29.} Die Dogmatiker nennen die Lehre von der Rechtfertigung auch wohl omnium fundamenta fundamentalissimum.\footnote{319) Die Zitate bei Baier-Walther III, 243 sq.}\nDie Schrift belehrt uns aber sehr nachdrücklich, und auf einzelne Lehren eingehend, darüber, welche Lehren der Glaube an die Vergebung der Sünden um Christi willen voraussetzt und in sich schließt.\n\n\begin{enumerate}\n    \item Die Schrift lehrt sehr klar, dass der seligmachende Glaube die Erkenntnis der Sünde und der Folge der Sünde, der ewigen Verdammnis, voraussetzt. Wo diese Sündenerkenntnis, das ist, die Erkenntnis der eigenen Verdammungswürdigkeit, nicht vorhanden ist, sondern noch Vertrauen auf eigene Würdigkeit sich findet, da kann auch kein Glaube an die göttliche Vergebung der Sünden um Christi willen vorhanden sein. Daher soll nach der von Christo vorgeschriebenen Lehrmethode unter allen Völkern erst Buße und dann Vergebung der Sünden gepredigt werden.\footnote{320) Luk. 24, 47.} Das illustriert Christus auch durch das Beispiel des Pharisäers und Zöllners. Christus verwirft sehr entschieden den Glauben des Pharisäers, der sich nicht des Zornes Gottes und der Verdammnis schuldig gab, sondern Gott dankte, dass er nicht sei wie andere Leute, das ist, sich vor Gott für besser hielt als Räuber, Ungerechte, Ehebrecher oder auch wie jener Zöllner.\footnote{321) Luk. 18, 9—14.} Hierher gehören alle Schriftauslagen von den „zerbrochenen Herzen“, bei denen Gott mit seiner Gnade einkehrt und wohnt.\footnote{322) Jes. 66, 2; 57, 15; Ps. 34, 19; 51, 19; Luk. 4, 18.}\n    \item Die Schrift lehrt ferner sehr bestimmt, dass der seligmachende Glaube die Erkenntnis der Person Christi in sich schließt, nämlich den Glauben, dass Christus \textit{θεάνθρωπος}, Gott und Mensch, ist. Die Frage Christi Matth. 22, 42: „Wie dünket euch um Christum? Wes Sohn ist er?“ hat nicht bloß „intellektuellen“, sondern einen sehr praktischen Wert. Dass ohne den Glauben an die wesentliche Gottheit Christi kein Glaube an Christum vorhanden ist, sagt Christus selbst. Er verwirft den Glauben des jüdischen Publikums, das ihn für Johannes den Täufer, Elias, Jeremias oder der Propheten einen hielt, und bestätigt den Glauben seiner Jünger, die ihn als des lebendigen Gottes Sohn durch des Vaters Offen-\n\end{enumerate}