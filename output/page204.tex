\begin{flushright}193\end{flushright}
\begin{center}Wesen und Begriff der Theologie.\end{center}

Donatisten, Calvinisten! --- Und wer mag alle die Sekten nennen,
die mit uns wieder auferstanden und neu geworden sein sollten?
Kurz, alles sollen wir sein, nur nicht, was wir allein sein zu wollen
selbst erklärten: Bekenner der Lehre der Reformation, Lutheraner.
Was konnten und mussten wir nun tun, wollten wir uns nicht zu
einer Sekte stempeln lassen? Wir mussten, solange man uns den
Charakter, treue Lutheraner zu sein, absprach, fort und fort das
teure Bekenntnis und die alten, unbestritten treuen Lehrer unserer
Kirche aufrufen, als unsere Zeugen für uns aufzutreten. Und wir
meinen, wir haben es in einer Weise getan, dass, wer es nur sehen
wollte, es auch sehen musste, dass wir jenen treuen Lehrern unserer
Kirche nicht blindlings, sondern in lebendiger Überzeugung gefolgt,
nicht ihre geistlosen Nachbeter und Nachtreter, sondern ihre Söhne
sind, so dass wir allezeit haben sagen können: „Ach glaube, darum
rede ich.“ --- Auf die Anlage, dass die amerikanisch-lutherische Kirche
die Symbole an die Stelle der Schrift gesetzt oder die Schrift nach
den Symbolen ausgelegt oder „Symbololatrie“ getrieben habe, ant-
wortet Walther: „So unvergleichlich wertvoll uns vor allem das
reine Bekenntnis unserer Kirche gewesen ist, so haben wir uns doch
selbst diesem nie als einem uns aufgelegten Lehrgesetz unterworfen,
sondern es vielmehr allein darum mit fröhlicher Danksagung gegen
Gott für seine unaussprechliche Gnade angenommen, weil wir darin
unser eigenes Bekenntnis gefunden haben. Gar manchen harten
Kampf hat auch unsere amerikanisch-lutherische Kirche mit den hiesi-
gen stolzen Sekten zu kämpfen gehabt, denen wir selbstverständlich
das Zeugnis unserer Väter nicht entgegenhalten konnten, und wer
Zeuge dieser Kämpfe gewesen ist, weiß, dass Gottes geschriebenes
Wort auch in unsern schwachen Händen sich als eine siegreiche Waffe
erwiesen hat.“ --- Über die Wirkung der alten Dogmatiker äußert
sich Walther so: „Übrigens kennen die uns nicht, welche un-
sere Theologie die des 17. Jahrhunderts nennen. So hoch wir die
immense Arbeit schätzen, welche die großen lutherischen Dogmatiker
dieser Periode getan haben, so sind doch eigentlich nicht sie es, zu
denen wir zurückgekehrt sind, sondern vor allem unsere teure Kon-
kordia und Luther, in welchem wir den Mann erkannt haben, den
Gott zum Moses seiner Kirche des Neuen Bundes erkoren hat, seine
in die Knechtschaft des Antichrists geratene Kirche, die Rauch- und
Feuersäule des goldreinen und lauteren Wortes Gottes voran, aus
derselben auszuführen. Die Dogmatiken jener Zeit, so unmerklich
reiche Schätze der Erkenntnis und Erfahrung auch darin aufgespeichert

\vspace{1em}
\noindent\small F. Bieber, Dogmatik. 1.