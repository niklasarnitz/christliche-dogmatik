kann, ist und bleibt ein Ärgernis für andere, die den Irrtum bei sich nicht diskontieren, sondern ihn auffassen, wie er lautet, ihn weiter tragen und wohl gar mit Berufung auf die „Väter“ weitere Trennungen in der Kirche anrichten. Um das Ärgernis, das die Abweichung von Gottes Wort naturgemäß anrichtet, möglichst abzutun, haben öffentliche Lehrer das Bedürfnis empfunden, früher vorgetragene Irrtümer öffentlich zu widerrufen. Deshalb hat Augustinus seine \textit{Retractationes} geschrieben\textsuperscript{365)} und deshalb bittet auch Luther, man wolle seine ersten Schriften „mit vielem Erbarmen“ lesen, weil sie von römischen irrigen Lehren noch nicht ganz frei seien.\textsuperscript{366)}\newline\newline 3. Neben der in einer Lehre das Zeugnis der Schrift beiseitesetzt, steht tatsächlich, wenn ihm selbst auch nicht klar bewusst, das ganze christliche Erkenntnisprinzip in Frage. Wir dürfen doch nicht vergessen, dass alle Artikel der christlichen Lehre ein gemeinschaftliches und in seiner Autorität unteilbares Erkenntnisprinzip haben, nämlich das Wort der Heiligen Schrift. Wenn wir nun aus Gründen der „Undenkbarkeit“, Irrationalität oder aus anderen in unserm Ego gelegenen Ursachen die Autorität der Schrift in gewissen Lehren beiseitesetzen, z. B. in den Lehren von der Taufe, vom Abendmahl, von der Bekehrung, von der Gnadenwahl, von der Inspiration der Schrift usw., so legen wir konsequenterweise auch die Autorität der Schrift beiseite, wenn sie uns von dem Gotteslamm sagt, dass der Welt Sünde trägt, und von dem Blut Christi, das uns rein macht von aller Sünde.\textsuperscript{367)} Darauf sieht Luther, wenn er warnend sagt: „Der heilige Geist [der in allen Worten der Schrift redet] lässt sich nicht trennen noch teilen, dass er ein Stück sollte wahrhaftig und das andere falsch lehren oder glauben lassen.“\textsuperscript{368)} Freilich setzt Luther auch hier hinzu: „Ohne wo Schwache sind, die bereit sind, sich unterweisen zu lassen und nicht halsstarriglich zu widersprechen.“ Zu den Schwachen, die bereit sind, sich unterweisen zu lassen, gehören alle, in deren Herzen noch der Glaube an die von\newline\newline \textsuperscript{365)} Über den Beweggrund spricht sich Augustinus im Prolog zu seinen \textit{Retractationes} so aus: De tam multis disputationibus meis sine dubio multa colligi possunt, quae, si non falsa, at certe videantur, sive etiam convincantur non necessaria. Quem vero Christus fidelium suorum non terruit, ubi ait: Omne verbum otiosum quodcunque dixerit homo, reddet pro eo rationem in die iudicii. Dann fährt er fort: Restat igitur, ut me ipsum iudicem sub magistro uno, cuius de offensionibus meis iudicium evadere cupio. Bd. Basil. I, 1.\newline \textsuperscript{366)} Opp. v. a. i. 15; Erl. S. XIV, 439; XIX, 293. 296 und oft.\newline \textsuperscript{367)} Joh. 1, 29; 1 Joh. 1, 7.\newline \textsuperscript{368)} Erl. S. XX, 1781.