\textsuperscript{41}\n\hfill Wesen und Begriff der Theologie.\n\nverständlich voraus, dass der Allmächtige, die Geschichte durchwaltende Gott nicht so in die „Geschichte der Menschheit“ eingreifen konnte oder wollte, dass er sofort nach dem Sündenfall der Menschheit Christum als den Erretter aus Sündenschuld und Tod offenbarte. Ein solcher Eingriff Gottes in die Geschichte der Menschheit liegt aber tatsächlich vor, wie die Schrift sehr klar und nachdrücklich berichtet. Wenn unmittelbar nach dem Sündenfall die göttliche Verheißung dahin lautet, dass der Weibesame der Schlange den Kopf zertreten, also des Teufels Werke, die Sünden-schuld und den Tod, unter den Menschen abtun werde, so ist damit bereits ausgesagt, dass kein anderes und nichts anderes das Menschengeschlecht aus Sündenschuld und Tod erretten werde als das Vertrauen auf das Werk des Weibesamens. Diese Absolutheit der christlichen Religion ist durch die ganze Schrift Alten Testaments gelehrt, wie Petrus Apost. 10, 43 und versichert, dass von Christo alle Propheten zeugen, dass durch seinen Namen alle, die an ihn glauben, Vergebung der Sünden empfangen sollen. Ebenso sagt Christus selbst, dass die ganze Schrift Alten Testaments ihn [Chris-tus] als Geber des ewigen Lebens bezeuge und insonderheit auch Abraham bereits an ihn geglaubt habe.\footnote{146} Was die Schrift von Bundesänderung und dem Altwerden (\foreignlanguage{greek}{παλαιοῦσθαι}) eines Bundes sagt,\footnote{147} bezieht sich nicht auf das Evangelium von Christo, sondern auf den zwischeneingekommenen Gesetzesbund von Sinai.\footnote{148} Kurz, nach der Schrift steht es fest, dass die christliche Religion von allem Anfang an in die Geschichte der Menschheit eingetreten ist, nicht als andern Religionen koordiniert, auch nicht als andere Religionen in sich aufnehmend und sie ergänzend, sondern als die absolute Religion in dem scharf ausgeprägten Sinne, dass sie den Weibesamen in einem Erlösungswerk als den einzigen Erretter von Sündenschuld und Tod für die ganze Menschheit darstellt und damit alle andern Religionen, wie sie Namen und Gestalt haben mögen, für Irrwege und für nicht existenzberechtigt erklärt. Wir sollten daher auch nicht von der christlichen Religion als der „höchsten“, „vollkommensten“ Religion, als der „Climax“ der Reli-gionen usw. reden, weil durch solche Ausdrücke die Vorstellung er-weckt wird, als ob zwischen dem Christentum und den nichtchristlichen Religionen nur ein gradueller Unterschied stattfinde, während der Unterschied doch nach dem Ursprung (God-made, man-made),\n\n\vspace{10pt}\n\n\textsuperscript{146} Joh. 5, 46, 39; 8, 56. \textsuperscript{147} Jer. 31, 31--34; Hebr. 8, 6--13. \textsuperscript{148} Gal. 3, 17 ff.