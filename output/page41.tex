30\par\begin{center}Wesen und Begriff der Theologie.\end{center}und die entschiedenste Adoption eines spekulativen Rationalismus seitens Calvins tritt uns da entgegen, wo Calvin zum Schutze seiner \emph{gratia particularis} über Christi Klage und Tränen vor Jerusalem (Matth. 23, 37; Luk. 19, 41 ff.) und über die zur Rettung des Volkes ausgebreiteten Arme Gottes (Jes. 65, 2; Röm. 10, 21) als eine zuverlässige Offenbarung des allgemeinen Gnadenwillens Gottes geradezu spottet, durch die Bemerkung, dass dadurch irrigerweise Menschliches auf Gott übertragen würde.\footnote{102) Quod humanum est ad Deum transferri. Inst. III. 24, 17.} Wer sorgfältig die betreffenden Partien in Calvins Institutiones liest, kann sich des Eindrucks nicht erwehren, dass Calvin im Interesse seiner rationalistischen Spekulation über den absoluten Gott zu einem fanatischen Bekämpfer und Verfolger aller Schriftaussagen, die auf die allgemeine Gnade Gottes in Christo lauten, wird. Die notwendige Folge der Wegdeutung der \emph{gratia universalis} ist die, dass sie das Evangelium von Christo \emph{praktisch unbrauchbar} macht. In dem vom Gesetz Gottes getroffenen Sünder kann der Glaube an den Sünderheiland nicht entstehen, solange die Vorstellung von einer \emph{gratia particularis} das Bewusstsein beherrscht. Dass es auch unter den calvinistischen Reformierten Kinder Gottes gibt, die sich ihrer Erlösung durch Christum freuen, kommt daher, dass manche unter ihnen die \emph{gratia particularis} nie angenommen haben, andere, die sie angenommen haben, unter den \emph{terrores conscientiae} ihre Zuflucht zur \emph{universalis gratia} nehmen, wozu inkonsequenterweise reformierte Lehrer selbst raten und so ihre Parteistellung in Bezug auf die \emph{gratia particularis} selbst verurteilen.\footnote{103) Vgl. Snedenborgs Darlegung, dass die seelsorgerische Praxis die calvinistischen Reformierten auf den lutherischen Standpunkt von der allgemeinen Gnade treibt, in Theologische Darlegung des luth. und ref. Lehrbegriffs. S. 269 ff.}Neben den calvinistischen gibt es auch \emph{arminianische Reformierte}. Diese wollen im Unterschied von den calvinistischen Reformierten die \emph{gratia universalis} festhalten, meinen aber, dies nur so tun zu können, dass sie die \emph{sola gratia} fahren lassen. Sie lehren eine menschliche Mitwirkung zur Entstehung des Glaubens.\footnote{104) Die Apol. Conf. demonstr., p. 162, behauptet, dass die Gnadenwirkung Gottes zur Bekehrung non posse exire in actum sine cooperatione liberae voluntatis humanae ac proinde, ut effectum habeat, pendere a libera voluntate.}