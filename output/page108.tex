Wesen und Begriff der Theologie.\hfill 97\n\nHieraus ergibt sich, dass zur Erhaltung am Plage ist, wenn es sich um ein Urteil über den persönlichen Glaubensstand einzelner Personen handelt, die sekundäre Fundamentalartikel leugnen. Einerseits ist freilich festzuhalten: Wer sekundäre Fundamentalartikel leugnet, stößt konsequenterweise auch die primären um wegen des engen Zusammenhangs, der zwischen ihnen besteht. Wir können dies ebenfalls an den Lehren von der Taufe und vom Abendmahl illustrieren. Wer leugnet, dass Gott durch Taufe und Abendmahl Vergebung der Sünden geben könne, weil Taufe und Abendmahl nur äußere Mittel seien, muss konsequenterweise auch die Sündenvergebung durch das Wort des Evangeliums leugnen, weil das Evangelium gleicherweise ein äußeres Mittel ist. Ein anderes Beispiel: Wer die Mitteilung der Eigenschaften (\textit{communicatio idiomatum}) in Christo leugnet auf Grund des Axioms, dass das Endliche des Unendlichen nicht fähig sei (\textit{Finitum non est capax infiniti}), leugnet konsequenterweise auch die Mitteilung der göttlichen Person des Sohnes Gottes an die menschliche Natur, das heißt, er leugnet die Menschwerdung (\textit{incarnatio}) des Sohnes Gottes. Die Erfahrung lehrt aber, dass hier eine „glückliche Inkonsequenz“ gibt, vornehmlich bei den sogenannten Laien, aber auch bei den Lehrern und gelehrten Theologen. Es ist, wie wir wiederholt erinnern müssen, mit der Logik nach dem Sündenfall bei uns Menschen schlecht bestellt, und diese Logik wird außerdem noch im Streit durch erregte Leidenschaften verschlechtert. Hierauf gründet sich Luthers relativ mildes Urteil über \textit{Nestorius}. Luther sagt bekanntlich: „Wiewohl nun, gründlich zu reden, aus Nestorius Meinung folgen muss, dass Christus ein purer Mensch und zwei Personen sei, so ist’s doch seine Meinung nicht gewesen. Denn der grobe, ungelehrte Mann sah das nicht, dass er unmögliche Dinge vorgab, dass er zugleich Christum ernstlich für Gott und Mensch in einer Person hielt und doch die Idiomata der Naturen nicht wollte derselben Person Christi zugeben. Das erste will er für wahr halten, aber das soll nicht wahr sein, das doch aus dem ersten folget. Damit er anzeigt, dass er selbst nicht versteht, was er verneinet.“\footnote{355) St. Z. XVI, 2230.} Luther führt dafür weitere Beispiele an, weil „solcher Unverstand nicht seltsam in der Welt“ ist.\footnote{356) St. Z. XVI, 2238.} Die römische Lehre vom Messopfer stößt an sich den Grund des Glaubens, die \textit{sola gratia}, um. Aber manche haben bei sich, für ihre Person, diese Konsequenz nicht ge-