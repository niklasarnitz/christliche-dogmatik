Wesen und Begriff der Theologie.\hfill 135

\noindent \dots behaup
\mbox{}ten „unmittelbaren Gewissheit“ --- teils tatsächlich, teils ausdrücklich zu. Aus einer fehlbaren Quelle und Norm kann aber keine Gewissheit kommen. Zum anderen stehen sie --- abermal im Widerspruch mit der behaupteten „unabhängigen“ Gewissheit --- außerhalb des Ich gelegene Gewissheitszeugen herbei, die sämtlich ebenfalls ungewiss sind.\footnote{Hier kann Zöckels verglichen werden, Zentralfragen, S. 159--166. Auch Kirn, Dogmatik, S. 1--6.} Dem „christlichen Erlebnis“ wird nämlich wegen der „möglichen Selbsttäuschung“ teils empfohlen, teils ausdrücklich zur Pflicht gemacht, auf eine ganze Anzahl außerhalb des Subjekts gelegene Faktoren gebührende Rücksicht zu nehmen, nämlich auf „die vornehmsten Weltanschauungsformen“, auf „den sonstigen Wahrheitsbesitz der Menschheit“, auf „die wirklichen Resultate der übrigen wissenschaftlichen Forschung“, als da sind: Geschichtswissenschaft, sonstige Religionswissenschaft („vergleichende Religionsforschung“), die naturwissenschaftliche Forschung usw. Durch diese herbeigezogenen auswärtigen Faktoren wird die Gewissheit noch mehr in unerreichbare Ferne gerückt, weil zugegebenermassen allgemeine gesicherte Resultate auf den genannten Gebieten bisher noch nicht vorliegen. So steht sich das der Gewissheit nachtrachtende „christliche Subjekt“ vor eine Sammlung von lauter Ungewissheiten gestellt. Dazu kommt auch noch die Forderung, diese Sammlung von Ungewissheiten „zu einem einheitlichen Ganzen zu verknüpfen“. Das ist wahrlich saure Arbeit, ein „Problem“ im vollsten Sinne des Worts. Der vielgebrauchte Ausdruck, der Theologe habe den Wahrheitsbegriff „herauszuarbeiten“, ist bezeichnend. Es ist das Wälzen des Sisyphussteins, das Wasseröpfen der Danaiden. Kurz, die Theologie, welche die Schrift als einzige Quelle und Norm der Theologie verlassen und ihre Zelte auf dem „frommen Selbstbewusstsein des theologisierenden Subjekts“ aufgeschlagen hat, ist eine Theologie der Ungewissheit.

Es ist zur Rettung der Gewissheit bei der Selbstbewusstseins-Theologie darauf hingewiesen worden, dass nach der Schrift der Christ seinen Christenstand und sein Stehen in der Wahrheit doch auch von seinen christlichen Werken „ablesen“, also durch Reflexion auf sein Ich erkennen könne. Das ist wahr. Das ist Schriftlehre.\footnote{Joh. 8, 47; 1 Joh. 3, 14; 2, 3, 4; Matth. 6, 14; 2 Petr. 1, 10.} Das ist auch Lehre des lutherischen Bekenntnisses.\footnote{Apol. m. 135, 184 f.} Wiewohl --- ebenfalls nach Schrift und Bekenntnis\footnote{1 Joh. 3, 20; Röm. 4, 16. F. C. 620, 43 ff.} --- „das Wetter so böse