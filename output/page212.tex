Büchern spricht sich Delitsch so aus: „Die symbolischen Bücher, heißt es jetzt, waren gut für damals, jetzt sind sie reif, abrogiiert zu werden, damit man die Lehrer, die darauf beeidet werden, nicht mehr dem Verdachte des Meineides aussetze. Denn die Lehren der Neologen widerstreiten schnurstracks den symbolischen Büchern, wenn man nicht etwa diese ebenso vergeistigend interpretieren will, wie man es mit der Bibel zu machen pflegt. Niemand ist ein Glied der lutherischen Kirche, als der die Schriftmäßigkeit dieses Bekenntnisses anerkennt und, wo er den Lehrerberuf hat, gemäß der Verpflichtung auf dasselbe lehrt. Dieses Bekenntnis gründet sich auf die Heilige Schrift Alten und Neuen Bundes und erkennt sowohl die alt- als neutestamentlichen Bücher des Kanons für eingegeben durch den heiligen Geist, für gleich ehrwürdig und unverbrüchlich, für übereinstimmig und beweiskräftig, für klar, vollkommen und hinreichend, von der in ihnen geoffenbarten, umgebenen Wahrheit die ihnen widerstreitende Lüge zu unterscheiden.“

Von den alten lutherischen Theologen sagt Delitsch: „Jene alten lutherischen Lehrer waren nicht bloß gelehrte, sondern auch geheiligte Theologen, unterwiesen in der Schule des Heiligen Geistes, erfüllt mit himmlischer Weisheit, süßem Troste und lebendiger Erkenntnis Gottes; Gottes Wort war eingepflanzt in ihr Herz, es war mit ihrem Glauben gemengt und in Saft und Kraft bei ihnen verwandelt. Gottes Wort, nicht menschliche Weisheit, auch nicht verstanden durch menschliche Weisheit, sondern erfahren durch göttliche Gnade, war das himmlische Feuer, an dem sie ihre Fackel entzündeten. So schaue doch, mein Volk, in den Spiegel deiner Ahnen, gedenke der vorigen Zeit bis daher und betrachte, was Gott getan hat an den alten Vätern! Frage deinen Vater, der wird dir’s verkündigen; deine Ältesten, die werden dir’s sagen, 5 Mos. 32, 7. So spricht der Herr: ,Treetet auf die Wege und schauet und fraget nach den vorigen Wegen, welches der gute Weg sei, und wandelt drinnen, so werdet ihr Ruhe finden für eure Seele', Jer. 6, 16. ,Ich predige euch Rückschritt, nämlich zum Worte Gottes, von dem ihr gefallen seid. Eure Aufklärung ist für mich eine finstere, sternenlose, schaurige Nacht; ihr habt in eurem Taumel die Begriffe verwechselt, sonst könntet ihr eine ägyptische Finsternis, die das Strafgericht Gottes ist, weil ihr das Licht der Reformation verworfen habt, nicht dem sonnigen Tage vergleichen.' Nach Darlegung der Lehren von der Rechtfertigung und den Gnadenmitteln, die durch die Reformation wieder ans Licht gebracht sind, schließt Delitsch seine Jubiläumsschrift also: