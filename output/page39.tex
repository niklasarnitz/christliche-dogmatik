Wesen und Begriff der Theologie. 28nicht ohne Wort und neben dem Wort, sondern gerade durch das Wort, also mittelbar, den Glauben an Christum zu wirken und zu erhalten.\footnote{95}\noindent \textbf{2. Eine weitere Besetzung des Schriftsworts auf Grund eines rationalistischen Axioms liegt auch der reformierten Leugnung der Realpräsenz des Leibes und Blutes Christi im Abendmahl zugrunde.}Dass die Schriftworte vom Abendmahl prima facie nicht auf die Abwesenheit, sondern auf die Anwesenheit des Leibes und Blutes Christi lauten, wird direkt und indirekt zugegeben. Nur müssten die Abendmahlsworte so gedeutet werden, dass sie sich mit dem „Glauben“ reimen. Auf die Frage nach dem Inhalt des „Glaubens“, nach dem die Abendmahlsworte zu deuten seien, bringen die alten und die neuen reformierten Lehrer nicht Schriftaussagen, sondern ein menschliches Dekret des Inhalts, dass Christi menschlicher Natur, wenn sie nicht zerstört werden solle, immer nur eine sichtbare und lokale Gegenwart (\textit{visibilis et localis praesentia}) zukommen könne. Christo nach seiner menschlichen Natur soll, wie Calvin uns sehr nachdrücklich belehrt, immer nur eine Gegenwart zukommen, die über die natürliche Größe des Leibes Christi (\textit{dimensio corporis, mensura corporis}) nicht hinausreiche, also nur über etwa sechs Fuß sich erstrecke und daher für die gleichzeitige Abendmahlsfeier an vielen Orten in der Welt jedenfalls nicht zureiche. Nicht nur Carlstadt und Zwingli, sondern gerade auch Calvin gründet seine Bekämpfung der Realpräsenz, auf die die Abendmahlsworte lauten, auf den Kanon, dass Christi Leib, wo er sei, notwendig stets räumliche Ausdehnung haben und sichtbar sein müsse.\footnote{96} Auch in den Joh. 20, 19 erwähnten geschlossenen Türen sei notwendig eine Öffnung hinzuzudenken, und Luk. 24, 31 sei so auszulegen, dass nicht Christus selbst nach seiner menschlichen Natur unsichtbar geworden sei, sondern der Emmausjünger Augen zugehalten wurden.\footnote{97} Kurz, der reformierten Leugnung der Realpräsenz des Leibes und Blutes Christi im Abendmahl liegt klar erkennbar die Tatsache zugrunde, dass ein menschliches Axiom gegen die Schriftaussagen geltend\footnote{95) Über die reformierte Selbstauffassung III, 188 ff.}\footnote{96) Inst. IV, 17, 19 besteht Calvin auf einer Gegenwart des Leibes Christi, \textit{quae nec mensuram illi suam habeat vel pluribus simul locis distrahat --- vel in pluribus simul locis ponitur}. Inst. IV, 17, 29: \textit{Haec est propria corporis veritas, ut spatio contineatur, ut suis dimensionibus constet, ut suam faciem habeat.}}\footnote{97) Inst. IV, 17, 29, am Ende.}