\hfill 27 \\ \centerline{Wesen und Begriff der Theologie.}\par an den Gnadenmitteln verstanden – stecken geblieben sei.\footnote{93) Zwinglis Antwort auf Luthers Schrift „Dass diese Worte“ um abgedruckt in der St. Louiser Ausgabe der Luthers Werken, XX, 1131 f.; „Ich [Zwingli] will die [Luther] vor die Augen stellen, dass du den weiten, herrlichen Schein des Evangelii nicht erkannt hast, du habest denn desselben wiederum vergessen.“} Die praktische Folge dieser Trennung der Gnadenoffenbarung und Gnadenwirksamkeit von den Gnadenmitteln ist der Rückfall in die römische „eingegossene Gnade“ (\emph{gratia infusa}) und somit ein Abfall von der christlichen Lehre von der Rechtfertigung. Denn alle Menschen, die sich von den äußeren Gnadenmitteln hinwegweisen lassen, gründen ihre Zuversicht zu Gott nicht auf die Vergebung der Sünden oder die gnädige Gesinnung Gottes (\emph{favor Dei propter Christum}), die durch Christi \emph{satisfactio vicaria} vorhanden ist, in der evangelischen Verheißung der Gnadenmittel dargeboten wird und auf Grund dieser objektiven Darbietung und Zusage geglaubt werden soll, sondern auf eine angeblich unmittelbar bewirkte innere Umwandlung, Erleuchtung und Erneuerung, also auf eine Gnade, die als eine dem Menschen inhärierende gute Qualität gefasst wird. Da aber der heilige Geist erklärtermaßen mit einer solchen unmittelbaren Gnadenoffenbarung und Gnadenwirkung sich nicht abgibt,\footnote{94) Vgl. III, 170 ff., 175 ff.} so sehen alle, die wirklich nach Zwinglis und Calvins Anweisung um die unmittelbare Erleuchtung und Erneuerung sich bemühen, Notgedrungen an die Stelle der wirklichen Geisteswirkung eigenes menschliches Produkt. Luthers öfters ausgesprochenes Urteil, dass „Papst und Schwärmer ein Ding“ sind, entstammt nicht der „übertreibenden Polemik des 16. Jahrhunderts“, sondern ist ein vollkommen sachliches Urteil. Dass es trotz der offiziellen Ablehnung der Gnadenmittel dennoch viele Kinder Gottes in den reformierten Gemeinschaften gibt, kommt von einer Konsequenz her, auf die Luther oft und sonderlich auch in den Schmalkaldischen Artikeln hingewiesen hat. Würden nämlich die offiziellen Leugner der Gnadenmittel ihre Theorie in die Praxis umsetzen, so müssten sie auch ihrerseits vom Evangelium in äußerlich geredetem und in äußerlich geschriebenem Wort schweigen, um nicht des Heiligen Geistes angeblich unmittelbare Wirksamkeit zu stören. Aber statt zu schweigen, sind sie in Wort und Schrift sehr geschäftig, und sofern sie dann das Evangelium von dem für die Sünden der Welt gekreuzigten Christus lehren, geben sie dem Heiligen Geist Gelegenheit,