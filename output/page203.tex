192\n\textsc{Wesen und Begriff der Theologie.}\n\nkönne, dass ihr die göttliche Torheit, das alte unveränderte Evangelium, gepredigt werde, von welchem Paulus und die Geschichte der Kirche aller Zeiten und jedes einzelnen Christentum bezeugt, dass es eine Kraft Gottes sei, die da selig macht alle, die daran glauben, die Juden vornehmlich und auch die Griechen. Ein Mensch, der dadurch für das Christentum gewonnen ist, dass ihm gezeigt wurde, wie das Christentum die schärfste Probe der Wissenschaft aushalte, ist noch nicht gewonnen, sein Glaube noch kein Glaube.“\n\n„Zur Widerlegung der Anklage, dass die Rückkehr zur Theologie Luthers und der Dogmatiker „eine sklavische Unterwerfung unter die Lehrentscheidungen der Dogmatiker oder Luthers oder der Symbole in sich schließe, erklärt Walther die folgende allgemeine Einladung: „Kommt und seht! Geht in unserer Gemeinschaft von Pfarre zu Pfarre und von Kirche zu Kirche und seht, ob da ein sogenannter toter Orthodoxismus und nicht vielmehr eine lebendige, unter inneren Kämpfen gereifte lebendige Erfahrungserkenntnis herrschend ist! Besucht unsere \textsc{Pastoralconferenzen}, welche regelmäßig zwischen unsern alljährlichen Synodalsammlungen gehalten werden, und seht, ob da jener geschäftige, der das Amtieren für ein Handwerk zum Broterwerb ansieht, und ob nicht vielmehr ein reges theologisches Leben und die Sorge sich kundgibt, zu wissen, wie ein Diener Christi wandeln solle in dem Hause Gottes, welches ist die Gemeinde des lebendigen Gottes. Nehmt an unsern \textsc{Synodalversammlungen} teil und seht, ob da ein iurare in verba magistri und nicht vielmehr jener Sinn Luthers sich zeigt: „Es sei denn, dass ich mit Zeugnissen der \textsc{Heiligen Schrift} oder mit öffentlichen, klaren und hellen Gründen und Ursachen überwunden und überwiesen werde, so kann und will ich nichts widerrufen.“ – Über die „Staatentheologie“, die Walther sonderlich wegen seiner Schrift „Die Stimme unserer Kirche in der Frage von Kirche und Amt“ \footnote{610} zugeschrieben worden ist, spricht er selbst sich so aus: „Als wir Lutheraner von Amerika wieder das alte gute Banner unserer Kirche entfalteten und uns um dasselbe wieder in geschlossenen Reihen scharten, während uns um ihr Zwinglianismus, Schwärmerei und Nationalismus unter lutherischer Flagge segelten, da hieß es alsbald: Wieder eine neue Sekte! Die einen riefen: Ihr seid auf dem Wege nach Rom! Die andern: Ihr seid Unionisten! noch andere: Ihr seid Independenten! Wieder andere: Ihr seid Pietisten, Schwärmer,\n\n\footnote{610} Erste Auflage 1852, 31874.