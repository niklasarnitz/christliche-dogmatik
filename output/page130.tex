Abschied nimmt, da tut er es mit den Worten: „Und nun, liebe Brüder, ich befehle euch Gott und dem Wort seiner Gnade, der da mächtig ist, euch zu erbauen und zu geben das Erbe unter allen, die geheiliget werden.“\footnote{429} Das haben wir alle, Pastoren und Professoren, immer wieder von Neuem im Glauben zu lernen. Was insonderheit das Kirchenregiment betrifft, so hat Schleiermacher auch die „Leitung“ der Kirche als letzten Zweck der Theologie genannt.\footnote{430} Das können wir uns gefallen lassen, sofern das rechte Leitungsmittel, Gottes Wort, gemeint ist. Die christliche Kirche wird lediglich mit Gottes Wort geleitet oder regiert. Die äußeren Ordnungen, welche kleinere oder größere kirchliche Körperschaften für gemeinschaftliche Zwecke treffen, sollten nicht „Regiermittel der christlichen Kirche“ genannt werden. Luther: „Die Christen kann man mitnichten, ohne allein mit Gottes Wort, regieren. Denn Christen müssen im Glauben regiert werden, nicht mit äußerlichen Werken. Glaube aber kann durch kein Menschenwort, sondern nur durch Gottes Wort kommen, wie St. Paulus sagt Röm. 10, 17: Der Glaube kommt durchs Hören, das Hören aber kommt durchs Wort Gottes.“\footnote{431}

\subsection*{15. Theologie und Wissenschaft.}

Wir müssen uns auf die Frage, ob die Theologie eine Wissenschaft sei, nicht eher einlassen, als bis eine Verständigung über den Begriff Wissenschaft stattgefunden hat, weil das Wort Wissenschaft in verschiedenem Sinne und noch öfter ohne allen Sinn gebraucht wird. Versteht man unter Wissenschaft \emph{ein geordnetes natürliches Wissen}, das heißt, ein Wissen, das der Mensch ohne die Offenbarung der heiligen Schrift auf dem Wege der Beobachtung der Natur und seiner selbst gewonnen hat, so ist die christliche Theologie nicht eine Wissenschaft. Der Grund hierfür liegt in der Tatsache, dass von dem spezifischen Inhalt der christlichen Lehre gebildet, nämlich von dem \emph{Evangelium von Christo}, weder das große Gebiet der Natur noch das Gewissen des Menschen Kunde gibt, wie die Schrift klar lehrt.\footnote{432} Aus der Offenbarung Gottes, die in der Natur und im menschlichen Gewissen vorliegt, lässt sich nicht die christliche, sondern nur die sogenannte natürliche Religion oder die Gesetzesreligion erkennen.\footnote{433} Ferner: Versteht man unter

\par\vspace{1em}
\begin{small}
429) Apost. 20, 32.\\
430) Der christliche Glaube, nach den Grundsätzen der evangelischen Kirche im Zusammenhange dargestellt, I. 16.\\
431) St. L. X, 406.\\
432) 1 Kor. 2, 6--16.\\
433) Röm. 1, 18 ff.; 2, 14. 15.
\end{small}