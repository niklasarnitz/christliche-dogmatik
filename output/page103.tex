barung erkannt hatten.\footnote{Matth. 16, 13--17: --- 1 Joh. 1, 1--4.} Wo man Christum allenfalls honoris causa Gott nennen, aber ihn nicht den ewigen wahren Gott sein lassen will, da kann kein seligmachender Glaube vorhanden sein. Die Unitarier und die auf unitarischen Wegen wandelnden modernen Theologen haben nach der Schrift ihren Standort extra ecclesiam.\footnote{1 Joh. 5, 12, 13; Apost. S. 77.} Was die Trinität betrifft oder die Erkenntnis, dass der unus Deus Vater, Sohn, Heiliger Geist ist, so ist nach der Schrift der Glaube an die drei Personen in der Weise ineinandergeschlossen, dass es keine Erkenntnis des Sohnes ohne den Vater gibt\footnote{Matth. 16, 17; 11, 27 a.} und keine Erkenntnis des Vaters ohne den Sohn,\footnote{Röm. 8, 15; 1 Kor. 12, 3; Joh. 16, 13--15.} und niemand den Vater kennen kann und den Sohn einen Herrn heißen kann ohne durch den heiligen Geist.\footnote{Dieser Punkt ist ausführlicher bei der Lehre von der Taufe behandelt, III, 297 ff.} Es ist dagegen eingewendet worden, dass wir den ersten Christen nicht wohl eine „Reflexion“ über Vater, Sohn und heiligen Geist zutrauen können. Man ist so weit gegangen, auch aus dieser angenommenen Ignoranz der ersten Christen „die Ungeschicklichkeit des matthäischen Taufbefehls“ (Tauset sie auf den Namen des Vaters und des Sohnes und des heiligen Geistes) beweisen zu wollen. Allein damit schiebt die moderne Theologie ihr eigenes Defizit in der christlichen Erkenntnis ganz ungeschickt in den ersten Christen unter.\footnote{1 Tim. 2, 5. 6; Joh. 1, 29.} Über die Offenbarung der Trinität auch im Alten Testament folgt ein besonderer Abschnitt bei der Lehre von Gott (De Deo).\par 3. Die fides salvifica schließt auch die Erkenntnis des Werkes Christi in sich. Nach der Schrift ist Christus Objekt der fides salvifica, nicht insofern er Lehrer des göttlichen Gesetzes, auch nicht insofern er vollkommenes Tugendvorbild ist, sondern insofern er der Mittler zwischen Gott und den Menschen ist, der sich selbst gegeben hat für alle zur Erlösung (ἀντιλυτρον), oder insofern er Gottes Lamm ist, das der Welt Sünde trägt.\footnote{Matth. 20, 28. (Das griechische Wort ist ἔργον.) 1 Kor. 6, 20. 7, 23. 1 Tim. 2, 6. Hebr. 9, 12.} Wer Christi satisfactio vicaria nicht glaubt, glaubt nicht an Christum im Sinne der Schrift, sondern gründet --- tertium non datur --- seine Versöhnung mit Gott irgendwie auf eigenes Tun oder eigene Würdigkeit und schreibt sich eo ipso von der von Christo erworbenen Gnade, wie die Schrift