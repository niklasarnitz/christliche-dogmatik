Wesen und Begriff der Theologie. schaft solcher kirchlichen Kreise, in denen die Theologie der Selbstgewissheit doctrinerell auf das Entscheidendste verworfen wird. Natürlich kann die Tatsache, dass die Theologie des Selbstbewusstseins nur eine Ausgestaltung der bösen Art ist, die sich in allen gefallenen Adamskindern findet, uns nicht von der Pflicht entbinden, sie mit großem Ernst zu bekämpfen. Diese Theologie ist in jeder Beziehung das Nicht, was sie zu sein vorgibt. Um kurz zu wiederholen: Zur Sicherstellung des wissenschaftlichen Charakters der Theologie ersonnen, versetzt sie seine Vertreter in die Rolle des Mannes, der zur Sicherung des wankenden Ich sich an das eigene Ich anklammert. Wir sahen ferner, dass die Ichtheologie die böteste Form der Abgötterei darstellt, nämlich die Selbstvergötterung. Der Ichtheologe spricht die Autorität, die er der Heiligen Schrift zuspricht, sich selbst zu. Und wie alle Abgötterei zuschanden macht,\footnote{497) Ps. 115, 4--8.} so speziell auch die Selbstvergötterung, in der das Wesen der Selbstgewissheitstheologie besteht. Ihr Resultat ist nicht Gewissheit, sondern Einbildung, Selbsttäuschung, Ungewissheit, die ausführlich gezeigt wurde. Wir sahen auch, dass die Ichtheologie sehr ansteckend wirkt. Schleiermacher wurde und wird als der Reformator des 19. Jahrhunderts bewundert. Und das hat einen doppelten Grund. Der durch den Sündenfall dezentralisierte Mensch macht sich selbst zum Zentrum der Dinge. Daher ist er leicht für eine Theologie zu gewinnen, die seinem Ich eine Herrscherstellung zuweist. Sodann trat und tritt die Ichtheologie, obwohl ihr Wesen Gottlosigkeit ist, mit dem Schein der Frömmigkeit auf. Die Schwärmer des 16. Jahrhunderts hätten die Kirche nicht zerrissen, wenn sie offen herausgesagt hätten, dass ihre Einfälle Produkte ihres menschlichen Ich seien. Stattdessen schrieben sie ihre Einfälle dem Heiligen Geist aufs Konto. Sie vergassen auch, um ihre Frömmigkeit zu bekräftigen, „Wulden von Tränen“, wie Luther gelegentlich bemerkt. So redeten und reden auch Schleiermacher und alle, die seine theologische Methode befolgen, vom frommen Selbstbewusstsein des theologisierenden Subjekts, vom wiedergeborenen Ich, vom christlichen Erlebnis usw. Und obwohl sie die Schrift als Gottes Wort und die \textit{satisfactio vicaria} verwerfen, so reden sie doch von großem Fortschritt in der Exegese und von einer tieferen Erforschung des Christtums. Dies alles täuscht und ist geeignet, auch „unschuldige Herzen“\footnote{498) Röm. 16, 18.} zu verführen. Auf dem Gebiet der Natur kann die fichtische Methode