\begin{center}
\begin{tabular}{lr}
\emph{Wesen und Begriff der Theologie.} & 80
\end{tabular}
\end{center}

Schriftwort nicht Glaube, sondern Einbildung sei. Aber wenn der Theologe, der ein Christ ist, sich auf die Situation besinnt, so erkennt er nach dem neuen Menschen, dass in der Berufung auf den \glqq Glauben\grqq oder das \glqq Glaubensbewusstsein\grqq eine Selbsttäuschung vorliegt, wenn dabei das Schriftwort als einzige Quelle und Norm der Theologie beigefügt wird.

Eine große Selbttäuschung liegt ferner vor in der Behauptung, dass es bei der Bestimmung, was christliche Lehre ist, nicht sowohl auf die Worte (gewöhnlich sagt man: auf den \glqq Buchstaben\grqq) der Schrift ankomme als auf ihren \glqq Inhalt\grqq. Diese Behauptung gehört in die große Zahl der Redeweisen, die von einer Generation auf die andere gedankenlos vererbt werden. Aber es wird uns damit eine logische und psychologische Unmöglichkeit zugemutet. Es steht doch so: Weil der Inhalt der Heiligen Schrift, wie der Inhalt jeder andern Schrift, nur durch die Worte der Schrift zum Ausdruck kommt, so ist auch ihr Inhalt nur deshalb gewiss, weil ihre Worte gewiss und zuverlässig sind. Können wir uns auf die Worte der Schrift nicht verlassen, so bleibt auch der Inhalt der Schrift, die Lehre der Schrift, auf dem Gebiet der Vernunft. Um uns dem Zustand der Ungewissheit zu entnehmen, der für heilsbegierige Seelen bitterer als der Tod ist,\footnote{266) Apologie, S. 191: \glqq Denn gute Gewissen schreien nach der Wahrheit und rechtem Unterricht aus Gottes Wort, und denselben ist der Tod nicht so bitter, als bitter ihnen ist, wo sie etwa in einem Stück zweifeln. Darum müssen sie suchen, wo sie Unterricht finden.\grqq} so verliert uns unser Heiland, dass die Schrift nicht gebrochen werden könne und verweist uns ausdrücklich auf seine Worte oder, was dasselbe ist, auf die Worte seiner Apostel. Die Belehrung und Ermahnung Joh.	hinspace8 lautet nicht: \glqq So ihr bleiben werdet an dem Inhalt meiner Rede\grqq, sondern: \glqq So ihr bleiben werdet an meiner Rede\grqq ($\lambda\acute{o}\gamma o\varsigma$), so 
\ldots werdet ihr die Wahrheit erkennen. Und ebenso lautet Joh.	hinspace17, 20 die Belehrung Christi nicht dahin, dass alle durch den Inhalt des apostolischen Wortes, sondern durch das apostolische Wort selbst ($\delta\iota\dot{\alpha} \lambda\acute{o}\gamma o\upsilon$ $\alpha\acute{\upsilon}\tau\tilde{\omega}\nu$) an ihn glauben werden. Dieser Gegenstand wird bei der Lehre von der heiligen Schrift weiter behandelt, namentlich auch bei dem Kapitel von den \glqq variae lectiones\grqq, die gegen die Zuverlässigkeit des Schriftwortes geltend gemacht werden. Es sei hier nur noch daran erinnert, dass gegen die versuchte Trennung des Schriftinhalts vom Schriftwort auch die christliche Erfahrung oder das christliche \glqq Erlebnis\grqq einen sehr entschiedenen Protest einlegt. Reden wie diese: