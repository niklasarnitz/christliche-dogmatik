\noindent\small Wesen und Begriff der Theologie.\hfill 67\par\medskipgesagt ist, zunächst gänzlich außer Augen zu tun und statt dessen das theologische Ich zu veranlassen, „in geschlossener Selbständigkeit“ über die christliche Lehre auszusagen. Hofmann meint\textsuperscript{236)} „jenes Verhältnis zu Gott [in Christo], nachdem ich seiner teilhaftig geworden, hat ein selbständiges Dasein in mir begonnen, welches, obwohl nur innerhalb der schriftgläubigen Kirche möglich, doch nicht von der Kirche abhängt noch von der Schrift, aus die sich die Kirche beruft, auch nicht an jener oder dieser die eigentliche und nächste Verbürgung seiner Wahrheit hat, sondern in sich selbst ruht und unmittelbar gewisse Wahrheit ist, von dem ihm selbst einwohnenden Geiste Gottes getragen und verbürgt. Dennoch will und muss dasselbe, wo man es sich zur Erkenntnis und Aussage [Lehrdarstellung] bringen lassen will, rein nur es selber bleiben, unversengt mit dem, ungestört durch das, was außer ihm, also außer uns, wo irgend gelegen ist. Und ob das außer uns Gelegene in noch so naher, in ursächlicher Beziehung steht zu dem in uns, und ob es sich als die gleiche Wahrheit unzweifelhaft zu erkennen gibt: hier gilt es, die eine nächste Aufgabe rein für sich, in geschlossener Selbständigkeit, zu vollziehen. Freilich werden, wo es recht hergeht, Schrift und Kirche ganz das Gleiche bieten, was wir in uns selbst erheben. Aber es dort aufzusuchen, ist eine zweite Aufgabe nach jener.“ Aus den letzten Worten Hofmanns geht hervor, dass Hofmann nachträglich Revision des Lehrprodukts nach der Schrift, als oberster Norm, verspricht. Dieses Versprechen hat Hofmann den Vorwurf der Inkonsequenz, das ist, den Vorwurf des Abfalls vom Jahweismus und des Rückfalls in „Biblizismus“ und „Intellektualismus“, eingetragen. Ganz neuerdings tadelt wieder Horst Stephan an Hofmann die Hinzufügung eines „Schriftbeweises“ mit der Begründung: „Damit nahm man der dogmatischen Methode die Einheitlichkeit; man kam im Grunde auf die biblizistische und konfessionalistische Dogmatik zurück.“\textsuperscript{236)} Und dieser Vorwurf der Inkonsequenz ist, vom erlebnis-theologischen Standpunkt Hofmanns aus angesehen, berechtigt. Hofmann leugnet ja sehr entschieden, dass die Heilige Schrift durch Inspiration Gottes unfehlbares Wort ist. Ist aber die Schrift nicht Gottes unfehlbares Wort, sondern muss in ihr der Theolog zwischen Wahrheit und Irrtum unterscheiden, so ist ihm die Schrift nicht mehr eine Norm,\par\vfill\noindent{\footnotesize\textbf{235)} Schriftbeweis I, 11.\\textbf{236)} Glaubenslehre, Gießen 1921, S. 21.}