Wesen und Begriff der Theologie. 125\n\nnachdrücklich ein, dass das Bleiben am Wort Christi die einzige Weise ist, durch die die christliche Wahrheitserkenntnis sich vermittelt. Der Apostel Paulus bezeugt jedem Lehrer, der nicht bei den gesunden Worten unseres Herrn Jesu Christi bleibt, dass er an Einbildungen leide, nichts wisse, in die Seuchen der Fragen und Wortstreitigkeiten verfallen sei (\textalpha{νοσῶν περὶ ζητήσεις καὶ λογομαχίας}), also an der christlichen Wahrheitsgewissheit vorbeigeht. Wir finden demnach in der Schrift wirklich die Frage von der „christlichen Gewissheit“ und speziell die Frage von der „christlichen Wahrheits-Gewissheit“ allseitig beantwortet.\n\nDie „erkenntnis-theoretische“ Belehrung der Schrift anzunehmen und zu befolgen, ist sowohl für alle Christen als auch insonderheit für die Theologen von großer praktischer Wichtigkeit. Wem die christliche Gewissheit am Herzen liegt, der flüchtet, sooft ihm die Gewissheit entschwinden will, in das Wort Christi, in die Heilige Schrift, hört, liest und bewegt das Wort in seinem Herzen, glaubt dem Wort durch Wirkung des Heiligen Geistes im Wort und unterwirft seinen Sinn dem gehörten und gelesenen Wort in der demütigen Gesinnung: „Rede, Herr, denn dein Knecht höret!“ So flüchtete Luther, wenn ihm die Gewissheit, sei es die „Heils-Gewissheit“, sei es die „Wahrheitsgewissheit“, entschwinden wollte, in die Schrift. Er sagt: \footnote{Predigt über Joh. 17, 1. St. L. VIII, 749 f.} „Ich weiß nicht, wie stark andere im Geist sind; aber so heilig kann ich nicht werden, wenn ich noch so gelehrt und voll Geistes wäre, als etliche sich dünken lassen. Noch widerfährt mir es allezeit, wenn ich ohne das Wort bin, nicht daran dachte noch damit umgehe, so ist kein Christus daheim, ja, auch keine Luft und Geist; aber sobald ich einen Psalmen oder Spruch der Schrift vor mich nehme, so leuchtet es und brennt es ins Herz, dass ich andern Mut und Sinne gewinne. Ich weiß auch, es soll’s ein jeglicher täglich also bei sich selbst erfahren.“ Luther gibt dabei jedem Christen und jedem Theologen weiterhin den Rat, dass man „sich mit den Gedanken an die Buchstaben [der Schrift] hefte, wie man sich mit der Faust an einen Baum oder Wand halten muss, auf dass wir nicht gleiten oder zu weit flattern und irrefahren mit eigenen Gedanken. Das mangelt unsern Schwärmern, dass sie meinen, wenn sie in ihre hohen geistlichen Gedanken fahren, so haben sie es troffen, und sehen nicht, wie sie ohne Wort des Holzweges fahren, lassen sich eitel Teufelsche verführen.“ In diesen Worten Luthers kommt schon zur Aussage, auf welche Weise wir der christlichen Gewissheit