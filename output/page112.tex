\centerline{101 \qquad Wesen und Begriff der Theologie.}

Christo erworbene Vergebung der Sünden wohnt. Aber die Sachlage ist auch für die Schwachen stets mit Gefahr verbunden, namentlich wenn Kontroversen sich erheben. Luther erinnert mit Recht daran, dass nicht nur die weltlichen, sondern auch die geistlichen Kriege gefährlich sind. Auch in den geistlichen Kriegen geht es nicht ohne Verwundete und Tote ab. Die Klimax der Gefahr tritt dann ein, wenn in Lehrkontroversen das klare Schriftwort auf den Plan gebracht wird und diesem Schriftwort gegenüber, in dem ja der heilige Geist wirksam ist, der Irrtum festgehalten wird. Dann kann der Fall eintreten, dass das „christliche Irren“, das ist, das Irren aus Schwachheit, wobei der Glaube noch besteht, aufhört und das „unchristliche Irren“ oder das Irrenwollen einsetzt, das den Glauben unmöglich macht.\textsuperscript{369}
Dies ist dann der Fall, der Tit. 3, 10. 11 so beschrieben wird: „Einen ketzerischen Menschen meide, wenn er einmal und abermal ermahnet ist, und wisse, dass ein solcher verkehrt ist und sündigt, als der sich selbst verurteilet hat“, \textgreek{ἁμαρτάνων ἀφ' ἑαυτοῦ}.\textsuperscript{370}

4. Wir sollten auch nicht vergessen, dass, wie jede Sünde auf dem Gebiet der Moral,\textsuperscript{371} so auch jeder Irrtum auf dem Gebiet der Lehre die Tendenz hat, sich durchzusetzen, das ist, andere Lehren in Mitleidenschaft zu ziehen und schließlich die ganze Lehre zu verderben. Das meint der Apostel, wenn er Gal. 5, 9 sagt, dass ein wenig Sauerteig den ganzen Teig versäure.\textsuperscript{372} Auf diese Tendenz jedes Irrtums sieht Luther, wenn er die christliche Lehre mit einem Ring vergleicht, der nicht mehr ganz ist, wenn er nur einen Bruch hat, und weiterhin von den Artikeln der christlichen Lehre sagt, „dass ein Artikel alle und alle Artikel einer sind, und dass, wenn man einen verloren hat, allmählich alle verloren werden“.\textsuperscript{373} Die Kirchengeschichte aller Zeiten liefert hierfür die Belege. An die Stelle

\begingroup
\small
\noindent \textsuperscript{369} Luthers klassisches Diktum in Bezug auf den „christlichen Irrtum:“ „Du kannst nicht sprechen: Ich will christlich irren. Ein christlicher Irrtum geschieht aus Unwissenheit.“ (St. V. XIX, 1132.)

\noindent \textsuperscript{370} \textgreek{Ἁμαρτάνων ἀφ' ἑαυτοῦ}, das nur hier vorkommt, kann gar nicht missverstanden werden. Es bezeichnet die \emph{innerliche Selbstverurteilung}, suopte iudicio condemnatus. Gottes Wort, das ihm vorgehalten wurde, hat ihn verurteilt und diese Verurteilung hat er selbst in seinem Gewissen empfangen. Luther z. B.: „der sündigt mit Bewusstsein seiner Schuld und Verurteilung.“

\noindent \textsuperscript{371} 1 Kor. 5, 6. Daher 2 Kor. 7, 1 die Mahnung zur Reinigung \textgreek{ἀπὸ παντὸς μολυσμοῦ σαρκὸς καὶ πνεύματος}.

\noindent \textsuperscript{372} Auch Meyer bezieht, wie Luther und unsere alten Theologen, Gal. 5, 9 auf das Gebiet der Lehre.

\noindent \textsuperscript{373} Zu Gal. 5, 9. St. V. IX, 642 ff.
\endgroup