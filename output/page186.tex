hauer, (579) Friedrich König, (580) Calov, (581) Duenstedt, (582) Baier, (583) Hoſſlaz. (584) Doch finden wir bei den Liebhabern der analytischen\begin{enumerate}\item[(579)] \emph{Hodosphia Christiana}, 1649. Dannhauer teilt in Bildern, die aus der Schrift genommen sind, die christliche Lehre in 12 Phänomena dar: Der Mensch ein Wanderer, die Seele das Ziel, die Kirche der Treuhüter, Gott das Ziel usw. Die beste Ausgabe ist die von 1713.\item[(580)] \emph{Theologia Positiva, Acroamatica}, 1664. Kurzer Leitfaden für theologische Vorlesungen, der auch Duenstedts Systema zugrunde liegt.\item[(581)] \emph{Systema Locorum Theologicorum}, 1655--1677, in 12 Bänden. Sorgfältig sind die ersten, weniger sorgfältig die letzten Bände gearbeitet. Calov ist der scharfsinnigste Theologe des 17. Jahrhunderts. Auch ist er Schriftgelehrte, was Buddeus an ihm rühmt (Isagoge Hist. Theol., 1730, S. 357). Noch heute ist ein klassisch-exegetisches Werk Calovs \emph{Biblia Illustrata}. Calov wird von Tholuc gehässig und ungerecht beurteilt und geradezu geschmäht (W.G I 11, 506). Gerechter Beurteilung Calovs in W.G II, 759f. von A. Kunze.\item[(582)] \emph{Theologia Didactico-polemica sive Systema Theologiae}, 1685, 1696, 1702, 1715. Von Tholuc der Buchhalter und Schriftsührer* der orthodoxen Theologie genannt (W.G I XII, 421). Das Urteil trifft sachlich nicht zu, weil Duenstedt mit eigenem Urteil gearbeitet hat. Um über Duenstedt urteilen zu können, muss man ihn gelesen und mit andern Dogmatikern verglichen haben. Auch Walchs Urteil (Bibliotheca Theol. I, 58), von Duenstedt seien \emph{meritissimi ecclesiae nostrae doctores} ob leviorem dissensum unter die Gegner gerechnet worden, ist ungerecht. Der Synkretismus der Helmstedter und des Musäus ist nicht lexior dissensus. Walch ist Indifferentist und der Aufnahme zu viel gefror’en. Selbst Tholuc, obwohl er ähnliche Bemerkungen nicht ganz unterdrücken kann, beschreibt Duenstedt als einen anspruchslosen, frommen Charakter ohne „bittere Leidenschaftlichkeit“ (W.G XII, 422). Dass Duenstedt auf den Schriftbeweis viel Sorgfalt verwandt habe, erkennt auch Walch an (a. a. O.).\item[(583)] \emph{Compendium Theologiae Positivae}, 1686 u. öfter. Baier ist von Musäus', seines Schwiegervaters, synergistischem Standpunkt infiziert. Die St. Louiser Ausgabe, 1879, ist nicht ein bloßer Abdruck von Baiers Kompendium, sondern durch eingereihte Zitate so erweitert, dass sie drei Bände (im ganzen 1332 Seiten groß) umfasst enthält. Für die eingereihten Zitate kommen auch zur Luther und die alten Theologen, sondern auch die neueren Theologen des 19. Jahrhunderts reichlich zu Worte, so dass den Studierenden ein Quellmaterial geboten ist, wodurch ihnen sowohl über die alte als über die moderne Theologie ein Urteil ermöglicht wird. Prof. Th. Bünger hat zu dieser Waltherschen Ausgabe ein Sach- und Namenregister geliefert, das mit vorzüglichem Fleiß und großer Sachkenntnis gearbeitet ist.\item[(584)] \emph{Examen Theologicum Acroamaticum}, 1707. „Der letzte genuin lutherische Dogmatiker.“ Hoſſlaz hat nicht bloß abgeschrieben, sondern hierin in eigener Darstellung, was von früher auf ihn auf sein Feld innerhalb der lutherischen Kirche von ihren Vertretern gelehrt worden ist. Nach der analytischen Methode bringt er die Lehre von der ewigen Seligkeit schon im ersten Teil und schließt sie vierten Teil mit der Lehre von der Kirche, vom Predigtamt, von der weltlichen Obrigkeit und vom Hausstand. Mit Recht wird auch Hoſſlaz nachgerühmt, dass er einen Sorgfalt-