Gegenwärtigen Chaos in diesen Kirchen. Doch auch unter freikirchlichen Verhältnissen, wie bei uns in den Vereinigten Staaten, ist der unchristlichen Forderung der „Lehrfreiheit“ an kirchlichen Lehranstalten in weitem Umfange nachgegeben worden, selbst in lutherisch sich nennenden Kreisen. Auch an unsern Staatsuniversitäten, die naturgemäß keine theologische Fakultät haben, fordern Professoren dennoch Lehrfreiheit für ihre „\textit{religiösen Ansichten}“.\footnote{Vgl. James D. Bales, American University Progress, 1916, unter dem Abschnitt „Academic Freedom“, S. 31 f.: „Problems may arise in discussion of \ldots religious and political beliefs; the university is responsible for pioneer thought and must adhere to facts, and like Huxley, let them lead where they will.“ Die Verwaltungsbeamten haben ihre Pflicht erfüllt, „when they appoint able and fearless men to the faculties and attend to the business details of university management“.} \par Bekanntlich wird in neuerer Zeit allgemein behauptet, daß zwei Dinge sich nicht miteinander vertragen: das Gebundensein an Christi Wort und die dem Theologen notwendig zukommende „\textit{innere Freiheit}“. Wenn dem Theologen zur Pflicht gemacht werde, sich an das Schriftwort schlechterdings gebunden zu erachten, so werde ihm das Schriftwort zu einem vom Himmel gefallenen Gesetzescoder, zu einem papiernen Papst u.f.m., und es liege in dieser Beziehung ein Rückzug auf den Katholizismus vor. Um dem „\textit{evangelischen}“ Geist des Protestantismus den unbedingt nötigen Spielraum zu gewähren, bleibe daher nur übrig, die Heilige Schrift als Quelle und Norm der Theologie fahren zu lassen und sich auf die „\textit{Lebenswärme}“ und „\textit{lebendige}“ Ich des theologisierenden Subjekts zurückzuziehen. So argumentiert in wesentlicher Übereinstimmung die gesamte moderne Theologie vom äußersten linken bis zum äußersten rechten Flügel. Es ist gesagt worden, daß die rechte Theologie der Gegenwart von „der Saulsrüstung der alten Theologie“, namentlich von der Verbalinspiration der Schrift, loskommen müsse. Dann könne sie wie David „\textit{mit Gott über die Mauern springen}“.\footnote{Vgl. Theodor Kaftan, Moderne Theologie des alten Glaubens?, S. 121.} \par Dagegen ist ernstlich daran zu erinnern, daß – nach Christi maßgeblicher Ansicht – jene zwei Dinge sich nicht nur sehr gut miteinander vertragen, sondern daß auch das eine Ding, und zwar das Bleiben am Schriftwort, das einzige Mittel ist, wodurch es zur Erkenntnis der Wahrheit und zur theologischen Freiheit kommt. Das ist es, was uns Christus dieserits verbiis Joh. 8, 31. 32 einschärft: „\textit{Ἐὰν ὑμεῖς μείνητε ἐν τῷ λόγῳ τῷ ἐμῷ \ldots} gewöhnliche v.\ d.\ Wahrheit, \textit{καὶ ἡ ἀλήθεια ἐλευθερώσει ὑμᾶς}.“ Zum andern weist Christus in demselben Zusammen-