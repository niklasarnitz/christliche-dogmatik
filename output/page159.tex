Eine Fortbildung der christlichen Lehre kann es deshalb nicht geben, weil die christliche Lehre mit der Lehre der Apostel völlig abgeschlossen Größe ist, die im Laufe der Zeit nicht fortzubilden, sondern völlig unverändert festzuhalten und zu lehren ist. Christi Lehrauftrag, den wir Matth. 28, 18–20 haben, deckt die ganze Zeit des Neuen Testaments bis an den jüngsten Tag und lautet dahin, dass seine Kirche die Völker halten lehre alles, was er (Christus) ihr befohlen hat. Dass wir aber Christi Lehre in der Lehre seiner Apostel haben, bezeugt uns ebenfalls Christus selbst, wenn er Joh. 17, 20 sagt, dass alle Glieder seiner Kirche bis an den jüngsten Tag durch das Apostel Wort an ihn glauben werden. Dessen waren sich auch die Apostel sehr klar bewusst. Paulus ermahnt die Gemeinden, die durch seine Lehrtätigkeit entstanden waren: „So stehet nun (στήκετε) liebe Brüder, und haltet an den Satzungen (παραδόσεις τᾶς παραδόσεις), die ihr gelehret seid, es sei durch unser Wort oder Epistel.“\footnote{505) 2 Thess. 2, 15.} Und dass Paulus dabei nicht bloß an temporäre Geltung seiner Lehre dachte, geht aus den Stellen hervor, in denen er ausdrücklich auf die zukünftigen Zeiten Rücksicht nimmt: „Dass weiß ich, dass nach meinem Abschied werden unter euch kommen greuliche Wölfe, die der Herde nicht verschonen werden. Auch aus euch selbst werden aufstehen“ (ἀναστήσονται, futurum) „Männer, die da verkehrte Lehren reden, die Jünger an sich zu ziehen.“\footnote{506) Apost. 20, 29–31.} „In den letzten Zeiten werden etliche vom Glauben abtreten“ (ἀποστήσονται, futurum) „und anhangen den verführerischen Geistern und Lehren der Teufel.“\footnote{507) 1 Tim. 4, 1 ff. Ebenso 2 Tim. 3, 1 ff.} Alles, was Paulus lehrt, lehrt er angesichts des Jüngsten Tages und damit als bis an den Jüngsten Tag geltend.\footnote{508) 2 Tim. 4, 1 ff.; 1 Tim. 6, 14. 15. Ebenso Petrus, 1 Petr. 5, 1–4.} Und die Lehre Pauli ist so unveränderliche göttliche Wahrheit, dass er über jeden, der sich an seinem Evangelium eine Änderung erlaubt, den Fluch ausspricht.\footnote{509) Gal. 1, 6–9; 5, 12.} Daher auch die Ermahnung des Apostels, von allen Anderslehrenden zu weichen und sie getrost für aufgeblasene Schwätzer und Nichtswisser zu halten.\footnote{510) Röm. 16, 17; 1 Tim. 6, 3. 4.} Besonders zu beachten für unsere Zeit ist die Tatsache, dass Paulus die Vollkommenheit der apostolischen Lehre auch höheren Lehrern gegenüber festhält, die unter dem Schein einer höheren philosophischen Erkenntnis und einer höheren Geistlichkeit die Lehre Christi ergänzen wollten. Er sagt von allen,