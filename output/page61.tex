\phantomsection\addcontentsline{toc}{section}{9. Die nähere Beschreibung der Theologie, als Tüchtigkeit gefaßt.}

\markboth{Wesen und Begriff der Theologie.}{50}

\section*{9. Die nähere Beschreibung der Theologie, als Tüchtigkeit gefaßt.}

Wir haben eine nähere Beschreibung der Theologie, als „Tüchtigkeit“ oder „persönliche Eigenschaft“ gefaßt, in den zahlreichen Schriftauslagen, welche die Personen beschreiben, die nach Gottes Willen und Ordnung mit dem Lehramt in der Kirche zu betrauen sind. Sehr richtig ist gesagt worden: „Da die Theologie, subjektiv betrachtet, dasjenige ist, was sie in denen sein soll, welche in der Kirche das Amt der Lehrer zu verwalten haben, so haben wir in der biblischen Beschreibung eines rechten Lehrers zugleich die eines rechten Theologen zu suchen und zu erkennen.“\footnote{178)}

Die folgenden Hauptmomente sind in der Schrift gegeben:

\begin{enumerate}
    \item Die theologische Tüchtigkeit ist eine geistliche Tüchtigkeit (\textit{habitus spiritualis, supernaturalis}), das heißt, eine Tüchtigkeit, die in jedem Falle außer natürlichen Gaben den persönlichen Glauben an Christum (den Glauben an die Vergebung der Sünden aus Gnaden um Christi \textit{satisfactio vicaria} willen, die Bekehrung oder Wiedergeburt) zur Voraussetzung hat. Ungläubige, wenn sie auch alle biblischen Lehren in ihren Geist aufgenommen haben und vermöge einer natürlichen Begabung lehren können, sind nicht Theologen im Sinne der heiligen Schrift. Dogmengeschichtlich ausgedrückt: Es gibt keine „\textit{theologia irregenitorum}“\footnote{179)}. Ausdrücklich wird die Tüchtigkeit zur Verwaltung des öffentlichen Lehramtes als eine geistliche Wirkung Gottes bezeichnet, 2 Kor. 3, 5: „Nicht daß wir tüchtig sind von uns selber, sondern ist ἱκανότης ἡμῶν ἐκ τοῦ θεοῦ, d.h. καὶ ἱκανώσει ἡμᾶς κτλ. Alle Ungläubigen sind Wohn- und Wirkungsstätten nicht des heiligen Geistes, sondern des Fürsten dieser Welt.\footnote{180)} Ferner sehen wir, daß in der Schrift die Amtsgaben immer nur in Verbindung mit dem persönlichen Christenstand und den Christentumsgaben auftreten. In der Beschreibung eines ἐπίσκοπος, 1 Tim. 3, 1 ff., steht das „lehrhaftig“ (διδακτικός) neben den Prädikaten „nicht ein Weinsäufer“ usw. Auch 2 Tim. 2, 1 ff. führt der Apostel Paulus die Tüchtigkeit, das öffentliche Lehramt zu verwalten, auf die Gnade
\end{enumerate}

\tiny
\begin{longtable}{@{}l}
    \endhead
    \footnotetext[178)]{So Walther nach dem Vorgang der älteren lutherischen Theologen, 2. u. W. 14, 10.}
    \footnotetext[179)]{Vgl. Walch, Bibliotheca Theol. II, 667 sqq. Baumgarten, Theol. Streitigkeiten III, 425 f. Neides Material über die Streitfrage bei Hollaz, Examen Prolog. I, qu. 18–21.}
    \footnotetext[180)]{Eph. 2, 3.}
\end{longtable}
\normalsize