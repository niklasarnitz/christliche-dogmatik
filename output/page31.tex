\section*{Wesen und Begriff der Theologie.}
Christenheit haben die Menschen aus sich selbst. Sie ist menschlichen Ursprungs, man-made religion. Die Schrift weist auf diesen menschlichen Ursprung der vielgestaltigen Gesetzesreligion sehr nachdrücklich hin. Was die Schrift hier sagt, stellen wir hier nochmals in etwas weiterer Ausführung unter den folgenden drei Punkten zusammen: Erstlich haben die Menschen auch nach dem Fall noch eine Kenntnis vom göttlichen Gesetz. Sie wissen um Gottes Gerechtigkeit, \textgreek{τὸ δικαίωμα τοῦ θεοῦ ἐπιγνόντες.}\footnote{57) Röm. 1, 32.} Das vom göttlichen Gesetz geforderte Werk steht, auch wenn sie das in der Schrift geoffenbarte Gesetz nicht haben, in ihren Herzen geschrieben, \textgreek{νόμος μὴ ἔχοντες ἑαυτοῖς εἰσιν νόμος, οἵτινες ἐνδείκνυνται τὸ ἔργον τοῦ νόμου γραπτὸν ἐν ταῖς καρδίαις αὐτῶν.}\footnote{58) Röm. 2, 14. 15. Apologie, S. 27, 7: Humana ratio naturaliter intelligit aliquo modo legem, habet enim idem iudicium scriptum in mente.} Zum andern haben sie ein böses Gewissen wegen ihrer Übertretungen des göttlichen Gesetzes. Sie „wissen, dass die solches tun“ -- nämlich die vorder genannten heidnischen Sünden -- „des Todes schuldig sind“, \textgreek{τὸ δικαίωμα τοῦ θεοῦ ἐπιγνόντες ὅτι οἱ τὰ τοιαῦτα πράσσοντες ἄξιοι θανάτου εἰσιν.}\footnote{59) Röm. 1, 32. Stöckhardt: S. 31: „Mit Hofmann hier an die von der Obrigkeit zu exekutierenden Todesstrafe zu denken, liegt ganz aus dem Wege.“ Philippi. S. 37. weist auf die heidnische Hadesscheibe mit ihren Strafen hin und urteilt: „Es ist demnach \textgreek{θάνατος} an unserer Stelle wohl von der mors aeterna zu interpretieren.“} Daher meinen sie drittens durch moralische Bestrebungen und durch von ihnen erdachte Gottesdienste und Opfer Gott versöhnen zu sollen und zu können, selbst wenn sie auf den Altar schreiben müssen: „Dem unbekannten Gott“, \textgreek{ἀγνώστῳ θεῷ.}\footnote{60) Apg. 17, 23.} Es sei nochmals auf die Worte der Apologie hingewiesen: Haec opinio legis [dass man durch Werke Gnade erlangen könne] haeret naturaliter in animis hominum, neque excuti potest, nisi unum divinitus docemur. Ferner: Sie de omnibus operibus iudicast mundus, quod sint propitiatio, qua placatur Deus.\footnote{61) Stöckhardt, S. 1 ff.} Der Apostel Paulus nennt die Gesetzesreligion ausdrücklich die Fleischreligion, wenn er den Galatern, sofern sie durch das Gesetz gerecht werden wollten, zuruft: „Ihr Geist (\textgreek{πνεύματι}) habt ihr angefangen, wollt ihr's denn nun im Fleisch (\textgreek{σαρκί}) vollenden?“\footnote{62) Gal. 3, 3.} Luther bemerkt hierzu in seinem Kommentar zum Galaterbrief: „Paulus sagt hier ‚Geist‘