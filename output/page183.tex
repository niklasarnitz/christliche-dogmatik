{\noindent\normalsize Wesen und Begriff der Theologie.\hfill 172}

Kampf Luthers um die sola gratia und um die sola Scriptura sowohl vom als dem reformierten Schwärmertum gegenüber. Daher die Warnung der Konkordienformel, den Christenstand durch irgend etwas, was in Menschen selbst gelegen ist (durch „aliquid in homine“) zu fundamentieren. Daher auch unser eigener heißer Kampf hier in den Vereinigten Staaten gegen die Ichtheologie der reformierten Sekten und gegen die Ichtheologie des Synergismus, der sogar unter lutherischem Namen Existenzberechtigung forderte und noch fordert. Christliche Heilsgewissheit und christliche Wahrheitsgewissheit ist nur möglich, wenn wir mit Luther aus unserem Ich heraus „über uns“ fahren oder --- was dasselbe ist --- einen Standpunkt außerhalb der Welt einnehmen. Das geschieht aber allein dadurch, dass wir in Übereinstimmung mit der von Christo festgesetzten Ordnung durch den Glauben an seinem (Christi) Wort bleiben, das wir im Wort seiner Apostel und Propheten, in der Heiligen Schrift, haben.

\section*{20. Theologie und Methode.}

Hat die Frage nach der „dogmatischen Methode“ den Sinn, woher der Theologe die christliche Lehre zu nehmen habe oder welches das der Theologie eigentümliche principium cognoscendi sei, so geht aus dem Dargelegten bereits hervor, dass jede Methode, wie sie sich auch nennen mag, zu verwerfen ist, die neben die Schrift noch ein anderes Erkenntnisprinzip stellt, sei es die Kirchenlehre oder das „Glaubensbewusstsein“ des Theologen oder irgendein anderes außerhalb der Schrift (extra Scripturam) gelegenes Prinzip. Was hier über „Methode“ noch zu sagen ist, betrifft vornehmlich die Frage, in welcher äußeren Gruppierung oder Reihenfolge die in der Schrift vorliegenden Lehren zum Zweck des Unterrichts (docendi causa) zur Darstellung kommen können oder sollen. Wir hätten diesen Abschnitt auch „Theologie und die äußere Ordnung der einzelnen Schriftlehren“ überschreiben können. Wir behalten aber den Ausdruck „Methode“ bei, weil es Sitte war und noch ist, namentlich von synthetischer und analytischer Methode zu reden, und im Anschluss hieran über die Richtigkeit, resp. Unrichtigkeit der Gruppierung der Lehren (innerhalb eines corpus doctrinae) das Nötige gesagt werden kann.

Unter den alten Lutherischen Theologen treten die einen für die synthetische, andere für die analytische Methode ein. Während