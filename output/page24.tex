\makebox[\linewidth][s]{\normalsize 14 \hfill \textmd{Wesen und Begriff der Theologie.}}

Religionen noch die nötige psychologische, geschichtliche und philosophische Betrachtung der Religionen gefehlt habe. Diesen Wissenszweigen sei erst in neuerer Zeit die gebührende Aufmerksamkeit gewidmet worden. Aber auch hier liegt eine Selbsttäuschung vor. Wir kommen auch vermittelst der Religionspsychologie, der Religionsgeschichte und der Religionsphilosophie nicht über die Zweizahl der wesentlich verschiedenen Religionen hinaus.

Was die psychologische Betrachtung der Religionen betrifft, so ist mit großer Energie auf die „Gleichartigkeit“ der „psychologischen Erscheinungen“ bei Nichtchristen und Christen hingewiesen worden. Weil die älteren Theologen diese Gleichartigkeit übersehen hätten, so sei es ihnen nicht möglich gewesen, die christliche Religion mit den nichtchristlichen unter ein Genus zu bringen.\footnote{38) So n. B. Kirn, Grundriss, S. 9.} Aber die behauptete Gleichartigkeit der psychologischen Phänomene bei Christen und Nichtchristen verschwindet sofort, sobald wir vergleichend prüfen. An die Stelle der Gleichartigkeit tritt der diametrale Gegensatz. Zu der nichtchristlichen Seele finden wir die folgenden seelischen Erscheinungen: das Schuldbewusstsein oder das böse Gewissen, die Furcht vor Strafe und damit die innerliche Flucht vor Gott, das Bestreben, durch eigene Werke die Strafe abzuwenden, und, weil das Streben nicht zum erstrebten Ziel führt, den Zustand der Todesfurcht und der Hoffnungslosigkeit.\footnote{39) Eph. 2, 12; Hebr. 2, 15. Mit Recht verweist Darleh auf Eph. 2, 12 die Ausnahmen, welche Zwingli, Pucer u. A. in Bezug auf einzelne Helden annahmen, in das Gebiet der „Träume“. Luther gegen Zwinglis Seligsprechung der heidnischen Heroen Herkules, Theseus, Sokrates usf. St. L. XX, 1767. Die Selbstbekenntnisse der Heiden über ihre Hoffnungslosigkeit bei Luthardt, Apost. Vortr. I, 2, Anm. 11.} In der christlichen Seele finden wir die entgegengesetzten Zustände und Bewegungen: das gute Gewissen durch den Glauben an die Versöhnung, die durch Christum geschehen ist,\footnote{40) Röm. 5, 1: \textgreek{εἰρήνην ἔχομεν πρὸς τὸν θεὸν διὰ τοῦ κυρίου ἡμῶν Ἰησοῦ Χριστοῦ}.} den ferner freudigen Zutritt zum Gott durch den vom Tod und Hoffnungseligkeit,\footnote{41) Röm. 5, 2: \textgreek{δι’ οὗ (Χριστοῦ) καὶ τὴν προσαγωγὴν ἐσχήκαμεν τῇ πίστει εἰς τὴν χάριν ταύτην}.} sondern Triumph über den Tod\footnote{42) 1 Kor. 15, 55: \textgreek{ποῦ σου, θάνατε, τὸ κέντρον;} Phil. 1, 23: \textgreek{Ἐπιθυμίαν ἔχω εἰς τὸ ἀναλῦσαι καὶ σὺν Χριστῷ εἶναι}.} und die gewisse Hoffnung des ewigen Lebens.\footnote{43) Röm. 5, 2: \textgreek{Καυχώμεθα ἐπ’ ἐλπίδι τῆς δόξης τοῦ θεοῦ}. So reduziert sich die „Gleichartigkeit“ der psycholo-}
