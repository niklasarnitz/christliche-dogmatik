werden derselben Kritik unterworfen. „Kritik“ ist ein wilder Ausdruck. Die Kritik nimmt tatsächlich die Gestalt entschiedener Anklagen an. Darauf glauben wir hier noch besonders hinweisen zu sollen, obwohl die Sache in andern Verbindungen schon vielfach behandelt wurde. Die Theologen des 16. und 17. Jahrhunderts werden offen angeklagt, dass sie die Heilige Schrift für Gottes Wort gehalten oder, wie es häufiger ausgedrückt wird, Schrift und Gottes Wort „identifiziert“ haben. Wir lesen: „Der Fehler . . . steckt in der mangelhaften oder mangelnden Unterscheidung zwischen Bibel und Wort Gottes.“\footnote{597) Bei Nitsch-Stephan, S. 245.}\n\n„Vollends ausgebildet wurde die Inspirationlehre von den protestantischen Scholastikern des 17. und 18. Jahrhunderts seit dem Vorbild, das Johann Gerhard bereits 1610 im Locus de Scriptura gab und 1625 in der Exegesis Uberior Loci de Scriptura weiterführte. Man glaubte, die Katholiken, die Sozinianer, die Arminianer und andere Parteien nur dann siegreich bekämpfen zu können, wenn man das göttliche Ansehen der Schrift auf den Buchstaben [auf die Worte der Schrift] ausdehne. Die Bibel war . . . ein göttliches Religionsbuch.“\footnote{598) u. a. L., S. 249.}\n\nEine zweite Anklage, die mit der vorstehenden Generalanklage zusammenhängt, lautet dahin, dass die alten Theologen, weil sie nur aus und nach der Schrift lehrten, einen schädlichen Einfluss auf die christliche Kirche ausgeübt hätten. Sie hätten „Intellektualismus“, totes Verstandeschristentum, gefördert. „Lebendiges“ Christentum könne nur durch das Lehren aus dem „christlichen Erlebnis“ oder dem „frommen Selbstbewusstsein“ des theologierenden Individuums erzielt werden. Dass auch Ihmels sowohl gegen die erste Kirche als auch gegen Luther und die Reformationszeit sowie gegen die alten Dogmatiker die Anklage erhebt, sie hätten „Intellektualismus“ gefördert, weil sie sich direkt auf die Schrift beriefen, wurde schon oben ausführlicher dargelegt.\footnote{599) Unter dem Abschnitt „Die nähere Beschreibung der Theologie, als Lehre gefällt“, S. 70 ff.}\n\nEine weitere Anklage lautet dahin, dass sich unter den alten Theologen von Luther an bis auf Hollaz inklusive fast gar keine Uneinigkeit in der Lehre finde. Denn dies und nichts anderes ist der Sinn des Vorwurfs, dass die einzelnen Dogmatiker an „selbständiger Reproduktion“ der christlichen Lehre fehle. Sie hätten die Sache so angesehen, als ob „inhaltlich“ an der überkommenen Glaubenslehre „nichts mehr geändert werden dürfte und der Fortschritt lediglich in einer bestimmteren begrifflichen Ausge-