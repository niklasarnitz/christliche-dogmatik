164 \hfill Wesen und Begriff der Theologie.

soviel es in den gegebenen Dingen tatsächlich ausgeprägt
vorliegt. Wo die Vermutung, die Hypothese oder die Spekulation
des Naturforschers anfängt, da hört die Naturwissenschaft auf. Um
noch einmal an Hoppes Ausspruch zu erinnern: „Die Natur ist nicht
so liebenswürdig, sich an das Schema des Lehrbuchs zu binden.“
Man hat die Sache bei uns hierzulande auch so verdeutlicht, dass
man zwischen einem Eisenbahnsystem und einem Gebirgssystem unter-
schieden hat. Ein Eisenbahnsystem können wir allerdings konstruie-
ren, insofern es noch nicht existiert, sondern erst erbaut werden soll.
Ein Gebirgssystem hingegen, weil es bereits unabhängig von unsern
Gedanken existiert, können wir nicht konstruieren, sondern nur nach
seiner gegebenen Beschaffenheit und nach dem bereits vorliegenden
Verhältnis der einzelnen Gebirgsketten zueinander beschreiben.
Dies findet nun seine Anwendung auf die Systembildung in der
Theologie. Der Theologe und überhaupt jeder, der in der Kirche
lehrend auftritt, hat es mit der gegebenen und unveränderlichen Tat-
sache des Wortes Gottes zu tun, das Christus seiner Kirche durch
die Apostel und Propheten gegeben hat. Gottes Wort ist eine so
unveränderliche und abgeschlossene Tatsache wie die Weltschöpfung.
Wie wir Menschen die geschaffene Welt nicht ändern können, sondern
als eine abgeschlossene Tatsache gehen lassen müssen, so sind wir in
der Kirche mit allem Lehren an Christi Wort gebunden. Christi
Wort ist unsere „Wahrnehmungswelt“, und zwar unsere einzige
„Wahrnehmungswelt“, so gewiss wir durch göttliche Ordnung ge-
bunden sind, an Christi Wort zu bleiben, allein aus Gottes
Munde und nicht aus dem eigenen Ich oder aus dem Ich anderer
Menschen zu lehren.\footnote{50) Röm. 8, 31. 32; Jer. 23, 16.} Was anderswoher stammt als aus Gottes
Munde, das ist Stroh unter dem Weizen, nicht Wahrheitserkenntnis,
sondern Einbildung, ein Produkt, das in Gottes Haufe, in der
christlichen Kirche, weder vorgetragen noch angehört werden soll. So
steht fest, dass in der Theologie nicht der geringste Raum gelassen ist
für die menschliche Spekulation oder, was auf dasselbe hinauskommt,
für eine Systembildung zum Zweck der „erkenntnismäßigen“ Er-
forschung der christlichen Wahrheit. Wo wir für unser beschränktes mensch-
liches Erkennen Lücken auf unserm „Wahrnehmungsgebiet“, das ist,
in dem geoffenbarten Wort Gottes, Zulage treten, da hüten wir uns
davor, die Lücken mit eigenen Gedanken auszufüllen. Da erinnern
wir uns, wie bereits durch den Hinweis auf 1 Kor. 13, 9 dargelegt