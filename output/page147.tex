136

\textls[100]{Wesen und Begriff der Theologie.}

werden kann" und "tatsächlich nicht selten so böse wird, dass der Christ, um nicht in Ungewissheit umzukommen, darauf angewiesen ist, seinen Christenstand und sein Stehen in der Wahrheit lediglich vermittels des Glaubens von dem objektiven Faktor des Schriftworts „abzulesen". Aber dabei ist und bleibt es, wie gesagt, Gottes Wille, dass die Christen mit allem Fleiß danach trachten sollen, dass sie auch an ihren Werken ein testimonium für ihr Stehen in der Gnade und in der Wahrheit haben.\footnote{480} Aber das gilt bekanntlich nur von den guten Werken der Christen. Prüfen wir aber die Werke oder die Früchte, welche am Baum „des frommen Selbstbewusstseins des theologisierenden Subjekts" gewachsen sind, so stellt sich sofort heraus, dass es böse Werke sind. Schleiermacher, der „Vater" der Selbstbewusstseinstheologie im 19. Jahrhundert, leugnet die Sündenschuld und die Tilgung der Sündenschuld durch die stellvertretende Genugtuung Christi, die ewige Gottheit Christi, die Dreieinigkeit, kurz, alle Grundartikel der christlichen Religion. Diese böse Beschaffenheit der Früchte des Schleiermacher’schen Selbstbewusstseins geben auch neuere Theologen zu. Sie bewundern zwar die Methode Schleiermachers und befolgen sie. Aber dem Inhalt des Schleiermacher’schen Selbstbewusstseins bringen sie „weitgehende Bedenken" entgegen. Doch, was für Lehrfrüchte haben auch die „lutherisch-konfessionellen" Theologen gebracht, die sich auf Schleiermachers Erlebnismethode eingelassen haben? Mit sichtlicher Schadenfreude notiert und publiziert der liberale Flügel der neueren Theologen die Tatsache, dass auch die neueren Lutheraner die Gottheit als Gottes Wort und Christi Werk als satisfactio vicaria aufgegeben haben, außerdem die Erbsünde, die ewige, unveränderliche Gottheit Christi, die „Zweijnaturenlehre", die Rechtfertigung als actus forensis, die Gnadenmittel als einzige Mittel der Darbietung der Vergebung der Sünden und der Entstehung und Erhaltung des Glaubens. Das sind aber lauter böse Werke, die der heilige Geist, der im Herzen der Christen wohnt, verabscheut. Sie können daher nicht die „christliche Gewissheit" stiften, sondern nur als Stützpunkte dienen für eine imaginäre Gewissheit des „menschlichen Selbst-

\textsuperscript{480)} Wir haben mit den alten Theologen die guten Werke testimonia Spiritus Sancti externa genannt, im Unterschied vom testimonium Spiritus Sancti internum, das in dem vom heiligen Geist gewirkten Glauben an Gottes Wort besteht und mit diesem Glauben zusammenfällt. Vgl. die eingehendere Darlegung unter den Abschnitten „Glaube und Zeugnis des heiligen Geistes", II, 554 ff., und „Die Rechtfertigung aus den Werken", II, 654 ff.