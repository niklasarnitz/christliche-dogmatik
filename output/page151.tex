\pagestyle{empty}
\begin{flushleft}
\begin{minipage}[t]{0.5\textwidth}
\textbf{140}
\end{minipage}%
\begin{minipage}[t]{0.5\textwidth}
\raggedleft
\emph{Wesen und Begriff der Theologie.}
\end{minipage}
\end{flushleft}

beschrieben worden, dass der Theologe, insofern er sich wirklich von der heiligen Schrift losgemacht hat, nicht mit Gott und der göttlichen Wahrheit, sondern mit sich selbst und seinen menschlichen Gedanken Verkehr pflegt oder, wie man es auch ausgedrückt hat, mit „Projektionen des menschlichen Ich“ umgeht. Es lässt sich wirklich nichts zugunsten des wissenschaftlichen Charakters der theologischen Lehrmethode sagen.\footnote{485) Schleiermacher, Christl. Glaube, I, \S 1, Anm. 2, S. 20.}

Als Resultat ergibt sich: Will die moderne Theologie wieder in Verbindung kommen mit nicht bloß eingebildeter, sondern wirklicher „christlicher Gewissheit“, und will sie auch wieder Kontakt gewinnen mit wirklicher Wissenschaft --- zur Wissenschaft gehört doch auch eine geordnete Gedankenwirtschaft ---, so muss sie --- es geht wirklich nicht anders --- durch sich selbst einen großen Strich machen. Sie muss den unchristlichen und unlogischen Gedanken der „Selbstgewissheit“ auf dem Gebiet der christlichen Wahrheitserkenntnis aufgeben und sich wieder auf den objektiven Grund stellen, auf dem die christliche Kirche tatsächlich erbaut ist, auf das Wort der Apostel und Propheten oder, was dasselbe ist, auf das Wort Christi.\footnote{486) Eph. 2, 20.} Es gibt keine Selbstgewissheit in der christlichen Theologie, sondern es steht so, wie Luther sagt: \emph{Homo est certus passive, sicut Verbum Dei est certus active}. Es gilt hier auch nicht die moderntheologische Berufung auf Christi Person im Gegensatz zu Christi Wort. Freilich ist Christi Person der Eckstein seiner Kirche.\footnote{487) Eph. 2, 20; 1 Petr. 2, 6.} Aber auf Christum als Eckstein können wir uns nur vermittelst des Glaubens an Christi Wort, das wir im Wort seiner Apostel und Propheten haben, gründen, wie in den unmittelbar davorhergehenden Worten zum Ausdruck kommt: „\textgreek{ἐποικοδομηθέντες ἐπὶ τῷ θεμελίῳ τῶν ἀποστόλων καὶ προφητῶν}“. Wer das Wort der
\footnotetext[488]{Frank beruft sich für den wissenschaftlichen Aufbau der Selbstgewissheit auf den Fichteschen Idealismus von der Deduktion durch das Subjekt. Er sagt: „es ist die bleibende Wahrheit des Fichteschen Idealismus“ (Christl. Gewissheit I, 61). C. F. Stäudlin beschreibt den Fichteschen Idealismus so in „Probleme der Philosophie und ihre Lösungen“, S. 96: „Nach Fichte besteht das Wesen der Dinge, wie bei Berkeley, in der bloßen Vorstellung; er geht aber dadurch noch über Berkeley hinaus, dass er nicht mehr nach einer äußeren Entstehungsursache dieser Vorstellungen sucht, sondern, von einer solchen ausdrücklich absehend, unsern Geist selbst als den alleinigen Urheber alles dessen ansieht, was er äußerlich wahrzunehmen glaubt.“ Frickel findet bei Fichte „nur etwa zwei“ Antersequenzen. S. Ulrici nennt den Fichteschen Idealismus „eine unsinnige Einseitigkeit“ (Re. 2 XV, 381).}