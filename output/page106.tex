mehr Lastern.\footnote{Luther zu den Worten οὐκ παρόμοια τῶι σατανᾶι: „Dieselbe Exkommunikation, von der der Apostel 1 Kor. 5, 5 redet.“} Und als unter den Korinthern etliche sagten, „die Auferstehung der Toten sei nichts“\footnote{1 Kor. 15, 12: ἀνάστασις νεκρῶν οὐκ ἔστιν.} belehrt sie der Apostel dahin, dass sie nichts von Gott wissen\footnote{1 Kor. 15, 34: ἀγνωσίαν θεοῦ τινες ἔχουσι. Dasselbe urteilt Christus von den Sadduzäern Matth. 22, 29: „Ihr irret und wisset die Schrift nicht noch die Kraft Gottes.“} und die ganze christliche Religion leugnen, die nun einmal den Glauben an die Auferstehung der Toten in sich schliesse.\footnote{Hierher gehört das ganze 15. Kapitel des ersten Korintherbriefes. Die ausführlichere Darstellung der Schriftlehre von der Auferstehung der Toten III, 600 ff.}\n\nWenn wir so von „Fundamentallehren“ reden, die der Glaube an die Vergebung der Sünden um Christi willen notwendig teils voraussetzt, teils in sich schliesst, so versteht sich von selbst, dass nicht die kirchlich-dogmatische Formulierung dieser Lehren gemeint ist. Daran haben auch die alten Dogmatiker reichlich erinnert;\footnote{Baier-Walther I, 61, nota e.} ebenso Luther, wenn er ausdrücklich darlegt, dass die Konzilien mit ihren kirchlichen Terminis „nichts Neues im Glauben gestellet“, sondern nur den alten Glauben, den die Christen vor allen Konzilien holten, „durch die heilige Schrift erhalten“ haben. Dies bezieht Luther auch auf das Nicänum. Die Christen haben stets die gleichwesentliche Gottheit Christi geglaubt, ehe Arius gegenüber der Terminus ὁμοούσιος in Aufnahme kam.\footnote{„Von Konfiliis u. Kirchen“, St. J. XVI, 2233. 2214.}\n\n\subsection*{Primäre und sekundäre Fundamentallehren.}\n\nDiese weitere Einteilung der fundamentalen Lehren braucht niemand zu erschrecken. Sie ist nicht eine Erfindung der orthodoxen Dogmatik, sondern sachlich berechtigt und praktisch wichtig. Es ist z. B. in den Streitigkeiten zwischen der lutherischen und der reformierten Kirche auch darüber verhandelt worden, ob Taufe und Abendmahl zum \textbf{Fundament} des christlichen Glaubens gehören.\footnote{Vgl. zur Litteratur II, Note 647. Ausser Nikolaus Hunnius’ Λύδομενους Theologica de Fundamentali Dissensu (1626) gehört vornehmlich hierher Joh. Süßmanns Calvinismus Irreconciliabilis (1646). Die gesamte Litteratur auch von reformierter Seite bei Bald., Bibliotheca Theologica II, 486 ff.} Die Schrift entscheidet diese Frage. Auf Grund der Schrift müssen wir sagen, dass neben dem Wort des Evangeliums auch beide Satra-