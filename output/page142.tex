\setcounter{footnote}{463}
\section*{Wesen und Begriff der Theologie.}

Das findet nun seine Anwendung auf die ganze Schar der neueren Theologen. So gewiss diese prinzipiell Christi Wort als Entscheidungsgrund und Basis der Gewissheit ausscheiden, so gewiss ist ihre Gewissheit eine eingebildete. Wie sie gegen die Schrift, die Gottes Wort ist, eine Barrikade errichten, so verbarrikadieren sie sich auch gegen das Wahrheitszeugnis des heiligen Geistes, das in dem Schriftwort als Gottes Wort wirksam ist. Sie sind darauf angewiesen, sich selbst ein Wahrheitszeugnis, das \textit{testimonium veritatis}, auszustellen auf Grund dessen, was sie auf dem Wege der „\emph{Selbstbesinnung}“ bei sich selbst empfinden, selbst tun und über sich selbst urteilen. Wir finden daher auch, dass die Vertreter der Selbstgewissheitstheologie eine entschiedene Antipathie gegen das Zeugnis des heiligen Geistes an den Tag legen. Teils erklären sie es ausdrücklich für logisch falsch, indem sie es mit den römischen Theologen unter die Rubrik „\emph{Zirkelbeweis}“ bringen, teils halten sie es doch für ungenügend. Und das ist von ihrem Standpunkt aus naturgemäß. Wer selbstgewiss ist, sieht in dem Zeugnis des heiligen Geistes ein unberechtigtes Konkurrenzunternehmen, ein unberechtigtes Scheineindringen in ein Geschäft, das der Theologe selbst zu besorgen habe. Dass der Theologie sich selbst gewiss machen müsse, spricht sehr klar z. B. auch Bößler aus, der dem rechts stehenden Teil der neueren Theologen zugesellt wird. Bößler will\footnote{Handbuch der theol. Wissenschaften II III. 65.} die Berufung der altprotestantischen Dogmatiker auf das \textit{testimonium Spiritus Sancti}\footnote{Der letzte Satz von Bößler selbst durch den Druck hervorgehoben.} nicht ganz verwerfen, bezeichnet sie aber als die Sache nicht deckend und fügt hinzu: „Es gibt eine von uns selbst abhängige, unserer Verantwortung anheimfallende freie Tat, eine moralisch notwendige, darum aber der Freiheit überlassene Konsequenz. Durch diese freie Tat erst schaffen wir selbst die Gewissheit.“\footnote{Schmalt. Rel. M. 312, 2.} Aber alles, was in der Theologie auf den Menschen selbst gebaut wird, sei es die „\emph{Heilsgewissheit}“, sei es die „\emph{Wahrheitsgewissheit}“, bricht zusammen, sobald und sooft die Donnerart des göttlichen Gesetzes auch die „Heiligen“ in einen Haufen schlägt, und lässt keinen recht haben, treibt sie allesamt in das Schrecken und Verzagten.\footnote{I, 133 f.} Frank hat in seinem „\emph{System der christlichen Gewissheit}“ gelegentlich an Archimedes’ \textgreek{ποῦ στῶ} erinnert.\footnotemark[468] Dies Diktum spricht aber gegen ihn und gegen jede Form der Ichtheologie. Es kommt nichts darauf an, ob