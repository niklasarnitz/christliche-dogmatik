daher so: Wo die Einigkeit in der christlichen Lehre vorhanden ist und erhalten bleibt, da ist das in keiner Weise das Resultat unserer Kraft, Weisheit und Geschicklichkeit, sondern ein Werk der göttlichen Gnade und Kraft allein. Das lehrt die Schrift, \footnotemark[127] und das kommt auch in unsern Kirchengebeten zum Ausdruck. \footnotemark[128]\subsection*{6. Das Christentum als absolute Religion.}Die christliche Religion ist allerdings die „absolute“, das ist, schlechthin vollkommene, Religion, die einer Ergänzung oder Verbesserung weder bedürftig noch fähig ist und daher auch nicht überboten werden kann. Aber dies Prädikat der Absolutheit kommt der christlichen Religion nicht zu, insofern sie ein „logisch vollkommenes Ganzes“ bildet. Das logisch vollkommene Ganze im Sinne des menschlichen Erkennens lehrt der Apostel Paulus ab, wenn er die religiöse Erkenntnis, die den Christen, den hohen Apostel eingeschlossen, in diesem Leben eignet, ausdrücklich eine fragmentarische nennt, \foreignlanguage{greek}{ἅρτι γινώσκομεν ἐκ μέρους}. \footnotemark[129] Die christliche Religion sollte ferner nicht absolut genannt werden, insofern sie die „vollkommenste Moral“ lehrt. Freilich lehrt die christliche Religion die vollkommenste Moral. Die christliche Moral kann unmöglich überboten werden, weil sie den Inhalt hat: „Du sollst lieben Gott, deinen Herrn, von ganzem Herzen, von ganzer Seele und von ganzem Gemüte“ und: „Du sollst deinen Nächsten lieben als dich selbst.“ \footnotemark[130] Aber diese vollkommene Moral ist erst eine Folge und Wirkung der christlichen Religion. Sowohl die Liebe zu Gott als auch die Liebe zum Nächsten ist eine Tochter des Glaubens, „dass Gott uns geliebet hat und gesandt seinen Sohn zur Versöhnung für unsere Sünden“. \footnotemark[131] Paulus begründet seine Ermahnung zu einem christlich- moralischen Leben so: „Ich ermahne euch durch die (in Christo erschienene) Barmherzigkeit Gottes, dass ihr eure Leiber begebet zum Opfer, das da lebendig, heilig und Gott wohlgefällig sei.“ \footnotemark[132]\setcounter{footnote}{126}\footnotetext[127]{Joh. 17, 11. 12. 15. 20. 21; Pf. 86, 11 usw.}\footnotetext[128]{J. B. in der Kirchenagende der Missourisynode, S. 44 f. 58 f. S. Walther, Pastorale, Anm. 1, S. 389 f.}\footnotetext[129]{1 Kor. 13, 12. Weiteres hierüber unter dem Abschnitt „Theologie und System“.}\footnotetext[130]{Matth. 22, 37--40.}\footnotetext[131]{1 Joh. 4, 9--21.}\footnotetext[132]{Röm. 12, 1. Hieran ist auch bei Nitzsch (Stephan, S. 147) erinnert: „Der erste Mangel der [sittlichen] Kantischen Fassung haftet auch allen den Deutungen an, welche die Vollkommenheit der christlichen Religion schon durch die}