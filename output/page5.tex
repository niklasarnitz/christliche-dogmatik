VI \hfill Vorwort.\par logischen Wissenschaft später selbst von der bezeugten Wahrheit abgewichen sind.\par Ich habe mich auch in dem vorliegenden Bande einer sachlichen Darstellung befleißigt. Wo an einigen Stellen scharfe Ausdrücke gebraucht worden sind, schienen sie von der Wichtigkeit der behandelten Sache gefordert zu sein. Es galt ins Licht zu stellen, dass eine Theologie, die die christliche Lehre nicht allein aus der heiligen Schrift, sondern aus dem Ich des theologisierenden Individuums beziehen und normieren will, weder christlich noch wissenschaftlich, sondern das Gegenteil von beidem ist. Dass ich eine theologische Inkonsequenz kenne, nach welcher die Möglichkeit vorliegt, dass jemand in seinem Herzen und vor Gott anders glaubt, als er in seinen Schriften schreibt, kommt auch in diesem Bande wiederholt zum Ausdruck.\par Wir amerikanischen Lutheraner \glqq{}streng konfessioneller Richtung\grqq{} haben nicht die geringste Ursache, uns über andere zu erheben. Wir würden sicherlich in demselben verkehrten Strom schwimmen, wenn uns Gottes Gnade nicht in ganz andere kirchliche Verhältnisse gestellt hätte. Wir -- die zweite und dritte Generation -- sind unter den denkbar günstigsten Verhältnissen theologisch geschult worden. Wir wurden quellenmäßig nicht nur mit der Theologie der alten Kirche, der Reformation und der Dogmatiker, sondern auch mit der Art und dem Resultat der modernen Theologie bekannt gemacht. Dazu kam die fortgehende Mahnung seitens unserer Lehrer, keine menschliche Autorität, auch nicht die Autorität Luthers und der symbolischen Bücher, an die Stelle der göttlichen Autorität der Schrift zu setzen. Die Mahnung im letzten Studienjahre lautete: \glqq{}Niemand von Ihnen trete in das Predigtamt, der in bezug auf die Schriftmäßigkeit irgendeiner Lehre der lutherischen Symbole noch Zweifel hat. Bei wem noch Zweifel sich finden, der unterrede sich freimütig mit irgendeinem seiner Lehrer.\grqq{} Schon von der ersten Predigt