\pagenumbering{arabic}
\setcounter{page}{132}
\section*{Wesen und Begriff der Theologie.}

Archimedes den Ausspruch: „$\Lambda\omicron\zeta$ $\pi\omicron\upsilon$ $\sigma\tau\omega$, $\kappa\alpha\iota$ $\tau\eta\nu$ $\kappa\omicron\sigma\mu\omicron\nu$ $\kappa\iota\nu\eta\sigma\omega$, „Gib mir einen festen Standpunkt, so will ich die Welt bewegen“, wirklich getan hat oder nicht. Jedenfalls soll damit gesagt sein, dass es einen Standpunkt außerhalb der Welt erfordern würde, um die Welt zu heben. Dies ist ein treffliches Bild der wichtigen Wahrheit auf geistlichem Gebiet, dass wir für unser persönliches Christentum und unsere Theologie einen außer uns und der ganzen Welt gelegenen Standpunkt nötig haben, um uns gegen Welt, Teufel und unser Ich behaupten zu können und den Sieg zu gewinnen. Bekanntlich haben wir alle diese Mächte gegen uns, speziell auch in Sachen der christlichen Lehre und ihrer Gewissheit. Diesen festen, außer uns gelegenen Standpunkt haben wir in Christi Wort. Christus sagt, wie wir uns bereits belehren ließen, von seinem Wort: „Himmel und Erde werden vergehen, aber meine Worte vergehen nicht.“ Zugleich hörten wir bereits, dass Christus dies sein Wort, das fester steht als Himmel und Erde, als den Standpunkt bezeichnet, auf den alle sich zu stellen haben, die in Wahrheit seine Jünger sein und die Wahrheit erkennen wollen. Diese Belehrung und Ermahnung Christi hat Luther verstanden. Aus eigener tiefer Erfahrung berät er dabei unaufhörlich alle Christen und insonderheit alle Theologen dahin, ja aus ihrem Ich herauszutreten und „durch das Wort über sich zu fahren“.\footnote{468) Luther. St. L. XI, 1727. 1736.} Durch diese Methode, die in diametralem Gegensatz zur Selbstgewissheitstheologie steht, kam es bei Luther zu der unerschütterlichen Wahrheitsgewissheit, wie sie sich z.\,B. in seiner „Antwort auf des Königs zu England Lästerschrift“ ausdrückt;\footnote{469) „So wahr Gott lebt, welcher König oder Fürst meint, dass sich der Luther vor ihm demütige der Meinung, als reue ihn seine Lehre und habe unrecht gelehrt und suche Gnade, der betrügt sich selbst weidlich und macht ihm selbst einen güldenen Traum, da er eitel Dreck finden wird, sobald er aufwacht. Der Lehre halben ist mir niemand so groß, ich halte ihn für eine Wasserblase und noch geringer; da wird nichts anders aus. \ldots Wen es gereuet, der lasse ab; wer sich fürchtet, der fliehe. Mein Rückhalter ist mir stark und gewiss genug, das weiß ich. Ob mir schon die ganze Welt anhinge und wiederum abfiele, das ist mir eben gleich, und denke: ist sie mir doch zuvor auch nicht angegangen, da ich allein war. Wer nicht will, der lasse es; wer nicht bleibt, der fahre immer hin.“ Das ist Luthers Wahrheitsgewissheit. Und diese Wahrheitsgewissheit
