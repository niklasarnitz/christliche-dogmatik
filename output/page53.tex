nach dem Wesen (Evangelium, Gesetz) und darum nach dem Resultat für die Menschen (Hoffnungslosigkeit, Gewissheit der Seligkeit) ein spezifischer ist. Das Christentum verhält sich zu allen nichtchristlichen Religionen nicht wie Licht zum Halbdunkel, sondern wie Licht zur Finsternis,	extsuperscript{149} nicht wie Gottesdienst zu einem wenigstens geringen Anfang von Gottesdienst, sondern wie Gottesdienst zum Dämonendienst,	extsuperscript{150} nicht wie Leben und ein Lebensanfang, sondern wie Leben und Tod.	extsuperscript{151} Das Christentum bietet nicht bloß die „höchste Befriedigung", sondern die einzige Befriedigung.	extsuperscript{152}
Es ist gegen den absoluten Charakter des Christentums auf den Unterschied zwischen dem Alten und Neuen Testament hingewiesen worden. Dieser Unterschied ist Tatsache, aber nur hinsichtlich der Klarheit und Fülle der Offenbarung, dass nur der Glaube an Christum ohne des Gesetzes Werke der einzige Lebensweg für die Menschheit ist. Christus ruft den Juden zu: „Werdet ihr nicht glauben, dass ich es sei, so werdet ihr sterben in euren Sünden."	extsuperscript{153} Aber gleichzeitig verwahrt er sich dabei gegen die Auffassung, dass er damit ein Novum lehre. Er erklärt sich für den eigentlichen Inhalt der Schrift Alten Testaments.	extsuperscript{154} Ebenso verwahrt sich Paulus gegen die irrige Meinung, dass er mit seiner Lehre von der Rechtfertigung aus Gnaden durch den Glauben an Christum unter Ausschluss des Gesetzes eine neue Weise der Gerechtigkeit vor Gott lehre; denn die Weise 	extgreek{χωρὶς νόμου} sei vom Gesetz und den Propheten bezeugt	extsuperscript{155} und die einzig richtige historische Auffassung der im Alten Testament gelehrten Religion.	extsuperscript{156}

\subsection*{7. Christliche Religion und christliche Theologie.}
Man unterscheidet in kirchlichem Sprachgebrauch christliche Religion und christliche Theologie, und zwar in der Weise, dass Religion (subjektiv genommen) die Gottesgelehrtheit aller Christen und Theologie (subjektiv genommen) die besondere Gottesgelehrtheit der Lehrer der Kirche bezeichnet. Diese Unterscheidung kann man sich gefallen lassen. Die Schrift lehrt sowohl,

\vspace{1em}
\noindent
	extsuperscript{149)} Eph. 5, 8: „	extgreek{ἦτε γάρ ποτε σκότος, νῦν δὲ φῶς ἐν κυρίῳ}." Derselbe Gegen	extgreek{stand} Joh. 2, 8; 60, 2.
\noindent
	extsuperscript{150)} 1 Kor. 10, 20; Apostelg. 26, 18.
\noindent
	extsuperscript{151)} Eph. 2, 1–5.
\noindent
	extsuperscript{152)} Röm. 5, 1 ff.; Gal. 2, 16.
\noindent
	extsuperscript{153)} Joh. 8, 24.
\noindent
	extsuperscript{154)} Röm. 3, 21 ff.
\noindent
	extsuperscript{155)} Röm. 5, 39.
\noindent
	extsuperscript{156)} Kap. 4 des Römerbriefs.