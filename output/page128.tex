\noindent 117
\hfill Wesen und Begriff der Theologie.

Manche Gottes aus dem Innern des Theologen lehrt. Aber indem diese Theologie auf das Leben aus Gottes Grunde und auf die unmittelbare Beziehung zur Seligkeit verzichtet, sollte sie auch auf den Anspruch verzichten, in der christlichen Kirche „ihren eigentlichen Lebensherd, ihr Recht und ihren Halt“ zu haben.\footnote{418}) Luthardt, Komp., S. 6.} Sie ist in der Kirche ein exotisches Gewächs und gehört nicht zu den Pflanzen, die der himmlische Vater gepflanzt hat,\footnote{419}) Matth. 15, 13.} weil in der Kirche Gottes nur Gottes Wort (\textgreek{\lambda\acute{o}\gamma\omicron\varsigma\  \theta\epsilon\omicron\tilde{v}}) geredet und dadurch der Menschen Seligkeit gesucht werden soll.\footnote{420}) 1 Petr. 4, 11: 1 Tim. 4, 6; 3. -- Tit. 1, 1. 2 (\textgreek{\epsilon\acute{\iota}\varsigma\  \epsilon\acute{\pi}\iota\gamma\nu\omega\sigma\iota\nu\  \alpha\acute{\lambda}\eta\theta\epsilon\acute{\iota}\alpha\varsigma}).} Walther zitiert aus Meisners Philosophia Sobria über den Zweck der Theologie: „Wer diesen Zweck [die Seligkeit des Menschen] nicht immer beabsichtigt und nicht in aller seiner Theorie [oder \textgreek{\gamma\nu\tilde{\omega}\sigma\iota\varsigma}, Erkenntnis] im Auge hat, der verdient den Namen eines wahren Theologen nicht.“\footnote{421}) S. u. 2B. 14, 76 f.}

Die Terminologie der alten Theologen über den Zweck der Theologie ist gar nicht übel. Sie sagen: \textit{Subjectum operationis theologiae est homo peccator, quatenus ad salutem aeternam perducendus est.}\footnote{422}) Voier I, 40.} Auch die bürgerliche Gesellschaft oder der Staat hat es mit dem \textit{homo peccator} zu tun, aber nicht insofern er zur Seligkeit zu führen ist, sondern insofern der Staat den Zweck hat, das leibliche Leben und die leiblichen Güter gegen die Ausbrüche der sündigen Menschennatur mit leiblichen Strafen zu schützen. Der Theologe hingegen und die christliche Kirche überhaupt geben sich nicht mit der bürgerlichen Bestrafung der Sünden ab, sondern lediglich mit der Offenbarung der Sündenschuld vor Gott durch die Predigt des göttlichen Gesetzes, um danach durch die Predigt des Evangeliums die Vergebung der Sünden und die Seligkeit zu vermitteln. Dieser Zweck der Theologie, die \textit{salus aeterna}, wird aber an den Menschen nicht auf mehreren Wegen, sondern nur auf einem einzigen Wege erreicht. Jeder Mensch, der zur \textit{salus aeterna} gelangt, erreicht dieses Ziel nur durch den Glauben an Christum oder, was dasselbe ist, durch den Glauben an das Evangelium von der Gnade Gottes in Christo, Joh. 3, 36: „\textgreek{\acute{O}\  \pi\iota\sigma\tau\epsilon\acute{\upsilon}\omega\nu\  \epsilon\acute{\iota}\varsigma\  \tau\acute{o}\nu\  \upsilon\acute{\iota}\acute{o}\nu\  \epsilon\acute{\chi}\epsilon\iota\  \zeta\omega\acute{\eta}\nu\  \alpha\acute{\iota}\acute{\omega}\nu\iota\omicron\nu\cdot\  \acute{o}\  \delta\epsilon\acute;\  \alpha\acute{\pi}\epsilon\iota\theta\tilde{\omega}\nu\  \tau\tilde{\omega}\  \upsilon\acute{\iota}\tilde{\omega}\  \omicron\acute{\upsilon}\kappa\  \acute{\partial}\psi\epsilon\tau\alpha\iota\  \zeta\omega\acute{\eta}\nu,\  \acute{\alpha}\lambda\lambda'\  \acute{\eta}\  \omicron\acute{\rho}\gamma\acute{\eta}\  \tau\omicron\tilde{v}\  \theta\epsilon\omicron\tilde{v}\  \mu\epsilon\acute{\nu}\epsilon\iota\  \epsilon\acute{\pi}\'\  \alpha\acute{\upsilon}\tau\acute{o}\nu}.“ So bezweckt die Theologie als „Mittelzweck“ (finis intermedius) vor allen Dingen die Erzeugung und Erhaltung des Glaubens an Christum, wie der Apostel Paulus von sich sagt, dass er sein Amt habe \textgreek{\epsilon\grave{\iota}\varsigma\  \upsilon\acute{\pi}\alpha\kappa\omicron\eta\nu\  \pi\iota\sigma\tau\epsilon\omega\varsigma}.\footnote{423}) Röm. 1, 5.} Freilich bezweckt die