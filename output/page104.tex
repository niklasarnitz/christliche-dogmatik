ausdrücklich lehrt.\footnote{330) Gal. 5, 4: \textgreek{καθαρηγθητε ἀπὸ τοῦ Χριστοῦ ... τῆς χάριτος ἐξεπέσατε.}} Doch innerhalb solcher Gemeinschaften, die die Vergebung der Sünden um Christi vollkommenen Verdienstes willen als Fundament des seligmachenden Glaubens verbieten (Rom), den noch gläubige Kinder Gottes sich finden, kommt daher, dass diese wider das kirchliche Verbot allein auf den für ihre Sünden gekreuzigten Christus ihr Vertrauen vor Gott setzen. Daher legt die Apologie einerseits dar, dass die römische Kirche allerdings im Fundament des christlichen Glaubens irrt\footnote{331) M. 156, 22.}; ,,Ziel Artikel bei unsern Widerfachern stehen den rechten Grund wieder, das Erkenntnis Christi und den Glauben. Denn sie verwerfen und verdammen den hohen, größten Artikel, da wir sagen, dass wir allein durch den Glauben, ohne alle Werke, Vergebung der Sünden durch Christum erlangen. Dagegen lehren sie vertrauen auf unsere Werke, damit Vergebung der Sünden zu verdienen, und geben anstatt Christi ihre Werke, Orden, Messe, wie auch die Juden, Heiden und Türken mit eigenen Werken vorhaben, selig zu werden. Jtem, sie lehren, die Sacramente machen fromm ex opere operato, ohne Glauben. Wer nun den Glauben nicht nötig achtet, der hat Christum bereits verloren. Jtem, sie richten Heiligendienst an, rufen sie an anstatt Christi als Mittler.“ Andererseits beweist dieselbe Apologie: Mansit tamen apud aliquos pios semper cognitio Christi.\footnote{332) M. 151, 271.} \noindent \textbf{4.} Die Schrift lehrt auch, dass der seligmachende Glaube stets Glaube an das Wort Christi ist. Gemeint ist das äußere Wort des Evangeliums, das Christus seiner Kirche zu predigen und zu lehren aufgetragen hat.\footnote{333) Mark. 16, 15. 16; Röm. 1, 17.} Dies Wort ist sowohl Objekt des Glaubens, \textgreek{πιστευετε τῷ εὐαγγελιῳ},\footnote{334) Mark. 1, 15.} als auch das Mittel, wodurch der Glaube entsteht, \textgreek{ἡ πίστις ἐξ ἀκοῆς}.\footnote{335) Röm. 10, 17.} Die Schrift verwirft den Glauben, der nicht Christi Wort, das im Wort seiner Apostel vorliegt (Joh. 17, 20), zum Objekt hat und nicht durch dies Wort gewirkt ist. Sie beschreibt solchen Glauben als eine Einbildung und ein Nichtwissen und als ein menschliches Machwerk (\textgreek{πλάνις ... ἐν σοφίᾳ ἀνθρώπων}).\footnote{336) 1 Tim. 6, 3; 1 Kor. 2, 1—5.} Luther nennt den Glauben, der sich nicht auf das äußere Wort gründet, einen Glauben ,,in die Luft''.\footnote{337) Vgl. die ausführlichere Darlegung II, 535, unter dem Abschnitt „Der seligmachende Glaube ist Glaube an die Gnade, die im Wort des Evangeliums dargeboten wird“. Wenn alte Dogmatiker zwischen dem Fundament} Dass