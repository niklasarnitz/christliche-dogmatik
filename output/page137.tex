126 \\ Wesen und Begriff der Theologie.\\ aus dem Wege gehen. Wir bleiben ungewiss, oder wir sinken wieder in Ungewissheit zurück, wenn wir in der Meinung, wir „könnten“ sie schon (Luthers Ausdruck), unfleißig mit der Schrift umgehen und so der Schrift nicht einmal äußerlich Gelegenheit geben, sich selbst als göttliche Wahrheit zu bezeugen. Überaus gefährlich gestaltet sich aber die Sachlage dann, wenn wir der Schrift gegenüber sogar eine kritische Haltung annehmen. Dann gehen wir an der christlichen Gewissheit nicht nur vorbei, sondern in diesem Falle erfahren wir auch die verblendende Wirkung des Wortes Christi, die Christus in den Worten beschreibt: „Ich preise dich, Vater und Herr Himmels und der Erde, dass du solches den Weisen und Klugen verborgen hast ($\Delta\pi\epsilon\chi\rho o\psi\alpha\varsigma$ $\tau\alpha\tilde{u}\tau\alpha$ $\dot{\alpha}\pi\dot{o}$ $\sigma\omicron\phi\tilde{\omega}\nu$ $\chi\alpha\dot{\imath}$ $\sigma\upsilon\nu\epsilon\tau\tilde{\omega}\nu$) und hast es den Unmündigen offenbaret.“\footnote{Matth. 11, 25.} Und abermal: „Ich bin zum Gericht auf diese Welt kommen, auf dass, die da nicht sehen, sehend werden und die da sehen, blind werden.“\footnote{Joh. 9, 39. vgl. Luther zu Matth. 13, 15. z. B. VII, 194 f.} Auch diese Warnung gehört zur allseitigen Behandlung der „Erkenntnis-theoretischen Frage“. Wir kommen später noch einmal auf diesen Punkt zurück.\\ Halten wir uns gegenwärtig, was die Schrift über die „erkenntnis-theoretische Frage“ lehrt, so gewinnen wir das richtige Urteil über die Bemühungen um die „Wahrheitsgewissheit“, denen wir im Lager der modernen Theologen begegnen. Trotz ihrer Reden von „unmittelbarer“ christlicher Gewissheit suchen sie doch nach einem zuverlässigen Grund oder Stützpunkt für die Gewissheit. Dieser gesuchte zuverlässige Stützpunkt ist auch die „turmfreie Burg“ genannt worden, eine Burg, in der der Christ sich „zuletzt zurück-“ zieht und in der er allen feindlichen Angriffen gegenüber geborgen ist. Wo ist die „turmfreie Burg“ zu finden? Christus verweist, wie wir gesehen haben, seine Kirche auf sein Wort, das wir im Wort seiner Apostel und Propheten haben. Christus verifiziert uns auch, dass sein Wort eine feste Burg sei, fester stehe als Himmel und Erde.\footnote{Matth. 24, 35; Mart. 13, 31; Luk. 21, 33.} Die christliche Kirche hat diese Weisung auch verstanden. Sie hat sich auf die heilige Schrift gestellt und von dieser Gewissheitsbasis aus den Angriffen der Feinde gegenüber sich behauptet. Von dieser Basis aus behauptete sich auch zu Worms der eine Mann Luther gegen die ganze Welt. Nach der Ansicht der modernen Theologen war Christus, die christliche Kirche und natürlich auch Luther in einem Irrtum befangen. Sie meinen, dass die bisher für „Sturmfrei-“ ge-