Wesen und Begriff der Theologie.\hfill 59

Nugtung willen zusagt, ohne Gesetz und ohne des Gesetzes Werke
(\textgreek{χωρὶς νόμου, χωρὶς ἔργων νόμου}), das ist, ohne jede eigene gute
moralische Beschaffenheit und ohne jede eigene Leistung auf seiten
des Menschen. Wieder durch Roms Fluch noch durch den Wider-
spruch des verkommenen Protestantismus, der die göttliche Recht-
fertigungsmethode als zu äußerlich und juridisch abweist, noch auch
durch den Widerspruch des eigenen natürlichen Herzens, dem ja die
„opinio legis“ naturaliter anhaftet, lässt sich der christliche Theologe
zu einer Minderung der in der göttlichen Rechtfertigungslehre
bewegen: „es falle Himmel und Erde, oder was nicht bleiben will“.ootnote{208) Schmalz, Art. \textgreek{μεταβολή}. 5.}
Jeder christliche Theologe ist als solcher stets ein praktischer
Theologe. Und gerade an der Zentrallehre von der Vergebung der
Sünden warnt ihn die eigene Erfahrung oder das eigene „Erlebnis“.
Er denkt daran, wie schrecklich es wäre, wenn der vom Gesetz Gottes
getroffene Sünder hinsichtlich der Vergebung seiner Sünden auf
menschliche Ansichten anstatt auf Gottes eigene geoffenbarte Lehre
angewiesen sein sollte. Und so durch alle Teile der christlichen
Lehre hindurch, inklusive der Lehren von der ewigen Verdammnis
und der ewigen Seligkeit. Kurz, der christliche Theologe lehrt aus
„Gottes Buch“, wie Luther die Schrift nennt, Gottes eigene
Lehre, \emph{doctrinam divinam}, im Gegensatz zu allen menschlichen Ge-
danken und Anschauungen.

Diese Beschaffenheit der in der christlichen Kirche vorzutragenden
Lehre, dass sie \emph{doctrina divina} sein müsse, ist durchweg in der
Schrift selbst gefordert. Die Schrift Alten und Neuen Testa-
ments ist voll von Warnungen vor allen Lehrern, die nicht lediglich
Gottes Wort lehren, sondern sich erlauben, eigene Anschauungen
vorzutragen. Wir lesen bei dem Propheten Jeremias die gewaltigen
Worte: „So spricht der Herr Zebaoth: Gehorchet nicht den Worten
der Propheten, so euch weissagen! Sie betrügen euch, denn sie pre-
digen ihres Herzens Gesicht“ (\textgreek{οὐκ ἀπὸ τοῦ στόματος κυρίου}),ootnote{209) Jer. 23, 16. Parallelen: Jer. 14, 14; 27, 14--16; Klagl. 2, 14; Hesek. 13, 2 ff.}
und nicht aus des Herrn Munde.“ Dasselbe Verbot
des Lehrens menschlicher Gedanken und Anschauungen und dieselbe
Bindung aller Lehrer an Gottes Mund ist durchweg im Neuen
Testament ausgedrückt. Wer in der Kirche, die ja „Gottes Haus“
ist,ootnote{210) 1 Tim. 3, 15.} lehrend den Mund auftut, soll Gottes Wort (\textgreek{λόγια θεοῦ})
lehren, wie es Petrus ausdrücklich vorschreibt.