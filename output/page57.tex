dern der Sache nach in der Schrift enthalten sind. Letzteres ist der Fall, wenn unter „Theologie“ verstanden wird:
\begin{enumerate}
    \item die Schrift denen eigenen sein soll, die zum öffentlichen Predigtamt in der Gemeinde bestellt werden. Das \emph{διδακτικός}, „Lehrhaftig“ (1 Tim. 3, 2), beschreibt die besondere Lehrtätigkeit eines \emph{ἐπίσκοπος}, der eine Gemeinde Gottes zu versorgen hat (1 Tim. 3, 5). Dies wird unter dem folgendem Abschnitt ausführlicher dargelegt;\footnote{Duenstedt, I. 13: \emph{Theologia acroamatica est, quae mysteria fidei accuratius et prolixius edocet, confirmat et contrarios sanae doctrinae errores refellit atque episcoporum et presbyterorum in ecclesia.} }
    \item die Gotteserkenntnis und Gotteslehre der Lehrer der zukünftigen öffentlichen Lehrer, also die Gottesgelehrtheit der Personen, die wir heute theologische Professoren nennen. Diese Tätigkeit übte auch Timotheus aus, wenn ihm 2 Tim. 2, 2 aufgetragen wird, dass er das, was er von dem Apostel Paulus gehört hat, treuen Menschen befehlen soll (\emph{ἀγαθός}), die tüchtig sein werden, auch andere zu lehren;\footnote{Duenstedt, I. 13: \emph{Theologia acroamatica est . . . imprimis eorum, qui in academiis non Christianos simpliciter, sed futuros Christianorum doctores informant et} κατ’ ἐξοχήν \emph{theologi dicuntur. Interessant ist, was Luther speziell über „Doktorpromotionen“ gelegentlich äußert. Er nennt einerseits die Doktorpromotionen „Larven“. Andererseits verwirft er die Sache nicht schlechthin, sondern sie gefällt ihm, wenn die zu Kreierenden selbst erkennen, dass die Sache an sich nichts sei, sondern sich mit den Doktorlarven schmücken lassen zu dem Dunk am Wort. (XXI a, 334. Vgl. auch II, 260.)} }
    \item die Gottesgelehrtheit und Gotteslehre, welche allen Christen zukommt. Luther bemerkt zu den Worten Joh. 3, 16 („Also hat Gott die Welt geliebet“ usw.): „Es sind solche Worte, die niemand kann ausgründen noch erschöpfen, und ja sollten, wo sie recht geglaubt würden, einen guten Theologen oder vielmehr einen starken, fröhlichen Christen machen, der da könnte recht reden und lehren von Christo, alle andere Lehre urteilen, ja, jedermann raten und trösten und alles leiden, was ihm vorkäme“;\footnote{St. 9, XL. 1193. Luther nennt aus den Salzburgern von Kapernaum „Theologen“ mit der Begründung, er habe zu tun und fleißig diskutiert, dass es genug wäre einem, der vier Jahre wäre ein Doktor gewesen“. (XII, 1185.) Und zu Röm. 12, 7 bemerkt Luther: „Siehe, was St. Paulus für Doktoren in der Schrift macht, nämlich alle, die den Glauben haben, und sonst niemand. Dieselbigen sollen richten und urteilen alle Lehre, und ihr Urteil [soll] gelten, es treffe gleich Papst, Konzil und alle Welt an.“ (XII, 335.) Auch Gerhard erwähnt, V. de Mat. Theol., § 4, dass das Wort Theologie gebraucht werde pro fide et religione Christiana, quae omnibus fidelibus, doctis aeque ac indoctis, communis est, ut sic Theologi dicantur, quicunque norunt fidei articulos}
\end{enumerate}
