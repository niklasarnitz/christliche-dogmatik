\textnormal{174}\hfill\textnormal{Wesen und Begriff der Theologie.} vertreter dieser Periode zu nennen: Melanchthon,\footnote{Loci Communes Rerum Theologicarum seu Hypotyposes Theologicae, 1521. Wie in späteren Ausgaben (1535, 1543, besonders 1548) die \emph{sola gratia} aufgegeben und Melanchthon der Vater des Synergismus und Majorismus in der lutherischen Kirche geworden ist, ist genau dargelegt von F. Bente in der \emph{Concordia Triglotta}, „Historical Introductions to the Symbolical Books“, S. 128 f.} Chemnitz,\footnote{Loci Theologici, \ldots\ \emph{quibus et Loci Communes D. Philippi Melanchthonis perspicue explicantur et quasi integrum Christianae doctrinae corpus ecclesiae Dei sincere proponitur}. Nach Chemnitz' Loci (1586) herausgegeben von Polycarp Leyser, 1591. Chemnitz' Loci sind nicht nur als Erweiterung, sondern auch als Korrektur der späteren Ausgaben von Melanchthons Loci gemeint. Doch war Chemnitz selbst mit seinen Loci nicht ganz zufrieden, sondern dachte an eine Neubearbeitung, zu der er aber nicht mehr Zeit und Kraft fand. Chemnitz' theologische Meisterschaft kommt zur vollen Geltung in seinem \emph{Examen Concilii Tridentini} und in seiner Schrift \emph{De Duabus Naturis in Christo}, 1571.} Hutter,\footnote{Loci Communes Theologici \emph{ex sacris literis diligenter eruti, veterum patrum testimoniis roborati et conformati ad methodum locorum Phil. Melanchthonis}, 1619. Also ebenfalls eine Art Kommentar zu Melanchthons Loci und ebenfalls erst nach seinem Tode (1616) von der Wittenberger Fakultät herausgegeben (1619). Hutter nennt Melanchthon \emph{magnum illum Germaniae nostrae Phoenicem, virum undiquaque doctissimum deque re literaria universa praeclarissimum meritum}. Zugleich weist Hutter auf des späteren Melanchthon traurigen (\emph{tristis}) Abfall von der reinen Lehre hin, aber nicht ohne die Bemerkung hinzuzufügen: \emph{Illud tamen dubitatum, quin sub unem vitae seria acta poenitentia huius etiam peccati veniam a Christo et aciverint et impetrauit}. Hutters Loci bieten den Studenten der Theologie ausführlich, was in seinem \emph{Compendium Locorum Theologicorum} für Gymnasien kurz zusammengefasst ist.} Gerhard.\footnote{Loci Theologici \emph{cum pro adstruenda veritate, tum pro destruenda quorumvis contradicentium falsitate}. 1610--1621 in 9 Bänden; Nachtrag 1625. Ausgabe von Cotta (1762--1781) in 20 Bänden, mit zwei Registerbänden von G. G. Müller, 1787. 1789. Ausgabe von Ch. Preuß (Berlin 1863--1865, Leipzig 1875) in 9 Bänden, mit einem Registerband von Lohe nach G. G. Müller, Leipzig 1885.} Auch der dänische Theologe Brochmand gehört hierher.\footnote{Universae Theologiae Systema, 1633; in 6. Aufl. 1658 in Ulm erschienen. \emph{Welch' Lobt mit Recht dies Werk} (Bibliotheca Theol. I, 57). Seine Ausführung: \emph{quamvis haud difficcax, casus conscientiae haud apte ad theologiam dogmaticam relatos esse}, ist unberechtigt, weil bei einer ausführlichen Darstellung der christlichen Lehre die Beantwortung von Gewissensfragen ganz in der Ordnung ist.}\par Die analytische Methode finden wir bei den Dogmatikern etwas vor und dann nach der Mitte des 17. Jahrhunderts: bei Dann=