Wesen und Begriff der Theologie. \hfill 104\n\n\subsection*{3. Offene Fragen und theologische Probleme.}\n\nOffene Fragen sind nicht solche, über die Menschen sich nicht einigen können, auch nicht solche, über die noch keine symbolische Entscheidung vorliegt, sondern nur solche Fragen, die die heilige Schrift selbst nicht beantwortet und so ipso facto lässt.\n\nEs liegt auf der Hand, dass die heilige Schrift als Quelle und Norm der christlichen Lehre abgesetzt wird, wenn nur das als allgemein verbindliche christliche Lehre gelten soll, worüber Menschen sich einigen können. Dadurch wird Christi Wort: „Lehret sie halten alles, was ich euch befohlen habe!“ umgesetzt in die Weisung: Lehret sie halten, wofür ihr die menschliche Zustimmung erlangen könnt! Dieser schriftwidrige Gedanke liegt den vielfachen Versuchen zugrunde, kirchliche Vereinigungen ohne Einigkeit in der christlichen Lehre zustande zu bringen. Beispiele hierfür haben wir in der „Evangelischen Allianz“ (seit 1846), in der reformierten Kirche, die (Zwingli und Calvin eingeschlossen) immer Neigung gezeigt hat, sich ohne Beseitigung der Lehrdifferenzen mit der lutherischen Kirche zu vereinigen. Das neueste Beispiel eines kirchlichen Zusammenschlusses ohne tatsächliche Übereinstimmung in der Lehre liegt hierzulande vor in der United Lutheran Church („Merger-Synoden“) \footnote{381}. Dieselbe Beiseitesetzung des Schriftprinzips tritt uns entgegen, wenn neuere Lutheraner die sonderbare Ansicht aussprechen, dass nur solche Lehren als in der lutherischen Kirche verbindlich anzusehen und zu behandeln seien, über die eine Entscheidung in den symbolischen Büchern der lutherischen Kirche vorliege. Dies ist der Sache nach der römische Irrtum, den Luther mit den bekannten Worten zurückweist: „Die christliche Kirche hat keine Macht [also auch nicht die lutherische], einigen Artikel des Glaubens zu setzen, hat’s noch nie getan, wird’s auch nimmermehr tun.“ \footnote{382} Hierauf kommt das Dorpater Gutachten vom Jahre 1866 hinaus \footnote{383}.\n\nHinzugegen ist festzuhalten, dass solche Fragen als offene anzuerkennen sind, die sich wohl bei dem Nachdenken über die in der Schrift vorliegenden Lehren aufdrängen, aber in der Schrift entweder gar nicht oder doch nicht klar beantwortet werden. Dass offene\n\n\addtocounter{footnote}{380}\n\footnote{381} Vgl. F. Bente, American Lutheranism, II, S. 9. Derselbe Utopismus und Indifferentismus fand sich in der früheren Generalsynode, S. 19. 48. 170, dem General Council, S. 195. 224, und der United Synod in the South, S. 232 ff.\n\footnote{382} Ev. Cl., 122. Opp. v. a. IV, 373: St. 3. XIX, 958.\n\footnote{383} Weiteres hierüber bei der Lehre von der heiligen Schrift unter dem Abschnitte „Schrift und Symbole“.