Zerfall erkannt habe, dass die Wahrheit nur eine, und zwar eine ewige, unabänderliche und, weil von Gott geoffenbart, keiner Sichtung oder Verbesserung bedürftig ist.“ Die Leipziger Gemeinden an drei Hauptlehrern erinnern, „welche durch die Reformation unter Gottes Beistand nach langem Dunkel, das sie umhüllte, wieder ans Licht gebracht sind: die Lehre vom Ansehen der Heiligen Schrift oder des Wortes Gottes, die Lehre von der Rechtfertigung, die Lehre von den Gnadenmitteln“. Über die Heilige Schrift sagt er: „Sie allein ist der Grund, auf dem die christliche Kirche den Pforten der Hölle trost, der Prüfstein, nach dem sie Wahrheit und Lüge unterscheidet, nach der sie richtet, aber auch gerichtet werden soll. Diesem Worte muss sie sich mit Ehrfurcht, mit Demut, mit Selbstverleugnung unbedingt unterwerfen. Sie ist über dieses Wort nicht als Richterin, sondern als Haushälterin gesetzt, von der Gott Rechenschaft fordern wird; sie soll, wo sie nicht den Fluch Gottes auf sich laden will, zu diesem Worte weder etwas hinzuthun noch davon tun; sie soll ohne alle Menschenfurcht und Menschengefälligkeit ihren Glauben an dies Wort bekennen und von aller Ungerechtigkeit oder ketzerischen Lehre nach dem ausdrücklichen Befehle Gottes abtreten.“Durch Missachtung der Heiligen Schrift ist Rom gefallen. „Die Väter unserer lutherischen Kirche aber bekämpfen nicht Antichristentum mit Antichristentum, sie ordneten der Heiligen Schrift nicht etwa eine andere Erkenntnisquelle, wie die Tradition, bei oder gar über. Sie setzten an die Stelle der langgewalteten Finsternis nicht das natürliche Licht der menschlichen Vernunft, auch nicht das übernatürliche einer unmittelbaren Erleuchtung, sondern das Licht der heiligen Schrift, ohne welches die menschliche Vernunft, sie mag philosophieren oder schwärmen, auf immer blind und unerleuchtet bleibt. Freilich, ihr [Neologen] scheidet zwischen Buchstaben und Geist. Ihr schmeichelt euch, dass Luther euer Patron sei. Nie aber versteht Luther unter dem Worte Gottes etwas von dem Buchstaben der heiligen Schrift Verschiedenes, nie die Eingebung eines inneren Lichts, die Einfälle der blinden Vernunft oder die Trugbilder des verkehrten Gefühls, sondern stets das \emph{geschriebene} Wort nach seinem einfachen Wortverstande, nach seinem klaren Sinne, mit Ausschluss aller menschlichen Vermittlung, Verfälschung und Vergeistigung --- die heilige Schrift, durch welche allein, aber durch welche auch immer Gott der heilige Geist wirkt, sie werde dem Hörer oder Leser ein Geruch des Lebens zum Leben oder ein Geruch des Todes zum Tode.“ über die Stellung zu den Symbolischen