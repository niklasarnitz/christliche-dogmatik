\noindent\makebox[\linewidth]{14\hspace*{\fill}Wesen und Begriff der Theologie.}Religionen noch die nötige psychologische, geschichtliche und philosophische Betrachtung der Religionen geschehen habe. Diesen Wissenszweigen sei erst in neuerer Zeit die gebührende Aufmerksamkeit gewidmet worden. Aber auch hier liegt eine Selbsttäuschung vor. Wir kommen auch vermittelst der Religionspsychologie, der Religionsgeschichte und der Religionsphilosophie nicht über die Zweizahl der wesentlich verschiedenen Religionen hinaus. Was die psychologische Betrachtung der Religionen betrifft, so ist mit großer Energie auf die „Gleichartigkeit“ der „psychologischen Erscheinungen“ bei Nichtchristen und Christen hingewiesen worden. Weil die älteren Theologen diese Gleichartigkeit übersehen hätten, so sei es ihnen nicht möglich gewesen, die christliche Religion mit den nichtchristlichen unter ein Genus zu bringen.\footnote{38) So z. B. Rinn, Grundriß 3, S. 9.} Aber die behauptete Gleichartigkeit der psychologischen Phänomene bei Christen und Nichtchristen verschwindet sofort, sobald wir vergleichend prüfen. An die Stelle der Gleichartigkeit tritt der diametrale Gegensatz. Zu der nichtchristlichen Seele finden wir die folgenden seelischen Erscheinungen: das Schuldbewusstsein oder das böse Gewissen, die Furcht vor Strafe und damit die innerliche Flucht vor Gott, das Bestreben, durch eigene Werke die Strafe abzuwenden, und, weil das Streben nicht zum erstrebten Ziel führt, den Zustand der Todesfurcht und der Hoffnungslosigkeit.\footnote{39) Eph. 2, 12; Hebr. 2, 15. Mit Recht verweist Darle auf Eph. 2, 12 die Ausnahmen, welche Zwingli, Bucer u. a. in bezug auf einzelne Heiden annahmen, in das Gebiet der „Träume“. Luther gegen Zwinglis Seligsprechung der heidnischen Heroen Herkules, Theseus, Sokrates usw. Sl. 8. XX, 1767. Die Selbsterkenntnisse der Heiden über ihre Hoffnungslosigkeit bei Luthardt, Apol. Vortr. I, 2, Anm. 11.} Zu der christlichen Seele finden wir die entgegengesetzten Zustände und Bewegungen: das gute Gewissen durch den Glauben an die Versöhnung, die durch Christum geschehen ist,\footnote{40) Röm. 5, 1: \textgreek{Ειρηνην εχομεν προς τον θεον δια του κυριου ημων Ίησου Χριστου.}} nicht innerliche Flucht vor Gott, sondern freudigen Zutritt vor Gott,\footnote{41) Röm. 5, 2: \textgreek{Δι' ου (Χριστου) και την προσαγωγην εσχηκαμεν τη πιστει εις την χαριν ταυτην.}} nicht Furcht vor dem Tode und Hoffnungslosigkeit, sondern Triumph über den Tod\footnote{42) 1 Kor. 15, 55: \textgreek{Που σου, θανατε, το κεντρον;} Phil. 1, 23: \textgreek{Επιθυμιαν εχων εις το αναλυσαι και συν Χριστω ειναι.}} und die gewisse Hoffnung des ewigen Lebens.\footnote{43) Röm. 5, 2: \textgreek{Καυχώμεθα επ' ελπιδι της δοξης του θεου.}} So reduziert sich die „Gleichartigkeit“ der psycho-