\textnormal{87}
\hfill Wesen und Begriff der Theologie.

Nun sind aber Gesetz und Evangelium ihrem Inhalte nach vollkom=
mene Gegensätze, „plus quam contradictoria.“\footnote{292} Wie denn? Das Gesetz versucht den Menschen, der es nicht vollkommen gehal=
ten hat.\footnote{293} Nach dem Evangelium rechnet Gott dem Menschen die Uebertrettung des Gesetzes nicht zu.\footnote{294} Was soll bei dieser Sachlage der Theologe tun? Er muß, um nicht selbst in Verwirrung zu bleiben und bei andern Verwirrung anzurichten, Gesetz und Evan=
gelium voneinander zu scheiden wissen. Die rechte Scheidung besteht darin, daß er beide auf die von der Schrift bestimmten Ge=
biete zu beschränken weiß, auf denen sie nach Gottes Willen und Ordnung Geltung haben.

1. Aus dem Gesetz soll der Theologe die Erkenntnis der Sünde lehren, \foreignlanguage{greek}{ὅτι διὰ γὰρ νόμου ἐπίγνωσις ἁμαρτίας}\footnote{295} die Ver=
gebung der Sünden hingegen oder die Rechtfertigung nur aus dem Evangelium, \foreignlanguage{greek}{λογιζόμεθα οὖν πίστει δικαιοῦσθαι ἄνθρωπον χωρὶς ἔργων νόμου.}\footnote{296} Wer die Vergebung der Sünden auch aus dem Gesetz lehrt, ist kein christlicher Theologe, sondern ein Verführer, der vom Christentum abfallen lehrt, \foreignlanguage{greek}{κατηργήθητε ἀπὸ Χριστοῦ οἵτινες ἐν νόμῳ δικαιοῦσθε, τῆς χάριτος ἐξεπέσατε.}\footnote{297} Er gehört zu der Klasse von Lehrern, in bezug auf welche der Apostel ihrer Schäd=
lichkeit wegen den Wunsch ausspricht: \foreignlanguage{greek}{„Ὄφελον καὶ ἀποκόψονται οἱ ἀναστατοῦντες ὑμᾶς.“}\footnote{298} Damit hängt die Scheidung von Gesetz und Evangelium nach der Beschaffenheit der Personen zusammen. Das Gesetz ist den Menschen vorzuhalten, die noch stolz und sicher sind, das heißt, sich nicht des Zornes Gottes und der ewigen Ver=
dammnis schuldig geben. Was das Gesetz sagt, so belehrt uns der Apostel, das sagt es denen, die unter dem Gesetz sind, \foreignlanguage{greek}{ἵνα πᾶν στόμα φραγῇ καὶ ὑπόδικος γένηται πᾶς ὁ κόσμος τῷ θεῷ.}\footnote{299} Das Evangelium hingegen ist denen vorzulegen, die durch das Gesetz gedemütigte und zerschlagene Herzen haben, \foreignlanguage{greek}{πτωχοὶ εὐαγγελί=
ζονται,}\footnote{300} \foreignlanguage{greek}{εὐαγγελίζεσθαι πτωχοῖς.}\footnote{301}

\rule{\textwidth}{0.4pt}

\footnotesize
\begin{enumerate}
\item[292)] Luther zum Galaterbrief. Ed. Erl. II, 105; St. L. IX, 447.
\item[293)] Gal. 3, 10; Röm. 3, 9–19; auch 2 Kor. 3, 9: \foreignlanguage{greek}{ἡ διακονία τῆς κατακρί=
σεως.}
\item[294)] 2 Kor. 5, 19: \foreignlanguage{greek}{μὴ λογιζόμενος αὐτοῖς τὰ παραπτώματα αὐτῶν.} Apok. 20, 24: \foreignlanguage{greek}{τὸ εὐαγγέλιον τῆς χάριτος τοῦ θεοῦ.} Auch 2 Kor. 2, 9: \foreignlanguage{greek}{ἡ διακονία τῆς δικαιοσύνης.}
\item[295)] Röm. 3, 20.
\item[296)] Röm. 3, 28.
\item[297)] Gal. 5, 4.
\item[298)] Gal. 5, 12.
\item[299)] Röm. 3, 19.
\item[300)] Matth. 11, 5.
\item[301)] Luc. 4, 18.
\end{enumerate}
